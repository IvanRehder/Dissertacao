% CAPÍTULO 5 – CONCLUSÕES
%   NÃO PODE TER APENAS UMA PÁGINA!
%   O assentamento do último tijolo é tão importante quanto o do primeiro

%1. Elabore um parágrafo que introduz o capítulo: Este capítulo apresenta (descreva o objetivo do capítulo...). É constituído de N seções a saber...
%2. O Capítulo Conclusões não “gosta” de novidades
%3. Responda a pergunta da pesquisa com os elementos que você pesquisou e desenvolveu
%4. Análise do atendimento dos objetivos específicos:
%   • Descreva SE e COMO os objetivos específicos foram atendidos. Utilize as informações e resultados já apresentados nos Capítulos 3 e 4.
%5. Principais resultados obtidos:
%   • reafirme os resultados mais importantes;
%   • retome o posicionamento do seu trabalho em relação à literatura.
%6. Limitações do trabalho
%   • descreva as dificuldades encontradas;
%   • analise a delimitação do trabalho (Cap. 1) e as limitações de sua contribuição.


%7. Propostas de desenvolvimentos futuros – aquilo que você faria se tivesse tempo.
%   • Esta seção não é uma lista com marcadores!
%   • Descreva 3 propostas baseadas nas limitações de seu trabalho e as detalhe a um ponto que um outro pesquisador possa retomar e desenvolver essa proposta.

At this final chapter, the goals will be revised along with the results collected. It will be divided in four sections, one for each goal and a final one for future works and suggestions, and each section will have four more subsections, one for each data source gathered and one for a final conclusion and comentaries for that goal.

%    \item Do BVI users feel present in the VE as if they were in the real world? \label{itm:obj_first}
%    \item BVI users rely on audio cues and haptic feedback to guide. But does it rely more on you type of information than the other? \label{itm:obj_second}
%    \item Do non BVI users have the same demands and skill as BVI users when designing assistive products? \label{itm:obj_third}

%%%%%%%%%%%%%%%%%%%%%%%%%%%%%%%%%%%%%%%%%%%%%%%%%%%%%%%%%%%%%%%%%%%%%%%

\section{Do BVI users feel present in the VE as if they were in the real world?}

\subsection{Answers based on the simulation data}

Analysing the time that each user took to complete each scene, it is not possible to infer a conclusion, because this data was not meant to measure this goal.

\subsection{Answers based on the subjective data}

This data also was not made to assess this goal, because there are no base of comparison with a subjective data from before the experiment, hence before the user started to use the virtual reality.

\subsection{Answers based on the physiological data}

According to the ECG data, there was a decrease in the mental workload during the experiment while the expectation was to be an increase instead. This difference proves that the users were impact by the experiment in the virtual reality, but does not represent the same situation outside the virtual reality.

The GSR data also showed a change when the users were using virtual reality. The results showed that the users were aroused or had an increase of the mental workload.

\subsection{Final conclusions}

The physiological data gathered was the only source of data to assess this goal, and they had opposite conclusion regarding the expectation. The ECG showed a decrease on the mental workload, while the GSR showed an increase.

The ECG data is less realiable than the GSR due to the sensibilty of the sensors used. It was noted that the ECG is very sensible to movements and to the position of the sensors in relationship with the data receiver. If something stands in the way between the sensor and receiver, such as a human body, that data is lost, causing the resulting analysis to be noiseness or to be made using corrections such as the one used.

So, this goal was partially achieved.

%%%%%%%%%%%%%%%%%%%%%%%%%%%%%%%%%%%%%%%%%%%%%%%%%%%%%%%%%%%%%%%%%%%%%%%

\section{Does BVI users rely more on a type of information than the other?}

\subsection{Answers based on the simulation data}

With the regards on the time, and since the Anova test showed that all of the data are differente between each other, one can look at the Table \ref{fig:barplot_duration_global} and notice that the method that the users took the shorter time was the "Mixture" method, which was along with the expectation that the BVI users rely on both of the information, but the second shortest was the "Virtual Cane", so it indicates that the BVI user rely more on an haptic source of information. But this data is not entirely reliable, since there was a couple of mistakes in the first experiment to close each simulation, hence increasing the final time of each user at the round.

\subsection{Answers based on the subjective data}

Analysing the Figures \ref{fig:boxplot_md_scene}, \ref{fig:boxplot_nasa_scene}, \ref{fig:boxplot_sagat_scene} and \ref{fig:boxplot_questionnaire_scene} one can notice that the haptic source of information is preferable for they have the best results in general for each questionnaire, but the Anova test disagree with that conclusion in some cases, but that can a conseguence of the fact that only 4 individuals of each group did the experiment. 

\subsection{Answers based on the physiological data}

Disconsidering that all of the ECG data were against the expected variation, according to the Anova test, only the "Audio" method can be concluded that is different the "Base" method and Figure \ref{fig:barplot_ecg_bpm_scene_blind} shows a similar conclusion.

The skin conductance Anova test resulted that only the "Virtual Cane" and the "Mixture" method are different than the "Base" method, and, also according with the skin data, they aroused more the user or have higher mental workload. Another conclusion from the Anova test is that "Audio" method has a similar workload than the "Base" method, and ironically this was the only one that could be said that arouses less or has a lesser mental workload.

\subsection{Final conclusions and comentaries}

The majority of the graphics showed an tendecy that haptics source of data are more favorable for the BVI users, but the conclusion draw by the hypothesis test did not support that analysis. This may be due to the small sample size.

One observation made during the experiment, is that the BVI users during the "Audio" and "Mixture" method did not used, or used only a few times, the audio guidance provided by the researcher. This does not discard that they did not rely on sound information, because the simulation has filled with audio cues. This may be because of their previous experience in navigation and mobility alone.

The final conclusion for this goal would be that they do rely more in a mixture of haptic and audio data, the first for obstacles detection at and short distance, the latter for a guidance and information gathered at bigger distances.

%%%%%%%%%%%%%%%%%%%%%%%%%%%%%%%%%%%%%%%%%%%%%%%%%%%%%%%%%%%%%%%%%%%%%%%

\section{Do non BVI users have the same demands and skill as BVI users when designing assistive products?}

\subsection{Answers based on the simulation data}

Results from the simulation data and the T-Test showed that the only time data that was different between the groups is the "Audio". Analysing the rest of the data one can conclude that the results had no difference.

Results from the Figure \ref{fig:boxplot_duration_scene} showed a rather similar average duration between the two groups going along with the conclusion from the T-Test, but there is the matter of the unreliability of this data meantioned before.

\subsection{Answers based on the subjective data}

The T-Test of each questionnaire showed that there are no differences between the groups, but the Figures \ref{fig:boxplot_md_scene}, \ref{fig:boxplot_nasa_scene} and \ref{fig:boxplot_sagat_scene} showed a noticible difference between the groups. This unmatched results may be because of the small sample number.

\subsection{Answers based on the physiological data}

The Figure \ref{fig:boxplot_ecg_sdnn_box_scene} indicates that there are not visual difference between the groups. The Figure \ref{fig:boxplot_ecg_bpm_scene} indicates a difference on the distribution and a rather similar average. The Figure \ref{fig:boxplot_gsr_scene} indicate a higher arousal or mental workload by the "blind" users. All the T-Test indicate that both groups have the same variation of workload and arouseness.

\subsection{Final conclusions and comentaries}

The T-Test results showed in general that all both groups had similar results, while the some graphics showed the opposite. This may happened because of may be for two reasons. First because of a small sample size. Second because of a tendecy of the "sight" sample. The sighted participant all were used to technology and to voluntiring for experiments, while the same can not be said for the BVI participants.

To close up, this goal is considered not achieved for lack of a bigger and more diverse sample size.

%%%%%%%%%%%%%%%%%%%%%%%%%%%%%%%%%%%%%%%%%%%%%%%%%%%%%%%%%%%%%%%%%%%%%%%

\section{Future works and suggestions}

For future works related to the this one it could be suggested:

\begin{itemize}
    \item Repeat the experiment in a real situation and compare with this one to verift the first goal;
    \item Repeat the experiment with more devices with different proporsions of haptic and audio information sources;
    \item Repeat the experiment with a bigger sample size and a more and diverse sample to verify if the results of the hypothesis test do remain the same;
\end{itemize}