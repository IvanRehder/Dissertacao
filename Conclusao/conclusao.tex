% CAPÍTULO 5 – CONCLUSÕES
%   NÃO PODE TER APENAS UMA PÁGINA!
%   O assentamento do último tijolo é tão importante quanto o do primeiro

%1. Elabore um parágrafo que introduz o capítulo: Este capítulo apresenta (descreva o objetivo do capítulo...). É constituído de N seções a saber...
%2. O Capítulo Conclusões não “gosta” de novidades
%3. Responda a pergunta da pesquisa com os elementos que você pesquisou e desenvolveu
%4. Análise do atendimento dos objetivos específicos:
%   • Descreva SE e COMO os objetivos específicos foram atendidos. Utilize as informações e resultados já apresentados nos Capítulos 3 e 4.
%5. Principais resultados obtidos:
%   • reafirme os resultados mais importantes;
%   • retome o posicionamento do seu trabalho em relação à literatura.
%6. Limitações do trabalho
%   • descreva as dificuldades encontradas;
%   • analise a delimitação do trabalho (Cap. 1) e as limitações de sua contribuição.


%7. Propostas de desenvolvimentos futuros – aquilo que você faria se tivesse tempo.
%   • Esta seção não é uma lista com marcadores!
%   • Descreva 3 propostas baseadas nas limitações de seu trabalho e as detalhe a um ponto que um outro pesquisador possa retomar e desenvolver essa proposta.

This work proposes the use of virtual reality to create an environment where concepts of assistive devices could be evaluated at early stages of development by blind and visual impaired (BVI) people.

In order to systematize this proposal, this work presents a method composed of five phases that guide the development of the virtual environment in parallel with the design of the assistive devices and the proposal of assessment methods.

In order to illustrate the proposed method and investigate two research questions related to this work, it describes an application of the method for the evaluation of four different solutions of assistive device in the environment of a hospital reception. For this example, it proposes as assessment method the use subjective questionnaires and physiological sensor. 

In order to evaluate situation awareness, this work proposes an adapted version of the SAGAT questionnaire, which was initially introduced for evaluating the situation awareness of air traffic controllers. 

Based on the results from the hospital reception example, the two research questions are discussed.



%    \item Do BVI users feel present in the VE as if they were in the real world? \label{itm:obj_first}
%    \item BVI users rely on audio cues and haptic feedback to guide. But does it rely more on your type of information than the other? \label{itm:obj_second}
%    \item Do non-BVI users have the same demands and skills as BVI users when designing assistive products? \label{itm:obj_third}

%%%%%%%%%%%%%%%%%%%%%%%%%%%%%%%%%%%%%%%%%%%%%%%%%%%%%%%%%%%%%%%%%%%%%%%

\subsection*{Is it possible to evaluate and compare concepts of assistive device from a human factors’ perspective in a virtual environment? What are the main limitations of the use of a virtual reality environment?
}

The example presented in this work showed that it is possible to evaluate both situation awareness and workload using experiments performed in a virtual environment. The tests performed in the virtual environment made possible the comparison of the assistive devices both qualitative and quantitative.

However, a number of limitations were identified during the development of this work, regarding both the virtual environment and the assessment techniques.

One of the most recurrent observations was the unsatisfactory quality of the sound system. According to blind participants, the headphone of the VIVE HMD does not provide sounds with a quality good enough for them to locate the source of a sound. A common comment was “I feel like the sound origin is inside my head”. This limitation may be solved by placing a real sound source in the real environment and use the HMD only for geolocalizing the participant in the virtual environment.

Another limitation is the actual position of the furniture. More than once, after a ‘first round’, the furniture was not precisely aligned with its virtual model. A future solution for this problem would be to use a locator on each piece of furniture.

Among the main limitations identified during this work, it worth also mentioning the failure in detecting collisions in the virtual environment, which could be solved by integrating sensors that monitor the position of each arm and leg of the user.

Regarding the assessment methods for evaluating human factors, the physiological sensors did not show any systematic difference among the methods under analysis. This result may be due to the presence of noise in the sensors data, compromising its quality. Another problem is that the low number of participants may compromise the statistical analysis, due to the large variability among users.

\subsection*{Do non-BVI users, when deprived from their vision, evaluate assistive devices in a similar way as BVI users?}

Comparing the results of experiments performed with blind and sighted participants, a number of differences were observed. Among the most important ones, is the relative evaluation of audio and haptic devices. Due to their enhanced sensitive to sounds, BVI users tend to evaluate audio solutions better than non-BVI. Also, the effect of repeating a task in the same environment, i.e., performing different rounds of the same experiment, may differ between sighted and blind users.

Generally, the results reinforce the importance of having BVI users involved in the design of assistive devices from the early stages of specification of requirements.

%%%%%%%%%%%%%%%%%%%%%%%%%%%%%%%%%%%%%%%%%%%%%%%%%%%%%%%%%%%%%%%%%%%%%%%

\section{Future works and suggestions}

The following topics are of interest for future research:

\begin{itemize}
    \item Perform a comparison between an evaluation campaign executed in a real environment and the same campaign executed in a virtual environment, with the purpose of assessing the main differences brough by the use of virtual reality;
    \item Further improve the virtual reality environment by providing better sound solutions, using different sources of sound in the real environment instead of using the sound from the HMD;
    \item Develop a solution for automatic collision detection in the virtual system in order to introduce performance metrics in the assessment.
    \item Repeat the experimental campaign with large sample of both BVI and non-BVI users, in order to improve the statistical analysis.
    \item Investigate the sources of noise of the physiological sensors and improve their data acquisition.
\end{itemize}