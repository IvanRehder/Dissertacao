% CAPÍTULO 5 – CONCLUSÕES
%   NÃO PODE TER APENAS UMA PÁGINA!
%   O assentamento do último tijolo é tão importante quanto o do primeiro

%1. Elabore um parágrafo que introduz o capítulo: Este capítulo apresenta (descreva o objetivo do capítulo...). É constituído de N seções a saber...
%2. O Capítulo Conclusões não “gosta” de novidades
%3. Responda a pergunta da pesquisa com os elementos que você pesquisou e desenvolveu
%4. Análise do atendimento dos objetivos específicos:
%   • Descreva SE e COMO os objetivos específicos foram atendidos. Utilize as informações e resultados já apresentados nos Capítulos 3 e 4.
%5. Principais resultados obtidos:
%   • reafirme os resultados mais importantes;
%   • retome o posicionamento do seu trabalho em relação à literatura.
%6. Limitações do trabalho
%   • descreva as dificuldades encontradas;
%   • analise a delimitação do trabalho (Cap. 1) e as limitações de sua contribuição.


%7. Propostas de desenvolvimentos futuros – aquilo que você faria se tivesse tempo.
%   • Esta seção não é uma lista com marcadores!
%   • Descreva 3 propostas baseadas nas limitações de seu trabalho e as detalhe a um ponto que um outro pesquisador possa retomar e desenvolver essa proposta.

In this final chapter, the goals will be revised along with the results collected. It will be divided into four sections, one for each goal and a final one for future works and suggestions, and each section will have four more subsections, one for each data source gathered and one for a conclusion and commentaries for that goal.

%    \item Do BVI users feel present in the VE as if they were in the real world? \label{itm:obj_first}
%    \item BVI users rely on audio cues and haptic feedback to guide. But does it rely more on your type of information than the other? \label{itm:obj_second}
%    \item Do non-BVI users have the same demands and skills as BVI users when designing assistive products? \label{itm:obj_third}

%%%%%%%%%%%%%%%%%%%%%%%%%%%%%%%%%%%%%%%%%%%%%%%%%%%%%%%%%%%%%%%%%%%%%%%

\begin{itemize}
    \item \textbf{Is it possible to evaluate and compare concepts of assistive device from a human factors’ perspective in a virtual environment? What are the main limitations of the use of a virtual reality environment?}

    Based on the gathered data, there was a variation in the mental workload and in the situation awareness during the experiment. This variation show that the users were impacted by the experiment in the virtual reality, but since no experiment outside the virtual reality was made, it is not possible to compare this data and verify that they are similar to one provided by a real scenario.

    Although there was variations inside the experiment, there was also some unexpected results. The heartbeat and the interbeat interval standard deviation did not show the same results as the NASA-TLX indicated. That could be by the fact that the parcipants walked the majority of the time, and that polluted the sensor data, leaving to unrelatable results. It could also be caused because the experiment was made using a virtual reality, and this may have "relaxed" the participants.

    As for the limitations, the participants complained about the sound. The integrate headphone of the VIVE HMD did not provide sounds with a quality good enought for they to locate. A common commentarie was "I feel like the sound origin is inside my head", which was not true. But this can be solved by placing a real sound source in the real environment and use the HMD only for geolocalizing the participant inside de virtual environment.

    Another limitation is the real time position of the furniture. More than once, after a "First round" a furniture was not well aligned with the its virtual model. That caused some frustation on the participant as well in the researchers that had to stop the experiment to fix their position. A solution for this it would be to install real time locator on each piece of furniture.

    \item \textbf{Do non-BVI users, when deprived from their vision, evaluate assistive devices in a similar way as BVI users?}

    \begin{itemize}
        \item Answers based on the simulation data

    Results from the simulation data and the T-Test showed that the only time data that was different between the groups is the "Audio". Analyzing the rest of the data one can conclude that the results had no difference.

    Graphically it is possible to notice a rather similar average duration between the two groups going along with the conclusion from the T-Test, but there is the matter of the unreliability of this data mentioned before.

    \item Answers based on the subjective data

    The T-Test of each questionnaire showed that there are no differences between the groups, but the graphically is noticeable a difference between the groups. These unmatched results may be because of the small sample number.

    \item Answers based on the physiological data

    The Figures indicate that there are the groups may have a similar average, but the variation of them are different in most cases. All the T-Tests indicate that both groups have the same variation of workload and arouse.

    
    \item Final conclusions and comentaries

    The T-Test results showed in general that both groups had similar results, while some graphics showed the opposite. This happened maybe because of two reasons. First because of a small sample size. Second because of a tendency of the "sight" sample. The sighted participant all were used to technology and volunteering for experiments, while the same can not be said for the BVI participants.

    \end{itemize}
\end{itemize}

%\subsection{Answers based on the simulation data}

%Analyzing the time that each user took to complete each scene, it is not possible to infer a conclusion, because this data was not meant to measure this goal.

%\subsection{Answers based on the subjective data}

%This data also was not made to assess this goal, because there is no base of comparison with subjective data from before the experiment, hence before the user started to use the virtual reality.

%\subsection{}

%\subsection{Final conclusions}

%%%%%%%%%%%%%%%%%%%%%%%%%%%%%%%%%%%%%%%%%%%%%%%%%%%%%%%%%%%%%%%%%%%%%%%

%\section{Does BVI users rely more on one type of information than the other?}
%
%\subsection{Answers based on the simulation data}
%
%With the regard to the time, and since the Anova test showed that all of the data are different from each other, one can look at the Table \ref{fig:barplot_duration_global} and notice that the method that the users took the shorter time was the "Mixture" method, which was along with the expectation that the BVI users rely on both of the information, but the second shortest was the "Virtual Cane", so it indicates that the BVI user relies more on a haptic source of information. But this data is not entirely reliable, since there were a couple of mistakes in the first experiment to close each simulation, hence increasing the final time of each user at the round.
%
%\subsection{Answers based on the subjective data}
%
%Analysing the Figures \ref{fig:boxplot_md_scene}, \ref{fig:boxplot_nasa_scene}, \ref{fig:boxplot_sagat_scene} and \ref{fig:boxplot_questionnaire_scene} one can notice that the haptic source of information is preferable for they have the best results in general for each questionnaire, but the Anova test disagree with that conclusion in some cases, but that can a conseguence of the fact that only 4 individuals of each group did the experiment. 
%
%\subsection{Answers based on the physiological data}
%
%Disconsidering that all of the ECG data were against the expected variation, according to the Anova test, only the "Audio" method can be concluded that is different from the "Base" method and Figure \ref{fig:barplot_ecg_bpm_scene_blind} shows a similar conclusion.
%
%The skin conductance Anova test resulted that only the "Virtual Cane" and the "Mixture" method are different than the "Base" method, and, also according to the skin data, they aroused more the user or have a higher mental workload. Another conclusion from the Anova test is that the "Audio" method has a similar workload to the "Base" method, and ironically this was the only one that could be said that arouses less or has a lesser mental workload.
%
%\subsection{Final conclusions and comentaries}
%
%The majority of the graphics showed a tendency that haptics sources of data are more favorable for the BVI users, but the conclusion drawn by the hypothesis test did not support that analysis. This may be due to the small sample size.
%
%One observation made during the experiment is that the BVI users during the "Audio" and "Mixture" method did not use, or used only a few times, the audio guidance provided by the researcher. This does not discard that they did not rely on sound information, because the simulation has filled with audio cues. This may be because of their previous experience in navigation and mobility alone.
%
%The conclusion for this goal would be that they do rely more on a mixture of haptic and audio data, the first for obstacle detection at and short distance, the latter for guidance and information gathered at bigger distances.
%
%%%%%%%%%%%%%%%%%%%%%%%%%%%%%%%%%%%%%%%%%%%%%%%%%%%%%%%%%%%%%%%%%%%%%%%

%\section{Do non-BVI users, when deprived from their vision, evaluate assistive devices in a similar way as BVI users?}
%
%%\subsection{Answers based on the simulation data}
%
%Results from the simulation data and the T-Test showed that the only time data that was different between the groups is the "Audio". Analyzing the rest of the data one can conclude that the results had no difference.
%
%Graphically it is possible to notice a rather similar average duration between the two groups going along with the conclusion from the T-Test, but there is the matter of the unreliability of this data mentioned before.
%
%\subsection{Answers based on the subjective data}
%
%The T-Test of each questionnaire showed that there are no differences between the groups, but the graphically is noticeable a difference between the groups. These unmatched results may be because of the small sample number.
%
%\subsection{Answers based on the physiological data}
%
%The Figures indicate that there are the groups may have a similar average, but the variation of them are different in most cases. All the T-Tests indicate that both groups have the same variation of workload and arouse.
%
%\subsection{Final conclusions and comentaries}
%
%The T-Test results showed in general that both groups had similar results, while some graphics showed the opposite. This happened maybe because of two reasons. First because of a small sample size. Second because of a tendency of the "sight" sample. The sighted participant all were used to technology and volunteering for experiments, while the same can not be said for the BVI participants.

%%%%%%%%%%%%%%%%%%%%%%%%%%%%%%%%%%%%%%%%%%%%%%%%%%%%%%%%%%%%%%%%%%%%%%%

\section{Future works and suggestions}

For future works related to this one it could be suggested:

\begin{itemize}
    \item Repeat the experiment in a real situation and compare it with this one to verify the first goal;
    \item Repeat the experiment with more devices with different proportions of haptic and audio information sources;
    \item Repeat the experiment with bigger sample size and a more diverse sample to verify if the results of the hypothesis test do remain the same;
\end{itemize}