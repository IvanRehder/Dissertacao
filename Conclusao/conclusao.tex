% CAPÍTULO 5 – CONCLUSÕES
%   NÃO PODE TER APENAS UMA PÁGINA!
%   O assentamento do último tijolo é tão importante quanto o do primeiro

%1. Elabore um parágrafo que introduz o capítulo: Este capítulo apresenta (descreva o objetivo do capítulo...). É constituído de N seções a saber...
%2. O Capítulo Conclusões não “gosta” de novidades
%3. Responda a pergunta da pesquisa com os elementos que você pesquisou e desenvolveu
%4. Análise do atendimento dos objetivos específicos:
%   • Descreva SE e COMO os objetivos específicos foram atendidos. Utilize as informações e resultados já apresentados nos Capítulos 3 e 4.
%5. Principais resultados obtidos:
%   • reafirme os resultados mais importantes;
%   • retome o posicionamento do seu trabalho em relação à literatura.
%6. Limitações do trabalho
%   • descreva as dificuldades encontradas;
%   • analise a delimitação do trabalho (Cap. 1) e as limitações de sua contribuição.


%7. Propostas de desenvolvimentos futuros – aquilo que você faria se tivesse tempo.
%   • Esta seção não é uma lista com marcadores!
%   • Descreva 3 propostas baseadas nas limitações de seu trabalho e as detalhe a um ponto que um outro pesquisador possa retomar e desenvolver essa proposta.

In this final chapter, the goals will be revised along with the results collected. It will be divided into four sections, one for each goal and a final one for future works and suggestions, and each section will have four more subsections, one for each data source gathered and one for a conclusion and commentaries for that goal.

%    \item Do BVI users feel present in the VE as if they were in the real world? \label{itm:obj_first}
%    \item BVI users rely on audio cues and haptic feedback to guide. But does it rely more on your type of information than the other? \label{itm:obj_second}
%    \item Do non-BVI users have the same demands and skills as BVI users when designing assistive products? \label{itm:obj_third}

%%%%%%%%%%%%%%%%%%%%%%%%%%%%%%%%%%%%%%%%%%%%%%%%%%%%%%%%%%%%%%%%%%%%%%%

\subsection*{Is it possible to evaluate and compare concepts of assistive device from a human factors’ perspective in a virtual environment? What are the main limitations of the use of a virtual reality environment?
}

As for the experiment used for to study this goal, the blind users were more affected by the rounds them by the methods, in both mental workload and situation awareness. And when impacted by the methods, the presence of a haptic device provoked negative conseguences on their perception or mental workload.

For the sighted users it were affected by both method and round in some cases, meaning that they were more sensible to the experiment than the blind users. There was no pattern in which method they performed better or not but in overall their performance was inferior than the perfomance from blind users.
    
Based on the gathered data, there was a variation in the mental workload and in the situation awareness during the experiment. This variation show that the users were impacted by the experiment in the virtual reality, but since no experiment outside the virtual reality was made, it is not possible to compare this data and verify that they are similar to one provided by a real scenario.

Some problems of the method was that the users walked approximately half of the experiment duration, and problably added some noise to the sensor data, leaving to unrelatable results. The heartbeat and the interbeat interval standard deviation did not show the same results as the NASA-TLX indicated. 

This could also be caused because the experiment was made using a virtual reality, a technology still unvisited by most of the participants. That could have made the participant anxious and risen their heartbeat at the beginning of the experiment.

As for the limitations, the participants complained about the sound. The integrate headphone of the VIVE HMD did not provide sounds with a quality good enought for them to locate. A common commentarie was "I feel like the sound origin is inside my head", which was not true. But this may be solved by placing a real sound source in the real environment and use the HMD only for geolocalizing the participant inside de virtual environment.

Another limitation is the real time position of the furniture. More than once, after a "First round" a furniture was not well aligned with the its virtual model. That caused some frustation on the participant as well in the researchers that had to stop the experiment to fix their position. A solution for this it would be to install real time locator on each piece of furniture.

\subsection*{Do non-BVI users, when deprived from their vision, evaluate assistive devices in a similar way as BVI users?}

Comparing the results from the analyzes of the "blind" sample and the "sight" sample one realizes that the groups felt different reaction. Most of the blind users felt a bigger impact between the rounds than by the methods, whilst the sight users felt a bigger impact by the method.

This may be biased, since most of the conception of the devices was made by a sighted researcher, even thought there was recommendations from BVI researchers. But this may only reinterate that sighted users have a difficulty imagine how a blind users perceives the world and how to develop ways to assist them.


%%%%%%%%%%%%%%%%%%%%%%%%%%%%%%%%%%%%%%%%%%%%%%%%%%%%%%%%%%%%%%%%%%%%%%%

\section{Future works and suggestions}

For future works related to this one it could be suggested:

\begin{itemize}
    \item Repeat the experiment in a real situation and compare it with this one to verify the first goal;
    
    This experiment was made exclusivily using virtual reality, hence it is not posibly to verify the efficiency or the quality of a experiment made using Virtual Reality.

    \item Repeat the experiment not using the sound from the HMD;
     
    The BVI users complained about the VIVE HMD sound system. They got confused sometimes and could figure it out if a sound source was coming from forwards or backwards. One alternative for this problem is to add a physical sound source at each point in the real environment where it was supose to be in the virtual environment. It still related to the virtual reality but it is more realistic. 

    \item Repeat the experiment with bigger sample size and a more diverse sample to verify if the results of the hypothesis test do remain the same;
    
    As commented before, most ANOVA tests showed one result and the figures showed a different conclusion. This happend because of the small sample size. If the sample size was bigger, maybe both conclusions would be the same.

\end{itemize}