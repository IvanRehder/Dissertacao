%\section{Motivation}

% Motivação: contextualização do problema ou da necessidade que motivou o trabalho. Essa seção deve preparar o leitor para a <Pergunta da Pesquisa>

According to the World Health Organisation (WHO), there are at least 2.2 billion people with some visual impairment degree \cite{world2019world}. Among them, 43,3 million are classified as blind and 295 million have moderate or severe vision impairment. In order to be fully integrated into our society, they rely on assistive devices, such as canes, braille speakers, among others \cite{bourne2021trends}. 

Although a range of products has already been proposed, incorporating different features, they do not entirely fulfill their aim. Among the problems, of the solutions available in the market, are the lack of practicality and portability, invasive and requiring too much effort to learn \cite{lozano2009electrotactile}.

The difficulty of using or learn how to use a device could be avoided if concepts from human factors, or ergonomics, were analysed during the product’s development, using appropriate methods. The early application of these methods and tests could be a gamechanger for the success of the product's user experience \cite{wolf2019towards}.

Motivated by the dissatisfaction of blind people with the currently available products, this dissertation starts from the hypothesis that a human-factors-centred design of assistive devices for blind and visually impaired people (BVIs) requires the involvement of BVIs in the design process in order to evaluate the product under design. The user has to test the product under development to provide feedback for the design team to improve the product.

In order to approach this problem, this work proposes using virtual reality (VR) as a tool for creating virtual environments, where proof of concepts or prototypes of assistive devices could be easily tested by BVIs. VR can be used to create specific, immersive and interactive situations that could help the user to learn and train \cite{farrell2018learning}, and the the developers to create more user-friendly products.

In a virtual environment, as long as the BVI is wearing a locating system, he/she can navigate the environment. Any information about the scenario, such as the position of objects and their distances to the user, is known and could be extracted from the virtual platform. As a consequence the designer can test different ways of translating this information into inputs before actually implementing a prototype of the assistive device, providing a flexible, safe and easy way to have it evaluated by different users.

As a second motivation, this dissertation considers the COVID-19 pandemic scenario that dominated the world during the last two years. In order to try to slow the rate at which the virus spread, WHO recommended strategies such as wearing face masks, washing hands regularly, social distancing, and avoiding touching surfaces that have not been disinfected \cite{who_2020}. The recommendations bring additional difficulties for BVI people as the touch is one of the senses they rely on to compensate for the visual impairment. The BVI depends on others to do daily activities \cite{jondani2021strategies}. The development of solutions that can guide a BVI in an environment respecting social distance and other recommendations is also considered in this work.

\section{Objectives}
\label{sec:objetivos}

 This dissertation proposes the use of virtual reality as a tool for evaluating proofs of concept of assistive devices for blind and visually impaired people from a human-factors perspective. The purpose is to provide a flexible and easily configured way of testing different concepts of assistive devices in order to support an agile and user-centered development.

 This goal is related to the following research questions, which are investigated in this work:

 \begin{itemize}
    \item Is it possible to evaluate and compare concepts of assistive devices from a human factors perspective in a virtual environment? What are the main limitations of the use of a virtual reality environment? \label{itm:obj_first}
    \item Do non-BVI users, when deprived of their vision, similarly evaluate assistive devices as BVI users? \label{itm:obj_second}
\end{itemize}
  
 To investigate these research questions and achieve the proposed goals, the following specific objectives are defined:
 
 \begin{itemize}
     \item Select a scenario for testing assistive devices and develop it in a virtual environment; \label{itm:subobj_first}
     \item Develop three concepts of assistive devices that use different senses to provide input to the BVI; \label{itm:subobj_second}
     \item Propose a set of methods for BVI to evaluate assistive devices from human factors perspective; \label{itm:subobj_third}
     \item Design and execute an experiment to evaluate the concepts of assistive devices in the virtual environment using the proposed methods. \label{itm:subobj_forth}
 \end{itemize}
 
\section{Resources and methods} 

This work adopts an experimental approach to evaluate the proposal of this dissertation and to investigate the questions stated in Section \ref{sec:objetivos}. 
The work is organized in the following steps, illustrated in Figure \ref{fig:steps_work}:

\begin{enumerate}[leftmargin = 6em, label = Step \arabic* -- ]
    \item Literature review 
    
    It is composed of two parts. The first is to review the fundamental concepts related to the topics covered in this work: human factors and virtual reality. The second part aims at contextualizing the dissertation’s proposal. It reviews recently published works on the development and evaluation of assistive devices for BVI people.
    
    \item Specification of examples of the virtual environment and assistive devices

    This step consists of specifying one example of a virtual environment and a few examples of assistive devices to test the proposed approach of using virtual reality for evaluating purposes. Considering the above-mentioned motivation related to the covid-19 pandemic, the chosen virtual environment is the reception of a health clinic. The assistive devices used as examples are: an audio system, a haptic belt and a virtual cane, which could be used as stand-alone devices or combined.
    
    \item Development of the specified virtual environment
    
    The virtual environment of a health clinic reception is developed in the Unit3D environment. The HTC VIVE VR Head Mounted Device (HMD) is used as a localizing system to define the user's position inside the virtual environment.
    
    \item Development of proofs of concept of the specified assistive devices
    
    The three examples of devices are developed using low-cost and available laboratory equipment. The audio guide is developed using the audio system of HTV VIVE HMD, while the virtual cane is developed using the HTC VIVE VR hand controller. Finally, the haptic belt is developed using an ESP32 microcontroller, eight vibrating motors 1027 and 3D printed pieces.
    
    \item Design and execution of the experiment
    
    The proposed experiment is based on the best practices and principles of the Design of Experiment (DoE) discipline.
    The following techniques and tools are used for evaluating human factors:

    \begin{enumerate}[label = \alph*)]
        \item Questionnaires adapted from the literature, such as NASA-TLX and SAGAT, or explicitly proposed for this work;
        \item Physiological sensors, such as GSR and ECG, to capture the body's response.
    \end{enumerate}
    
    \item Analysis of results
    
    The results of the experiment are graphically and statistically analysed to estimate the user's mental workload and situation awareness. In the statistical analysis, the outputs are verified for normality distribution, then pairwise compared using the Student’s T-Test, and their variance inside the group is verified using an ANOVA. Finally, when needed, a Fisher’s Least Squared Difference Test is done to verify similarities between pairs.

\end{enumerate}

\begin{figure}[!htb]
    \centering
    %\tikzstyle{every node}=[font=\large]

    \tikzstyle{start} = [rectangle, rounded corners, minimum width=3cm, minimum height=1.0cm,text centered, draw=black, fill=white!30, text width=3cm]
    \tikzstyle{steps} = [rectangle, minimum width=3cm, minimum height=1.0cm, text centered, draw=black, fill=white!30, text width=3cm]
    
    \tikzstyle{arrow} = [ccmDBlue, rounded corners, line width = 1mm, ->]
    
    \begin{tikzpicture}[node distance=2.5cm]
        \centering
        \node (start) [start] {Dissertation};

        \node (review) [steps, below of=start] {Literature Review};
        \draw [arrow] (start) to node[midway,right]{} (review);

        \node (examples) [steps, below of=review] {Examples specification};
        \draw [arrow] (review) to node[midway,right]{} (examples);
        
        \node (ve) [steps, below of=examples, left of = examples] {VE Development};
        \draw [arrow] (examples) to ++(0,-1.25cm) to ++(-2.5cm,0) to node[midway,right]{} (ve.north);
        
        \node (devices) [steps, below of=examples, right of = examples] {Devices development};
        \draw [arrow] (examples) to ++(0,-1.25cm) to ++(2.5cm,0) to node[midway,right]{} (devices.north);

        \node (experiment) [steps, below of=devices, left of = devices] {Experiment};
        \draw [arrow] (ve) to ++(0,-1.25cm) to ++(2.5cm,0) to node[midway,right]{} (experiment.north);
        \draw [arrow] (devices) to ++(0,-1.25cm) to ++(-2.5cm,0) to node[midway,right]{} (experiment.north);

        \node (analysis) [steps, below of=experiment] {Analysis of results};
        \draw [arrow] (experiment) to node[midway,right]{} (analysis);
    \end{tikzpicture}
        \centering
        \caption{Steps of this work}
        \label{fig:steps_work}
\end{figure}
\FloatBarrier

\section{Research boundaries}

% Delimitação da pesquisa: é o recorte do seu trabalho

The concepts of assistive devices presented as part of this work are used only as examples for investigating the research questions presented in Section \ref{sec:objetivos}. The challenges related to their full development up to high Technology Readiness Levels (TRLs), as well as their feasibility as commercial products, are out of the scope of this work.

% Estrutura do texto

\section{Structure of the text}

This dissertation is organized into seven additional chapters as follows.

Chapter \ref{ch:fundamentacao} introduces the concepts and techniques that are used in this work. It starts with a review of human factors, emphasizing mental workload and situational awareness, and introduces some human factors’ evaluation tools and techniques. Then, it presents the definitions of virtual reality (VR) and Extended Reality (XR) and, to conclude, discusses the concept of co-design.

Chapter \ref{ch:revisao} is dedicated to the state of the art. It brings a review of the literature and discusses published research works that are related to this dissertation. It covers the proposal and evaluation of BVI devices with emphasis on human factors analysis or virtual reality.

Chapter \ref{ch:metodologia} details the proposal of this dissertation describing how virtual reality could be used to integrate BVI users into the design process of assistive design. It illustrates the proposed method by applying it to evaluate three different assistive devices (audio guide, virtual cane and haptic belt), as well as their mixed-use, in the environment of a hospital reception. 

Chapter \ref{ch:resultados} describes the experiment designed to evaluate the dissertation’s proposal and analyses the results in order to investigate the research questions of Section \ref{sec:objetivos}

Finally, Chapter \ref{ch:conclusao} summarizes the main conclusions of this work and discusses future work.
