%% Abstract (Resumo em inglês)

Society has developed technology to create autonomous vehicles and to connect different devices and machinery to exchange data and optimize production efficiency.  With this technology, soon, it will be possible to achieve better methods to guide blind and visually impaired (BVI) users in their daily activities. The available products in the market have several limitations and do not satisfy BVI users. We believe that one of the reasons behind this problem is that they are not members of the development team or are not consulted by these. 

The lack of an efficient solution for BVI users' navigation became even more significant with the SARS-CoV2 pandemic, in which people had to avoid contact with one another and not touch another surface.

The purpose of this dissertation is to use virtual reality (VR) to test and evaluate different designs of BVI products. Also to verify if BVI and non-BVI users have the same mental demand and situation awareness when using assistive products. The idea is to use VR as a testing ground where a BVI user can try different assistive solutions in different scenarios. By doing so, the user becomes part of the product design and evaluation, resulting in better and more user-friendly products. The proposed method includes not only the setup of the virtual environment but also the use of physiological sensors and subjective tests to assess the mental workload and situational awareness in different situations.

To illustrate the proposed method, a case study is proposed, in which the navigation of BVI users inside a medical clinic is studied. This case study is chosen due to the current undergoing SARS-CoV-2 pandemic and the impact on BVI people, so the simulated clinic is also applying COVID health protocols.

The scenes were made using Unity3D, a widely used development platform for virtual reality applications. The VR device was the Tobii Eye Tracking VR, a head-mounted display for virtual reality developed using the HTC VIVE. This VR device is used for defining the user position and orientation inside the virtual environment. Based on the current situation in the virtual environment, inputs are provided to the user using aural commands and haptics devices. To assess the mental workload, physiological sensors, from TEA Captiv T-Sens, are used. Among them, are an electrocardiogram sensor (ECG), to gather heart-rate and heart-rate variance data, and a galvanic skin reaction sensor (GSR), to collect skin conductance. Besides these sensors, the users are also expected to answer mental workload assessment tests and situation awareness questionnaires.

Among the proposed method's expected benefits are the flexibility and agility to create different scenarios, and also the possibility to test all of them in the same physical room. The method could not only speed the design of new solutions but also improve the overall quality of the products and verify the need of a BVI user in the development team of an assistive product.