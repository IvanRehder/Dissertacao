%% Resumo

A sociedade alcançou tecnologia para criar veículos autônomos e conectar diferentes aparelhos e máquinas umas às outras a fim de trocar informações e otimizar a eficiência de produção. Com essa tecnologia, logo será possível obter melhores métodos para orientar usuários cegos e deficientes visuais (CDV) nas suas atividades diárias. Os produtos que estão disponíveis no mercado hoje em dia possuem um número de limitações e não agradam os usuários CDV. Acredita-se que uma das razões desse problema é a ausência do envolvimento de indivíduos CDV no desenvolvimento desses produtos. A falta de uma solução eficiente para a navegação desse público tornou-se mais grave com a pandemia da SARS-CoV 2, quando pessoas eram instruídas a praticar isolamento social e evitar contato em superfícies que possam estar contaminadas. O objetivo desse trabalho é propor um método para avaliação de opções de design para produtos assistivos para CDV baseados em Realidade Virtual (RV). A ideia é usar o RV como um campo de teste, onde o usuário pode experimentar diferentes soluções em diferentes cenários. Com isso, ele se torna integrante do design e da avaliação, resultando em um produto melhor e com uma interface mais simples. O método proposto inclui, além da montagem do ambiente virtual, o uso de sensores fisiológicos e testes subjetivos que aferem a carga mental e a consciência situacional nas diferentes situações e produtos que estão em desenvolvimento. Para ilustrar o método proposto, é estudado a navegação de indivíduos CDV em um hospital que usa protocolos COVID-19. Esse estudo de caso foi escolhido devido a ocorrente pandemia e a situação crítica que ela causa à população CDV. O cenário virtual foi feito usando Unity3D, uma plataforma de desenvolvimento de aplicações para realidade virtual largamente utilizada. O aparelho RV é o Tobbi Eye Tracking VR. São óculos que foram desenvolvidos usando o HTC VIVE. Esses óculos são utilizados para definir a posição e orientação do usuário no ambiente virtual do Unity. Para inferir a carga mental, foram utilizados os sensores fisiológicos da TEA Capitv T-Sens. Eles são o eletrocardiograma (ECG), usado para coletar a frequência cardíaca e a variância cardíaca, e o GSR (Galvanic skin reaction, reação galvânica da pele), para captar a condutância da pele. Além desses sensores, os voluntários também responderam os testes NASA-TLX, também para verificar a carga mental, e uma versão adaptada do SAGAT, para determinar a consciência situacional. Entre os benefícios esperados pelo método é a flexibilidade e a agilidade para se criar diferentes cenários e também a possibilidade de testar eles no mesmo espaço físico. Isso pode acelerar o design de novas soluções e melhorar a qualidade dos produtos. Outro resultado esperado da pesquisa é a identificação de características chaves dos produtos que causam o aumento ou diminuição da carga mental ou da consciência situacional nos usuários CDV.
