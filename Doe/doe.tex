In this chapter, the  design of the experiment is detailed.

\section{Hypotheses}
The experiment was designed to verify the following hypotheses:
\begin{enumerate}[label = $H_\arabic*$:]
    \setlength\itemindent{2ex}
    \item The time of the sighted partipants will be higher than the mental demand of the blinded particpants;
    \item The mental workload of the sighted partipants will be higher than the mental demand of the blinded particpants;
    \item The situation awareness of the sighted partipants will be higher than the mental demand of the blinded particpants;
    \item There is a method, or a configuration of methods, that is more efficient, or less efficient, for navigation and mobility of the blind participants.
\end{enumerate}

\section{Participants' profile}
The participants can be classified in two groups.
\begin{itemize}
    \item Blind participants;

        This groups was composed of 4 participants with age from 26 to 56, with an average of 44 years, all male and all of them had superior education, with one exception that was at the time completing his course.

    \item Sighted participants.
    
        This groups was composed of 4 participants with age from 22 to 31, with an average of 28 years, three males and one women, all of them had superior education and were used to participate in research experiments.
\end{itemize}
    


\section{Experimental procedure}
The experiment was composed of 4 steps:

\begin{enumerate}[label = Step \arabic* --]
    \setlength\itemindent{7ex}
    \item ~\nameref{subsec:briefing}
    \item ~\nameref{subsec:experiment}; \label{itm:experiment}
    %\begin{enumerate}[label = Step 2.\arabic* --]
    %    \setlength\itemindent{9ex}
    %    \item ~\nameref{subsubsec:method};
    %    \item ~\nameref{subsubsec:first_visit};
    %    \item ~\nameref{subsubsec:return};
    %\end{enumerate}
    \item ~\nameref{subsec:conclusion}.
\end{enumerate}

Each one of this steps are that ahead, describing the main activities of each one.

\subsection{Briefing and presentation of the equipment}
\label{subsec:briefing}

The first step is the moment that the participant learns more details about the experiment, its reasons and its goals. This step has the following substeps:

\begin{enumerate}[label = 1.\arabic* --]
    \setlength\itemindent{4ex}
    \item Presentation of the reasons and goals of the experiment;
    \item Signature, or recording, of the free and informed consent forma;
    \item Presentation and positioning of the physiological sensors;
    \item Presentation of the guidance devices and the VR HMD;
    \item Answer the first questionnaire.
\end{enumerate}


\subsection{Experiment}
\label{subsec:experiment}

At this moment the participant will properly begin the experiment. Each participant tested each guidance method twice, one at the "First visit" and another for the "Return". Since it were 5 different methods, each participant repeated the procedure presented on the Subsection \ref{subsec:virtual_world_creation} 10 times.

\begin{enumerate}[label = 2.\arabic* --]
    \setlength\itemindent{4ex}
    \item Randomize scenes, done without the participant's knowledge;
    \item Present the device to the partipant and allow it to train;
    
    This substep is important for the participant to understand how the device he/she is going to test works. At this moment no data is gathered from the partipant and he/she is informed of this so he/she can test the device without worring of being evaluated

    \item First visit round \label{itm:first_visit_round};
    
    The first scene that the participant will be evaluated while using the device. The participant answered to the SAGAT questionnaire while doing the activities listed on the Subsection \ref{subsec:virtual_world_creation}.

    \item Anwer the NASA-TLX questionnaire for the "First visit";
    \item Return round;
    
    After answering the NASA-TLX questionnaire, the participant entered the same scene as before with the same device. The scene layout was preserved from the "First visit" to the "Return" but a few things were altered such as the presence, or not, of a television or random people talking.
    
    \item Anwer the NASA-TLX questionnaire for the "Return" and the guidance method questionnaire; \label{itm:return_questionnaire}
    
    \item Repeat substeps \ref{itm:first_visit_round} to \ref{itm:return_questionnaire} until all the guidance methods are completed tested;
\end{enumerate}

\subsection{Conclusion}
\label{subsec:conclusion}

After all the scenes are completed, the devices and physiological sensores are removed from the participant and them they were released to go.


\section{Data Processing}

In this section the processing method of the gathered data is presented.

\subsection{Data gathered}

During the experiment the following data were gathered:

\begin{itemize}
    \item Time on each scene;
    \item NASA-TLX answers;
    \item Sagat answers;
    \item Guidance method questionnaire;
    \item ECG data;
    \item GSR data;
    \item Temperature data;
\end{itemize}

\subsection{ANOVA statiscal model}
%7.3.2 Modelos estatísticos para verificação das hipóteses

The statistical model used for the hypotheses tests is presented at Equation \ref{eq:statistical_model}

\begin{equation}
    \label{eq:statistical_model}
    V_{ij} = \mu + \tau_j + \beta_i +e_{ij}
\end{equation}

where:

\begin{itemize}
    \item $y_{ij}$ - Is the one the observation from the participant $i$ of the method $j$;
    \item $\mu$ - Is the average of all the observations, the global average;
    \item $\tau_j$ - Is the difference between the global average and the method $j$ average. It represents the influece the method on the observation;
    \item $\beta_i$ - It represents the influece the participant on the observation;
    \item $e_{ij}$ - Is the difference between the global average and one observations. It represents the sistematic error of the observation.
\end{itemize}

Each participant has an uncontrolable influence over the observations. This influence can vary with the age, the education, the profession and other individualities.

\subsection{Hypotheses verification}

Before each test, an average between the rounds is calculated to be used in the hypotheses tests.

\begin{enumerate}[label = $H_\arabic*$:]
    \setlength\itemindent{2ex}
    \item The time of the sighted partipants will be higher than the mental demand of the blinded particpants;
    \item The mental workload of the sighted partipants will be higher than the mental demand of the blinded particpants;
    \item The situation awareness of the sighted partipants will be higher than the mental demand of the blinded particpants;
    
    To verify the hypothesis from $H_1$ to $H_3$ a Student T-Test is analyzed for each of this variables:
    \begin{itemize}
        \item For $H_1$ the time average is analyzed between the "blind sample" and the "sight sample" for each method;
        \item For $H_2$ the mental demand, the NASA-TLX average score, the average heartbeat average, the average standard deviation interbeat interval and the average skin conductace is analyzed between the "blind sample" and the "sight sample" for each method.
        \item For $H_3$ the Sagat score average is analyzed between the "blind sample" and the "sight sample" for each method;
    \end{itemize}

    \item There is a method, or a configuration of methods, that is more efficient, or less efficient, for navigation and mobility of the blind participants.
    
    To verify the $H_4$ hypothesis an ANOVA test using the statiscal model presented in the Equation \ref{eq:statistical_model} is analyzed and, if necessary, a Fisher LSD test is analyzed to check a pairwise relationship between the methods.

    All the collected variables are analyzed for this matter.
\end{enumerate}