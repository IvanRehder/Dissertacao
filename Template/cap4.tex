%\section{Conclusão}

Neste trabalho realizou-se o projeto de uma metodologia de controle subótimo redundante da junta passiva de um manipulador com três graus de liberdade instantaneamente. Para este propósito usou-se nas formulações o vetor gradiente de uma função escalar que estima o acoplamento entre a junta passiva e as ativas desse manipulador. Aqui a redundância
foi usada da melhor maneira possível sem focalizar o efeito global. Portanto, este método deve ser denominado de \emph{controle ótimo local por redundância}. A principal vantagem dessa formulação é a computação em tempo real, que é
necessária para o controle do manipulador experimental. Além disso esse método pode ser usado com diferentes tipos de controladores, uma vez que as alterações são feitas nas equações dinâmicas do manipulador.

A consequência direta observada nessa formulação é a redução dos torques na fase de controle da junta passiva, e consequente redução da energia elétrica gasta. Isso ocorre devido ao fato de que ao longo da trajetória do manipulador
o índice de acoplamento de torque tende a ser maximizado, e portanto, menor é o torque necessário nos atuadores para se conseguir o posicionamento da junta passiva do manipulador.

Outros resultados indiretos obtidos são: um movimento mais uniforme e suave do manipulador e um tempo de acomodação menor tanto no posicionamento da junta passiva quanto das ativas, conforme podemos obervar nos gráficos de desempenho dos resultados apresentados. Isso ocorre porque a maximização do acoplamento entre as juntas facilita o controle. Assim
ocorrem menos picos de torque, e como as juntas ativas tem ``menos trabalho'' para posicionar a passiva estas se movem menos na direçao contrária ao movimento daquelas, diminuindo assim as velocidades alcançadas e os tempos de posicionamento.

Uma extensão deste trabalho pode ser a implementação de um \emph{controle ótimo global por redundância} da junta passiva do manipulador. Para isto pode-se fazer o planejamento \emph{off-line} da trajetória das juntas de modo a minimizar a energia consumida. Alguns estudos foram feitos nesse sentido, usando o Princípio Mínimo de Pontryagin, mas sem resultados satisfatórios até o momento.


