%CAPÍTULO 4: ANÁLISE DOS RESULTADOS E DISCUSSÃO

%1. Elabore um parágrafo que introduz o capítulo: Este capítulo apresenta (descreva o objetivo do capítulo ...). É constituído de N seções a saber...
%2. Caso vc tenha aplicado a sua contribuição (modelo, produto, processo etc.) em um caso (empresa, laboratório, simulação etc.), apresente a descrição e análise dos resultados. Na seção de discussão cabem as análises de cenários What-If ou de sensibilidade. Exemplo: se o parâmetro X aumentar de N para N+1, o resultado poderia mudar de Y para Z?
%3. Elabore um parágrafo que conclui o capítulo e introduz o capítulo seguinte.

Throughout the experiment, three data sources were gathered from the participants, and this chapter will show their values, explain the process to analyze the data and discuss their results. This chapter is divided into two sections, each one related to one of the objectives:

\begin{itemize}
    \item Evaluation of assistive device from a human factors’ perspective in a virtual environment;
    \item Comparison between BVI users and sighted users.
\end{itemize}

From this point, the data from the blind participants will be called the “Blind” sample and the data from the sighted participants will be called the “Sight” sample.

\section{Evaluation of assistive device from a human factors’ perspective in a virtual environment}

%\input{Resultados/SimulationData}
\subsection{Subjective data}

There were 3 different questionnaires in this experiment. Each of these questionnaires was meant to verify one of the experiment goals:

\begin{itemize}
    \item \nameref{subsubsec:results_nasa_tlx_1};
    
        Meant to verify the mental workload of the user. Is expected that after each “First” round, the mental workload would decrease and that one of the methods would have the least mental workload.

    \item \nameref{subsubsec:results_adapted_sagat_1};
    
        Meant to verify the situation awareness and the mental map of the user. Is expected to notice an increase from the “First” round to the “Return” round at each method.

    \item \nameref{subsubsec:results_questionnaires}.

        Meant to assess the user experience with each method.

\end{itemize}

\subsubsection{NASA-TLX}
\label{subsubsec:results_nasa_tlx_1}

The NASA-TLX provides two information that are relevant to the workload analysis. The first one is the score attributed to the ‘mental demand’ dimension. The second one is the average obtained from the scores of the six dimensions of NASA-TLX. The two analyses are presented in the next subsections.

\paragraph{Analysis of the mental demand scale}\mbox{}\\

Table \ref{tab:md_table_blind} presents the ‘mental demand’ score attributed by each blind participant to each guidance method. The ‘base’ method refers to the guidance method that the person uses in his/her daily life (e.g., white cane). 


\begin{table}[!htb]
\centering
\caption{Score of NASA-TLX mental demand for the blind participants.}
\label{tab:md_table_blind}
\begin{tabular}{llrrrrr}
\toprule
     &        & Base & Audio & \begin{tabular}[c]{@{}l@{}}Haptic\\ Belt\end{tabular} & \begin{tabular}[c]{@{}l@{}}Virtual\\ Cane\end{tabular} & Mixture \\
Participant & Round &      &       &                                                       &                                                        &         \\
\midrule
001C & First &    3 &     1 &                                                    14 &                                                      3 &       6 \\
     & Return &    1 &     1 &                                                    10 &                                                      2 &       6 \\
002C & First &    5 &     1 &                                                     1 &                                                     10 &      12 \\
     & Return &    1 &     1 &                                                     1 &                                                     10 &       3 \\
003C & First &    5 &     5 &                                                     5 &                                                      8 &       1 \\
     & Return &    3 &     1 &                                                     1 &                                                      2 &       1 \\
004C & First &    9 &    10 &                                                    15 &                                                     10 &      10 \\
     & Return &    7 &    10 &                                                    14 &                                                      8 &      10 \\
\bottomrule
\end{tabular}
\end{table}



The mean value obtained for each guidance method is illustrated in Figure \ref{fig:barplot_md_avg_5_scene_blind}. It shows a systematic reduction on the perceived mental workload between the rounds for all methods, confirming that the participants get familiar with the devices after the first use. It also shows that although the haptic belt obtained the largest mean, it also had the largest variation, showing that the effort required from the user may vary significantly.

\begin{figure}[!htb]
    \centering
    \includegraphics[width = 0.8\linewidth]{Resultados/Nasa/Figuras/png/barplot_md_avg_5_scene_blind.png}
    \caption{Mean and standard deviation of mental demand of blind participants for each method.}
    \label{fig:barplot_md_avg_5_scene_blind}
\end{figure}

Figure \ref{fig:boxplot_md_blind_scene}  presents a box plot of the mental demand score grouped by method. This Figure shows that there may be two different groups: one associated with lower demand, composed of ‘base’ and ‘audio’, and another with higher demand, composed of ‘haptic belt’, ‘virtual cane’ and ‘mixture’. It indicates that maybe a guidance method that uses vibration as input is not intuitive. Following, Figure \ref{fig:boxplot_md_blind_rounds} presents a box plot of the mental demand grouped by the rounds, confirming the general tendency to reduce the required ‘mental demand’. 

\begin{figure}[!htb]
    \centering
    \begin{minipage}{0.45\textwidth}
        \centering
        \includegraphics[width = 0.8\linewidth]{Resultados/Nasa/Figuras/png/boxplot_md_blind_scene.png}
        \caption{Boxplot of the mental demand of the blind participants grouped by method.}
        \label{fig:boxplot_md_blind_scene}
    \end{minipage}
    \begin{minipage}{0.45\textwidth}
        \centering
        \includegraphics[width = 0.8\linewidth]{Resultados/Nasa/Figuras/png/boxplot_md_blind_rounds.png}
        \caption{Boxplot of the mental demand of the blind participants grouped by round.}
        \label{fig:boxplot_md_blind_rounds}
    \end{minipage}
\end{figure}

In order to support the statistical analysis, Figures \ref{fig:qqplot_md_avg_two_way_blind} and \ref{fig:residplot_md_avg_two_way_blind} presents the QQ-plot and the residual plot of the ‘mental demand’ data, confirming that the data follow a normal distribution and the residues are homogenous.

Following, Figures \ref{fig:qqplot_md_avg_two_way_blind} and \ref{fig:residplot_md_avg_two_way_blind} shows the distribution and variance of the Table \ref{tab:md_table_blind}. These Figures shows that the data are normally distributed and that the methods have a similar variance. The Table \ref{tab:blocanova_md_avg_two_way_blind} shows the Anova test p-values of the mental demand of the ‘blind” sample between the guidance methods. The method’s and the round’s p-values indicates that there is no influence from them in the mental demand. The interaction between the methods and the round also does not influences the mental demand.

\begin{figure}[!htb]
    \centering
    \begin{minipage}{0.45\textwidth}
        \centering
        \includegraphics[width = 0.8\linewidth]{Resultados/Nasa/Figuras/png/qqplot_md_avg_two_way_blind.png}
        \caption{QQ plot of the mental demand of the blind participants on each method.}
        \label{fig:qqplot_md_avg_two_way_blind}
    \end{minipage}
    \begin{minipage}{0.45\textwidth}
        \centering
        \includegraphics[width = 0.8\linewidth]{Resultados/Nasa/Figuras/png/residplot_md_avg_two_way_blind.png}
        \caption{Residual plot of the mental demand score the blind participants on each method.}
        \label{fig:residplot_md_avg_two_way_blind}
    \end{minipage}
\end{figure}

Following, the statistical model of Equation 5.1 is used for the analysis of variance (ANOVA): 

\begin{equation}
    \label{eq:statistical_model}
    y_{ijk} = \mu + \tau_i + \beta_j + \msout{\omega_k} + (\tau\beta_{ij}) + e
\end{equation}

where:

\begin{itemize}
    \item $y_{ij}$ - output variable for method $i$, round $j$ and participant $k$;
    \item $\mu$ - mean of all the observations;
    \item $\tau_i$ - variance from method $i$;
    \item $\beta_j$ - variance from round $j$;
    \item \sout{$\omega_k$} - variance from participant k, which treated as a block;
    \item $\tau\beta_{ij}$ - combined variance from the interaction between method i and round j;
    \item $e$ - residual error.
\end{itemize}

The results of ANOVA are presented in Table \ref{tab:blocanova_md_avg_two_way_blind}. ANOVA tests the hypothesis that the means of independent groups of data are equal or not. In the literature, a p-value of 0.05 is commonly adopted as a threshold to confirm the hypothesis. A p-value < 0.05 indicates that the means of the groups are statistically different with 95\% of confidence. According to this criterion neither method or round have a significant influence on the mental demand.

However, due to the low number of participants, the threshold of 0.1 could also be considered. In this case, it indicates, with 90\% confidence, that the mean of the ‘first’ and ‘return’ rounds are different. For guidance method, the p-value of 0.170 is close to the threshold but slightly higher, suggesting that the means may be different but this hypothesis is not statistically confirmed with the current data. 


\begin{table}[!htb]
\centering
\caption{Anova p-value for the mental demand average on each method for blinded users.}
\label{tab:blocanova_md_avg_two_way_blind}
\begin{tabular}{lrrrrl}
\toprule
               Source &  Squared sum &  DOF & Squared average &     F & \begin{tabular}[c]{@{}l@{}}P-Value \\ $(F_{0} > F)$\end{tabular} \\
\midrule
Participants (Blocks) &      298.475 &    3 &          99.492 & 8.133 &                                                                  \\
         \    Methods &       85.150 &    4 &          21.288 & 1.740 &                                                            0.170 \\
          \    Rounds &       42.025 &    1 &          42.025 & 3.436 &                                                            0.075 \\
     \    Interaction &        2.850 &    4 &           0.712 & 0.058 &                                                            0.993 \\
   Experimental Error &      330.275 &   27 &          12.232 &       &                                                                  \\
                Total &      758.775 &   39 &                 &       &                                                                  \\
\bottomrule
\end{tabular}
\end{table}



In order to conclude the analysis of the NASA-TLX mental demand, Table \ref{tab:md_var_average_group_blind} brings the average of difference between the mental demand of the ‘first’ and ‘return’ rounds. Unexpectedly, it shows that the largest variation is obtained to the ‘base’, i.e., the guidance method the participant is used to and, therefore, should not present a significant variation. The method with the lower variation was ‘audio’, probably because it already had a very low score in the first round. 


\begin{table}[!htb]
\centering
\caption{Mental demand variation grouped by participant and visual condition}
\label{tab:md_var_average_group_blind}
\begin{tabular}{lrrrrrr}
\toprule
{} &  Base & Audio & \begin{tabular}[c]{@{}l@{}}Haptic\\ Belt\end{tabular} & \begin{tabular}[c]{@{}l@{}}Virtual\\ Cane\end{tabular} & Mixture \\
Visual Condition &       &       &                                                       &                                                        &         \\
\midrule
Blind            &  -2.5 &  -1.0 &                                                  -2.2 &                                                   -2.2 &    -2.2 \\
\bottomrule
\end{tabular}
\end{table}



\FloatBarrier

%%%%%%%%%%%%%%%%%%%%%%%%%%%%%%%%%%%%%%%%%%%%%%%%%%%%%%%%%%%%%%%%%%%%%%%%%%%
%%%%%%%%%%%%%%%%%%%%%%%%%%%%%%%%%%%%%%%%%%%%%%%%%%%%%%%%%%%%%%%%%%%%%%%%%%%
%%%%%%%%%%%%%%%%%%%%%%%%%%%%%%%%%%%%%%%%%%%%%%%%%%%%%%%%%%%%%%%%%%%%%%%%%%%
%%%%%%%%%%%%%%%%%%%%%%%%%%%%%%%%%%%%%%%%%%%%%%%%%%%%%%%%%%%%%%%%%%%%%%%%%%%


\paragraph{Analysis of the NASA-TLX score}\mbox{}\\

This section repeats the analysis steps of the previous section but now considering the mean value of all dimension of NASA-TLX, referred in this text as global score. Table \ref{tab:nasa_table_blind} presents the global score of each blind participant. 


\begin{table}[!htb]
\centering
\caption{NASA-TLX score felled by the blinded participants.}
\label{tab:nasa_table_blind}
\begin{tabular}{llrrrrr}
\toprule
     &        &  Base &  Audio & \begin{tabular}[c]{@{}l@{}}Haptic\\ Belt\end{tabular} & \begin{tabular}[c]{@{}l@{}}Virtual\\ Cane\end{tabular} & Mixture \\
Participant & Round &       &        &                                                       &                                                        &         \\
\midrule
001C & First & 4.833 &  4.000 &                                                 8.833 &                                                  5.167 &   6.333 \\
     & Return & 4.167 &  4.000 &                                                 6.667 &                                                  4.500 &   6.167 \\
002C & First & 6.333 &  4.833 &                                                 4.833 &                                                  9.000 &   7.000 \\
     & Return & 4.500 &  4.833 &                                                 4.833 &                                                  7.000 &   5.167 \\
003C & First & 4.000 &  4.000 &                                                 5.333 &                                                  6.667 &   3.500 \\
     & Return & 4.000 &  3.833 &                                                 3.667 &                                                  3.500 &   3.500 \\
004C & First & 9.833 & 10.000 &                                                12.667 &                                                  9.667 &  11.000 \\
     & Return & 8.667 &  9.167 &                                                11.667 &                                                  9.333 &  10.833 \\
\bottomrule
\end{tabular}
\end{table}



Figure \ref{fig:barplot_nasa_avg_5_scene_blind} brings the corresponding barplot with the mean value and standard deviation for each guidance method and each round. In a qualitative comparison with Figure \ref{fig:barplot_md_avg_5_scene_blind}, the differences between the methods are confirmed but softened. It is possible to notice that the mean score of ‘audio’ and ‘base’ are still lower than that of the other methods. The differences between ‘first’ and ‘return’ rounds are also reduced. However, the standard deviation among are also considerably reduced for all methods, and especially for the haptic belt.

\begin{figure}[!htb]
    \centering
    \includegraphics[width = 0.8\linewidth]{Resultados/Nasa/Figuras/png/barplot_nasa_avg_5_scene_blind.png}
    \caption{Barplot of the average NASA-TLX score of the blind participants on each method.}
    \label{fig:barplot_nasa_avg_5_scene_blind}
\end{figure}

Figure \ref{fig:boxplot_nasa_blind_scene} presents the box plot with the NASA-TLX global score grouped by method. Similar to what happened for the ‘mental demand’, it shows that it possible to split the methods in two different groups: ’base’ and ‘audio’, which requires a lower level of workload, and another group, which requires a higher level. Figure \ref{fig:boxplot_nasa_blind_rounds} presents a box plot with the NASA-TLX global score grouped by the rounds, apparently showing that the two groups are still different. 

\begin{figure}[!htb]
    \centering
    \begin{minipage}{0.45\textwidth}
        \centering
        \includegraphics[width = 0.8\linewidth]{Resultados/Nasa/Figuras/png/boxplot_nasa_blind_scene.png}
        \caption{QQ plot of the NASA-TLX score of the blind participants on each method.}
        \label{fig:boxplot_nasa_blind_scene}
    \end{minipage}
    \begin{minipage}{0.45\textwidth}
        \centering
        \includegraphics[width = 0.8\linewidth]{Resultados/Nasa/Figuras/png/boxplot_nasa_blind_rounds.png}
        \caption{Residual plot of the NASA-TLX score the blind participants on each method.}
        \label{fig:boxplot_nasa_blind_rounds}
    \end{minipage}
\end{figure}

%The Table \ref{tab:nasa_average_group_blind} shows the average NASA-TLX score in the “blind” sample and is possible to notice how the average score by the “blind” sample was lower during the “Audio” and the “Base” methods.
%
%
\begin{table}[!htb]
\centering
\caption{Average NASA-TLX score of the blind participants}
\label{tab:nasa_average_group_blind}
\begin{tabular}{lrrrrrr}
\toprule
{} &  Base & Audio & \begin{tabular}[c]{@{}l@{}}Haptic\\ Belt\end{tabular} & \begin{tabular}[c]{@{}l@{}}Virtual\\ Cane\end{tabular} &  Mixture \\
Visual Condition &       &       &                                                       &                                                        &          \\
\midrule
Blind            &  5.79 &  5.58 &                                                  7.31 &                                                   6.85 &    6.688 \\
\bottomrule
\end{tabular}
\end{table}



Figures \ref{fig:qqplot_nasa_avg_two_way} and \ref{fig:residplot_nasa_avg_two_way} presents the QQ plot and residual distribution of the NASA-TLX global score, showing that apparently the data are normally distributed, but the residuals are not so homogeneous as in the previous case, showing that the participants have different variability among them.

\begin{figure}[!htb]
    \centering
    \begin{minipage}{0.45\textwidth}
        \centering
        \includegraphics[width = 0.8\linewidth]{Resultados/Nasa/Figuras/png/qqplot_nasa_avg_two_way.png}
        \caption{QQ plot of the NASA-TLX score variation of the blind participants on each method.}
        \label{fig:qqplot_nasa_avg_two_way}
    \end{minipage}
    \begin{minipage}{0.45\textwidth}
        \centering
        \includegraphics[width = 0.8\linewidth]{Resultados/Nasa/Figuras/png/residplot_nasa_avg_two_way.png}
        \caption{Residual plot of the NASA-TLX score variation the blind participants on each method.}
        \label{fig:residplot_nasa_avg_two_way}
    \end{minipage}
\end{figure}

Table \ref{tab:blocanova_nasa_avg_two_way} brings the p-value resulting from ANOVA. In this case, both the method and round where appointed as significant variables that influence the mean value of the NASA-TLX global score. 


\begin{table}[!htb]
\centering
\caption{Anova p-value for the NASA-TLX score on each method for blinded users.}
\label{tab:blocanova_nasa_avg_two_way}
\begin{tabular}{lrrrrl}
\toprule
               Source &  Squared sum &  DOF & Squared average &      F & \begin{tabular}[c]{@{}l@{}}P-Value \\ $(F_{0} > F)$\end{tabular} \\
\midrule
Participants (Blocks) &      211.041 &    3 &          70.347 & 51.869 &                                                                  \\
         \    Methods &       17.185 &    4 &           4.296 &  3.168 &                                                          0.029** \\
          \    Rounds &        7.951 &    1 &           7.951 &  5.862 &                                                          0.022** \\
     \    Interaction &        2.115 &    4 &           0.529 &  0.390 &                                                            0.814 \\
   Experimental Error &       36.619 &   27 &           1.356 &        &                                                                  \\
                Total &      274.910 &   39 &                 &        &                                                                  \\
\bottomrule
\end{tabular}
\end{table}



Finally, Table \ref{tab:lsd_nasa_avg_two_way} presents the results of a pairwise Fisher LSD test comparing each pair of guidance method. The results show that only ‘audio’ is similar ‘base’, all the other methods are different among each other

\input{Resultados/Nasa/Tabelas/lsd_nasa_avg_two_way}

Table \ref{tab:nasa_var_group_blind} shows the difference in the NASA-TLX global score between the first and return rounds. It shows that the ‘audio’ difference is the lowest among all methods, while the highest difference is for the ‘virtual cane’.

%The Table \ref{tab:nasa_var_group_blind} shows the average of the NASA-TLX score variation between the rounds. This table shows that the variation from the “Audio” was the lowest variation and the highest variation was the "Virtual Cane".

\input{Resultados/Nasa/Tabelas/nasa_var_group_blind}

%The Figures \ref{fig:qqplot_nasa_var} and \ref{fig:residplot_nasa_var} shows the distribution and variance of the NASA-TLX score variation of the Table \ref{tab:nasa_table_blind}. These Figures shows that the data are normally distributed and that the methods have a similar variance.
%The Table \ref{tab:blocanova_nasa_var} shows the Anova test p-value of the NASA-TLX score of the "blind" sample between the guidance methods. The p-value indicates that there are no difference between the variation of any method. 
%
%
\begin{table}[!htb]
\centering
\caption{Anova p-value for the NASA-TLX score variation on each method for blinded users.}
\label{tab:blocanova_nasa_var}
\begin{tabular}{lrrrrr}
\toprule
Source & P-Value \\
\midrule
Method &   0.402 \\
\bottomrule
\end{tabular}
\end{table}


%
%\begin{figure}[!htb]
%    \centering
%    \begin{minipage}{0.45\textwidth}
%        \centering
%        \includegraphics[width = 0.8\linewidth]{Resultados/Nasa/Figuras/png/qqplot_nasa_var.png}
%        \caption{Bar plot of the average NASA-TLX score of the blind participants on each method.}
%        \label{fig:qqplot_nasa_var}
%    \end{minipage}
%    \begin{minipage}{0.45\textwidth}
%        \centering
%        \includegraphics[width = 0.8\linewidth]{Resultados/Nasa/Figuras/png/residplot_nasa_var.png}
%        \caption{Bar plot of the average NASA-TLX score of the sighted participants on each method.}
%        \label{fig:residplot_nasa_var}
%    \end{minipage}
%\end{figure}
%
%%The Table \ref{tab:lsdbloc_nasa_var} presents the conclusion of a pairwise Fisher LSD test of the blind NASA-TLX score between all the guidance methods. The results show that all methods have similar variations.
%
%%\input{Resultados/Nasa/Tabelas/lsdbloc_nasa_var}
%
%To close up, according to the LSD test at Table \ref{tab:lsd_nasa_avg_two_way} only the "Audio" method has a NASA-TLX score that could be said to be similar to the "Base" method, which indicates that the existance of an haptic device increased the NASA-TLX score and that the round has some impact on the score, which means that there was a learning effect from the "First" to the "Return" round. Probably this effect was reflected in the other dimensions of the NASA-TLX.
%
%The \ref{tab:blocanova_nasa_avg_two_way} concludes that the rounds and the interaction between the rounds and the methods have no influence on the variation of the NASA-TLX score.
%
\FloatBarrier

\subsubsection{Adapted SAGAT}
\label{subsubsec:results_adapted_sagat_1}

This section discusses the results of the adapted SAGAT questionnaire, which aims at assessing the participant situation awareness and mental map of the environment. 

For each question of the SAGAT questionnaire, the participant could score 1 point or a fraction of it. The total score achieved by each blind participant is presented in the Table \ref{tab:sagat_table_blind}. Figure  \ref{fig:barplot_sagat_avg_5_scene_blind} illustrates the corresponding bar plot, indicating the mean and standard deviation for each guidance method and each round. This figure shows clearly that the participants improved its situation awareness in the return round, when they already had some information about the environment. Also, it is possible to observe that the worst situation awareness is obtained in the ‘first’ round for the ‘virtual cane’. However, on the ‘return’ round, the SAGAT mean score becomes equivalent to that of the ‘audio’ method.


\begin{table}[!htb]
\centering
\caption{SAGAT global score felled by the blinded participants.}
\label{tab:sagat_table_blind}
\begin{tabular}{llrrrrr}
\toprule
     &        &   Base &  Audio & \begin{tabular}[c]{@{}l@{}}Haptic\\ Belt\end{tabular} & \begin{tabular}[c]{@{}l@{}}Virtual\\ Cane\end{tabular} & Mixture \\
Participant & Round &        &        &                                                       &                                                        &         \\
\midrule
001C & First &   6.25 &   5.50 &                                                  5.33 &                                                   5.83 &   3.500 \\
     & Return &   6.25 &   6.50 &                                                  8.50 &                                                   5.50 &   5.500 \\
002C & First &   6.75 &   4.50 &                                                  3.99 &                                                   4.50 &   6.250 \\
     & Return &   5.25 &   5.00 &                                                  4.00 &                                                   6.50 &   8.500 \\
003C & First &   7.25 &   7.50 &                                                  7.49 &                                                   4.66 &   9.000 \\
     & Return &  10.00 &  10.00 &                                                  8.50 &                                                   9.00 &   9.000 \\
004C & First &   7.50 &   6.00 &                                                  7.66 &                                                   4.99 &   6.500 \\
     & Return &   9.00 &   6.00 &                                                  9.25 &                                                   7.25 &   9.000 \\
\bottomrule
\end{tabular}
\end{table}



\begin{figure}[!htb]
    \centering
    \includegraphics[width = 0.8\linewidth]{Resultados/Sagat/Figuras/png/barplot_sagat_avg_5_scene_blind.png}
    \caption{Barplot of the average SAGAT score of the blind participants on each method.}
    \label{fig:barplot_sagat_avg_5_scene_blind}
\end{figure}

Figure \ref{fig:boxplot_sagat_blind_scene} brings the boxplot of the SAGAT score grouped by guidance method. It shows that the methods can be divided in two groups. The first one is composed of ‘base’, ‘haptic belt’ and the ‘mixture’. This group received scores higher than the second group, composed of ‘audio’ and ‘virtual cane’. Following, Figure \ref{fig:boxplot_sagat_blind_rounds} shows the boxplot of the data grouped by round and confirms the general improvement of situation awareness from the ‘first’ to the ‘return’ round. 

\begin{figure}[!htb]
    \centering
    \begin{minipage}{0.45\textwidth}
        \centering
        \includegraphics[width = 0.8\linewidth]{Resultados/Sagat/Figuras/png/boxplot_sagat_blind_scene.png}
        \caption{Boxplot of the SAGAT score of the blind participants grouped by method.}
        \label{fig:boxplot_sagat_blind_scene}
    \end{minipage}
    \begin{minipage}{0.45\textwidth}
        \centering
        \includegraphics[width = 0.8\linewidth]{Resultados/Sagat/Figuras/png/boxplot_sagat_blind_rounds.png}
        \caption{Boxplot of the SAGAT score of the blind participants grouped by round.}
        \label{fig:boxplot_sagat_blind_rounds}
    \end{minipage}
\end{figure}

%The Table \ref{tab:sagat_average_group_blind} shows the average SAGAT score in the “blind” sample and is possible to notice how the average score by the “blind” sample was lower during the “Audio” and the “Base” methods.
%
%
\begin{table}[!htb]
\centering
\caption{SAGAT score average grouped by participant and visual condition}
\label{tab:sagat_average_group_blind}
\begin{tabular}{lrrrrrr}
\toprule
{} &  Base & Audio & \begin{tabular}[c]{@{}l@{}}Haptic\\ Belt\end{tabular} & \begin{tabular}[c]{@{}l@{}}Virtual\\ Cane\end{tabular} &  Mixture \\
Visual Condition &       &       &                                                       &                                                        &          \\
\midrule
Blind            &  7.28 &  6.38 &                                                  6.84 &                                                   6.03 &    7.156 \\
\bottomrule
\end{tabular}
\end{table}



Proceeding to the statistical analysis of the data, Figures \ref{fig:qqplot_sagat_avg_two_way_blind} and \ref{fig:residplot_sagat_avg_two_way_blind} present the QQ plot and the residual distribution, which confirms the normal distribution assumption and the homogeneity of variances

\begin{figure}[!htb]
    \centering
    \begin{minipage}{0.45\textwidth}
        \centering
        \includegraphics[width = 0.8\linewidth]{Resultados/Sagat/Figuras/png/qqplot_sagat_avg_two_way_blind.png}
        \caption{QQ plot of the SAGAT score of the blind participants on each method.}
        \label{fig:qqplot_sagat_avg_two_way_blind}
    \end{minipage}
    \begin{minipage}{0.45\textwidth}
        \centering
        \includegraphics[width = 0.8\linewidth]{Resultados/Sagat/Figuras/png/residplot_sagat_avg_two_way_blind.png}
        \caption{Residual plot of the SAGAT score the blind participants on each method.}
        \label{fig:residplot_sagat_avg_two_way_blind}
    \end{minipage}
\end{figure}

Finally, Table \ref{tab:blocanova_sagat_avg_two_way_blind} shows the Anova test p-value of the SAGAT score. It indicates that the round is a significant variable that influences the value of the SAGAT score. The same cannot be said for the method, which, apparently, has no significant influence.


\begin{table}[!htb]
\centering
\caption{Anova p-value for the SAGAT score on each method for blinded users.}
\label{tab:blocanova_sagat_avg_two_way_blind}
\begin{tabular}{lrrrrr}
\toprule
          Source & P-Value \\
\midrule
    \    Methods &   0.277 \\
     \    Rounds & 0.002** \\
\    Interaction &   0.834 \\
\bottomrule
\end{tabular}
\end{table}



%The Table \ref{tab:lsd_sagat_avg_two_way} presents the conclusion of a pairwise Fisher LSD test of the blind NASA-TLX score between all the guidance methods. The results show that only the "Audio" has a similar NASA-TLX score as the "Base" method, as it was also posible to notice at Figure \ref{fig:boxplot_sagat_blind_scene}.

%\input{Resultados/Sagat/Tabelas/lsd_sagat_avg_two_way}

Finally, Table \ref{tab:sagat_var_group_blind} brings the mean difference in the SAGAT score between the first and return round for each guidance method. It shows that the ‘base’ and ‘audio’ methods have the lowest difference, while the highest one was obtained for the ‘virtual cane’.


\begin{table}[!htb]
\centering
\caption{Adapted Sagat global score variation grouped by participant and visual Condition}
\label{tab:sagat_var_group_blind}
\begin{tabular}{lrrrrrr}
\toprule
{} &  Base &  Audio & \begin{tabular}[c]{@{}l@{}}Haptic\\ Belt\end{tabular} & \begin{tabular}[c]{@{}l@{}}Virtual\\ Cane\end{tabular} & Mixture \\
Visual Condition &       &        &                                                       &                                                        &         \\
\midrule
Blind            &  8.93 &  15.66 &                                                 23.49 &                                                  44.30 &   32.90 \\
\bottomrule
\end{tabular}
\end{table}



%The Figures \ref{fig:qqplot_sagat_var_blind} and \ref{fig:residplot_sagat_var_blind} shows the distribution and variance of the SAGAT score variation of the Table \ref{tab:sagat_table_blind}. These Figures shows that the data are normally distributed and that the methods have a similar variance.
%The Table \ref{tab:blocanova_sagat_var_blind} shows the Anova test p-value of the SAGAT score of the "blind" sample between the guidance methods. The p-value indicates that there are no difference between the variation in any method. 
%
%
\begin{table}[!htb]
\centering
\caption{Anova p-value for the SAGAT score variation on each method for blinded users.}
\label{tab:blocanova_sagat_var_blind}
\begin{tabular}{lrrrrr}
\toprule
               Source &  Squared sum &  DOF & Squared average &     F & \begin{tabular}[c]{@{}l@{}}P-Value \\ $(F_{0} > F)$\end{tabular} \\
\midrule
Participants (blocks) &     1176.902 &    3 &         782.885 & 0.473 &                                                                  \\
               Method &     3131.542 &    4 &         392.301 & 0.944 &                                                            0.472 \\
   Experimental error &     9956.458 &   12 &         829.705 &       &                                                                  \\
                Total &    14264.902 &   19 &                 &       &                                                                  \\
\bottomrule
\end{tabular}
\end{table}


%
%\begin{figure}[!htb]
%    \centering
%    \begin{minipage}{0.45\textwidth}
%        \centering
%        \includegraphics[width = 0.8\linewidth]{Resultados/Sagat/Figuras/png/qqplot_sagat_var_sight.png}
%        \caption{QQ plot of the SAGAT score variation of the blind participants on each method.}
%        \label{fig:qqplot_sagat_var_blind}
%    \end{minipage}
%    \begin{minipage}{0.45\textwidth}
%        \centering
%        \includegraphics[width = 0.8\linewidth]{Resultados/Sagat/Figuras/png/residplot_sagat_var_sight.png}
%        \caption{Residual plot of the SAGAT score variation of the blind participants on each method.}
%        \label{fig:residplot_sagat_var_blind}
%    \end{minipage}
%\end{figure}
%
%%The Table \ref{tab:lsdbloc_nasa_var} presents the conclusion of a pairwise Fisher LSD test of the blind NASA-TLX score between all the guidance methods. The results show that all methods have similar variations.
%
%%\input{Resultados/Nasa/Tabelas/lsdbloc_nasa_var}
%
%To close up, according to the ANOVA test at Table \ref{tab:blocanova_sagat_avg_two_way_blind} the methods caused no reaction on the SAGAT score, but the rounds did. That means that the participants were able in all methods to learn a little about their environment and that learning impacted their environmental perception in the next round. The fact that the test has not found any influence of the methods on the SAGAT score may be because of the small sample size, since it is posible to notice a difference between the methods at Figure \ref{fig:boxplot_sagat_blind_scene}. Also the interaction between method and round caused no influence in the Sagat score. According to the ANOVA test at Table \ref{tab:blocanova_sagat_var_blind}, the methods did not influenced the SAGAT score.
%
\FloatBarrier

\subsubsection{Guidance method's questionnaire.}
\label{subsubsec:results_questionnaires}

The data from the questionnaire for evaluating the user experience with each guidance method is also analysed. The higher the score, the more satisfied the user is with the method. It is essential to observe that this analysis does not include the base method as the questions are specific about each method and the base may vary among the participants. Also, there is no distinction between first and return rounds. Each questionnaire is answered only once for each method.

Table \ref{tab:questionnaire_average_blind} presents the score attributed to each method by each participant. The mean values are plotted in Figure \ref{fig:barplot_questionnaire_scene_blind} and show a dissatisfaction with the methods that only use vibration for communicating with the participant, i.e., the haptic belt and the virtual cane. 


\begin{table}[!htb]
\centering
\caption{ Guidance method questionnaire score felled by the blinded participants.}
\label{tab:questionnaire_average_blind}
\begin{tabular}{llrrrrr}
\toprule
{} &  Audio &  \begin{tabular}[c]{@{}l@{}}Haptic\\ Belt\end{tabular} &  \begin{tabular}[c]{@{}l@{}}Virtual\\ Cane\end{tabular} &  Mixture \\
Participant &        &                                                        &                                                         &          \\
\midrule
001C        &  0.774 &                                                  0.543 &                                                   0.629 &    0.865 \\
002C        &  0.857 &                                                  0.743 &                                                   0.543 &    0.935 \\
003C        &  0.929 &                                                  0.571 &                                                   0.543 &    0.745 \\
004C        &  0.881 &                                                  0.486 &                                                   0.400 &    0.730 \\
\bottomrule
\end{tabular}
\end{table}



\begin{figure}[!htb]
    \centering
    \includegraphics[width = \textwidth]{Resultados/Questionario/Figuras/pdf/barplot_questionnaire_scene_blind.pdf}
    \caption{Barplot of the average questionaire score of the blind participants on each method.}
    \label{fig:barplot_questionnaire_scene_blind}
\end{figure}

Figure \ref{fig:boxplot_quest_blind_scene} brings the questionnaire boxplot, which clearly shows the difference between two groups: haptic belt and virtual cane, and audio and mixture. 

\begin{figure}[!htb]
    \centering
    \includegraphics[width = 0.45\textwidth]{Resultados/Questionario/Figuras/pdf/boxplot_questionnaire_scene_blind.pdf}
    \caption{Boxplot of the questionaire score of the blind participants grouped by the methods.}
    \label{fig:boxplot_quest_blind_scene}
\end{figure}

%The Table \ref{tab:questionnaire_average_group_blind} show the the average questionnaire score on each method. It also shows a disatisfaction with the haptic devices alone.
%
%
\begin{table}[!htb]
\centering
\caption{Guidance method questionnaire average score for the blind participants.}
\label{tab:questionnaire_average_group_blind}
\begin{tabular}{lrrrrr}
\toprule
{} & Audio & Haptic Belt & Virtual Cane & Mixture \\
Visual Condition &       &             &              &         \\
\midrule
Blind            &  0.86 &        0.59 &         0.53 &    0.82 \\
\bottomrule
\end{tabular}
\end{table}



Figures \ref{fig:qqplot_questionnaire_blind} and \ref{fig:residplot_questionnaire_blind} show that the data follows a normal distribution. However, the residual variance is not strictly homogenous among the participants. 

\begin{figure}[!thb]
    \centering
    \begin{minipage}{0.45\textwidth}
        \centering
        \includegraphics[width = \textwidth]{Resultados/Questionario/Figuras/pdf/qqplot_questionnaire_blind.pdf}
        \caption{QQ plot of the questionnaire score of the blind participants on each method.}
        \label{fig:qqplot_questionnaire_blind}
    \end{minipage}
    \begin{minipage}{0.075\textwidth}
        \hfill
    \end{minipage}
    \begin{minipage}{0.45\textwidth}
        \centering
        \includegraphics[width = \textwidth]{Resultados/Questionario/Figuras/pdf/residplot_questionnaire_blind.pdf}
        \caption{Residual plot of the questionnaire score the blind participants on each method.}
        \label{fig:residplot_questionnaire_blind}
    \end{minipage}
\end{figure}

The results of ANOVA are presented in Table  \ref{tab:blocanova_questionnaire_blind} and it shows that the method, with a p-value of 0.001, is indeed a significant variable that affects the user's satisfaction.


\begin{table}[!htb]
\centering
\caption{Anova p-value for the questionnaire score on each method for blinded users.}
\label{tab:blocanova_questionnaire_blind}
\begin{tabular}{lrrrrr}
\toprule
Source & P-Value \\
\midrule
Method & 0.001** \\
\bottomrule
\end{tabular}
\end{table}



In order to complement the ANOVA analysis, the pairwise comparison of the methods obtained from the Fisher LSD test is presented in Table \ref{tab:lsd_questionnaire_blind}. The results show that audio and mixture are equivalent from the perspective of user satisfaction. All the other comparisons indicate there is a difference between the methods.

\input{Resultados/Questionario/Tabelas/lsd_questionnaire_blind.tex}

Additional to the scores, the participants also expressed their dissatisfaction with the answers to the open questions of the questionnaire, where they commented that the haptic belt and the virtual cane are confusing, are not precise enough, and are very different from what they are used to.

\FloatBarrier

\subsection{Physiological data}

During the experiment, data from two physiological sensors were captured: ECG and GSR. As commonly found in the literature, these data are used to assess mental workload. The corresponding analysis is presented in this section.

\begin{itemize}
    \item \nameref{subsubsec:results_ecg_1};
    
        Two features are extracted from the ECG, heartrate (BPM) and heartrate variance (SDNN).
    
        Is expected that the heartrate slight decrease from the "First" to the "Return" round. The heartrate variance is expected to slight increase from the "First" to the "Return" round.
    

    \item \nameref{subsubsec:results_gsr_temp_1};
    
        Is expected that the GSR average to increase at every “First” round and then a slight decrease in the next round.

\end{itemize}
\subsubsection{Electrocardiogram (ECG) data}
\label{subsubsec:results_ecg_1}

The ECG analysis is divided into two different types

\begin{itemize}
    \item Heart rate;
    
        This analysis checks the heartbeat frequency;

    \item Heart rate variance.
    
        This analysis checks the heartbeat frequency variance and it is done by analyzing the variation of the interval between beats.

\end{itemize}

At the beginning of each experience, a baseline data was gathered to establish a comparison between the normal state of the user and the scenes’ induced state. After the data gathering, an algorithm in Python was used to read the data and separate it accordingly to each participant, method and round. The algorithm followed the steps above:

\begin{itemize}
    \item Outliers remotion;
        Since the participants moved during the whole experience a lot of noise was collected by the sensors
    \item Normalization between -1 and 1;
    \item Peak detection;
        If the results were appropriate:
        \begin{itemize}
            \item Heartbeat interval calculation;
            \item File save to be used in Kubius HRV Standard.
        \end{itemize} 
        If the results were not appropriate:
        \begin{itemize}
            \item Tune peak detection method’s parameters;
            \item Heartbeat interval calculation;
            \item File save to be used in the next software.
        \end{itemize}    
\end{itemize}

This judgment was made by analyzing the plotted ECG signal and the detected peaks. Kubios HRV Standard is a heart rate variability (HRV) analysis software for personal non-commercial use. The Kubios HRV Standard makes it possible to use your HR monitor to examine the health of the cardiovascular system or to evaluate stress and recovery \cite{kubios}. At Kubius, the file with the intervals was analyzed and the results were saved in a report file to be read in python again. Back in python the results were plotted, tabled and statistically tested as the other data. In Appendix D there is a diagram with a pseudo-algorithm of this process.

This analysis was made by comparing the baseline values with the values of each round individually and between the round values themselves.

\paragraph{Analysis of the heartbeat frequency (BPM)}\mbox{}\\

Table \ref{tab:bpm_table_blind}  presents the heart rate of each blind participant for each guidance method. It is possible to observe that there is no systematic difference among the methods. Also, there are significant differences among the participants, which some of them presenting values significantly lower than others.

\input{Resultados/ECG/Tabelas/bpm_table_blind.tex}

Figure \ref{fig:barplot_ecg_bpm_5_scene_blind} presents the mean heart rate. It shows a slight increase in the heartrate between the rounds, with the exception of the ‘base’ method, indicating that the participants felt the ‘return’ round more demandful.

\begin{figure}[!htb]
    \centering
    \includegraphics[width = 0.8\linewidth]{Resultados/ECG/Figuras/png/barplot_ecg_bpm_5_scene_blind.png}
    \caption{Barplot of the average BPM of the blind participants on each method.}
    \label{fig:barplot_ecg_bpm_5_scene_blind}
\end{figure}

%The Table \ref{tab:bpm_average_group_blind} show the average heartbeat frequency variation between the rounds of each group. As it was shown in the Figure \ref{fig:barplot_ecg_bpm_5_scene_blind}, only the "Base" method has a negative average variaton between the rounds. It is also posible to see that the Virtual Cane variation was the highest, hence it was also the highest mental workload.
% 
%\input{Resultados/ECG/Tabelas/bpm_average_group_blind.tex}

Figures \ref{fig:boxplot_ecg_bpm_blind_scene} and \ref{fig:boxplot_ecg_bpm_blind_rounds} brings the corresponding boxplot, grouped by method and round. In both cases, it is not possible to observe significant differences among the methods or rounds.

\begin{figure}[!htb]
    \centering
    \begin{minipage}{0.45\textwidth}
        \centering
        \includegraphics[width = 0.8\linewidth]{Resultados/ECG/Figuras/png/boxplot_ecg_bpm_blind_scene.png}
        \caption{Boxplot of the BPM of the blind participants grouped by method.}
        \label{fig:boxplot_ecg_bpm_blind_scene}
    \end{minipage}
    \begin{minipage}{0.45\textwidth}
        \centering
        \includegraphics[width = 0.8\linewidth]{Resultados/ECG/Figuras/png/boxplot_ecg_bpm_blind_rounds.png}
        \caption{Boxplot of the BPM of the blind participants grouped by round.}
        \label{fig:boxplot_ecg_bpm_blind_rounds}
    \end{minipage}
\end{figure}

The Figures  and  shows the distribution and variance of the Table . These Figures shows that the data are normally distributed but the participants had different  that the methods have a similar variance.

Figures \ref{fig:qqplot_bpm_two_way} and \ref{fig:residplot_bpm_two_way} bring the QQ Plot and residual distribution. Particularly, the last one shows that the participant does not have similar variance, which may jeopardize the results of ANOVA. Considering this limitation, Table \ref{tab:bpm_table_blind} brings the p-value obtained by ANOVA, which confirmed the previous analysis, as it does not indicate a significant influence of either the guidance method or the round in the participants heartrate. 

\begin{figure}[!htb]
    \centering
    \begin{minipage}{0.45\textwidth}
        \centering
        \includegraphics[width = 0.8\linewidth]{Resultados/ECG/Figuras/png/qqplot_bpm_two_way_blind.png}
        \caption{QQ plot of the BPM of the blind participants on each method.}
        \label{fig:qqplot_bpm_two_way}
    \end{minipage}
    \begin{minipage}{0.45\textwidth}
        \centering
        \includegraphics[width = 0.8\linewidth]{Resultados/ECG/Figuras/png/residplot_bpm_two_way_blind.png}
        \caption{Residual plot of the BPM score the blind participants on each method.}
        \label{fig:residplot_bpm_two_way}
    \end{minipage}
\end{figure}

\input{Resultados/ECG/Tabelas/blocanova_bpm_two_way_blind.tex}

%\input{Resultados/ECG/Tabelas/lsd_bpm_two_way.tex}

%The Table \ref{tab:lsd_bpm_two_way} presents the conclusion of a pairwise Fisher LSD test of the blind heart rate frequency variation between all the guidance methods. The results show that the only the "Base" and "Haptic belt" have simila reaction.

\FloatBarrier

%%%%%%%%%%%%%%%%%%%%%%%%%%%%%%%%%%%%%%%%%%%%%%%%%%%%%%%%%%%%%%%%%%%%%%%%%%%%
%%%%%%%%%%%%%%%%%%%%%%%%%%%%%%%%%%%%%%%%%%%%%%%%%%%%%%%%%%%%%%%%%%%%%%%%%%%%
%%%%%%%%%%%%%%%%%%%%%%%%%%%%%%%%%%%%%%%%%%%%%%%%%%%%%%%%%%%%%%%%%%%%%%%%%%%%
%%%%%%%%%%%%%%%%%%%%%%%%%%%%%%%%%%%%%%%%%%%%%%%%%%%%%%%%%%%%%%%%%%%%%%%%%%%%
%
%
\paragraph{Analysis of the heartbeat variance (SDNN)}\mbox{}\\
%
The Table \ref{tab:sdnn_table_blind} presents the standard deviation of the interbeat interval by each participant on each scenes. As it was with the Table \ref{tab:bpm_table_blind}, it is not posible to draw a pattern inside this Table. Different participant had increase, or decrease, with different methods.

\input{Resultados/ECG/Tabelas/sdnn_table_blind.tex}

Inside the barplot Figure \ref{fig:barplot_ecg_sdnn_5_scene_blind} shows the average SDNN in each method. It is posible to notice that some method had an increase and some a decrease in the SDNN. The ones that indicate an increase would mean that the participant felt a lesser mental workload in the "Return" round, whilst the deacrese means the opposite.

\begin{figure}[!htb]
    \centering
    \includegraphics[width = 0.8\linewidth]{Resultados/ECG/Figuras/png/barplot_ecg_sdnn_5_scene_blind.png}
    \caption{Barplot of the average SDNN of the blind participants on each method.}
    \label{fig:barplot_ecg_sdnn_5_scene_blind}
\end{figure}

The Table \ref{tab:sdnn_average_group_blind} presents the average SDNN variation between the rounds. It shows that only the "Audio" and the "Haptic Belt" methods shown a increase in the mental workload.

\input{Resultados/ECG/Tabelas/sdnn_average_group_blind.tex}

The Figures \ref{fig:boxplot_ecg_sdnn_blind_scene} presents the distribution of each method SDNN. It noticeable that the "Base" method has a different SDNN than the rest. The "Virtual Cane" also has a different distribution from the rest. The Figure \ref{fig:boxplot_ecg_sdnn_blind_rounds} presents the SDNN grouped by the rounds. It shows a slight difference between the rounds.


\begin{figure}[!htb]
    \centering
    \begin{minipage}{0.45\textwidth}
        \centering
        \includegraphics[width = 0.8\linewidth]{Resultados/ECG/Figuras/png/boxplot_ecg_sdnn_blind_scene.png}
        \caption{Boxplot of the SDNN of the blind participants grouped by method.}
        \label{fig:boxplot_ecg_sdnn_blind_scene}
    \end{minipage}
    \begin{minipage}{0.45\textwidth}
        \centering
        \includegraphics[width = 0.8\linewidth]{Resultados/ECG/Figuras/png/boxplot_ecg_sdnn_blind_rounds.png}
        \caption{Boxplot of the SDNN of the blind participants grouped by round.}
        \label{fig:boxplot_ecg_sdnn_blind_rounds}
    \end{minipage}
\end{figure}

The Figures \ref{fig:qqplot_sdnn_two_way} and \ref{fig:residplot_sdnn_two_way} shows the distribution and variance of the Table \ref{tab:sdnn_table_blind}. These Figures shows that the data are normally distributed but the participants had different  that the methods have a similar variance.
The Table \ref{tab:blocdanova_sdnn_two_way} shows the ANOVA test p-value of the heartbeat interval variance of the “blind” sample. The p-value indicates that there is no effect of any factor.

\input{Resultados/ECG/Tabelas/blocdanova_sdnn_two_way.tex}

\begin{figure}[!htb]
    \centering
    \begin{minipage}{0.45\textwidth}
        \centering
        \includegraphics[width = 0.8\linewidth]{Resultados/ECG/Figuras/png/qqplot_sdnn_two_way.png}
        \caption{QQ plot of the SDNN of the blind participants on each method.}
        \label{fig:qqplot_sdnn_two_way}
    \end{minipage}
    \begin{minipage}{0.45\textwidth}
        \centering
        \includegraphics[width = 0.8\linewidth]{Resultados/ECG/Figuras/png/residplot_sdnn_two_way.png}
        \caption{Residual plot of the SDNN of the blind participants on each method.}
        \label{fig:residplot_sdnn_two_way}
    \end{minipage}
\end{figure}

%\input{Resultados/ECG/Tabelas/lsd_sdnn_two_way.tex}

The Table \ref{tab:blocdanova_sdnn_two_way} does not prove that any method or round has some influence in the heartbeat interval variance, thus in the Mental Workload. Although, in the Figure \ref{fig:boxplot_ecg_sdnn_blind_scene} it is posible to notice that the "Base" method has a different distribution. As it has already commented before, maybe the result of the anova test is a conseguence of a small sample size.

\FloatBarrier

\subsubsection{Galvanic skin reaction and temperature data;}
\label{subsubsec:results_gsr_temp_1}

The GSR analysis is based in the average level of the signal during each run of the experiment and its comparison to the participant baseline, collected before each round. For the experiment, the GSR sensor was worn on the left hand for right-handed participant and on the right hand for left-handed participants. One of the blind participants had the GSR sensor removed during the experiment because it was not appropriately fixed.

Table \ref{tab:gsr_table_blind} presents the GSR mean values for each of the three remaining participants. For all the participants, the baseline was smaller than the values obtained during the experiment, as expected. Moreover, in most of the cases, the skin conductance has risen from the ‘first’ to the ‘return’, indicating an increase in the mental workload.


\begin{table}[!htb]
\centering
\caption{Average GSR felled by the blind participants [$\mu$S].}
\label{tab:gsr_table_blind}
\begin{tabular}{lllrrrrrr}
\toprule
     &        & Baseline &  Base & Audio & \begin{tabular}[c]{@{}l@{}}Haptic\\ Belt\end{tabular} & \begin{tabular}[c]{@{}l@{}}Virtual\\ Cane\end{tabular} & Mixture \\
Participant & Round &          &       &       &                                                       &                                                        &         \\
\midrule
001C & First &     0.37 &  0.48 &  1.03 &                                                  3.14 &                                                   3.79 &    3.90 \\
     & Return &          &  0.83 &  1.58 &                                                  2.81 &                                                   4.04 &    4.57 \\
003C & First &     0.30 &  0.56 &  0.56 &                                                  0.62 &                                                   0.85 &    1.09 \\
     & Return &          &  0.62 &  0.63 &                                                  0.65 &                                                   0.92 &    1.06 \\
004C & First &     1.24 &  2.34 &  3.07 &                                                  3.49 &                                                   2.28 &    2.23 \\
     & Return &          &  2.57 &  2.95 &                                                  3.20 &                                                   2.21 &    2.24 \\
\bottomrule
\end{tabular}
\end{table}



Table \ref{tab:gsr_var_blind} brings the percentual increase in the GSR mean when compared to the baseline value. Figure \ref{fig:barplot_gsr_avg_5_scene_blind} shows the corresponding barplot. Apparently, the presence of a haptic device causes an increase in the skin conductance, hence its mental workload. Also, it is possible to observe the increase in GSR mean between the two rounds, excepted for the haptic belt.


\begin{table}[!htb]
\centering
\caption{Average GSR variation in relation to the baseline in each round of the blind participants [$\mu$S].}
\label{tab:gsr_var_blind}
\begin{tabular}{lllrrrrrr}
\toprule
     &        &      Base &     Audio & \begin{tabular}[c]{@{}l@{}}Haptic\\ Belt\end{tabular} & \begin{tabular}[c]{@{}l@{}}Virtual\\ Cane\end{tabular} &    Mixture \\
Participant & Round &           &           &                                                       &                                                        &            \\
\midrule
001C & First &   30.58\% &  176.54\% &                                              746.10\% &                                               920.72\% &   951.71\% \\
     & Return &  125.29\% &  327.42\% &                                              656.99\% &                                               988.93\% &  1132.39\% \\
003C & First &   85.36\% &   84.23\% &                                              104.19\% &                                               182.35\% &   258.80\% \\
     & Return &  105.34\% &  109.23\% &                                              112.95\% &                                               202.35\% &   249.72\% \\
004C & First &   89.62\% &  148.53\% &                                              182.84\% &                                                84.33\% &    80.69\% \\
     & Return &  108.22\% &  138.64\% &                                              159.00\% &                                                78.73\% &    81.61\% \\
\bottomrule
\end{tabular}
\end{table}



\begin{figure}[!htb]
    \centering
    \includegraphics[width = 0.8\linewidth]{Resultados/GSR/Figuras/png/barplot_gsr_avg_5_scene_blind.png}
    \caption{Barplot of the average SDNN of the blind participants on each method.}
    \label{fig:barplot_gsr_avg_5_scene_blind}
\end{figure}

Figure \ref{fig:boxplot_gsr_avg_blind_scene} presents the boxplot of the percentual variation in the skin conductance for each method. The ‘base’ method has the lowest variation among all methods. Also, the introduction of vibration increases the method variance. Figure \ref{fig:boxplot_gsr_avg_blind_rounds} presents the GSR grouped by the rounds. In this case there is no apparent different between the rounds.

\begin{figure}[!htb]
    \centering
    \begin{minipage}{0.45\textwidth}
        \centering
        \includegraphics[width = 0.8\linewidth]{Resultados/GSR/Figuras/png/boxplot_gsr_avg_blind_scene.png}
        \caption{Boxplot of the GSR of the blind participants grouped by method.}
        \label{fig:boxplot_gsr_avg_blind_scene}
    \end{minipage}
    \begin{minipage}{0.45\textwidth}
        \centering
        \includegraphics[width = 0.8\linewidth]{Resultados/GSR/Figuras/png/boxplot_gsr_avg_blind_rounds.png}
        \caption{Boxplot of the GSR of the blind participants grouped by round.}
        \label{fig:boxplot_gsr_avg_blind_rounds}
    \end{minipage}
\end{figure}

Figures \ref{fig:qqplot_gsr_two_way_blind} and \ref{fig:residplot_gsr_two_way_blind} shows the QQ plot and the residual distribution. The Table \ref{tab:blocanova_gsr_two_way_blind} shows the ANOVA test p-value for the GSR percentual variance. Although the p-value for the method is not below the threshold of 0.05, it close to it, indicating that probably the GSR is affected by it. 

The Figures  shows the distribution and variance of the Table \ref{tab:gsr_var_blind}. These Figures shows that the data are normally distributed but the participants had different  that the methods have a similar variance.

\begin{figure}[!htb]
    \centering
    \begin{minipage}{0.45\textwidth}
        \centering
        \includegraphics[width = 0.8\linewidth]{Resultados/GSR/Figuras/png/qqplot_gsr_two_way_blind.png}
        \caption{QQ plot of the SDNN of the blind participants on each method.}
        \label{fig:qqplot_gsr_two_way_blind}
    \end{minipage}
    \begin{minipage}{0.45\textwidth}
        \centering
        \includegraphics[width = 0.8\linewidth]{Resultados/GSR/Figuras/png/residplot_gsr_two_way_blind.png}
        \caption{Residual plot of the SDNN of the blind participants on each method.}
        \label{fig:residplot_gsr_two_way_blind}
    \end{minipage}
\end{figure}


\begin{table}[!htb]
\centering
\caption{Anova p-value for the mental demand average on each method for blinded users.}
\label{tab:blocanova_gsr_two_way_blind}
\begin{tabular}{lrrrrl}
\toprule
               Source &  Squared sum &  DOF & Squared average &      F & \begin{tabular}[c]{@{}l@{}}P-Value \\ $(F_{0} > F)$\end{tabular} \\
\midrule
Participants (Blocks) &  1499470.825 &    2 &      749735.412 & 14.528 &                                                                  \\
         \    Methods &   599028.542 &    4 &      149757.136 &  2.902 &                                                            0.051 \\
          \    Rounds &     6756.031 &    1 &        6756.031 &  0.131 &                                                            0.722 \\
     \    Interaction &     8702.285 &    4 &        2175.571 &  0.042 &                                                            0.996 \\
   Experimental Error &   928919.342 &   18 &       51606.630 &        &                                                                  \\
                Total &  3042877.025 &   29 &                 &        &                                                                  \\
\bottomrule
\end{tabular}
\end{table}


%
%%\input{Resultados/GSR/Tabelas/lsd_gsr_two_way.tex}
%
%The Table \ref{tab:blocanova_gsr_two_way_blind} does not prove that any method or round has some influence in the skin conductance variation, thus in the Mental Workload. Although, in the Figure \ref{fig:boxplot_gsr_avg_blind_scene} it is posible to notice that the "Base" and the "Audio" method have a different distribution. As it has already commented before, maybe the result of the anova test is a conseguence of a small sample size.
%
\FloatBarrier

 

\subsection{Final Remarks}

% Mental demand
% • ANOVA não deu resultado
% • Grafico - Presença de haptico aumenta a demanda mental

% NASA TLX
% • ANOVA - Efeito do método e to round.
% •• Metodo - 3 grupos
% ••• Base e Audio menores Carga mental
% ••• Haptic alta Carga mental
% 
% SAGAT
% ANOVA - Round tem efeito no SAGAT
%
% Satisfação
% ANOVA mostrou que o método fez diferença na satisfação do usuario
% Audio e Mix os preferidos
% Participantes comentando que o haptic e o virtual eram imprecisos e confusos

% BPM
% ANOVA - não deu resultado
% Gráfico mostra uma pequena diferença entre os métodos

% SDNN
% ANOVA - não deu resultado
% Gráfico - "Base" tem uma variância menor

% GSR
% ANOVA - não deu resultado
% Grafico - Base e Audio

The “Audio” method showed a higher performance among the other methods and the use of a haptic device decreased the performance of all methods. This probably happened because the participants are already used to use sound to guide themselves, especially environmental sounds. The environment sounds used inside the scenes that gave hints about locations where always the same (telephone ringing, laptop keyboard sounds, exterior noise, door opening and closing). It is likely that the participants felt more secure when it only had to focus on the sounds around him/her. This is reinforced by the fact that, during the “Audio” only guidance, half of the participants did not called for any command, or used only a few times the audio command option.

The fact that the haptic devices caused a higher average and a higher variation is probably due to the fact that the users had to learn and get used with them. Besides, for being just conceptual, their precision was not as big as they were expecting. That explains why their results were not as good as the “Base” or “Audio” methods and these results are correctly related to the satisfaction questionnaires, which scored them as the unsatisfied devices.

The ANOVA test from the mental demand did not prove any influence of the method nor the rounds, but the same test for the NASA-TLX score proved a influence of these both factors. That means that the methods influenced the score in other NASA-TLX dimension.

The SAGAT's ANOVA proved that the round did influenced the participants average score, meaning that they did learn from one round to another and that effect was a natural thing, not an effect caused by the devices.

The other ANOVA tests were inconclusive, but most of their boxplots most showed a difference between the methods. In these cases the reason is the small sample size and the sample’s variety. The blind participants age range was from 26 to 56, with an average of 43.5 years, and the education range was from High School to 2 graduations. That may impact the user experience as well in the questionnaires' answer.

But all the participants showed a great enthusiasm before, during and after the research. They also recommend some modifications that would bring more realism for they. And of course, they made some complaints, such as:

\begin{itemize}
    \item The speakers inside the HMD were not could enough for some to give them the precise location of its origin;
    \item The HMD was big enough to cover have of the participant’s face and that gave them a strange sensation, since some of them use the air or the wind feeling on the face to give them hints about the location of walls or other high obstacles;
    \item As said before, the precision of the vibration was not good for them to use the devices. That is mainly because of how the HMD position the user inside the virtual environment. \\
    The user is represented as a vertical capsule, and the HMD is positioned on the top end of that capsule. If the user tilts his/her head down, as if they were facing the ground, the capsule rotates in relation to the HMD point making the virtual body of the user occupy a total different space from the reality. The Figure \ref{fig:user_envelope} represents that situation. 

    \begin{figure}[!htb]
        \centering
        \begin{minipage}{0.45\textwidth}
            \centering
            \includegraphics[width = 0.8\linewidth]{Resultados/envelope1.png}
            \subcaption{The user's capsule while the participant is straight and looking forward.}
            \label{fig:user_straight}
        \end{minipage}
        \begin{minipage}{0.45\textwidth}
            \centering
            \includegraphics[width = 0.8\linewidth]{Resultados/envelope2.png}
            \subcaption{The user's capsule while the participant is straight but looking down.}
            \label{fig:user_looking_down}
        \end{minipage}
        \caption{Two different capsule positions based on the user's head position.}
        \label{fig:user_envelope}
    \end{figure}
    
    \item The vibration from the haptic belt was not intense enough sometimes.
\end{itemize}

\section{Comparison between BVI users and sighted users}

In this section, the second goal of this experiment, “do non-BVI users, when deprived from their vision, evaluate assistive devices in a similar way as BVI users?”, will be linked with the gathered data and then compared with the results of the first goal's section. It is expected that both results would be different. As was the last section, this section will also be divided in the same subsections.

\subsection{Subjective data}

Only two of the questionnaires will be analyzed, the NASA-TLX and the Adapted SAGAT, and it is expected that for:

\begin{itemize}
    \item \nameref{subsubsec:results_nasa_tlx_2};
    
        There will be a noticeable difference between the sight sample mental workload and the blind sample mental workload.

    \item \nameref{subsubsec:results_adapted_sagat_2};
    
        Is expected to notice a difference between the “blind” sample and the “sight” sample.

    \item \nameref{subsubsec:results_questionnaires}.

        Meant to assess the user experience with each method.

\end{itemize}

\subsubsection{NASA-TLX}
\label{subsubsec:results_nasa_tlx_2}

\paragraph{Analysis of the mental demand scale}\mbox{}\\

The Table \ref{tab:md_table_noBase} presents the ‘mental demand’ score of all participants, while the corresponding barplot is presented in Figure \ref{fig:barplot_md_avg_4_scene_blind_sight}. It is interesting to observe that sighted people gave a higher score to audio, as they are not so familiar to use sounds as source of guidance.


\begin{table}[!htb]
\centering
\caption{Mental demand felled by the participants.}
\label{tab:md_table_noBase}
\begin{tabular}{lllrrrrr}
\toprule
    &       &        & Audio & \begin{tabular}[c]{@{}l@{}}Haptic\\ Belt\end{tabular} & \begin{tabular}[c]{@{}l@{}}Virtual\\ Cane\end{tabular} & Mixture \\
Participant & \begin{tabular}[c]{@{}l@{}}Visual\\ Condition\end{tabular} & Round &       &                                                       &                                                        &         \\
\midrule
001C & Blind & First &     1 &                                                    14 &                                                      3 &       6 \\
    &       & Return &     1 &                                                    10 &                                                      2 &       6 \\
002C & Blind & First &     1 &                                                     1 &                                                     10 &      12 \\
    &       & Return &     1 &                                                     1 &                                                     10 &       3 \\
003C & Blind & First &     5 &                                                     5 &                                                      8 &       1 \\
    &       & Return &     1 &                                                     1 &                                                      2 &       1 \\
004C & Blind & First &    10 &                                                    15 &                                                     10 &      10 \\
    &       & Return &    10 &                                                    14 &                                                      8 &      10 \\
001 & Sight & First &    12 &                                                    11 &                                                      5 &       9 \\
    &       & Return &    13 &                                                    13 &                                                      5 &      10 \\
003 & Sight & First &    18 &                                                    18 &                                                     16 &      10 \\
    &       & Return &    12 &                                                    15 &                                                     11 &       8 \\
004 & Sight & First &    17 &                                                    20 &                                                     12 &      20 \\
    &       & Return &    12 &                                                    15 &                                                     10 &      15 \\
005 & Sight & First &     4 &                                                    12 &                                                     10 &      13 \\
    &       & Return &     6 &                                                    10 &                                                      6 &      12 \\
\bottomrule
\end{tabular}
\end{table}



 %The Figures \ref{fig:barplot_md_avg_4_scene_blind} and \ref{fig:barplot_md_avg_4_scene_sight} show a systematic reduction on the perceived mental demand in all methods between the rounds for both groups. But the Figure \ref{fig:barplot_md_avg_4_scene} shows that the average of each method was very different between the two groups.

\begin{figure}[!htb]
    \centering
    \begin{minipage}{\textwidth}
        \centering
        \includegraphics[width = 0.8\linewidth]{Resultados/Nasa/Figuras/png/barplot_md_avg_4_scene_blind.png}
        \subcaption{Blind participants}
        \label{fig:barplot_md_avg_4_scene_blind}
    \end{minipage}
    \begin{minipage}{\textwidth}
        \centering
        \includegraphics[width = 0.8\linewidth]{Resultados/Nasa/Figuras/png/barplot_md_avg_4_scene_sight.png}
        \subcaption{Sight participants}
        \label{fig:barplot_md_avg_4_scene_sight}
    \end{minipage}
    \caption{Barplot of the average mental demand on each method and round.}
    \label{fig:barplot_md_avg_4_scene_blind_sight}
\end{figure}
%\begin{figure}[!htb]
%    \centering
%    \includegraphics[width = 0.8\linewidth]{Resultados/Nasa/Figuras/png/barplot_md_avg_4_scene.png}
%    \caption{Barplot of the average mental demand of both participants on each method.}
%    \label{fig:barplot_md_avg_4_scene}
%\end{figure}

Figures \ref{fig:boxplot_noBase_md_4_scene} and \ref{fig:boxplot_noBase_md_4_rounds} presents the box plot for both groups, organized by method and round. It is clear that the mental demand is systematically higher for sighted people, which is something expected. But while blind participants considered the ‘audio’ method less demanding, sighted participants gave preference to the virtual cane. For both groups, we observe a decrease in the mental demand.

\begin{figure}[!htb]
    \centering
    \begin{minipage}{0.45\textwidth}
        \centering
        \includegraphics[width = 0.8\linewidth]{Resultados/Nasa/Figuras/png/boxplot_noBase_md_4_scene.png}
        \caption{Boxplot of the mental demand of the participants grouped by method.}
        \label{fig:boxplot_noBase_md_4_scene}
    \end{minipage}
    \begin{minipage}{0.45\textwidth}
        \centering
        \includegraphics[width = 0.8\linewidth]{Resultados/Nasa/Figuras/png/boxplot_noBase_md_4_rounds.png}
        \caption{Boxplot of the mental demand of the participants grouped by round.}
        \label{fig:boxplot_noBase_md_4_rounds}
    \end{minipage}
\end{figure}

%The Table \ref{tab:md_average_group_noBase} shows the average mental demand of both samples and is possible to notice how the average perceived mental demand by the sight sample was higher in every method.
%
%
\begin{table}[!htb]
\centering
\caption{Mental demand average grouped by participant and visual condition}
\label{tab:md_average_group_noBase}
\begin{tabular}{lrrrrrr}
\toprule
{} &  Audio & \begin{tabular}[c]{@{}l@{}}Haptic\\ Belt\end{tabular} & \begin{tabular}[c]{@{}l@{}}Virtual\\ Cane\end{tabular} &  Mixture \\
Visual Condition &        &                                                       &                                                        &          \\
\midrule
Blind            &   3.75 &                                                  7.62 &                                                   6.62 &    6.125 \\
Sight            &  11.75 &                                                 14.25 &                                                   9.38 &   12.125 \\
\bottomrule
\end{tabular}
\end{table}



Figures \ref{fig:qqplot_md_avg_two_way_sight} and \ref{fig:residplot_md_avg_two_way_sight} show the QQ plot and residual distribution for the sighted data, confirming that the data is normally distributed and participants have similar variance. Table \ref{tab:blocanova_md_avg_two_way_blind_sight} brings the results of ANOVA. Different from the blind participants, in the case of sighted ones, the p-value for ‘method’ is below the threshold of 0.05, confirming it as a significant variable for the mental demand. In the case of ‘round’, data from both sighted and blind participants resulted in the same p-value of 0.075, which is close to the traditional threshold of 0.05, but slightly higher. 

\begin{table}
    \caption{Anova p-value for the mental demand average on each method'}
    \label{tab:blocanova_md_avg_two_way_blind_sight}
\begin{minipage}{0.45\textwidth}
    \subcaption{Blind participants}
    
\centering
\begin{tabular}{ll}
\toprule
          Source & P-Value \\
\midrule
    \    Methods &   0.170 \\
     \    Rounds &   0.075 \\
\    Interaction &   0.993 \\
\bottomrule
\end{tabular}

\end{minipage}
\begin{minipage}{0.45\textwidth}
    \subcaption{Sight participants}
    
\centering
\begin{tabular}{ll}
\toprule
          Source & P-Value \\
\midrule
    \    Methods & 0.049** \\
     \    Rounds &   0.075 \\
\    Interaction &   0.990 \\
\bottomrule
\end{tabular}
    
\end{minipage}
\end{table}

\begin{figure}[!htb]
    \centering
    \begin{minipage}{0.45\textwidth}
        \centering
        \includegraphics[width = 0.8\linewidth]{Resultados/Nasa/Figuras/png/qqplot_md_avg_two_way_sight.png}
        \caption{QQ plot of the mental demand of the sight participants on each method.}
        \label{fig:qqplot_md_avg_two_way_sight}
    \end{minipage}
    \begin{minipage}{0.45\textwidth}
        \centering
        \includegraphics[width = 0.8\linewidth]{Resultados/Nasa/Figuras/png/residplot_md_avg_two_way_sight.png}
        \caption{Residual plot of the mental demand score the sighted participants on each method.}
        \label{fig:residplot_md_avg_two_way_sight}
    \end{minipage}
\end{figure}

%The Table \ref{tab:lsd_md_avg_two_way_sight} presents the conclusion of a pairwise Fisher LSD test of the previous ANOVA test. The results show that only the "Audio" has a similar mental demand as the "Mixture" method.
%
%
\begin{table}[!htb]
\centering
\caption{Cross validation p-value for the mental demand average on each method for sighted users.}
\label{tab:lsd_md_avg_two_way_sight}
\begin{tabular}{rcllr}
\toprule
      \multicolumn{3}{c}{Method} &                          \multicolumn{2}{c}{Analysis} \\
\midrule
       Audio & $X$ & Haptic Belt &        $H_1 : \mu_{Audio} \ne \mu_{Haptic Belt}$ & ** \\
      Audio & $X$ & Virtual Cane &       $H_1 : \mu_{Audio} \ne \mu_{Virtual Cane}$ & ** \\
           Audio & $X$ & Mixture &                $H_0 : \mu_{Audio} = \mu_{Mixture}$ &  \\
Haptic Belt & $X$ & Virtual Cane & $H_1 : \mu_{Haptic Belt} \ne \mu_{Virtual Cane}$ & ** \\
     Haptic Belt & $X$ & Mixture &      $H_1 : \mu_{Haptic Belt} \ne \mu_{Mixture}$ & ** \\
    Virtual Cane & $X$ & Mixture &     $H_1 : \mu_{Virtual Cane} \ne \mu_{Mixture}$ & ** \\
\bottomrule
\end{tabular}
\end{table}


%
%The Table \ref{tab:md_var_average_group} shows the average of the mental demand variation between the rounds. This table shows that the mental demand variation from the “Audio” has the lower variation, and the rest are similar variations.
%
%
\begin{table}[!htb]
\centering
\caption{Mental demand variation grouped by participant and visual condition}
\label{tab:md_var_average_group}
\begin{tabular}{lrrrrrr}
\toprule
{} &  Base & Audio & \begin{tabular}[c]{@{}l@{}}Haptic\\ Belt\end{tabular} & \begin{tabular}[c]{@{}l@{}}Virtual\\ Cane\end{tabular} & Mixture \\
Visual Condition &       &       &                                                       &                                                        &         \\
\midrule
Blind            &  -2.5 &  -1.0 &                                                  -2.2 &                                                   -2.2 &    -2.2 \\
Sight            &  -1.0 &  -2.0 &                                                  -2.0 &                                                   -2.8 &    -1.8 \\
\bottomrule
\end{tabular}
\end{table}


%
%The Figures \ref{fig:qqplot_md_var_sight} and \ref{fig:residplot_md_var_sight} shows the distribution and variance of the mental demand variation of the Table \ref{tab:md_table_blind}. These Figures shows that the data are normally distributed and that the methods have a similar variance.
%The Table \ref{tab:blocanova_md_var_sight} shows the Anova test p-value of the mental demand of the "sight" sample between the guidance methods. The p-value indicates that there is no influence of the methods in the variation of mental demand between the rounds. 
%
%
\begin{table}[!htb]
\centering
\caption{Anova p-value for the mental demand variation on each method for sighted users.}
\label{tab:blocanova_md_var_sight}
\begin{tabular}{lrrrrr}
\toprule
Source & P-Value \\
\midrule
Method &   0.900 \\
\bottomrule
\end{tabular}
\end{table}


%
%\begin{figure}[!htb]
%    \centering
%    \begin{minipage}{0.45\textwidth}
%        \centering
%        \includegraphics[width = 0.8\linewidth]{Resultados/Nasa/Figuras/png/qqplot_md_var_sight.png}
%        \caption{Residual plot of the mental demand variation of the blind participants on each method.}
%        \label{fig:qqplot_md_var_sight}
%    \end{minipage}
%    \begin{minipage}{0.45\textwidth}
%        \centering
%        \includegraphics[width = 0.8\linewidth]{Resultados/Nasa/Figuras/png/residplot_md_var_sight.png}
%        \caption{Residual plot of the mental demand variation of the sighted participants on each method.}
%        \label{fig:residplot_md_var_sight}
%    \end{minipage}
%\end{figure}
%
%%The Table \ref{tab:lsdbloc_mental_demand_var} presents the conclusion of a pairwise Fisher LSD test of the blind mental demand between all the guidance methods. The results show that all methods have similar variations.
%
%%\input{Resultados/Nasa/Tabelas/lsdbloc_mental_demand_var.tex}
%
%To close up, according to the ANOVA test at Table \ref{tab:lsd_md_avg_two_way_sight} the method do have influence on the mental demand of the sighted participant and that the "Audio" and the "Mixture" method have the same mental demand for them. This differs from the result of the previous section that was the ANOVA did not prove any effect and that the "Audio" and "Mixture" methods could not be said to be similar. Although for the "blind" users, they were also the methods that caused the lowest mental demand.
%
%There is no influence in the tested methods in the participants mental demand variation, as shown in the Table \ref{tab:blocanova_md_var_sight}.

\FloatBarrier

%%%%%%%%%%%%%%%%%%%%%%%%%%%%%%%%%%%%%%%%%%%%%%%%%%%%%%%%%%%%%%%%%%%%%%%%%%%%
%%%%%%%%%%%%%%%%%%%%%%%%%%%%%%%%%%%%%%%%%%%%%%%%%%%%%%%%%%%%%%%%%%%%%%%%%%%%
%%%%%%%%%%%%%%%%%%%%%%%%%%%%%%%%%%%%%%%%%%%%%%%%%%%%%%%%%%%%%%%%%%%%%%%%%%%%
%%%%%%%%%%%%%%%%%%%%%%%%%%%%%%%%%%%%%%%%%%%%%%%%%%%%%%%%%%%%%%%%%%%%%%%%%%%%


\paragraph{Analysis of the NASA-TLX score}\mbox{}\\

Table \ref{tab:nasa_table_noBase} brings the NASA-TLX global score of all participants, while the corresponding barplot is presented in Figure \ref{fig:barplot_nasa_avg_4_scene}.


\begin{table}[!htb]
\centering
\caption{NASA-TLX score felled by the participants.}
\label{tab:nasa_table_noBase}
\begin{tabular}{lllrrrrr}
\toprule
    &       &        &  Audio & \begin{tabular}[c]{@{}l@{}}Haptic\\ Belt\end{tabular} & \begin{tabular}[c]{@{}l@{}}Virtual\\ Cane\end{tabular} & Mixture \\
Participant & \begin{tabular}[c]{@{}l@{}}Visual\\ Condition\end{tabular} & Round &        &                                                       &                                                        &         \\
\midrule
001C & Blind & First &  4.000 &                                                 8.833 &                                                  5.167 &   6.333 \\
    &       & Return &  4.000 &                                                 6.667 &                                                  4.500 &   6.167 \\
002C & Blind & First &  4.833 &                                                 4.833 &                                                  9.000 &   7.000 \\
    &       & Return &  4.833 &                                                 4.833 &                                                  7.000 &   5.167 \\
003C & Blind & First &  4.000 &                                                 5.333 &                                                  6.667 &   3.500 \\
    &       & Return &  3.833 &                                                 3.667 &                                                  3.500 &   3.500 \\
004C & Blind & First & 10.000 &                                                12.667 &                                                  9.667 &  11.000 \\
    &       & Return &  9.167 &                                                11.667 &                                                  9.333 &  10.833 \\
001 & Sight & First & 10.167 &                                                 9.833 &                                                  7.000 &   9.000 \\
    &       & Return & 11.000 &                                                10.833 &                                                  6.167 &   9.333 \\
003 & Sight & First &  9.833 &                                                10.167 &                                                  9.500 &   6.500 \\
    &       & Return &  6.667 &                                                 9.667 &                                                  7.833 &   4.833 \\
004 & Sight & First & 14.833 &                                                13.667 &                                                 11.500 &  15.833 \\
    &       & Return & 11.833 &                                                11.833 &                                                 10.833 &  12.167 \\
005 & Sight & First &  7.667 &                                                 9.000 &                                                  8.000 &   9.667 \\
    &       & Return &  7.667 &                                                 8.667 &                                                  7.667 &   6.000 \\
\bottomrule
\end{tabular}
\end{table}



From Figure \ref{fig:barplot_nasa_avg_4_scene} it is possible to see that, similar to blind participants, sighted participants also consider that the workload of the return round was lower than that of the first round. However, similar to what happened for the mental demand, sighted participants considered ‘virtual cane’ as the method with the lowest workload, while, for  blind participants, it was the ‘audio’.

\begin{figure}[!htb]
    \centering
    \begin{minipage}{\textwidth}
        \centering
        \includegraphics[width = 0.8\linewidth]{Resultados/Nasa/Figuras/png/barplot_nasa_avg_4_scene_blind.png}
        \subcaption{Blind participants.}
        \label{fig:barplot_nasa_avg_4_scene_blind}
    \end{minipage}
    \begin{minipage}{\textwidth}
        \centering
        \includegraphics[width = 0.8\linewidth]{Resultados/Nasa/Figuras/png/barplot_nasa_avg_4_scene_sight.png}
        \subcaption{Sight participants.}
        \label{fig:barplot_nasa_avg_4_scene_sight}
    \end{minipage}
    \caption{Barplot of the NASA-TLX score on each method and round.}
    \label{fig:barplot_nasa_avg_4_scene}
\end{figure}
%\begin{figure}[!htb]
%    \centering
%    \includegraphics[width = 0.8\linewidth]{Resultados/Nasa/Figuras/png/barplot_nasa_avg_4_scene.png}
%    \caption{Barplot of the NASA-TLX score of both participants on each method.}
%    \label{fig:barplot_nasa_avg_4_scene}
%\end{figure}

Figures \ref{fig:boxplot_noBase_nasa_4_scene} and \ref{fig:boxplot_noBase_nasa_4_rounds} present the boxplots of NASA-TLX global score. Again, it is possible to see that sighted people usually give higher workload scores than blind ones. The influence of the round is approximately the same. But the order of preference of the methods are different.

\begin{figure}[!htb]
    \centering
    \begin{minipage}{0.45\textwidth}
        \centering
        \includegraphics[width = 0.8\linewidth]{Resultados/Nasa/Figuras/png/boxplot_noBase_nasa_4_scene.png}
        \caption{Boxplot of the NASA-TLX score of the participants grouped by method.}
        \label{fig:boxplot_noBase_nasa_4_scene}
    \end{minipage}
    \begin{minipage}{0.45\textwidth}
        \centering
        \includegraphics[width = 0.8\linewidth]{Resultados/Nasa/Figuras/png/boxplot_noBase_nasa_4_rounds.png}
        \caption{Boxplot of the NASA-TLX score of the participants grouped by round.}
        \label{fig:boxplot_noBase_nasa_4_rounds}
    \end{minipage}
\end{figure}

%The Table \ref{tab:nasa_average_group_noBase} shows the average NASA-TLX score of both samples and is possible to notice how the average perceived NASA-TXL average by the sight sample was also higher in every method.
%
%
\begin{table}[!htb]
\centering
\caption{Average NASA-TLX score grouped by participant and visual condition}
\label{tab:nasa_average_group_noBase}
\begin{tabular}{lrrrrrr}
\toprule
{} & Audio & \begin{tabular}[c]{@{}l@{}}Haptic\\ Belt\end{tabular} & \begin{tabular}[c]{@{}l@{}}Virtual\\ Cane\end{tabular} &  Mixture \\
Visual Condition &       &                                                       &                                                        &          \\
\midrule
Blind            &  5.58 &                                                  7.31 &                                                   6.85 &    6.688 \\
Sight            &  9.96 &                                                 10.46 &                                                   8.56 &    9.167 \\
\bottomrule
\end{tabular}
\end{table}


Figures \ref{fig:qqplot_nasa_avg_two_way_sight} and \ref{fig:residplot_nasa_avg_two_way_sight} bring the QQ plot and residual distribution of the data from sighted participants, showing that ANOVA can be used. The p-values for both groups are presented in Table \ref{tab:blocanova_nasa_avg_two_way_blind_sight}. It confirms the influence of the round for both sighted and blind people. In the case of the method, the p-value of ‘blind’ is lower than the threshold of 0.5, while that of ‘sighted’ is slightly higher.

\begin{table}
    \caption{Anova p-value for the mental demand average on each method'}
    \label{tab:blocanova_nasa_avg_two_way_blind_sight}
    \begin{minipage}{0.45\textwidth}
        \subcaption{Blind participants}
        \input{Resultados/Nasa/Tabelas/blocanova_nasa_avg_two_way_blindSemBegin.tex}
    \end{minipage}
    \begin{minipage}{0.45\textwidth}
        \subcaption{Sight participants}
        
\centering
\begin{tabular}{ll}
\toprule
          Source & P-Value \\
\midrule
    \    Methods &   0.086 \\
     \    Rounds & 0.034** \\
\    Interaction &   0.688 \\
\bottomrule
\end{tabular}
    
    \end{minipage}
\end{table}


\begin{figure}[!htb]
    \centering
    \begin{minipage}{0.45\textwidth}
        \centering
        \includegraphics[width = 0.8\linewidth]{Resultados/Nasa/Figuras/png/qqplot_nasa_avg_two_way_sight.png}
        \caption{QQ plot of the NASA-TLX score of the sight participants on each method.}
        \label{fig:qqplot_nasa_avg_two_way_sight}
    \end{minipage}
    \begin{minipage}{0.45\textwidth}
        \centering
        \includegraphics[width = 0.8\linewidth]{Resultados/Nasa/Figuras/png/residplot_nasa_avg_two_way_sight.png}
        \caption{Residual plot of the NASA-TLX score the sight participants on each method.}
        \label{fig:residplot_nasa_avg_two_way_sight}
    \end{minipage}
\end{figure}

%The Table \ref{tab:lsd_nasa_avg_two_way_sight} presents the conclusion of a pairwise Fisher LSD test of the previous ANOVA test and it shows that all the method had an effect in the NASA-TLX score.
%
%\input{Resultados/Nasa/Tabelas/lsd_nasa_avg_two_way_sight.tex}

%The Table \ref{tab:nasa_var_group} shows the average of the NASA-TLX score variation between the rounds. This table shows that the score variation from the “Audio” has the lower variation, and the rest are similar variations.

%
\begin{table}[!htb]
\centering
\caption{NASA-TLX score grouped by participant and visual Condition.}
\label{tab:nasa_var_group}
\begin{tabular}{lrrrrrr}
\toprule
{} &     Base &    Audio & \begin{tabular}[c]{@{}l@{}}Haptic\\ Belt\end{tabular} & \begin{tabular}[c]{@{}l@{}}Virtual\\ Cane\end{tabular} &  Mixture \\
Visual Condition &          &          &                                                       &                                                        &          \\
\midrule
Blind            &  -13.7\% &   -3.1\% &                                               -15.9\% &                                                -21.5\% &   -7.6\% \\
Sight            &   -1.4\% &  -11.1\% &                                                -3.0\% &                                                 -9.9\% &  -20.8\% \\
\bottomrule
\end{tabular}
\end{table}



%The Figures \ref{fig:qqplot_nasa_var_sight} and \ref{fig:residplot_nasa_var_sight} shows the distribution and variance of the NASA-TLX score variation of the Table \ref{tab:md_table_blind}. These Figures shows that the data are normally distributed and that the methods have a similar variance.

%The Table \ref{tab:blocanova_nasa_var_sight} shows the Anova test p-value of the NASA-TLX score of the "sight" sample between the guidance methods and it proves that there is no influence of the methods in the variation of score between the rounds. 

%
\begin{table}[!htb]
\centering
\caption{Anova p-value for the NASA score variation on each method for sighted users.}
\label{tab:blocanova_nasa_var_sight}
\begin{tabular}{lrrrrr}
\toprule
               Source &  Squared sum &  DOF & Squared average &     F & \begin{tabular}[c]{@{}l@{}}P-Value \\ $(F_{0} > F)$\end{tabular} \\
\midrule
Participants (blocks) &       15.436 &    3 &           2.229 & 3.517 &                                                                  \\
               Method &        6.686 &    3 &           5.145 & 1.523 &                                                            0.274 \\
   Experimental error &       13.168 &    9 &           1.463 &       &                                                                  \\
                Total &       35.290 &   15 &                 &       &                                                                  \\
\bottomrule
\end{tabular}
\end{table}



%\begin{figure}[!htb]
%    \centering
%    \begin{minipage}{0.45\textwidth}
%        \centering
%        \includegraphics[width = 0.8\linewidth]{Resultados/Nasa/Figuras/png/qqplot_nasa_var_sight.png}
%        \caption{Residual plot of the variation NASA-TLX score of the blind participants on each method.}
%        \label{fig:qqplot_nasa_var_sight}
%    \end{minipage}
%    \begin{minipage}{0.45\textwidth}
%        \centering
%        \includegraphics[width = 0.8\linewidth]{Resultados/Nasa/Figuras/png/residplot_nasa_var_sight.png}
%        \caption{Residual plot of the variation NASA-TLX score of the sighted participants on each method.}
%        \label{fig:residplot_nasa_var_sight}
%    \end{minipage}
%\end{figure}

%The Table \ref{tab:lsdbloc_mental_demand_var} presents the conclusion of a pairwise Fisher LSD test of the blind mental demand between all the guidance methods. The results show that all methods have similar variations.

%\input{Resultados/Nasa/Tabelas/lsdbloc_mental_demand_var.tex}

%To close up, according to the ANOVA test at Table \ref{tab:blocanova_nasa_avg_two_way_sight} the methods do not an effect on the score, but rounds do. The blind users felt an impact on both the method and the round.
%
%There is no influence in the tested methods in the participants NASA-TLX score variation, as shown in the Table \ref{tab:blocanova_nasa_var_sight}.

\FloatBarrier

\subsubsection{Adapted SAGAT}
\label{subsubsec:results_adapted_sagat_2}

Table\ref{tab:sagat_table_noBase} presents the SAGAT score of all participants. The corresponding barplot is presented in Figure \ref{fig:barplot_sagat_avg_4_scene_blind_sight}.


\begin{table}[!htb]
\centering
\caption{SAGAT global score felled by the participants.}
\label{tab:sagat_table_noBase}
\begin{tabular}{lllrrrrr}
\toprule
    &       &        &  Audio & \begin{tabular}[c]{@{}l@{}}Haptic\\ Belt\end{tabular} & \begin{tabular}[c]{@{}l@{}}Virtual\\ Cane\end{tabular} & Mixture \\
Participant & \begin{tabular}[c]{@{}l@{}}Visual\\ Condition\end{tabular} & Round &        &                                                       &                                                        &         \\
\midrule
001C & Blind & First &  5.500 &                                                 5.330 &                                                  5.830 &   3.500 \\
    &       & Return &  6.500 &                                                 8.500 &                                                  5.500 &   5.500 \\
002C & Blind & First &  4.500 &                                                 3.990 &                                                  4.500 &   6.250 \\
    &       & Return &  5.000 &                                                 4.000 &                                                  6.500 &   8.500 \\
003C & Blind & First &  7.500 &                                                 7.490 &                                                  4.660 &   9.000 \\
    &       & Return & 10.000 &                                                 8.500 &                                                  9.000 &   9.000 \\
004C & Blind & First &  6.000 &                                                 7.660 &                                                  4.990 &   6.500 \\
    &       & Return &  6.000 &                                                 9.250 &                                                  7.250 &   9.000 \\
001 & Sight & First &  4.500 &                                                 4.330 &                                                  2.660 &   6.500 \\
    &       & Return &  6.000 &                                                 5.000 &                                                  5.000 &   4.500 \\
003 & Sight & First &  6.750 &                                                 5.990 &                                                  3.990 &   6.750 \\
    &       & Return &  6.000 &                                                 7.250 &                                                  6.250 &   7.500 \\
004 & Sight & First &  7.250 &                                                 7.990 &                                                  5.990 &   8.250 \\
    &       & Return &  7.750 &                                                 9.500 &                                                  8.250 &   7.000 \\
005 & Sight & First &  3.000 &                                                 3.160 &                                                  3.990 &   4.000 \\
    &       & Return &  3.750 &                                                 3.000 &                                                  2.000 &   6.000 \\
\bottomrule
\end{tabular}
\end{table}



Figure \ref{fig:barplot_sagat_avg_4_scene_blind_sight}. shows that the SAGAT score for sighted participants is on average lower than that of blind participants, which is expected as they are not used to navigate without vision. Also, the increase in situation awareness from the first to the return round is lower. In the case of the mixture method, the SAGAT score did not improve at all. For both groups, the ‘virtual cane’ was the method with lowest score in the first round.

\begin{figure}[!htb]
    \centering
    \begin{minipage}{\textwidth}
        \centering
        \includegraphics[width = 0.8\linewidth]{Resultados/Sagat/Figuras/png/barplot_sagat_avg_4_scene_blind.png}
        \subcaption{Blind participants.}
        \label{fig:barplot_sagat_avg_4_scene_blind}
    \end{minipage}
    \begin{minipage}{\textwidth}
        \centering
        \includegraphics[width = 0.8\linewidth]{Resultados/Sagat/Figuras/png/barplot_sagat_avg_4_scene_sight.png}
        \subcaption{Sight participants.}
        \label{fig:barplot_sagat_avg_4_scene_sight}
    \end{minipage}
    \caption{Barplot of the SAGAT score on each method and round.}
    \label{fig:barplot_sagat_avg_4_scene_blind_sight}
\end{figure}
%\begin{figure}[!htb]
%    \centering
%    \includegraphics[width = 0.8\linewidth]{Resultados/Sagat/Figuras/png/barplot_sagat_avg_4_scene.png}
%    \caption{Barplot of the SAGAT score of both participants on each method.}
%    \label{fig:barplot_sagat_avg_4_scene}
%\end{figure}

Figures \ref{fig:boxplot_sagat_4_scene} and \ref{fig:boxplot_sagat_4_rounds} bring the boxplots. According to Figure \ref{fig:boxplot_sagat_4_scene}, both groups presented a higher situation awareness with ‘mixture’ and ‘haptic’. On the other hand, Figure \ref{fig:boxplot_sagat_4_rounds} confirms that the difference between the rounds is greater for blind participants. 

\begin{figure}[!htb]
    \centering
    \begin{minipage}{0.45\textwidth}
        \centering
        \includegraphics[width = 0.8\linewidth]{Resultados/Sagat/Figuras/png/boxplot_sagat_4_scene.png}
        \caption{Boxplot of the Sagat score of the participants grouped by method.}
        \label{fig:boxplot_sagat_4_scene}
    \end{minipage}
    \begin{minipage}{0.45\textwidth}
        \centering
        \includegraphics[width = 0.8\linewidth]{Resultados/Sagat/Figuras/png/boxplot_sagat_4_rounds.png}
        \caption{Boxplot of the Sagat score of the participants grouped by round.}
        \label{fig:boxplot_sagat_4_rounds}
    \end{minipage}
\end{figure}

%The Table \ref{tab:sagat_average_group_noBase} shows the average SAGAT score of both samples and is possible to notice how the average score by the blind users was higher in every method.
%
%
\begin{table}[!htb]
\centering
\caption{Adapted Sagat average global score grouped by participant and visual Condition.}
\label{tab:sagat_average_group_noBase}
\begin{tabular}{lrrrrrr}
\toprule
{} & Audio & \begin{tabular}[c]{@{}l@{}}Haptic\\ Belt\end{tabular} & \begin{tabular}[c]{@{}l@{}}Virtual\\ Cane\end{tabular} &  Mixture \\
Visual Condition &       &                                                       &                                                        &          \\
\midrule
Blind            &  6.38 &                                                  6.84 &                                                   6.03 &    7.156 \\
Sight            &  5.62 &                                                  5.78 &                                                   4.77 &    6.312 \\
\bottomrule
\end{tabular}
\end{table}



Figures \ref{fig:qqplot_sagat_avg_two_way_sight} and \ref{fig:residplot_sagat_avg_two_way_sight} brings the QQ plot and residual distribution. It is clear that the residuals variance is not equal among the participants. Table \ref{tab:blocanova_sagat_avg_two_way_blind_sight} brings the p-value from ANOVA. While for blind participants the round is a significant factor and the method is not, for sighted participants the result is the opposite, showing a significant influence of the method and not of the round.

\begin{table}
    \caption{Anova p-value for the SAGAT score on each method}
    \label{tab:blocanova_sagat_avg_two_way_blind_sight}
\begin{minipage}{0.45\textwidth}
    \subcaption{Blind participants}
    
\centering
\begin{tabular}{ll}
\toprule
          Source & P-Value \\
\midrule
    \    Methods &   0.277 \\
     \    Rounds & 0.002** \\
\    Interaction &   0.834 \\
\bottomrule
\end{tabular}

\end{minipage}
\begin{minipage}{0.45\textwidth}
    \subcaption{Sight participants}
    
\centering
\begin{tabular}{ll}
\toprule
          Source & P-Value \\
\midrule
    \    Methods & 0.035** \\
     \    Rounds &   0.095 \\
\    Interaction &   0.578 \\
\bottomrule
\end{tabular}
    
\end{minipage}
\end{table}

\begin{figure}[!htb]
    \centering
    \begin{minipage}{0.45\textwidth}
        \centering
        \includegraphics[width = 0.8\linewidth]{Resultados/Sagat/Figuras/png/qqplot_sagat_avg_two_way_sight.png}
        \caption{QQ plot of the mental demand of the sight participants on each method.}
        \label{fig:qqplot_sagat_avg_two_way_sight}
    \end{minipage}
    \begin{minipage}{0.45\textwidth}
        \centering
        \includegraphics[width = 0.8\linewidth]{Resultados/Sagat/Figuras/png/residplot_sagat_avg_two_way_sight.png}
        \caption{Residual plot of the mental demand score the sight participants on each method.}
        \label{fig:residplot_sagat_avg_two_way_sight}
    \end{minipage}
\end{figure}

%%%%%%%%%%%%%%%%%%%%%%%%%%%%%%%%%%%%%%%%%%%%%%%%%%%%%

%The Table \ref{tab:lsd_sagat_avg_two_way_sight} presents the conclusion of a pairwise Fisher LSD test of the previous ANOVA test. The results show that only the "Audio" and the "Haptic Belt" had a similar SAGAT score. This is different than the result from the ANOVA of the blind users, which indicated for them that the rounds had an effect.

%\input{Resultados/Sagat/Tabelas/lsd_sagat_avg_two_way_sight.tex}

%The Table \ref{tab:sagat_var_group_noBase} shows the average of the SAGAT score variation between the rounds. This table and the Figure \ref{fig:boxplot_sagat_4_rounds} show that, besides the higher average score, the blind users also had a higher variation between the rounds.

%
\begin{table}[!htb]
\centering
\caption{Adapted Sagat global score variation grouped by participant and visual Condition}
\label{tab:sagat_var_group_noBase}
\begin{tabular}{lrrrrrr}
\toprule
{} &  Audio & \begin{tabular}[c]{@{}l@{}}Haptic\\ Belt\end{tabular} & \begin{tabular}[c]{@{}l@{}}Virtual\\ Cane\end{tabular} & Mixture \\
Visual Condition &        &                                                       &                                                        &         \\
\midrule
Blind            &  15.66 &                                                 23.49 &                                                  44.30 &   32.90 \\
Sight            &  13.53 &                                                 12.59 &                                                  33.12 &    3.80 \\
\bottomrule
\end{tabular}
\end{table}





%\begin{figure}[!htb]
%    \centering
%    \begin{minipage}{0.45\textwidth}
%        \centering
%        \includegraphics[width = 0.8\linewidth]{Resultados/Sagat/Figuras/png/qqplot_sagat_var_sight.png}
%        \caption{Residual plot of the variation SAGAT score of the blind participants on each method.}
%        \label{fig:qqplot_sagat_var_sight}
%    \end{minipage}
%    \begin{minipage}{0.45\textwidth}
%        \centering
%        \includegraphics[width = 0.8\linewidth]{Resultados/Sagat/Figuras/png/residplot_sagat_var_sight.png}
%        \caption{Residual plot of the variation SAGAT score of the sighted participants on each method.}
%        \label{fig:residplot_sagat_var_sight}
%    \end{minipage}
%\end{figure}

%The Table \ref{tab:lsdbloc_mental_demand_var} presents the conclusion of a pairwise Fisher LSD test of the blind mental demand between all the guidance methods. The results show that all methods have similar variations.

%\input{Resultados/Nasa/Tabelas/lsdbloc_mental_demand_var.tex}

%Figures \ref{tab:sagat_average_group_noBase} and \ref{tab:sagat_var_group_noBase} brings the QQ plot and residual distribution. It is clear that the residuals variance is not equal among the participants. Table \ref{tab:blocanova_sagat_avg_two_way} brings the p-value from ANOVA. While for blind participants the round is a significant factor and the method is not, for sighted participants the result is the opposite, showing a significant influence of the method and not of the round.
%
%To close up, according to the Tables  with the Figures \ref{fig:boxplot_sagat_4_scene} and \ref{fig:boxplot_sagat_4_rounds} the blind user scored a higher SAGAT score than the sight user with the same conditions and devices. Besides that, the ANOVA and the LSD Fisher Test at Tables \ref{tab:blocanova_sagat_avg_two_way_sight} and \ref{tab:lsd_sagat_avg_two_way_sight} show that for the sight user the methods impact more their score, whilst the blind user were affected more with the rounds.
%
%There is no influence in the tested methods in the participants mental demand variation, as shown in the Table \ref{tab:blocanova_sagat_var_sight}.

\FloatBarrier

\subsubsection{Guidance method's questionnaire.}
\label{subsec:results_questionnaires}

Finally, the Questionnaire is analyzed to give an idea about the impressions of the users with each device. This is an important evaluation to seek their impressions of each method. The higher the score, the more the user was satisfaction with that method. The Table \ref{tab:questionnaire_average_blind} shows the score of each method and they are plotted in the Figure \ref{fig:barplot_questionnaire_scene_blind}. The Figure show a disatisfaction with the haptic devices alone.


\begin{table}[!htb]
\centering
\caption{ Guidance method questionnaire score felled by the blinded participants.}
\label{tab:questionnaire_average_blind}
\begin{tabular}{llrrrrr}
\toprule
{} &  Audio &  \begin{tabular}[c]{@{}l@{}}Haptic\\ Belt\end{tabular} &  \begin{tabular}[c]{@{}l@{}}Virtual\\ Cane\end{tabular} &  Mixture \\
Participant &        &                                                        &                                                         &          \\
\midrule
001C        &  0.774 &                                                  0.543 &                                                   0.629 &    0.865 \\
002C        &  0.857 &                                                  0.743 &                                                   0.543 &    0.935 \\
003C        &  0.929 &                                                  0.571 &                                                   0.543 &    0.745 \\
004C        &  0.881 &                                                  0.486 &                                                   0.400 &    0.730 \\
\bottomrule
\end{tabular}
\end{table}



\begin{figure}[!htb]
    \centering
    \includegraphics[width = 0.8\linewidth]{Resultados/Questionario/Figuras/png/barplot_questionnaire_scene_blind.png}
    \caption{Barplot of the average questionaire score of the blind participants on each method.}
    \label{fig:barplot_questionnaire_scene_blind}
\end{figure}

\begin{figure}[!htb]
    \centering
    \includegraphics[width = 0.6\linewidth]{Resultados/Questionario/Figuras/png/boxplot_questionnaire_scene_blind.png}
    \caption{Boxplot of the questionaire score of the blind participants grouped by method.}
    \label{fig:boxplot_quest_blind_scene}
\end{figure}

The Table \ref{tab:questionnaire_average_group_blind} show the the average questionnaire score on each method. It also shows a disatisfaction with the haptic devices alone.


\begin{table}[!htb]
\centering
\caption{Guidance method questionnaire average score for the blind participants.}
\label{tab:questionnaire_average_group_blind}
\begin{tabular}{lrrrrr}
\toprule
{} & Audio & Haptic Belt & Virtual Cane & Mixture \\
Visual Condition &       &             &              &         \\
\midrule
Blind            &  0.86 &        0.59 &         0.53 &    0.82 \\
\bottomrule
\end{tabular}
\end{table}



The Figures \ref{fig:qqplot_sagat_avg_two_way} and \ref{fig:residplot_sagat_avg_two_way} shows the distribution and variance of the Table \ref{tab:sagat_table_blind}. These Figures shows that the data are normally distributed and that the methods have a similar variance.
The Table \ref{tab:blocanova_sagat_avg_two_way} shows the Anova test p-value of the SAGAT score of the "blind" sample. The p-values indicates that the method have influence on the questionnaire score. Meaning that the participants had differents level os satisfaction about each method.


\begin{table}[!htb]
\centering
\caption{Anova p-value for the questionnaire score on each method for blinded users.}
\label{tab:blocanova_questionnaire}
\begin{tabular}{lrrrrr}
\toprule
            Source &  Squared sum &  DOF & Squared average &      F & \begin{tabular}[c]{@{}l@{}}P-Value \\ $(F_{0} > F)$\end{tabular} \\
\midrule
   Between factors &        0.329 &    3 &           0.110 & 15.677 &                                                            0.001 \\
    Between blocks &        0.042 &    3 &           0.014 &  2.014 &                                                            0.183 \\
Experimental error &        0.063 &    9 &           0.007 &        &                                                                  \\
             Total &        0.434 &   15 &                 &        &                                                                  \\
\bottomrule
\end{tabular}
\end{table}



\begin{figure}[!htb]
    \centering
    \begin{minipage}{0.45\textwidth}
        \centering
        \includegraphics[width = 0.8\linewidth]{Resultados/Questionario/Figuras/png/qqplot_questionnaires.png}
        \caption{QQ plot of the questionnaire score of the blind participants on each method.}
        \label{fig:qqplot_sagat_avg_two_way}
    \end{minipage}
    \begin{minipage}{0.45\textwidth}
        \centering
        \includegraphics[width = 0.8\linewidth]{Resultados/Questionario/Figuras/png/residplot_questionnaires.png}
        \caption{Residual plot of the questionnaire score the blind participants on each method.}
        \label{fig:residplot_questionnaires}
    \end{minipage}
\end{figure}


The Table \ref{tab:lsd_questionnaire} presents the conclusion of a pairwise Fisher LSD test of the blind NASA-TLX score between all the guidance methods. The results show that only the "Audio" and "Mixture" have the same statistically result and that there is a difference between the both "Haptic Belt" and "Virtual Cane".


\begin{table}[!htb]
\centering
\caption{Cross validation p-value for the questionnaire score on each method for blinded users.}
\label{tab:lsd_questionnaire}
\begin{tabular}{lr}
\toprule
                         Method &                                           Analysis \\
\midrule
       Audio versus Haptic Belt &        $H_1 : \mu_{Audio} \ne \mu_{Haptic Belt}**$ \\
      Audio versus Virtual Cane &       $H_1 : \mu_{Audio} \ne \mu_{Virtual Cane}**$ \\
           Audio versus Mixture &                $H_0 : \mu_{Audio} = \mu_{Mixture}$ \\
Haptic Belt versus Virtual Cane & $H_1 : \mu_{Haptic Belt} \ne \mu_{Virtual Cane}**$ \\
     Haptic Belt versus Mixture &          $H_0 : \mu_{Haptic Belt} = \mu_{Mixture}$ \\
    Virtual Cane versus Mixture &     $H_1 : \mu_{Virtual Cane} \ne \mu_{Mixture}**$ \\
\bottomrule
\end{tabular}
\end{table}



The LSD Table \ref{tab:lsd_questionnaire} confirms the information of the Figure \ref{fig:boxplot_quest_blind_scene} that the “Audio” and the ”Mixture” methods were the most favorite by the blind participants, whilst the “Haptic Belt” and “Virtual Cane” were the most unfavorite devices. The participants did comment about those two last devices, saying that they were not precise enough, confusing and very different from what they are used to use.

\FloatBarrier
\subsection{Physiological data}

The same sensors used for the first objective are used for the second objective. The expectations for all of the results is a difference between the “blind” sample and the “sight” sample. This subsection was divided in the same way as before:

\begin{itemize}
    \item \nameref{subsubsec:results_ecg_2};
    
        Two features are extracted from the ECG, heartrate (BPM) and heartrate variance (SDNN).
    
        Is expected that the heartrate increases at every “First” round and then a slight decrease in the next round. The heartrate variance is expected to decrease in the “First” round and a slight increase in the next round.    

    \item \nameref{subsubsec:results_gsr_temp_2};
    
        Is expected that the GSR average to increase at every “First” round and then a slight decrease in the next round.

\end{itemize}
\subsubsection{Electrocardiogram (ECG) data}
\label{subsubsec:results_ecg_2}

\paragraph{Analysis of the heartbeat frequency (BPM)}\mbox{}\\

The Table \ref{tab:bpm_table_noBase} presents the average heart rate by each participant on each scenes and they are plotted in the Figures \ref{fig:barplot_ecg_bpm_4_scene_blind} to \ref{fig:barplot_ecg_bpm_4_scene}.

\input{Resultados/ECG/Tabelas/bpm_table_noBase.tex}

The Figures \ref{fig:barplot_ecg_bpm_4_scene_blind} show and increase between all methods, whilst the Figure \ref{fig:barplot_ecg_bpm_4_scene_sight} also presents a decrease in two methods. The Figure \ref{fig:barplot_ecg_bpm_4_scene} shows that in most methods, the average BPM of the sight users was slight higher than the blind users.

\begin{figure}[!htb]
    \centering
    \begin{minipage}{\textwidth}
        \centering
        \includegraphics[width = 0.8\linewidth]{Resultados/ECG/Figuras/png/barplot_ecg_bpm_4_scene_blind.png}
        \caption{Barplot of the average BPM of the blind participants on each method and round.}
        \label{fig:barplot_ecg_bpm_4_scene_blind}
    \end{minipage}
    \begin{minipage}{\textwidth}
        \centering
        \includegraphics[width = 0.8\linewidth]{Resultados/ECG/Figuras/png/barplot_ecg_bpm_4_scene_sight.png}
        \caption{Barplot of the average BPM of the sight participants on each method and round.}
        \label{fig:barplot_ecg_bpm_4_scene_sight}
    \end{minipage}
\end{figure}
\begin{figure}[!htb]
    \centering
    \includegraphics[width = 0.8\linewidth]{Resultados/ECG/Figuras/png/barplot_ecg_bpm_4_scene.png}
    \caption{Barplot of the average BPM of both participants on each method.}
    \label{fig:barplot_ecg_bpm_4_scene}
\end{figure}

The Figure \ref{fig:boxplot_ecg_bpm_4_scene} and \ref{fig:boxplot_ecg_bpm_4_rounds} presents a box plot with the average BPM of both groups by method and rounds in that order. These figures show the reaction of the sight user was very different from the blind users, in all methods and rounds.

\begin{figure}[!htb]
    \centering
    \begin{minipage}{0.45\textwidth}
        \centering
        \includegraphics[width = 0.8\linewidth]{Resultados/ECG/Figuras/png/boxplot_ecg_bpm_4_scene.png}
        \caption{Boxplot of the average BPM of the participants grouped by method.}
        \label{fig:boxplot_ecg_bpm_4_scene}
    \end{minipage}
    \begin{minipage}{0.45\textwidth}
        \centering
        \includegraphics[width = 0.8\linewidth]{Resultados/ECG/Figuras/png/boxplot_ecg_bpm_4_rounds.png}
        \caption{Boxplot of the average BPM of the participants grouped by round.}
        \label{fig:boxplot_ecg_bpm_4_rounds}
    \end{minipage}
\end{figure}
 
The Table \ref{tab:bpm_average_group_noBase} shows the average heartrate of both samples and is possible to notice how the average score by the blind users was higher in every method, apart of the "Haptic Belt".

\input{Resultados/ECG/Tabelas/bpm_average_group_noBase.tex}

The Figures \ref{fig:qqplot_bpm_two_way_sight} and \ref{fig:residplot_bpm_two_way_sight} shows the distribution and variance of sighted participants of the Table \ref{tab:bpm_table_noBase}. These Figures shows that the data are normally distributed and that the methods have a similar variance.
The Table \ref{tab:blocanova_bpm_two_way_sight} shows the ANOVA test p-values of the average heartrate of the "sight" sample between the guidance methods and they show that the methods had an effect on the score.

\input{Resultados/ECG/Tabelas/blocanova_bpm_two_way_sight.tex}

\begin{figure}[!htb]
    \centering
    \begin{minipage}{0.45\textwidth}
        \centering
        \includegraphics[width = 0.8\linewidth]{Resultados/ECG/Figuras/png/qqplot_bpm_two_way_sight.png}
        \caption{QQ plot of the BPM of the sight participants on each method.}
        \label{fig:qqplot_bpm_two_way_sight}
    \end{minipage}
    \begin{minipage}{0.45\textwidth}
        \centering
        \includegraphics[width = 0.8\linewidth]{Resultados/ECG/Figuras/png/residplot_bpm_two_way_sight.png}
        \caption{Residual plot of the BPM score the sight participants on each method.}
        \label{fig:residplot_bpm_two_way_sight}
    \end{minipage}
\end{figure}

%\input{Resultados/ECG/Tabelas/lsd_bpm_two_way_sight.tex}

%The Table \ref{tab:lsd_bpm_two_way_sight} presents the conclusion of a pairwise Fisher LSD test of the blind heart rate frequency variation between all the guidance methods and it shows that all methods had different effect on the heartrate.

According to the ANOVA test at Table \ref{tab:blocanova_bpm_two_way_sight} there was no effect of the methods neither the rounds or their interaction, despite the fact that the Figure \ref{fig:boxplot_ecg_bpm_4_scene} showed a big difference between the methods. So the methods did not influence the sighted user mental workload. The same conclusion was driven in the section \ref{subsubsec:results_ecg_1} in the BPM part.

\FloatBarrier

%%%%%%%%%%%%%%%%%%%%%%%%%%%%%%%%%%%%%%%%%%%%%%%%%%%%%%%%%%%%%%%%%%%%%%%%%%%%
%%%%%%%%%%%%%%%%%%%%%%%%%%%%%%%%%%%%%%%%%%%%%%%%%%%%%%%%%%%%%%%%%%%%%%%%%%%%
%%%%%%%%%%%%%%%%%%%%%%%%%%%%%%%%%%%%%%%%%%%%%%%%%%%%%%%%%%%%%%%%%%%%%%%%%%%%
%%%%%%%%%%%%%%%%%%%%%%%%%%%%%%%%%%%%%%%%%%%%%%%%%%%%%%%%%%%%%%%%%%%%%%%%%%%%
%
%
\paragraph{Analysis of the heartbeat variance (SDNN)}\mbox{}\\
%
The Table \ref{tab:sdnn_table_noBase} presents the average heartbeat variance by each participant on each scenes and they are plotted in the Figures \ref{fig:barplot_ecg_sdnn_4_scene_blind} to \ref{fig:barplot_ecg_sdnn_4_scene}.

\input{Resultados/ECG/Tabelas/sdnn_table_noBase.tex}

The Figures \ref{fig:barplot_ecg_sdnn_4_scene_blind} and \ref{fig:barplot_ecg_sdnn_4_scene_sight} do not show a pattern. Some methods caused a decrease while other caused an increase on the SDNN. The Figure \ref{fig:barplot_ecg_sdnn_4_scene} that in all methods the SDNN of the sight users were higher than the blind users. That means that the sight users had a lower mental workload than the blind users.

\begin{figure}[!htb]
    \centering
    \begin{minipage}{\textwidth}
        \centering
        \includegraphics[width = 0.8\linewidth]{Resultados/ECG/Figuras/png/barplot_ecg_sdnn_4_scene_blind.png}
        \caption{Barplot of the average SDNN of the blind participants on each method and round.}
        \label{fig:barplot_ecg_sdnn_4_scene_blind}
    \end{minipage}
    \begin{minipage}{\textwidth}
        \centering
        \includegraphics[width = 0.8\linewidth]{Resultados/ECG/Figuras/png/barplot_ecg_sdnn_4_scene_sight.png}
        \caption{Barplot of the average SDNN of the sight participants on each method and round.}
        \label{fig:barplot_ecg_sdnn_4_scene_sight}
    \end{minipage}
\end{figure}
\begin{figure}[!htb]
    \centering
    \includegraphics[width = 0.8\linewidth]{Resultados/ECG/Figuras/png/barplot_ecg_sdnn_4_scene.png}
    \caption{Barplot of the average SDNN of both participants on each method.}
    \label{fig:barplot_ecg_sdnn_4_scene}
\end{figure}

The Figure \ref{fig:boxplot_ecg_sdnn_4_scene} and \ref{fig:boxplot_ecg_sdnn_4_rounds} presents a box plot with the average SDNN of both groups by method and rounds in that order. These figures show the reaction of the sight user was a little different from the blind users. In all methods and rounds the SDNN from the sight users appear to be higher, hence with lower mental workload.

\begin{figure}[!htb]
    \centering
    \begin{minipage}{0.45\textwidth}
        \centering
        \includegraphics[width = 0.8\linewidth]{Resultados/ECG/Figuras/png/boxplot_ecg_sdnn_4_scene.png}
        \caption{Boxplot of the average SDNN of the participants grouped by method.}
        \label{fig:boxplot_ecg_sdnn_4_scene}
    \end{minipage}
    \begin{minipage}{0.45\textwidth}
        \centering
        \includegraphics[width = 0.8\linewidth]{Resultados/ECG/Figuras/png/boxplot_ecg_sdnn_4_rounds.png}
        \caption{Boxplot of the average SDNN of the participants grouped by round.}
        \label{fig:boxplot_ecg_sdnn_4_rounds}
    \end{minipage}
\end{figure}
 
The Table \ref{tab:sdnn_average_group_noBase} shows the average heartbeat variance of both samples and is possible to notice how the average score by the sight users was higher in every method, reinforcing the Figures \ref{fig:barplot_ecg_sdnn_4_scene} to \ref{fig:boxplot_ecg_sdnn_4_rounds} conclusions.

\input{Resultados/ECG/Tabelas/sdnn_average_group_noBase.tex}

The Figures \ref{fig:qqplot_sdnn_two_way_sight} and \ref{fig:residplot_sdnn_two_way_sight} shows the distribution and variance of sighted participants of the Table \ref{tab:sdnn_table_noBase}. These Figures shows that the data are normally distributed and that the methods have a similar variance.
The Table \ref{tab:blocanova_sdnn_two_way_sight} shows the ANOVA test p-values of the average heartbeat variance of the "sight" sample between the guidance methods and they show that the methods had an effect on the score.

\input{Resultados/ECG/Tabelas/blocanova_sdnn_two_way_sight.tex}

\begin{figure}[!htb]
    \centering
    \begin{minipage}{0.45\textwidth}
        \centering
        \includegraphics[width = 0.8\linewidth]{Resultados/ECG/Figuras/png/qqplot_sdnn_two_way_sight.png}
        \caption{QQ plot of the average SDNN of the sight participants on each method.}
        \label{fig:qqplot_sdnn_two_way_sight}
    \end{minipage}
    \begin{minipage}{0.45\textwidth}
        \centering
        \includegraphics[width = 0.8\linewidth]{Resultados/ECG/Figuras/png/residplot_sdnn_two_way_sight.png}
        \caption{Residual plot of the average SDNN score the sight participants on each method.}
        \label{fig:residplot_sdnn_two_way_sight}
    \end{minipage}
\end{figure}


%\input{Resultados/ECG/Tabelas/lsd_sdnn_two_way_sight.tex}
%
%The Table \ref{tab:lsd_sdnn_two_way_sight} presents the conclusion of a pairwise Fisher LSD test of the blind heart rate frequency variation between all the guidance methods and it shows that all methods had different effect on the heartrate, appart of the "Virtual Cane" and "Mixture", which presented similar SDNN.

According to the ANOVA test at Table \ref{tab:blocanova_sdnn_two_way_sight} there was no effect on the interbeat variance. So the methods did not influence the sighted user mental workload. The same conclusion was driven in the section \ref{subsubsec:results_ecg_1} in the SDNN part.

Also, the sight user had a higher SDNN, which means lower Mental Workload, a unexpected result based on the expectation and on the previous notes. 

\FloatBarrier

\subsubsection{Galvanic skin response and temperature data;}
\label{subsubsec:results_gsr_temp_2}

Table \ref{tab:gsr_table_noBase} presents the average skin conductance for both groups, while the percentual variation related to the baseline is presented in Table \ref{tab:gsr_var_blind}.


\begin{table}[!htb]
\centering
\caption{Average GSR felled by the participants [$\mu$S].}
\label{tab:gsr_table_noBase}
\begin{tabular}{lllrrrrrr}
\toprule
    &       &        & Baseline &  Audio & \begin{tabular}[c]{@{}l@{}}Haptic\\ Belt\end{tabular} & \begin{tabular}[c]{@{}l@{}}Virtual\\ Cane\end{tabular} & Mixture \\
Participant & Visual Condition & Round &          &        &                                                       &                                                        &         \\
\midrule
001C & Blind & First &     0.37 &   1.03 &                                                  3.14 &                                                   3.79 &    3.90 \\
    &       & Return &          &   1.58 &                                                  2.81 &                                                   4.04 &    4.57 \\
003C & Blind & First &     0.30 &   0.56 &                                                  0.62 &                                                   0.85 &    1.09 \\
    &       & Return &          &   0.63 &                                                  0.65 &                                                   0.92 &    1.06 \\
004C & Blind & First &     1.24 &   3.07 &                                                  3.49 &                                                   2.28 &    2.23 \\
    &       & Return &          &   2.95 &                                                  3.20 &                                                   2.21 &    2.24 \\
001 & Sight & First &     4.27 &  15.19 &                                                 15.67 &                                                  15.19 &   14.15 \\
    &       & Return &          &  14.95 &                                                 15.09 &                                                  15.72 &   21.52 \\
004 & Sight & First &     2.60 &  11.18 &                                                 12.60 &                                                  12.92 &   10.34 \\
    &       & Return &          &  11.97 &                                                 12.25 &                                                  13.47 &   10.16 \\
005 & Sight & First &     0.47 &   1.58 &                                                  1.44 &                                                   1.37 &    1.33 \\
    &       & Return &          &   1.53 &                                                  1.47 &                                                   1.49 &    1.33 \\
\bottomrule
\end{tabular}
\end{table}




\begin{table}[!htb]
\centering
\caption{Average GSR variation in relation to the baseline in each round [$\mu$S].}
\label{tab:gsr_var_noBase}
\begin{tabular}{lllrrrrrr}
\toprule
    &       &        &     Audio & \begin{tabular}[c]{@{}l@{}}Haptic\\ Belt\end{tabular} & \begin{tabular}[c]{@{}l@{}}Virtual\\ Cane\end{tabular} &    Mixture \\
Participant & Visual Condition & Round &           &                                                       &                                                        &            \\
\midrule
001C & Blind & First &  176.54\% &                                              746.10\% &                                               920.72\% &   951.71\% \\
    &       & Return &  327.42\% &                                              656.99\% &                                               988.93\% &  1132.39\% \\
003C & Blind & First &   84.23\% &                                              104.19\% &                                               182.35\% &   258.80\% \\
    &       & Return &  109.23\% &                                              112.95\% &                                               202.35\% &   249.72\% \\
004C & Blind & First &  148.53\% &                                              182.84\% &                                                84.33\% &    80.69\% \\
    &       & Return &  138.64\% &                                              159.00\% &                                                78.73\% &    81.61\% \\
001 & Sight & First &  255.76\% &                                              266.93\% &                                               255.69\% &   231.52\% \\
    &       & Return &  250.18\% &                                              253.32\% &                                               268.25\% &   403.90\% \\
004 & Sight & First &  329.08\% &                                              383.54\% &                                               395.83\% &   297.05\% \\
    &       & Return &  359.53\% &                                              370.35\% &                                               417.17\% &   289.96\% \\
005 & Sight & First &  239.16\% &                                              207.74\% &                                               193.85\% &   184.71\% \\
    &       & Return &  227.06\% &                                              214.91\% &                                               219.59\% &   185.86\% \\
\bottomrule
\end{tabular}
\end{table}



The barplots of the two groups are presented in Figure \ref{fig:barplot_gsr_avg_4_scene_blind_sight}. If the variation between the round and the Baseline is positive, it means that the user had an increase on his/her Mental Workload or stress. While the GSR varied for the blind participants, increasing for methods with vibration, the same does not happen for sighted participants. Also, the variance of GSR data for blind participants is significantly higher than that of sighted ones. The same conclusion can be drawn from the boxplots in Figures \ref{fig:boxplot_ecg_sdnn_4_scene} and \ref{fig:boxplot_ecg_sdnn_4_rounds}. 

\begin{figure}[!htb]
    \centering
    \begin{minipage}{\textwidth}
        \centering
        \includegraphics[width = \textwidth]{Resultados/GSR/Figuras/pdf/barplot_gsr_avg_4_scene_blind.pdf}
        \subcaption{Blind participants.}
        \label{fig:barplot_gsr_avg_4_scene_blind}
    \end{minipage}
    \begin{minipage}{\textwidth}
        \centering
        \includegraphics[width = \textwidth]{Resultados/GSR/Figuras/pdf/barplot_gsr_avg_4_scene_sight.pdf}
        \subcaption{Sight participants.}
        \label{fig:barplot_gsr_avg_4_scene_sight}
    \end{minipage}
    \caption{Barplot of the average GSR on each method and round.}
    \label{fig:barplot_gsr_avg_4_scene_blind_sight}
\end{figure}

\begin{figure}[!htb]
    \centering
    \begin{minipage}{0.45\textwidth}
        \centering
        \includegraphics[width = \textwidth]{Resultados/GSR/Figuras/pdf/boxplot_gsr_avg_4_scene.pdf}
        \caption{Boxplot of the average GSR of the participants grouped by method.}
        \label{fig:boxplot_gsr_avg_4_scene}
    \end{minipage}
    \begin{minipage}{0.075\textwidth}
        \hfill
    \end{minipage}
    \begin{minipage}{0.45\textwidth}
        \centering
        \includegraphics[width = \textwidth]{Resultados/GSR/Figuras/pdf/boxplot_gsr_avg_4_rounds.pdf}
        \caption{Boxplot of the average GSR of the participants grouped by round.}
        \label{fig:boxplot_gsr_avg_4_rounds}
    \end{minipage}
\end{figure}

Figures \ref{fig:qqplot_gsr_two_way_sight} and \ref{fig:residplot_gsr_two_way_sight} bring the QQ Plot and residual distribution. The results from ANOVA are presented in Table \ref{tab:blocanova_gsr_two_way_blind_sight}. In the case of blind participants, the p-value for the method is just slightly over the threshold, indicating a possible influence of the method. The same does not happen with sighted participants, where the p-value of the method factor is the highest and well above the 0.05 threshold.
 
%The Table \ref{tab:gsr_average_group_noBase} shows the average skin conductance variation of both samples. It also shows that the presence of a haptic device increases the GSR, whilst the sight user had a basically constant GSR.
%
%
\begin{table}[!htb]
\centering
\caption{Average GSR variation grouped by participant and visual condition}
\label{tab:gsr_average_group_noBase}
\begin{tabular}{lrrrrr}
\toprule
{} &     Audio & \begin{tabular}[c]{@{}l@{}}Haptic\\ Belt\end{tabular} & \begin{tabular}[c]{@{}l@{}}Virtual\\ Cane\end{tabular} &   Mixture \\
Visual Condition &           &                                                       &                                                        &           \\
\midrule
Blind            &  164.10\% &                                              327.01\% &                                               409.57\% &  459.15\% \\
Sight            &  276.80\% &                                              282.80\% &                                               291.73\% &  265.50\% \\
\bottomrule
\end{tabular}
\end{table}



\begin{table}[!htb]
    \caption{Anova p-value for the skin conductance average on each method}
    \label{tab:blocanova_gsr_two_way_blind_sight}
\begin{minipage}{0.45\textwidth}
    \subcaption{Blind participants}
    \input{Resultados/GSR/Tabelas/blocanova_gsr_two_way_blindsemBegin.tex}
\end{minipage}
\begin{minipage}{0.45\textwidth}
    \subcaption{Sight participants}
    \input{Resultados/GSR/Tabelas/blocanova_gsr_two_way_sightsemBegin.tex}
\end{minipage}
\end{table}

\begin{figure}[!htb]
    \centering
    %\vspace{-15.0cm}
    \begin{minipage}{0.45\textwidth}
        \centering
        \includegraphics[width = \textwidth]{Resultados/GSR/Figuras/pdf/qqplot_gsr_two_way_sight.pdf}
        \caption{QQ plot of the average skin conductance of the sight participants on each method.}
        \label{fig:qqplot_gsr_two_way_sight}
    \end{minipage}
    \begin{minipage}{0.075\textwidth}
        \hfill
    \end{minipage}
    \begin{minipage}{0.45\textwidth}
        \centering
        \includegraphics[width = \textwidth]{Resultados/GSR/Figuras/pdf/residplot_gsr_two_way_sight.pdf}
        \caption{Residual plot of the average skin conductance score the sight participants on each method.}
        \label{fig:residplot_gsr_two_way_sight}
    \end{minipage}
\end{figure}

\FloatBarrier 

\subsection{Final Remarks}

% Mental demand
% • ANOVA - Audio e Mix     VS   não deu resultado
% • Grafico - Não tem uma relação específica VS  Presença de haptico aumenta a demanda mental

% NASA TLX
% • ANOVA - Round  VS  - Efeito do método e do round.
% •• Difere do ANOVA MD     Metodo - 3 grupos
% •••                       Base e Audio menores Carga mental
% ••• Sem padrão no tipo de dispositivo  Haptic alta Carga mental
% 
% SAGAT
% ANOVA - Método tem efeito     VS   Round tem efeito no SAGAT
% Nota menor nas mesmas condições
%

% BPM
% ANOVA - não deu resultado VS IDEM
% Gráfico mostra uma diferença entre os métodos
% Maior BPM que os cegos

% SDNN
% ANOVA - não deu resultado VS IDEM
% Gráfico - "Base" tem uma variância menor
% Valor de SDNN maior que o dos cegos

% GSR
% ANOVA - não deu resultado VS IDEM
% Grafico - métodos muito parecidos VS metodos com impactos diferentes

Differently than the blind users, the results from the mental demand discipline of the NASA-TLX proved that the sight users felt a higher mental demand than the blind users. 

The overall NASA-TLX score also proved a different conclusion than the one in the Section \ref{subsubsec:results_nasa_tlx_1}. For the sighted users, the round impacted more the overall score than the methods, whilst for the blind user were the opposite. This may be because the overall score is composed of 6 dimensions. Probably for the sighted user the mental demand score was higher and for the blind user it was not. But even so, the average score of the sight user was higher than the score of the blind user.

The Adapted SAGAT questionnaire for the sight users proved that the method impacts their situation awareness. The conclusion was different than the one proved by the blind users in the Section \ref{subsubsec:results_adapted_sagat_1}, who felt a bigger impact between the rounds than between the methods. The sight performance was also poorer than the blind user.

These conclusion show that the sighted users were more sensible to the methods than the blind users, although the effects were different. The blind users were more impacted by the methods than by the rounds, and when impacted by the methods, it was not possible to detect a pattern about the presence or not of a haptic device, as happened with the blind users.

The ECG sensors shown a difference in the heartrate between the methods, but the ANOVA test was not able to prove that difference, the same conclusion of the blind users in the Section \ref{subsubsec:results_ecg_1}. Another observation is that the heartrate frequency of the sighte user was higher than the blind users, meaning that their mental workload was probably higher.

According to the ANOVA test, The heartbeat variance also was not impact by the method or by the rounds, the same conclusion for the blind users in the Section \ref{subsubsec:results_ecg_1}. Graphically there was a small difference in the methods. Despite the results of the heartrate, the variance of the sight user was higher than the results from the blind user, meaning that the mental workload of the "sight" sample was higher than the one of "blind" sample.

The GSR ANOVA test also did not detect any impact from the methods or from the rounds, as it happened with the blind users in the Section \ref{subsubsec:results_gsr_temp_1}. Graphically the sight user variations were very similar in all methods. This is a different effect than the one observed in the blind users, which graphically showed different GSR distributions on different methods. The sight GSR also show a small variation between the rounds and methods, which means that the sight user were not stressed or had a low mental workload during the experiment.

Despite the proved and not proved tests, there is a consideration to be made. The sight sample group profiling. As already explained before, the profile of the “blind” sample group was very wide and that can impact negatively in their performance. But the opposite effect may had happened with the “sight” sample group. This group was composed basically by researchers and engineer students, people that are typically involved with computers and technological devices, aging from 22 to 31 with an average of 27.5 years. This may biased the results with better performance when using the HMD and being able to feel present inside a virtual environment.

Besides these results, the “sighted” sample also commented the experiment. They all felt a lot more insecurity when walking, exploring and even when hand guided by the researcher before the start of the round. The “blind” sample group was already used to bumping their body when exploring new closed quarters. The “sighted” group did not want that to happen and approached the furniture with a lot more caution. They also noticed the lack of precision of the haptic devices, but they did rely more on then to navigate.

%The processing of each data collected is rather similar and follows these steps:
%\begin{enumerate}
%    \item Separate the Blind sample and the Sight sample;
%    \item Check if the samples are normally distributed; \label{itm:results_shapiro} \\
%        If the data is normally distributed then it is possible to use other statistical analyses and verify the results statistically.
%    \item Check if the "blind" sample is statistically different then the "sight" sample; \label{itm:results_t_test} \\ 
%        This is one of the goals. To verify that the workload and the situation awareness of the blind participants are different from the sighted participants.
%    \item For the physiological data:
%    \begin{enumerate}
%        \item Calculate the variation between the scene and the baseline  in each method;
%        \item Calculate the variation between the scenes in each method;
%        \item Calculate the average variation in the group in each method.
%    \end{enumerate}
%    \item For the other data:
%    \begin{enumerate}
%        \item Calculate the variation between the scenes in each method; \label{itm:results_average_method_particpant}
%        \item Calculate the average variation in the group in each method.
%        \label{itm:results_average_method}
%    \end{enumerate}
%        The variation mentioned above is calculated as Equation \ref{eq:variation}:
%    \begin{equation}
%        \label{eq:variation}
%        Var_{ij} = \frac{Obs_j - Obs_i}{Obs_i}*100 [\%]
%    \end{equation}
%        Where:\\
%        $Var_{ij}$ = Variation between the two sequential observations \\
%        $Obs_i$ = First observation \\
%        $Obs_j$ = Second observation 
%    
%    \item Check if there are methods that are statistically different from the rest. \\
%        This is done by doing an analyses of variance (ANOVA) in the data. For this analyses is required that the observerd residues are normally distributed and the variance is constant. The residues are all plotted in the figures of the Appendix \ref{ap:figures}. The variance constancy can be verified by the box plot presented in the next sections.
%
%\end{enumerate}