Table \ref{tab:bpm_table_blind}  presents the heart rate of each blind participant for each guidance method. It is possible to observe that there is no systematic difference among the methods. Also, there are significant differences among the participants, which some of them presenting values significantly lower than others.


\begin{table}[!htb]
\centering
\caption{Average BPM felled by the blinded participants [BPM].}
\label{tab:bpm_table_blind}
\begin{tabular}{llrrrrr}
\toprule
     &        &   Base &  Audio & \begin{tabular}[c]{@{}l@{}}Haptic\\ Belt\end{tabular} & \begin{tabular}[c]{@{}l@{}}Virtual\\ Cane\end{tabular} & Mixture \\
Participant & Round &        &        &                                                       &                                                        &         \\
\midrule
001C & First &  75.75 &  60.71 &                                                 71.17 &                                                  59.07 &   68.24 \\
     & Return &  71.05 &  58.61 &                                                 66.22 &                                                  64.20 &   70.76 \\
002C & First &  48.69 &  38.67 &                                                 48.74 &                                                  46.89 &   52.23 \\
     & Return &  52.46 &  47.58 &                                                 58.97 &                                                  56.75 &   58.25 \\
003C & First &  68.37 &  69.89 &                                                 70.95 &                                                  69.41 &   66.94 \\
     & Return &  67.34 &  67.44 &                                                 69.68 &                                                  68.82 &   67.37 \\
004C & First &  75.09 &  73.55 &                                                 73.70 &                                                  71.94 &   74.03 \\
     & Return &  74.74 &  74.79 &                                                 74.02 &                                                  72.69 &   67.34 \\
\bottomrule
\end{tabular}
\end{table}



Figure \ref{fig:barplot_ecg_bpm_5_scene_blind} presents the mean heart rate. It shows a slight increase in the heartrate between the rounds, with the exception of the ‘base’ method, indicating that the participants felt the ‘return’ round more demandful.

\begin{figure}[!htb]
    \centering
    \includegraphics[width = 0.8\linewidth]{Resultados/ECG/Figuras/png/barplot_ecg_bpm_5_scene_blind.png}
    \caption{Barplot of the average BPM of the blind participants on each method.}
    \label{fig:barplot_ecg_bpm_5_scene_blind}
\end{figure}

%The Table \ref{tab:bpm_average_group_blind} show the average heartbeat frequency variation between the rounds of each group. As it was shown in the Figure \ref{fig:barplot_ecg_bpm_5_scene_blind}, only the "Base" method has a negative average variaton between the rounds. It is also posible to see that the Virtual Cane variation was the highest, hence it was also the highest mental workload.
% 
%
\begin{table}[!htb]
\centering
\caption{ECG average BPM  for each method of the blind participants.}
\label{tab:bpm_average_group_blind}
\begin{tabular}{lrrrrr}
\toprule
{} &   Base & Audio & Haptic Belt & Virtual Cane & Mixture \\
Visual Condition &        &       &             &              &         \\
\midrule
Blind            &  -0.58 &  1.40 &        1.09 &         3.79 &    0.57 \\
\bottomrule
\end{tabular}
\end{table}



Figures \ref{fig:boxplot_ecg_bpm_blind_scene} and \ref{fig:boxplot_ecg_bpm_blind_rounds} brings the corresponding boxplot, grouped by method and round. In both cases, it is not possible to observe significant differences among the methods or rounds.

\begin{figure}[!htb]
    \centering
    \begin{minipage}{0.45\textwidth}
        \centering
        \includegraphics[width = 0.8\linewidth]{Resultados/ECG/Figuras/png/boxplot_ecg_bpm_blind_scene.png}
        \caption{Boxplot of the BPM of the blind participants grouped by method.}
        \label{fig:boxplot_ecg_bpm_blind_scene}
    \end{minipage}
    \begin{minipage}{0.45\textwidth}
        \centering
        \includegraphics[width = 0.8\linewidth]{Resultados/ECG/Figuras/png/boxplot_ecg_bpm_blind_rounds.png}
        \caption{Boxplot of the BPM of the blind participants grouped by round.}
        \label{fig:boxplot_ecg_bpm_blind_rounds}
    \end{minipage}
\end{figure}

The Figures  and  shows the distribution and variance of the Table . These Figures shows that the data are normally distributed but the participants had different  that the methods have a similar variance.

Figures \ref{fig:qqplot_bpm_two_way} and \ref{fig:residplot_bpm_two_way} bring the QQ Plot and residual distribution. Particularly, the last one shows that the participant does not have similar variance, which may jeopardize the results of ANOVA. Considering this limitation, Table \ref{tab:bpm_table_blind} brings the p-value obtained by ANOVA, which confirmed the previous analysis, as it does not indicate a significant influence of either the guidance method or the round in the participants heartrate. 

\begin{figure}[!htb]
    \centering
    \begin{minipage}{0.45\textwidth}
        \centering
        \includegraphics[width = 0.8\linewidth]{Resultados/ECG/Figuras/png/qqplot_bpm_two_way_blind.png}
        \caption{QQ plot of the BPM of the blind participants on each method.}
        \label{fig:qqplot_bpm_two_way}
    \end{minipage}
    \begin{minipage}{0.45\textwidth}
        \centering
        \includegraphics[width = 0.8\linewidth]{Resultados/ECG/Figuras/png/residplot_bpm_two_way_blind.png}
        \caption{Residual plot of the BPM score the blind participants on each method.}
        \label{fig:residplot_bpm_two_way}
    \end{minipage}
\end{figure}


\begin{table}[!htb]
\centering
\caption{Anova p-value for the BPM on each method for blinded users.}
\label{tab:blocanova_bpm_two_way_blind}
\begin{tabular}{lrrrrr}
\toprule
          Source & P-Value \\
\midrule
    \    Methods &   0.100 \\
     \    Rounds &   0.371 \\
\    Interaction &   0.894 \\
\bottomrule
\end{tabular}
\end{table}



%\input{Resultados/ECG/Tabelas/lsd_bpm_two_way.tex}

%The Table \ref{tab:lsd_bpm_two_way} presents the conclusion of a pairwise Fisher LSD test of the blind heart rate frequency variation between all the guidance methods. The results show that the only the "Base" and "Haptic belt" have simila reaction.

\FloatBarrier