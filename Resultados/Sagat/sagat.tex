\subsection{Adapted SAGAT}
\label{subsec:results_adapted_sagat}

In this subsection, the Sagat questionnaire is analyzed. Its result may give an idea of the mental map the participant is drawing. For each question a participant could score 1 point or a fraction of it. The total score of each participant is presented on the Table \ref{tab:sagat} and they are plotted in the Figures \ref{fig:barplot_sagat_scene_blind} and \ref{fig:barplot_sagat_scene_sight}. It is visually noticeable that both of the groups perform better the second time they visit the room.


\begin{table}[!htb]
\centering
\caption{Adapted Sagat global score by participant and guidance method.}
\label{tab:sagat}
\begin{tabular}{lllrrrrr}
\toprule
    &       &        &  Base & Audio & \begin{tabular}[c]{@{}l@{}}Haptic\\ Belt\end{tabular} & \begin{tabular}[c]{@{}l@{}}Virtual\\ Cane\end{tabular} & Mixture \\
Participant & \begin{tabular}[c]{@{}l@{}}Visual\\ Condition\end{tabular} & Round &       &       &                                                       &                                                        &         \\
\midrule
001 & Sight & First &  1.00 &  0.45 &                                                  0.43 &                                                   0.27 &   0.650 \\
    &       & Return &  1.00 &  0.60 &                                                  0.50 &                                                   0.50 &   0.450 \\
001C & Blind & First &  0.62 &  0.55 &                                                  0.53 &                                                   0.58 &   0.350 \\
    &       & Return &  0.62 &  0.65 &                                                  0.85 &                                                   0.55 &   0.550 \\
002C & Blind & First &  0.68 &  0.45 &                                                  0.40 &                                                   0.45 &   0.625 \\
    &       & Return &  0.53 &  0.50 &                                                  0.40 &                                                   0.65 &   0.850 \\
003 & Sight & First &  1.00 &  0.68 &                                                  0.60 &                                                   0.40 &   0.675 \\
    &       & Return &  1.00 &  0.60 &                                                  0.72 &                                                   0.62 &   0.750 \\
003C & Blind & First &  0.72 &  0.75 &                                                  0.75 &                                                   0.47 &   0.900 \\
    &       & Return &  1.00 &  1.00 &                                                  0.85 &                                                   0.90 &   0.900 \\
004 & Sight & First &  1.00 &  0.72 &                                                  0.80 &                                                   0.60 &   0.825 \\
    &       & Return &  1.00 &  0.78 &                                                  0.95 &                                                   0.82 &   0.700 \\
004C & Blind & First &  0.75 &  0.60 &                                                  0.77 &                                                   0.50 &   0.650 \\
    &       & Return &  0.90 &  0.60 &                                                  0.93 &                                                   0.72 &   0.900 \\
005 & Sight & First &  1.00 &  0.30 &                                                  0.32 &                                                   0.40 &   0.400 \\
    &       & Return &  1.00 &  0.38 &                                                  0.30 &                                                   0.20 &   0.600 \\
\bottomrule
\end{tabular}
\end{table}



\begin{figure}[!htb]
    \centering
    \begin{minipage}{\textwidth}
        \centering
        \includegraphics[width = 0.8\linewidth]{Resultados/Sagat/Figuras/png/barplot_sagat_avg_scene_blind.png}
        %\resizebox{0.6\linewidth}{!}{
        %%% Creator: Matplotlib, PGF backend
%%
%% To include the figure in your LaTeX document, write
%%   \input{<filename>.pgf}
%%
%% Make sure the required packages are loaded in your preamble
%%   \usepackage{pgf}
%%
%% Figures using additional raster images can only be included by \input if
%% they are in the same directory as the main LaTeX file. For loading figures
%% from other directories you can use the `import` package
%%   \usepackage{import}
%%
%% and then include the figures with
%%   \import{<path to file>}{<filename>.pgf}
%%
%% Matplotlib used the following preamble
%%   \usepackage{fontspec}
%%
\begingroup%
\makeatletter%
\begin{pgfpicture}%
\pgfpathrectangle{\pgfpointorigin}{\pgfqpoint{15.190505in}{8.690562in}}%
\pgfusepath{use as bounding box, clip}%
\begin{pgfscope}%
\pgfsetbuttcap%
\pgfsetmiterjoin%
\pgfsetlinewidth{0.000000pt}%
\definecolor{currentstroke}{rgb}{1.000000,1.000000,1.000000}%
\pgfsetstrokecolor{currentstroke}%
\pgfsetstrokeopacity{0.000000}%
\pgfsetdash{}{0pt}%
\pgfpathmoveto{\pgfqpoint{0.000000in}{-0.000000in}}%
\pgfpathlineto{\pgfqpoint{15.190505in}{-0.000000in}}%
\pgfpathlineto{\pgfqpoint{15.190505in}{8.690562in}}%
\pgfpathlineto{\pgfqpoint{0.000000in}{8.690562in}}%
\pgfpathclose%
\pgfusepath{}%
\end{pgfscope}%
\begin{pgfscope}%
\pgfsetbuttcap%
\pgfsetmiterjoin%
\definecolor{currentfill}{rgb}{1.000000,1.000000,1.000000}%
\pgfsetfillcolor{currentfill}%
\pgfsetlinewidth{0.000000pt}%
\definecolor{currentstroke}{rgb}{0.000000,0.000000,0.000000}%
\pgfsetstrokecolor{currentstroke}%
\pgfsetstrokeopacity{0.000000}%
\pgfsetdash{}{0pt}%
\pgfpathmoveto{\pgfqpoint{1.140505in}{1.191562in}}%
\pgfpathlineto{\pgfqpoint{15.090505in}{1.191562in}}%
\pgfpathlineto{\pgfqpoint{15.090505in}{6.476562in}}%
\pgfpathlineto{\pgfqpoint{1.140505in}{6.476562in}}%
\pgfpathclose%
\pgfusepath{fill}%
\end{pgfscope}%
\begin{pgfscope}%
\pgfpathrectangle{\pgfqpoint{1.140505in}{1.191562in}}{\pgfqpoint{13.950000in}{5.285000in}}%
\pgfusepath{clip}%
\pgfsetbuttcap%
\pgfsetmiterjoin%
\definecolor{currentfill}{rgb}{0.651961,0.093137,0.093137}%
\pgfsetfillcolor{currentfill}%
\pgfsetlinewidth{0.000000pt}%
\definecolor{currentstroke}{rgb}{0.000000,0.000000,0.000000}%
\pgfsetstrokecolor{currentstroke}%
\pgfsetstrokeopacity{0.000000}%
\pgfsetdash{}{0pt}%
\pgfpathmoveto{\pgfqpoint{1.489255in}{1.191562in}}%
\pgfpathlineto{\pgfqpoint{2.884255in}{1.191562in}}%
\pgfpathlineto{\pgfqpoint{2.884255in}{4.454550in}}%
\pgfpathlineto{\pgfqpoint{1.489255in}{4.454550in}}%
\pgfpathclose%
\pgfusepath{fill}%
\end{pgfscope}%
\begin{pgfscope}%
\pgfpathrectangle{\pgfqpoint{1.140505in}{1.191562in}}{\pgfqpoint{13.950000in}{5.285000in}}%
\pgfusepath{clip}%
\pgfsetbuttcap%
\pgfsetmiterjoin%
\definecolor{currentfill}{rgb}{0.651961,0.093137,0.093137}%
\pgfsetfillcolor{currentfill}%
\pgfsetlinewidth{0.000000pt}%
\definecolor{currentstroke}{rgb}{0.000000,0.000000,0.000000}%
\pgfsetstrokecolor{currentstroke}%
\pgfsetstrokeopacity{0.000000}%
\pgfsetdash{}{0pt}%
\pgfpathmoveto{\pgfqpoint{4.976755in}{1.191562in}}%
\pgfpathlineto{\pgfqpoint{6.371755in}{1.191562in}}%
\pgfpathlineto{\pgfqpoint{6.371755in}{4.589235in}}%
\pgfpathlineto{\pgfqpoint{4.976755in}{4.589235in}}%
\pgfpathclose%
\pgfusepath{fill}%
\end{pgfscope}%
\begin{pgfscope}%
\pgfpathrectangle{\pgfqpoint{1.140505in}{1.191562in}}{\pgfqpoint{13.950000in}{5.285000in}}%
\pgfusepath{clip}%
\pgfsetbuttcap%
\pgfsetmiterjoin%
\definecolor{currentfill}{rgb}{0.651961,0.093137,0.093137}%
\pgfsetfillcolor{currentfill}%
\pgfsetlinewidth{0.000000pt}%
\definecolor{currentstroke}{rgb}{0.000000,0.000000,0.000000}%
\pgfsetstrokecolor{currentstroke}%
\pgfsetstrokeopacity{0.000000}%
\pgfsetdash{}{0pt}%
\pgfpathmoveto{\pgfqpoint{8.464255in}{1.191562in}}%
\pgfpathlineto{\pgfqpoint{9.859255in}{1.191562in}}%
\pgfpathlineto{\pgfqpoint{9.859255in}{3.965796in}}%
\pgfpathlineto{\pgfqpoint{8.464255in}{3.965796in}}%
\pgfpathclose%
\pgfusepath{fill}%
\end{pgfscope}%
\begin{pgfscope}%
\pgfpathrectangle{\pgfqpoint{1.140505in}{1.191562in}}{\pgfqpoint{13.950000in}{5.285000in}}%
\pgfusepath{clip}%
\pgfsetbuttcap%
\pgfsetmiterjoin%
\definecolor{currentfill}{rgb}{0.651961,0.093137,0.093137}%
\pgfsetfillcolor{currentfill}%
\pgfsetlinewidth{0.000000pt}%
\definecolor{currentstroke}{rgb}{0.000000,0.000000,0.000000}%
\pgfsetstrokecolor{currentstroke}%
\pgfsetstrokeopacity{0.000000}%
\pgfsetdash{}{0pt}%
\pgfpathmoveto{\pgfqpoint{11.951755in}{1.191562in}}%
\pgfpathlineto{\pgfqpoint{13.346755in}{1.191562in}}%
\pgfpathlineto{\pgfqpoint{13.346755in}{4.697539in}}%
\pgfpathlineto{\pgfqpoint{11.951755in}{4.697539in}}%
\pgfpathclose%
\pgfusepath{fill}%
\end{pgfscope}%
\begin{pgfscope}%
\pgfpathrectangle{\pgfqpoint{1.140505in}{1.191562in}}{\pgfqpoint{13.950000in}{5.285000in}}%
\pgfusepath{clip}%
\pgfsetbuttcap%
\pgfsetmiterjoin%
\definecolor{currentfill}{rgb}{0.144608,0.218137,0.424020}%
\pgfsetfillcolor{currentfill}%
\pgfsetlinewidth{0.000000pt}%
\definecolor{currentstroke}{rgb}{0.000000,0.000000,0.000000}%
\pgfsetstrokecolor{currentstroke}%
\pgfsetstrokeopacity{0.000000}%
\pgfsetdash{}{0pt}%
\pgfpathmoveto{\pgfqpoint{2.884255in}{1.191562in}}%
\pgfpathlineto{\pgfqpoint{4.279255in}{1.191562in}}%
\pgfpathlineto{\pgfqpoint{4.279255in}{5.009952in}}%
\pgfpathlineto{\pgfqpoint{2.884255in}{5.009952in}}%
\pgfpathclose%
\pgfusepath{fill}%
\end{pgfscope}%
\begin{pgfscope}%
\pgfpathrectangle{\pgfqpoint{1.140505in}{1.191562in}}{\pgfqpoint{13.950000in}{5.285000in}}%
\pgfusepath{clip}%
\pgfsetbuttcap%
\pgfsetmiterjoin%
\definecolor{currentfill}{rgb}{0.144608,0.218137,0.424020}%
\pgfsetfillcolor{currentfill}%
\pgfsetlinewidth{0.000000pt}%
\definecolor{currentstroke}{rgb}{0.000000,0.000000,0.000000}%
\pgfsetstrokecolor{currentstroke}%
\pgfsetstrokeopacity{0.000000}%
\pgfsetdash{}{0pt}%
\pgfpathmoveto{\pgfqpoint{6.371755in}{1.191562in}}%
\pgfpathlineto{\pgfqpoint{7.766755in}{1.191562in}}%
\pgfpathlineto{\pgfqpoint{7.766755in}{5.391792in}}%
\pgfpathlineto{\pgfqpoint{6.371755in}{5.391792in}}%
\pgfpathclose%
\pgfusepath{fill}%
\end{pgfscope}%
\begin{pgfscope}%
\pgfpathrectangle{\pgfqpoint{1.140505in}{1.191562in}}{\pgfqpoint{13.950000in}{5.285000in}}%
\pgfusepath{clip}%
\pgfsetbuttcap%
\pgfsetmiterjoin%
\definecolor{currentfill}{rgb}{0.144608,0.218137,0.424020}%
\pgfsetfillcolor{currentfill}%
\pgfsetlinewidth{0.000000pt}%
\definecolor{currentstroke}{rgb}{0.000000,0.000000,0.000000}%
\pgfsetstrokecolor{currentstroke}%
\pgfsetstrokeopacity{0.000000}%
\pgfsetdash{}{0pt}%
\pgfpathmoveto{\pgfqpoint{9.859255in}{1.191562in}}%
\pgfpathlineto{\pgfqpoint{11.254255in}{1.191562in}}%
\pgfpathlineto{\pgfqpoint{11.254255in}{5.114090in}}%
\pgfpathlineto{\pgfqpoint{9.859255in}{5.114090in}}%
\pgfpathclose%
\pgfusepath{fill}%
\end{pgfscope}%
\begin{pgfscope}%
\pgfpathrectangle{\pgfqpoint{1.140505in}{1.191562in}}{\pgfqpoint{13.950000in}{5.285000in}}%
\pgfusepath{clip}%
\pgfsetbuttcap%
\pgfsetmiterjoin%
\definecolor{currentfill}{rgb}{0.144608,0.218137,0.424020}%
\pgfsetfillcolor{currentfill}%
\pgfsetlinewidth{0.000000pt}%
\definecolor{currentstroke}{rgb}{0.000000,0.000000,0.000000}%
\pgfsetstrokecolor{currentstroke}%
\pgfsetstrokeopacity{0.000000}%
\pgfsetdash{}{0pt}%
\pgfpathmoveto{\pgfqpoint{13.346755in}{1.191562in}}%
\pgfpathlineto{\pgfqpoint{14.741755in}{1.191562in}}%
\pgfpathlineto{\pgfqpoint{14.741755in}{5.634780in}}%
\pgfpathlineto{\pgfqpoint{13.346755in}{5.634780in}}%
\pgfpathclose%
\pgfusepath{fill}%
\end{pgfscope}%
\begin{pgfscope}%
\pgfsetbuttcap%
\pgfsetroundjoin%
\definecolor{currentfill}{rgb}{0.000000,0.000000,0.000000}%
\pgfsetfillcolor{currentfill}%
\pgfsetlinewidth{0.803000pt}%
\definecolor{currentstroke}{rgb}{0.000000,0.000000,0.000000}%
\pgfsetstrokecolor{currentstroke}%
\pgfsetdash{}{0pt}%
\pgfsys@defobject{currentmarker}{\pgfqpoint{0.000000in}{-0.048611in}}{\pgfqpoint{0.000000in}{0.000000in}}{%
\pgfpathmoveto{\pgfqpoint{0.000000in}{0.000000in}}%
\pgfpathlineto{\pgfqpoint{0.000000in}{-0.048611in}}%
\pgfusepath{stroke,fill}%
}%
\begin{pgfscope}%
\pgfsys@transformshift{2.884255in}{1.191562in}%
\pgfsys@useobject{currentmarker}{}%
\end{pgfscope}%
\end{pgfscope}%
\begin{pgfscope}%
\definecolor{textcolor}{rgb}{0.000000,0.000000,0.000000}%
\pgfsetstrokecolor{textcolor}%
\pgfsetfillcolor{textcolor}%
\pgftext[x=2.884255in,y=1.094339in,,top]{\color{textcolor}\rmfamily\fontsize{38.016000}{45.619200}\selectfont Audio}%
\end{pgfscope}%
\begin{pgfscope}%
\pgfsetbuttcap%
\pgfsetroundjoin%
\definecolor{currentfill}{rgb}{0.000000,0.000000,0.000000}%
\pgfsetfillcolor{currentfill}%
\pgfsetlinewidth{0.803000pt}%
\definecolor{currentstroke}{rgb}{0.000000,0.000000,0.000000}%
\pgfsetstrokecolor{currentstroke}%
\pgfsetdash{}{0pt}%
\pgfsys@defobject{currentmarker}{\pgfqpoint{0.000000in}{-0.048611in}}{\pgfqpoint{0.000000in}{0.000000in}}{%
\pgfpathmoveto{\pgfqpoint{0.000000in}{0.000000in}}%
\pgfpathlineto{\pgfqpoint{0.000000in}{-0.048611in}}%
\pgfusepath{stroke,fill}%
}%
\begin{pgfscope}%
\pgfsys@transformshift{6.371755in}{1.191562in}%
\pgfsys@useobject{currentmarker}{}%
\end{pgfscope}%
\end{pgfscope}%
\begin{pgfscope}%
\definecolor{textcolor}{rgb}{0.000000,0.000000,0.000000}%
\pgfsetstrokecolor{textcolor}%
\pgfsetfillcolor{textcolor}%
\pgftext[x=6.371755in,y=1.094339in,,top]{\color{textcolor}\rmfamily\fontsize{38.016000}{45.619200}\selectfont Haptic Belt}%
\end{pgfscope}%
\begin{pgfscope}%
\pgfsetbuttcap%
\pgfsetroundjoin%
\definecolor{currentfill}{rgb}{0.000000,0.000000,0.000000}%
\pgfsetfillcolor{currentfill}%
\pgfsetlinewidth{0.803000pt}%
\definecolor{currentstroke}{rgb}{0.000000,0.000000,0.000000}%
\pgfsetstrokecolor{currentstroke}%
\pgfsetdash{}{0pt}%
\pgfsys@defobject{currentmarker}{\pgfqpoint{0.000000in}{-0.048611in}}{\pgfqpoint{0.000000in}{0.000000in}}{%
\pgfpathmoveto{\pgfqpoint{0.000000in}{0.000000in}}%
\pgfpathlineto{\pgfqpoint{0.000000in}{-0.048611in}}%
\pgfusepath{stroke,fill}%
}%
\begin{pgfscope}%
\pgfsys@transformshift{9.859255in}{1.191562in}%
\pgfsys@useobject{currentmarker}{}%
\end{pgfscope}%
\end{pgfscope}%
\begin{pgfscope}%
\definecolor{textcolor}{rgb}{0.000000,0.000000,0.000000}%
\pgfsetstrokecolor{textcolor}%
\pgfsetfillcolor{textcolor}%
\pgftext[x=9.859255in,y=1.094339in,,top]{\color{textcolor}\rmfamily\fontsize{38.016000}{45.619200}\selectfont Virtual Cane}%
\end{pgfscope}%
\begin{pgfscope}%
\pgfsetbuttcap%
\pgfsetroundjoin%
\definecolor{currentfill}{rgb}{0.000000,0.000000,0.000000}%
\pgfsetfillcolor{currentfill}%
\pgfsetlinewidth{0.803000pt}%
\definecolor{currentstroke}{rgb}{0.000000,0.000000,0.000000}%
\pgfsetstrokecolor{currentstroke}%
\pgfsetdash{}{0pt}%
\pgfsys@defobject{currentmarker}{\pgfqpoint{0.000000in}{-0.048611in}}{\pgfqpoint{0.000000in}{0.000000in}}{%
\pgfpathmoveto{\pgfqpoint{0.000000in}{0.000000in}}%
\pgfpathlineto{\pgfqpoint{0.000000in}{-0.048611in}}%
\pgfusepath{stroke,fill}%
}%
\begin{pgfscope}%
\pgfsys@transformshift{13.346755in}{1.191562in}%
\pgfsys@useobject{currentmarker}{}%
\end{pgfscope}%
\end{pgfscope}%
\begin{pgfscope}%
\definecolor{textcolor}{rgb}{0.000000,0.000000,0.000000}%
\pgfsetstrokecolor{textcolor}%
\pgfsetfillcolor{textcolor}%
\pgftext[x=13.346755in,y=1.094339in,,top]{\color{textcolor}\rmfamily\fontsize{38.016000}{45.619200}\selectfont Mixture}%
\end{pgfscope}%
\begin{pgfscope}%
\definecolor{textcolor}{rgb}{0.000000,0.000000,0.000000}%
\pgfsetstrokecolor{textcolor}%
\pgfsetfillcolor{textcolor}%
\pgftext[x=8.115505in,y=0.569392in,,top]{\color{textcolor}\rmfamily\fontsize{38.016000}{45.619200}\selectfont Scene}%
\end{pgfscope}%
\begin{pgfscope}%
\pgfsetbuttcap%
\pgfsetroundjoin%
\definecolor{currentfill}{rgb}{0.000000,0.000000,0.000000}%
\pgfsetfillcolor{currentfill}%
\pgfsetlinewidth{0.803000pt}%
\definecolor{currentstroke}{rgb}{0.000000,0.000000,0.000000}%
\pgfsetstrokecolor{currentstroke}%
\pgfsetdash{}{0pt}%
\pgfsys@defobject{currentmarker}{\pgfqpoint{-0.048611in}{0.000000in}}{\pgfqpoint{-0.000000in}{0.000000in}}{%
\pgfpathmoveto{\pgfqpoint{-0.000000in}{0.000000in}}%
\pgfpathlineto{\pgfqpoint{-0.048611in}{0.000000in}}%
\pgfusepath{stroke,fill}%
}%
\begin{pgfscope}%
\pgfsys@transformshift{1.140505in}{1.191562in}%
\pgfsys@useobject{currentmarker}{}%
\end{pgfscope}%
\end{pgfscope}%
\begin{pgfscope}%
\definecolor{textcolor}{rgb}{0.000000,0.000000,0.000000}%
\pgfsetstrokecolor{textcolor}%
\pgfsetfillcolor{textcolor}%
\pgftext[x=0.632340in, y=1.008346in, left, base]{\color{textcolor}\rmfamily\fontsize{38.016000}{45.619200}\selectfont \(\displaystyle {0.0}\)}%
\end{pgfscope}%
\begin{pgfscope}%
\pgfsetbuttcap%
\pgfsetroundjoin%
\definecolor{currentfill}{rgb}{0.000000,0.000000,0.000000}%
\pgfsetfillcolor{currentfill}%
\pgfsetlinewidth{0.803000pt}%
\definecolor{currentstroke}{rgb}{0.000000,0.000000,0.000000}%
\pgfsetstrokecolor{currentstroke}%
\pgfsetdash{}{0pt}%
\pgfsys@defobject{currentmarker}{\pgfqpoint{-0.048611in}{0.000000in}}{\pgfqpoint{-0.000000in}{0.000000in}}{%
\pgfpathmoveto{\pgfqpoint{-0.000000in}{0.000000in}}%
\pgfpathlineto{\pgfqpoint{-0.048611in}{0.000000in}}%
\pgfusepath{stroke,fill}%
}%
\begin{pgfscope}%
\pgfsys@transformshift{1.140505in}{2.302366in}%
\pgfsys@useobject{currentmarker}{}%
\end{pgfscope}%
\end{pgfscope}%
\begin{pgfscope}%
\definecolor{textcolor}{rgb}{0.000000,0.000000,0.000000}%
\pgfsetstrokecolor{textcolor}%
\pgfsetfillcolor{textcolor}%
\pgftext[x=0.632340in, y=2.119150in, left, base]{\color{textcolor}\rmfamily\fontsize{38.016000}{45.619200}\selectfont \(\displaystyle {0.2}\)}%
\end{pgfscope}%
\begin{pgfscope}%
\pgfsetbuttcap%
\pgfsetroundjoin%
\definecolor{currentfill}{rgb}{0.000000,0.000000,0.000000}%
\pgfsetfillcolor{currentfill}%
\pgfsetlinewidth{0.803000pt}%
\definecolor{currentstroke}{rgb}{0.000000,0.000000,0.000000}%
\pgfsetstrokecolor{currentstroke}%
\pgfsetdash{}{0pt}%
\pgfsys@defobject{currentmarker}{\pgfqpoint{-0.048611in}{0.000000in}}{\pgfqpoint{-0.000000in}{0.000000in}}{%
\pgfpathmoveto{\pgfqpoint{-0.000000in}{0.000000in}}%
\pgfpathlineto{\pgfqpoint{-0.048611in}{0.000000in}}%
\pgfusepath{stroke,fill}%
}%
\begin{pgfscope}%
\pgfsys@transformshift{1.140505in}{3.413171in}%
\pgfsys@useobject{currentmarker}{}%
\end{pgfscope}%
\end{pgfscope}%
\begin{pgfscope}%
\definecolor{textcolor}{rgb}{0.000000,0.000000,0.000000}%
\pgfsetstrokecolor{textcolor}%
\pgfsetfillcolor{textcolor}%
\pgftext[x=0.632340in, y=3.229955in, left, base]{\color{textcolor}\rmfamily\fontsize{38.016000}{45.619200}\selectfont \(\displaystyle {0.4}\)}%
\end{pgfscope}%
\begin{pgfscope}%
\pgfsetbuttcap%
\pgfsetroundjoin%
\definecolor{currentfill}{rgb}{0.000000,0.000000,0.000000}%
\pgfsetfillcolor{currentfill}%
\pgfsetlinewidth{0.803000pt}%
\definecolor{currentstroke}{rgb}{0.000000,0.000000,0.000000}%
\pgfsetstrokecolor{currentstroke}%
\pgfsetdash{}{0pt}%
\pgfsys@defobject{currentmarker}{\pgfqpoint{-0.048611in}{0.000000in}}{\pgfqpoint{-0.000000in}{0.000000in}}{%
\pgfpathmoveto{\pgfqpoint{-0.000000in}{0.000000in}}%
\pgfpathlineto{\pgfqpoint{-0.048611in}{0.000000in}}%
\pgfusepath{stroke,fill}%
}%
\begin{pgfscope}%
\pgfsys@transformshift{1.140505in}{4.523975in}%
\pgfsys@useobject{currentmarker}{}%
\end{pgfscope}%
\end{pgfscope}%
\begin{pgfscope}%
\definecolor{textcolor}{rgb}{0.000000,0.000000,0.000000}%
\pgfsetstrokecolor{textcolor}%
\pgfsetfillcolor{textcolor}%
\pgftext[x=0.632340in, y=4.340759in, left, base]{\color{textcolor}\rmfamily\fontsize{38.016000}{45.619200}\selectfont \(\displaystyle {0.6}\)}%
\end{pgfscope}%
\begin{pgfscope}%
\pgfsetbuttcap%
\pgfsetroundjoin%
\definecolor{currentfill}{rgb}{0.000000,0.000000,0.000000}%
\pgfsetfillcolor{currentfill}%
\pgfsetlinewidth{0.803000pt}%
\definecolor{currentstroke}{rgb}{0.000000,0.000000,0.000000}%
\pgfsetstrokecolor{currentstroke}%
\pgfsetdash{}{0pt}%
\pgfsys@defobject{currentmarker}{\pgfqpoint{-0.048611in}{0.000000in}}{\pgfqpoint{-0.000000in}{0.000000in}}{%
\pgfpathmoveto{\pgfqpoint{-0.000000in}{0.000000in}}%
\pgfpathlineto{\pgfqpoint{-0.048611in}{0.000000in}}%
\pgfusepath{stroke,fill}%
}%
\begin{pgfscope}%
\pgfsys@transformshift{1.140505in}{5.634780in}%
\pgfsys@useobject{currentmarker}{}%
\end{pgfscope}%
\end{pgfscope}%
\begin{pgfscope}%
\definecolor{textcolor}{rgb}{0.000000,0.000000,0.000000}%
\pgfsetstrokecolor{textcolor}%
\pgfsetfillcolor{textcolor}%
\pgftext[x=0.632340in, y=5.451564in, left, base]{\color{textcolor}\rmfamily\fontsize{38.016000}{45.619200}\selectfont \(\displaystyle {0.8}\)}%
\end{pgfscope}%
\begin{pgfscope}%
\definecolor{textcolor}{rgb}{0.000000,0.000000,0.000000}%
\pgfsetstrokecolor{textcolor}%
\pgfsetfillcolor{textcolor}%
\pgftext[x=0.576784in,y=3.834062in,,bottom,rotate=90.000000]{\color{textcolor}\rmfamily\fontsize{38.016000}{45.619200}\selectfont Sagat score average}%
\end{pgfscope}%
\begin{pgfscope}%
\pgfpathrectangle{\pgfqpoint{1.140505in}{1.191562in}}{\pgfqpoint{13.950000in}{5.285000in}}%
\pgfusepath{clip}%
\pgfsetrectcap%
\pgfsetroundjoin%
\pgfsetlinewidth{2.710125pt}%
\definecolor{currentstroke}{rgb}{0.260000,0.260000,0.260000}%
\pgfsetstrokecolor{currentstroke}%
\pgfsetdash{}{0pt}%
\pgfpathmoveto{\pgfqpoint{2.186755in}{3.899148in}}%
\pgfpathlineto{\pgfqpoint{2.186755in}{5.079378in}}%
\pgfusepath{stroke}%
\end{pgfscope}%
\begin{pgfscope}%
\pgfpathrectangle{\pgfqpoint{1.140505in}{1.191562in}}{\pgfqpoint{13.950000in}{5.285000in}}%
\pgfusepath{clip}%
\pgfsetrectcap%
\pgfsetroundjoin%
\pgfsetlinewidth{2.710125pt}%
\definecolor{currentstroke}{rgb}{0.260000,0.260000,0.260000}%
\pgfsetstrokecolor{currentstroke}%
\pgfsetdash{}{0pt}%
\pgfpathmoveto{\pgfqpoint{5.674255in}{3.779736in}}%
\pgfpathlineto{\pgfqpoint{5.674255in}{5.422339in}}%
\pgfusepath{stroke}%
\end{pgfscope}%
\begin{pgfscope}%
\pgfpathrectangle{\pgfqpoint{1.140505in}{1.191562in}}{\pgfqpoint{13.950000in}{5.285000in}}%
\pgfusepath{clip}%
\pgfsetrectcap%
\pgfsetroundjoin%
\pgfsetlinewidth{2.710125pt}%
\definecolor{currentstroke}{rgb}{0.260000,0.260000,0.260000}%
\pgfsetstrokecolor{currentstroke}%
\pgfsetdash{}{0pt}%
\pgfpathmoveto{\pgfqpoint{9.161755in}{3.735304in}}%
\pgfpathlineto{\pgfqpoint{9.161755in}{4.267102in}}%
\pgfusepath{stroke}%
\end{pgfscope}%
\begin{pgfscope}%
\pgfpathrectangle{\pgfqpoint{1.140505in}{1.191562in}}{\pgfqpoint{13.950000in}{5.285000in}}%
\pgfusepath{clip}%
\pgfsetrectcap%
\pgfsetroundjoin%
\pgfsetlinewidth{2.710125pt}%
\definecolor{currentstroke}{rgb}{0.260000,0.260000,0.260000}%
\pgfsetstrokecolor{currentstroke}%
\pgfsetdash{}{0pt}%
\pgfpathmoveto{\pgfqpoint{12.649255in}{3.552021in}}%
\pgfpathlineto{\pgfqpoint{12.649255in}{5.808343in}}%
\pgfusepath{stroke}%
\end{pgfscope}%
\begin{pgfscope}%
\pgfpathrectangle{\pgfqpoint{1.140505in}{1.191562in}}{\pgfqpoint{13.950000in}{5.285000in}}%
\pgfusepath{clip}%
\pgfsetrectcap%
\pgfsetroundjoin%
\pgfsetlinewidth{2.710125pt}%
\definecolor{currentstroke}{rgb}{0.260000,0.260000,0.260000}%
\pgfsetstrokecolor{currentstroke}%
\pgfsetdash{}{0pt}%
\pgfpathmoveto{\pgfqpoint{3.581755in}{4.176849in}}%
\pgfpathlineto{\pgfqpoint{3.581755in}{6.190182in}}%
\pgfusepath{stroke}%
\end{pgfscope}%
\begin{pgfscope}%
\pgfpathrectangle{\pgfqpoint{1.140505in}{1.191562in}}{\pgfqpoint{13.950000in}{5.285000in}}%
\pgfusepath{clip}%
\pgfsetrectcap%
\pgfsetroundjoin%
\pgfsetlinewidth{2.710125pt}%
\definecolor{currentstroke}{rgb}{0.260000,0.260000,0.260000}%
\pgfsetstrokecolor{currentstroke}%
\pgfsetdash{}{0pt}%
\pgfpathmoveto{\pgfqpoint{7.069255in}{4.037998in}}%
\pgfpathlineto{\pgfqpoint{7.069255in}{6.224895in}}%
\pgfusepath{stroke}%
\end{pgfscope}%
\begin{pgfscope}%
\pgfpathrectangle{\pgfqpoint{1.140505in}{1.191562in}}{\pgfqpoint{13.950000in}{5.285000in}}%
\pgfusepath{clip}%
\pgfsetrectcap%
\pgfsetroundjoin%
\pgfsetlinewidth{2.710125pt}%
\definecolor{currentstroke}{rgb}{0.260000,0.260000,0.260000}%
\pgfsetstrokecolor{currentstroke}%
\pgfsetdash{}{0pt}%
\pgfpathmoveto{\pgfqpoint{10.556755in}{4.385125in}}%
\pgfpathlineto{\pgfqpoint{10.556755in}{5.843056in}}%
\pgfusepath{stroke}%
\end{pgfscope}%
\begin{pgfscope}%
\pgfpathrectangle{\pgfqpoint{1.140505in}{1.191562in}}{\pgfqpoint{13.950000in}{5.285000in}}%
\pgfusepath{clip}%
\pgfsetrectcap%
\pgfsetroundjoin%
\pgfsetlinewidth{2.710125pt}%
\definecolor{currentstroke}{rgb}{0.260000,0.260000,0.260000}%
\pgfsetstrokecolor{currentstroke}%
\pgfsetdash{}{0pt}%
\pgfpathmoveto{\pgfqpoint{14.044255in}{4.732251in}}%
\pgfpathlineto{\pgfqpoint{14.044255in}{6.190182in}}%
\pgfusepath{stroke}%
\end{pgfscope}%
\begin{pgfscope}%
\pgfsetrectcap%
\pgfsetmiterjoin%
\pgfsetlinewidth{0.803000pt}%
\definecolor{currentstroke}{rgb}{0.000000,0.000000,0.000000}%
\pgfsetstrokecolor{currentstroke}%
\pgfsetdash{}{0pt}%
\pgfpathmoveto{\pgfqpoint{1.140505in}{1.191562in}}%
\pgfpathlineto{\pgfqpoint{1.140505in}{6.476562in}}%
\pgfusepath{stroke}%
\end{pgfscope}%
\begin{pgfscope}%
\pgfsetrectcap%
\pgfsetmiterjoin%
\pgfsetlinewidth{0.803000pt}%
\definecolor{currentstroke}{rgb}{0.000000,0.000000,0.000000}%
\pgfsetstrokecolor{currentstroke}%
\pgfsetdash{}{0pt}%
\pgfpathmoveto{\pgfqpoint{15.090505in}{1.191562in}}%
\pgfpathlineto{\pgfqpoint{15.090505in}{6.476562in}}%
\pgfusepath{stroke}%
\end{pgfscope}%
\begin{pgfscope}%
\pgfsetrectcap%
\pgfsetmiterjoin%
\pgfsetlinewidth{0.803000pt}%
\definecolor{currentstroke}{rgb}{0.000000,0.000000,0.000000}%
\pgfsetstrokecolor{currentstroke}%
\pgfsetdash{}{0pt}%
\pgfpathmoveto{\pgfqpoint{1.140505in}{1.191562in}}%
\pgfpathlineto{\pgfqpoint{15.090505in}{1.191562in}}%
\pgfusepath{stroke}%
\end{pgfscope}%
\begin{pgfscope}%
\pgfsetrectcap%
\pgfsetmiterjoin%
\pgfsetlinewidth{0.803000pt}%
\definecolor{currentstroke}{rgb}{0.000000,0.000000,0.000000}%
\pgfsetstrokecolor{currentstroke}%
\pgfsetdash{}{0pt}%
\pgfpathmoveto{\pgfqpoint{1.140505in}{6.476562in}}%
\pgfpathlineto{\pgfqpoint{15.090505in}{6.476562in}}%
\pgfusepath{stroke}%
\end{pgfscope}%
\begin{pgfscope}%
\definecolor{textcolor}{rgb}{0.000000,0.000000,0.000000}%
\pgfsetstrokecolor{textcolor}%
\pgfsetfillcolor{textcolor}%
\pgftext[x=8.115505in,y=6.584273in,,base]{\color{textcolor}\rmfamily\fontsize{38.016000}{45.619200}\selectfont Sagat score average for blind users between rounds}%
\end{pgfscope}%
\begin{pgfscope}%
\pgfsetbuttcap%
\pgfsetmiterjoin%
\definecolor{currentfill}{rgb}{1.000000,1.000000,1.000000}%
\pgfsetfillcolor{currentfill}%
\pgfsetfillopacity{0.800000}%
\pgfsetlinewidth{1.003750pt}%
\definecolor{currentstroke}{rgb}{0.800000,0.800000,0.800000}%
\pgfsetstrokecolor{currentstroke}%
\pgfsetstrokeopacity{0.800000}%
\pgfsetdash{}{0pt}%
\pgfpathmoveto{\pgfqpoint{12.811305in}{7.457562in}}%
\pgfpathlineto{\pgfqpoint{15.017171in}{7.457562in}}%
\pgfpathquadraticcurveto{\pgfqpoint{15.090505in}{7.457562in}}{\pgfqpoint{15.090505in}{7.530896in}}%
\pgfpathlineto{\pgfqpoint{15.090505in}{8.517228in}}%
\pgfpathquadraticcurveto{\pgfqpoint{15.090505in}{8.590562in}}{\pgfqpoint{15.017171in}{8.590562in}}%
\pgfpathlineto{\pgfqpoint{12.811305in}{8.590562in}}%
\pgfpathquadraticcurveto{\pgfqpoint{12.737972in}{8.590562in}}{\pgfqpoint{12.737972in}{8.517228in}}%
\pgfpathlineto{\pgfqpoint{12.737972in}{7.530896in}}%
\pgfpathquadraticcurveto{\pgfqpoint{12.737972in}{7.457562in}}{\pgfqpoint{12.811305in}{7.457562in}}%
\pgfpathclose%
\pgfusepath{stroke,fill}%
\end{pgfscope}%
\begin{pgfscope}%
\pgfsetbuttcap%
\pgfsetmiterjoin%
\definecolor{currentfill}{rgb}{0.651961,0.093137,0.093137}%
\pgfsetfillcolor{currentfill}%
\pgfsetlinewidth{0.000000pt}%
\definecolor{currentstroke}{rgb}{0.000000,0.000000,0.000000}%
\pgfsetstrokecolor{currentstroke}%
\pgfsetstrokeopacity{0.000000}%
\pgfsetdash{}{0pt}%
\pgfpathmoveto{\pgfqpoint{12.884638in}{8.187228in}}%
\pgfpathlineto{\pgfqpoint{13.617972in}{8.187228in}}%
\pgfpathlineto{\pgfqpoint{13.617972in}{8.443895in}}%
\pgfpathlineto{\pgfqpoint{12.884638in}{8.443895in}}%
\pgfpathclose%
\pgfusepath{fill}%
\end{pgfscope}%
\begin{pgfscope}%
\definecolor{textcolor}{rgb}{0.000000,0.000000,0.000000}%
\pgfsetstrokecolor{textcolor}%
\pgfsetfillcolor{textcolor}%
\pgftext[x=13.911305in,y=8.187228in,left,base]{\color{textcolor}\rmfamily\fontsize{26.400000}{31.680000}\selectfont First}%
\end{pgfscope}%
\begin{pgfscope}%
\pgfsetbuttcap%
\pgfsetmiterjoin%
\definecolor{currentfill}{rgb}{0.144608,0.218137,0.424020}%
\pgfsetfillcolor{currentfill}%
\pgfsetlinewidth{0.000000pt}%
\definecolor{currentstroke}{rgb}{0.000000,0.000000,0.000000}%
\pgfsetstrokecolor{currentstroke}%
\pgfsetstrokeopacity{0.000000}%
\pgfsetdash{}{0pt}%
\pgfpathmoveto{\pgfqpoint{12.884638in}{7.675729in}}%
\pgfpathlineto{\pgfqpoint{13.617972in}{7.675729in}}%
\pgfpathlineto{\pgfqpoint{13.617972in}{7.932395in}}%
\pgfpathlineto{\pgfqpoint{12.884638in}{7.932395in}}%
\pgfpathclose%
\pgfusepath{fill}%
\end{pgfscope}%
\begin{pgfscope}%
\definecolor{textcolor}{rgb}{0.000000,0.000000,0.000000}%
\pgfsetstrokecolor{textcolor}%
\pgfsetfillcolor{textcolor}%
\pgftext[x=13.911305in,y=7.675729in,left,base]{\color{textcolor}\rmfamily\fontsize{26.400000}{31.680000}\selectfont Return}%
\end{pgfscope}%
\end{pgfpicture}%
\makeatother%
\endgroup%
    
        %}
        \caption{Bar plot of the average Sagat score of the blind participants on each method.}
        \label{fig:barplot_sagat_scene_blind}
    \end{minipage}
    \begin{minipage}{\textwidth}
        \centering
        \includegraphics[width = 0.8\linewidth]{Resultados/Sagat/Figuras/png/barplot_sagat_avg_scene_sight.png}
        %\resizebox{0.6\linewidth}{!}{
        %%% Creator: Matplotlib, PGF backend
%%
%% To include the figure in your LaTeX document, write
%%   \input{<filename>.pgf}
%%
%% Make sure the required packages are loaded in your preamble
%%   \usepackage{pgf}
%%
%% and, on pdftex
%%   \usepackage[utf8]{inputenc}\DeclareUnicodeCharacter{2212}{-}
%%
%% or, on luatex and xetex
%%   \usepackage{unicode-math}
%%
%% Figures using additional raster images can only be included by \input if
%% they are in the same directory as the main LaTeX file. For loading figures
%% from other directories you can use the `import` package
%%   \usepackage{import}
%%
%% and then include the figures with
%%   \import{<path to file>}{<filename>.pgf}
%%
%% Matplotlib used the following preamble
%%   \usepackage{url}
%%   \usepackage{unicode-math}
%%   \setmainfont{DejaVu Serif}
%%   \usepackage{fontspec}
%%
\begingroup%
\makeatletter%
\begin{pgfpicture}%
\pgfpathrectangle{\pgfpointorigin}{\pgfqpoint{12.196638in}{14.522618in}}%
\pgfusepath{use as bounding box, clip}%
\begin{pgfscope}%
\pgfsetbuttcap%
\pgfsetmiterjoin%
\pgfsetlinewidth{0.000000pt}%
\definecolor{currentstroke}{rgb}{1.000000,1.000000,1.000000}%
\pgfsetstrokecolor{currentstroke}%
\pgfsetstrokeopacity{0.000000}%
\pgfsetdash{}{0pt}%
\pgfpathmoveto{\pgfqpoint{0.000000in}{0.000000in}}%
\pgfpathlineto{\pgfqpoint{12.196638in}{0.000000in}}%
\pgfpathlineto{\pgfqpoint{12.196638in}{14.522618in}}%
\pgfpathlineto{\pgfqpoint{0.000000in}{14.522618in}}%
\pgfpathclose%
\pgfusepath{}%
\end{pgfscope}%
\begin{pgfscope}%
\pgfsetbuttcap%
\pgfsetmiterjoin%
\definecolor{currentfill}{rgb}{1.000000,1.000000,1.000000}%
\pgfsetfillcolor{currentfill}%
\pgfsetlinewidth{0.000000pt}%
\definecolor{currentstroke}{rgb}{0.000000,0.000000,0.000000}%
\pgfsetstrokecolor{currentstroke}%
\pgfsetstrokeopacity{0.000000}%
\pgfsetdash{}{0pt}%
\pgfpathmoveto{\pgfqpoint{1.246638in}{1.104418in}}%
\pgfpathlineto{\pgfqpoint{12.096638in}{1.104418in}}%
\pgfpathlineto{\pgfqpoint{12.096638in}{11.674418in}}%
\pgfpathlineto{\pgfqpoint{1.246638in}{11.674418in}}%
\pgfpathclose%
\pgfusepath{fill}%
\end{pgfscope}%
\begin{pgfscope}%
\pgfpathrectangle{\pgfqpoint{1.246638in}{1.104418in}}{\pgfqpoint{10.850000in}{10.570000in}}%
\pgfusepath{clip}%
\pgfsetbuttcap%
\pgfsetmiterjoin%
\definecolor{currentfill}{rgb}{0.651961,0.093137,0.093137}%
\pgfsetfillcolor{currentfill}%
\pgfsetlinewidth{0.000000pt}%
\definecolor{currentstroke}{rgb}{0.000000,0.000000,0.000000}%
\pgfsetstrokecolor{currentstroke}%
\pgfsetstrokeopacity{0.000000}%
\pgfsetdash{}{0pt}%
\pgfpathmoveto{\pgfqpoint{1.517888in}{1.104418in}}%
\pgfpathlineto{\pgfqpoint{2.602888in}{1.104418in}}%
\pgfpathlineto{\pgfqpoint{2.602888in}{7.565115in}}%
\pgfpathlineto{\pgfqpoint{1.517888in}{7.565115in}}%
\pgfpathclose%
\pgfusepath{fill}%
\end{pgfscope}%
\begin{pgfscope}%
\pgfpathrectangle{\pgfqpoint{1.246638in}{1.104418in}}{\pgfqpoint{10.850000in}{10.570000in}}%
\pgfusepath{clip}%
\pgfsetbuttcap%
\pgfsetmiterjoin%
\definecolor{currentfill}{rgb}{0.651961,0.093137,0.093137}%
\pgfsetfillcolor{currentfill}%
\pgfsetlinewidth{0.000000pt}%
\definecolor{currentstroke}{rgb}{0.000000,0.000000,0.000000}%
\pgfsetstrokecolor{currentstroke}%
\pgfsetstrokeopacity{0.000000}%
\pgfsetdash{}{0pt}%
\pgfpathmoveto{\pgfqpoint{4.230388in}{1.104418in}}%
\pgfpathlineto{\pgfqpoint{5.315388in}{1.104418in}}%
\pgfpathlineto{\pgfqpoint{5.315388in}{7.556100in}}%
\pgfpathlineto{\pgfqpoint{4.230388in}{7.556100in}}%
\pgfpathclose%
\pgfusepath{fill}%
\end{pgfscope}%
\begin{pgfscope}%
\pgfpathrectangle{\pgfqpoint{1.246638in}{1.104418in}}{\pgfqpoint{10.850000in}{10.570000in}}%
\pgfusepath{clip}%
\pgfsetbuttcap%
\pgfsetmiterjoin%
\definecolor{currentfill}{rgb}{0.651961,0.093137,0.093137}%
\pgfsetfillcolor{currentfill}%
\pgfsetlinewidth{0.000000pt}%
\definecolor{currentstroke}{rgb}{0.000000,0.000000,0.000000}%
\pgfsetstrokecolor{currentstroke}%
\pgfsetstrokeopacity{0.000000}%
\pgfsetdash{}{0pt}%
\pgfpathmoveto{\pgfqpoint{6.942888in}{1.104418in}}%
\pgfpathlineto{\pgfqpoint{8.027888in}{1.104418in}}%
\pgfpathlineto{\pgfqpoint{8.027888in}{6.101692in}}%
\pgfpathlineto{\pgfqpoint{6.942888in}{6.101692in}}%
\pgfpathclose%
\pgfusepath{fill}%
\end{pgfscope}%
\begin{pgfscope}%
\pgfpathrectangle{\pgfqpoint{1.246638in}{1.104418in}}{\pgfqpoint{10.850000in}{10.570000in}}%
\pgfusepath{clip}%
\pgfsetbuttcap%
\pgfsetmiterjoin%
\definecolor{currentfill}{rgb}{0.651961,0.093137,0.093137}%
\pgfsetfillcolor{currentfill}%
\pgfsetlinewidth{0.000000pt}%
\definecolor{currentstroke}{rgb}{0.000000,0.000000,0.000000}%
\pgfsetstrokecolor{currentstroke}%
\pgfsetstrokeopacity{0.000000}%
\pgfsetdash{}{0pt}%
\pgfpathmoveto{\pgfqpoint{9.655388in}{1.104418in}}%
\pgfpathlineto{\pgfqpoint{10.740388in}{1.104418in}}%
\pgfpathlineto{\pgfqpoint{10.740388in}{8.767105in}}%
\pgfpathlineto{\pgfqpoint{9.655388in}{8.767105in}}%
\pgfpathclose%
\pgfusepath{fill}%
\end{pgfscope}%
\begin{pgfscope}%
\pgfpathrectangle{\pgfqpoint{1.246638in}{1.104418in}}{\pgfqpoint{10.850000in}{10.570000in}}%
\pgfusepath{clip}%
\pgfsetbuttcap%
\pgfsetmiterjoin%
\definecolor{currentfill}{rgb}{0.144608,0.218137,0.424020}%
\pgfsetfillcolor{currentfill}%
\pgfsetlinewidth{0.000000pt}%
\definecolor{currentstroke}{rgb}{0.000000,0.000000,0.000000}%
\pgfsetstrokecolor{currentstroke}%
\pgfsetstrokeopacity{0.000000}%
\pgfsetdash{}{0pt}%
\pgfpathmoveto{\pgfqpoint{2.602888in}{1.104418in}}%
\pgfpathlineto{\pgfqpoint{3.687888in}{1.104418in}}%
\pgfpathlineto{\pgfqpoint{3.687888in}{8.166110in}}%
\pgfpathlineto{\pgfqpoint{2.602888in}{8.166110in}}%
\pgfpathclose%
\pgfusepath{fill}%
\end{pgfscope}%
\begin{pgfscope}%
\pgfpathrectangle{\pgfqpoint{1.246638in}{1.104418in}}{\pgfqpoint{10.850000in}{10.570000in}}%
\pgfusepath{clip}%
\pgfsetbuttcap%
\pgfsetmiterjoin%
\definecolor{currentfill}{rgb}{0.144608,0.218137,0.424020}%
\pgfsetfillcolor{currentfill}%
\pgfsetlinewidth{0.000000pt}%
\definecolor{currentstroke}{rgb}{0.000000,0.000000,0.000000}%
\pgfsetstrokecolor{currentstroke}%
\pgfsetstrokeopacity{0.000000}%
\pgfsetdash{}{0pt}%
\pgfpathmoveto{\pgfqpoint{5.315388in}{1.104418in}}%
\pgfpathlineto{\pgfqpoint{6.400388in}{1.104418in}}%
\pgfpathlineto{\pgfqpoint{6.400388in}{8.541731in}}%
\pgfpathlineto{\pgfqpoint{5.315388in}{8.541731in}}%
\pgfpathclose%
\pgfusepath{fill}%
\end{pgfscope}%
\begin{pgfscope}%
\pgfpathrectangle{\pgfqpoint{1.246638in}{1.104418in}}{\pgfqpoint{10.850000in}{10.570000in}}%
\pgfusepath{clip}%
\pgfsetbuttcap%
\pgfsetmiterjoin%
\definecolor{currentfill}{rgb}{0.144608,0.218137,0.424020}%
\pgfsetfillcolor{currentfill}%
\pgfsetlinewidth{0.000000pt}%
\definecolor{currentstroke}{rgb}{0.000000,0.000000,0.000000}%
\pgfsetstrokecolor{currentstroke}%
\pgfsetstrokeopacity{0.000000}%
\pgfsetdash{}{0pt}%
\pgfpathmoveto{\pgfqpoint{8.027888in}{1.104418in}}%
\pgfpathlineto{\pgfqpoint{9.112888in}{1.104418in}}%
\pgfpathlineto{\pgfqpoint{9.112888in}{7.565115in}}%
\pgfpathlineto{\pgfqpoint{8.027888in}{7.565115in}}%
\pgfpathclose%
\pgfusepath{fill}%
\end{pgfscope}%
\begin{pgfscope}%
\pgfpathrectangle{\pgfqpoint{1.246638in}{1.104418in}}{\pgfqpoint{10.850000in}{10.570000in}}%
\pgfusepath{clip}%
\pgfsetbuttcap%
\pgfsetmiterjoin%
\definecolor{currentfill}{rgb}{0.144608,0.218137,0.424020}%
\pgfsetfillcolor{currentfill}%
\pgfsetlinewidth{0.000000pt}%
\definecolor{currentstroke}{rgb}{0.000000,0.000000,0.000000}%
\pgfsetstrokecolor{currentstroke}%
\pgfsetstrokeopacity{0.000000}%
\pgfsetdash{}{0pt}%
\pgfpathmoveto{\pgfqpoint{10.740388in}{1.104418in}}%
\pgfpathlineto{\pgfqpoint{11.825388in}{1.104418in}}%
\pgfpathlineto{\pgfqpoint{11.825388in}{8.616856in}}%
\pgfpathlineto{\pgfqpoint{10.740388in}{8.616856in}}%
\pgfpathclose%
\pgfusepath{fill}%
\end{pgfscope}%
\begin{pgfscope}%
\pgfsetbuttcap%
\pgfsetroundjoin%
\definecolor{currentfill}{rgb}{0.000000,0.000000,0.000000}%
\pgfsetfillcolor{currentfill}%
\pgfsetlinewidth{0.803000pt}%
\definecolor{currentstroke}{rgb}{0.000000,0.000000,0.000000}%
\pgfsetstrokecolor{currentstroke}%
\pgfsetdash{}{0pt}%
\pgfsys@defobject{currentmarker}{\pgfqpoint{0.000000in}{-0.048611in}}{\pgfqpoint{0.000000in}{0.000000in}}{%
\pgfpathmoveto{\pgfqpoint{0.000000in}{0.000000in}}%
\pgfpathlineto{\pgfqpoint{0.000000in}{-0.048611in}}%
\pgfusepath{stroke,fill}%
}%
\begin{pgfscope}%
\pgfsys@transformshift{2.602888in}{1.104418in}%
\pgfsys@useobject{currentmarker}{}%
\end{pgfscope}%
\end{pgfscope}%
\begin{pgfscope}%
\definecolor{textcolor}{rgb}{0.000000,0.000000,0.000000}%
\pgfsetstrokecolor{textcolor}%
\pgfsetfillcolor{textcolor}%
\pgftext[x=2.602888in,y=1.007196in,,top]{\color{textcolor}\rmfamily\fontsize{31.680000}{38.016000}\selectfont Audio}%
\end{pgfscope}%
\begin{pgfscope}%
\pgfsetbuttcap%
\pgfsetroundjoin%
\definecolor{currentfill}{rgb}{0.000000,0.000000,0.000000}%
\pgfsetfillcolor{currentfill}%
\pgfsetlinewidth{0.803000pt}%
\definecolor{currentstroke}{rgb}{0.000000,0.000000,0.000000}%
\pgfsetstrokecolor{currentstroke}%
\pgfsetdash{}{0pt}%
\pgfsys@defobject{currentmarker}{\pgfqpoint{0.000000in}{-0.048611in}}{\pgfqpoint{0.000000in}{0.000000in}}{%
\pgfpathmoveto{\pgfqpoint{0.000000in}{0.000000in}}%
\pgfpathlineto{\pgfqpoint{0.000000in}{-0.048611in}}%
\pgfusepath{stroke,fill}%
}%
\begin{pgfscope}%
\pgfsys@transformshift{5.315388in}{1.104418in}%
\pgfsys@useobject{currentmarker}{}%
\end{pgfscope}%
\end{pgfscope}%
\begin{pgfscope}%
\definecolor{textcolor}{rgb}{0.000000,0.000000,0.000000}%
\pgfsetstrokecolor{textcolor}%
\pgfsetfillcolor{textcolor}%
\pgftext[x=5.315388in,y=1.007196in,,top]{\color{textcolor}\rmfamily\fontsize{31.680000}{38.016000}\selectfont Haptic Belt}%
\end{pgfscope}%
\begin{pgfscope}%
\pgfsetbuttcap%
\pgfsetroundjoin%
\definecolor{currentfill}{rgb}{0.000000,0.000000,0.000000}%
\pgfsetfillcolor{currentfill}%
\pgfsetlinewidth{0.803000pt}%
\definecolor{currentstroke}{rgb}{0.000000,0.000000,0.000000}%
\pgfsetstrokecolor{currentstroke}%
\pgfsetdash{}{0pt}%
\pgfsys@defobject{currentmarker}{\pgfqpoint{0.000000in}{-0.048611in}}{\pgfqpoint{0.000000in}{0.000000in}}{%
\pgfpathmoveto{\pgfqpoint{0.000000in}{0.000000in}}%
\pgfpathlineto{\pgfqpoint{0.000000in}{-0.048611in}}%
\pgfusepath{stroke,fill}%
}%
\begin{pgfscope}%
\pgfsys@transformshift{8.027888in}{1.104418in}%
\pgfsys@useobject{currentmarker}{}%
\end{pgfscope}%
\end{pgfscope}%
\begin{pgfscope}%
\definecolor{textcolor}{rgb}{0.000000,0.000000,0.000000}%
\pgfsetstrokecolor{textcolor}%
\pgfsetfillcolor{textcolor}%
\pgftext[x=8.027888in,y=1.007196in,,top]{\color{textcolor}\rmfamily\fontsize{31.680000}{38.016000}\selectfont Virtual Cane}%
\end{pgfscope}%
\begin{pgfscope}%
\pgfsetbuttcap%
\pgfsetroundjoin%
\definecolor{currentfill}{rgb}{0.000000,0.000000,0.000000}%
\pgfsetfillcolor{currentfill}%
\pgfsetlinewidth{0.803000pt}%
\definecolor{currentstroke}{rgb}{0.000000,0.000000,0.000000}%
\pgfsetstrokecolor{currentstroke}%
\pgfsetdash{}{0pt}%
\pgfsys@defobject{currentmarker}{\pgfqpoint{0.000000in}{-0.048611in}}{\pgfqpoint{0.000000in}{0.000000in}}{%
\pgfpathmoveto{\pgfqpoint{0.000000in}{0.000000in}}%
\pgfpathlineto{\pgfqpoint{0.000000in}{-0.048611in}}%
\pgfusepath{stroke,fill}%
}%
\begin{pgfscope}%
\pgfsys@transformshift{10.740388in}{1.104418in}%
\pgfsys@useobject{currentmarker}{}%
\end{pgfscope}%
\end{pgfscope}%
\begin{pgfscope}%
\definecolor{textcolor}{rgb}{0.000000,0.000000,0.000000}%
\pgfsetstrokecolor{textcolor}%
\pgfsetfillcolor{textcolor}%
\pgftext[x=10.740388in,y=1.007196in,,top]{\color{textcolor}\rmfamily\fontsize{31.680000}{38.016000}\selectfont Mixture}%
\end{pgfscope}%
\begin{pgfscope}%
\definecolor{textcolor}{rgb}{0.000000,0.000000,0.000000}%
\pgfsetstrokecolor{textcolor}%
\pgfsetfillcolor{textcolor}%
\pgftext[x=6.671638in,y=0.525820in,,top]{\color{textcolor}\rmfamily\fontsize{31.680000}{38.016000}\selectfont Scene}%
\end{pgfscope}%
\begin{pgfscope}%
\pgfsetbuttcap%
\pgfsetroundjoin%
\definecolor{currentfill}{rgb}{0.000000,0.000000,0.000000}%
\pgfsetfillcolor{currentfill}%
\pgfsetlinewidth{0.803000pt}%
\definecolor{currentstroke}{rgb}{0.000000,0.000000,0.000000}%
\pgfsetstrokecolor{currentstroke}%
\pgfsetdash{}{0pt}%
\pgfsys@defobject{currentmarker}{\pgfqpoint{-0.048611in}{0.000000in}}{\pgfqpoint{-0.000000in}{0.000000in}}{%
\pgfpathmoveto{\pgfqpoint{-0.000000in}{0.000000in}}%
\pgfpathlineto{\pgfqpoint{-0.048611in}{0.000000in}}%
\pgfusepath{stroke,fill}%
}%
\begin{pgfscope}%
\pgfsys@transformshift{1.246638in}{1.104418in}%
\pgfsys@useobject{currentmarker}{}%
\end{pgfscope}%
\end{pgfscope}%
\begin{pgfscope}%
\definecolor{textcolor}{rgb}{0.000000,0.000000,0.000000}%
\pgfsetstrokecolor{textcolor}%
\pgfsetfillcolor{textcolor}%
\pgftext[x=0.587096in, y=0.937270in, left, base]{\color{textcolor}\rmfamily\fontsize{31.680000}{38.016000}\selectfont \(\displaystyle {0.0}\)}%
\end{pgfscope}%
\begin{pgfscope}%
\pgfsetbuttcap%
\pgfsetroundjoin%
\definecolor{currentfill}{rgb}{0.000000,0.000000,0.000000}%
\pgfsetfillcolor{currentfill}%
\pgfsetlinewidth{0.803000pt}%
\definecolor{currentstroke}{rgb}{0.000000,0.000000,0.000000}%
\pgfsetstrokecolor{currentstroke}%
\pgfsetdash{}{0pt}%
\pgfsys@defobject{currentmarker}{\pgfqpoint{-0.048611in}{0.000000in}}{\pgfqpoint{-0.000000in}{0.000000in}}{%
\pgfpathmoveto{\pgfqpoint{-0.000000in}{0.000000in}}%
\pgfpathlineto{\pgfqpoint{-0.048611in}{0.000000in}}%
\pgfusepath{stroke,fill}%
}%
\begin{pgfscope}%
\pgfsys@transformshift{1.246638in}{2.306408in}%
\pgfsys@useobject{currentmarker}{}%
\end{pgfscope}%
\end{pgfscope}%
\begin{pgfscope}%
\definecolor{textcolor}{rgb}{0.000000,0.000000,0.000000}%
\pgfsetstrokecolor{textcolor}%
\pgfsetfillcolor{textcolor}%
\pgftext[x=0.587096in, y=2.139260in, left, base]{\color{textcolor}\rmfamily\fontsize{31.680000}{38.016000}\selectfont \(\displaystyle {0.1}\)}%
\end{pgfscope}%
\begin{pgfscope}%
\pgfsetbuttcap%
\pgfsetroundjoin%
\definecolor{currentfill}{rgb}{0.000000,0.000000,0.000000}%
\pgfsetfillcolor{currentfill}%
\pgfsetlinewidth{0.803000pt}%
\definecolor{currentstroke}{rgb}{0.000000,0.000000,0.000000}%
\pgfsetstrokecolor{currentstroke}%
\pgfsetdash{}{0pt}%
\pgfsys@defobject{currentmarker}{\pgfqpoint{-0.048611in}{0.000000in}}{\pgfqpoint{-0.000000in}{0.000000in}}{%
\pgfpathmoveto{\pgfqpoint{-0.000000in}{0.000000in}}%
\pgfpathlineto{\pgfqpoint{-0.048611in}{0.000000in}}%
\pgfusepath{stroke,fill}%
}%
\begin{pgfscope}%
\pgfsys@transformshift{1.246638in}{3.508398in}%
\pgfsys@useobject{currentmarker}{}%
\end{pgfscope}%
\end{pgfscope}%
\begin{pgfscope}%
\definecolor{textcolor}{rgb}{0.000000,0.000000,0.000000}%
\pgfsetstrokecolor{textcolor}%
\pgfsetfillcolor{textcolor}%
\pgftext[x=0.587096in, y=3.341250in, left, base]{\color{textcolor}\rmfamily\fontsize{31.680000}{38.016000}\selectfont \(\displaystyle {0.2}\)}%
\end{pgfscope}%
\begin{pgfscope}%
\pgfsetbuttcap%
\pgfsetroundjoin%
\definecolor{currentfill}{rgb}{0.000000,0.000000,0.000000}%
\pgfsetfillcolor{currentfill}%
\pgfsetlinewidth{0.803000pt}%
\definecolor{currentstroke}{rgb}{0.000000,0.000000,0.000000}%
\pgfsetstrokecolor{currentstroke}%
\pgfsetdash{}{0pt}%
\pgfsys@defobject{currentmarker}{\pgfqpoint{-0.048611in}{0.000000in}}{\pgfqpoint{-0.000000in}{0.000000in}}{%
\pgfpathmoveto{\pgfqpoint{-0.000000in}{0.000000in}}%
\pgfpathlineto{\pgfqpoint{-0.048611in}{0.000000in}}%
\pgfusepath{stroke,fill}%
}%
\begin{pgfscope}%
\pgfsys@transformshift{1.246638in}{4.710388in}%
\pgfsys@useobject{currentmarker}{}%
\end{pgfscope}%
\end{pgfscope}%
\begin{pgfscope}%
\definecolor{textcolor}{rgb}{0.000000,0.000000,0.000000}%
\pgfsetstrokecolor{textcolor}%
\pgfsetfillcolor{textcolor}%
\pgftext[x=0.587096in, y=4.543240in, left, base]{\color{textcolor}\rmfamily\fontsize{31.680000}{38.016000}\selectfont \(\displaystyle {0.3}\)}%
\end{pgfscope}%
\begin{pgfscope}%
\pgfsetbuttcap%
\pgfsetroundjoin%
\definecolor{currentfill}{rgb}{0.000000,0.000000,0.000000}%
\pgfsetfillcolor{currentfill}%
\pgfsetlinewidth{0.803000pt}%
\definecolor{currentstroke}{rgb}{0.000000,0.000000,0.000000}%
\pgfsetstrokecolor{currentstroke}%
\pgfsetdash{}{0pt}%
\pgfsys@defobject{currentmarker}{\pgfqpoint{-0.048611in}{0.000000in}}{\pgfqpoint{-0.000000in}{0.000000in}}{%
\pgfpathmoveto{\pgfqpoint{-0.000000in}{0.000000in}}%
\pgfpathlineto{\pgfqpoint{-0.048611in}{0.000000in}}%
\pgfusepath{stroke,fill}%
}%
\begin{pgfscope}%
\pgfsys@transformshift{1.246638in}{5.912378in}%
\pgfsys@useobject{currentmarker}{}%
\end{pgfscope}%
\end{pgfscope}%
\begin{pgfscope}%
\definecolor{textcolor}{rgb}{0.000000,0.000000,0.000000}%
\pgfsetstrokecolor{textcolor}%
\pgfsetfillcolor{textcolor}%
\pgftext[x=0.587096in, y=5.745230in, left, base]{\color{textcolor}\rmfamily\fontsize{31.680000}{38.016000}\selectfont \(\displaystyle {0.4}\)}%
\end{pgfscope}%
\begin{pgfscope}%
\pgfsetbuttcap%
\pgfsetroundjoin%
\definecolor{currentfill}{rgb}{0.000000,0.000000,0.000000}%
\pgfsetfillcolor{currentfill}%
\pgfsetlinewidth{0.803000pt}%
\definecolor{currentstroke}{rgb}{0.000000,0.000000,0.000000}%
\pgfsetstrokecolor{currentstroke}%
\pgfsetdash{}{0pt}%
\pgfsys@defobject{currentmarker}{\pgfqpoint{-0.048611in}{0.000000in}}{\pgfqpoint{-0.000000in}{0.000000in}}{%
\pgfpathmoveto{\pgfqpoint{-0.000000in}{0.000000in}}%
\pgfpathlineto{\pgfqpoint{-0.048611in}{0.000000in}}%
\pgfusepath{stroke,fill}%
}%
\begin{pgfscope}%
\pgfsys@transformshift{1.246638in}{7.114368in}%
\pgfsys@useobject{currentmarker}{}%
\end{pgfscope}%
\end{pgfscope}%
\begin{pgfscope}%
\definecolor{textcolor}{rgb}{0.000000,0.000000,0.000000}%
\pgfsetstrokecolor{textcolor}%
\pgfsetfillcolor{textcolor}%
\pgftext[x=0.587096in, y=6.947220in, left, base]{\color{textcolor}\rmfamily\fontsize{31.680000}{38.016000}\selectfont \(\displaystyle {0.5}\)}%
\end{pgfscope}%
\begin{pgfscope}%
\pgfsetbuttcap%
\pgfsetroundjoin%
\definecolor{currentfill}{rgb}{0.000000,0.000000,0.000000}%
\pgfsetfillcolor{currentfill}%
\pgfsetlinewidth{0.803000pt}%
\definecolor{currentstroke}{rgb}{0.000000,0.000000,0.000000}%
\pgfsetstrokecolor{currentstroke}%
\pgfsetdash{}{0pt}%
\pgfsys@defobject{currentmarker}{\pgfqpoint{-0.048611in}{0.000000in}}{\pgfqpoint{-0.000000in}{0.000000in}}{%
\pgfpathmoveto{\pgfqpoint{-0.000000in}{0.000000in}}%
\pgfpathlineto{\pgfqpoint{-0.048611in}{0.000000in}}%
\pgfusepath{stroke,fill}%
}%
\begin{pgfscope}%
\pgfsys@transformshift{1.246638in}{8.316358in}%
\pgfsys@useobject{currentmarker}{}%
\end{pgfscope}%
\end{pgfscope}%
\begin{pgfscope}%
\definecolor{textcolor}{rgb}{0.000000,0.000000,0.000000}%
\pgfsetstrokecolor{textcolor}%
\pgfsetfillcolor{textcolor}%
\pgftext[x=0.587096in, y=8.149210in, left, base]{\color{textcolor}\rmfamily\fontsize{31.680000}{38.016000}\selectfont \(\displaystyle {0.6}\)}%
\end{pgfscope}%
\begin{pgfscope}%
\pgfsetbuttcap%
\pgfsetroundjoin%
\definecolor{currentfill}{rgb}{0.000000,0.000000,0.000000}%
\pgfsetfillcolor{currentfill}%
\pgfsetlinewidth{0.803000pt}%
\definecolor{currentstroke}{rgb}{0.000000,0.000000,0.000000}%
\pgfsetstrokecolor{currentstroke}%
\pgfsetdash{}{0pt}%
\pgfsys@defobject{currentmarker}{\pgfqpoint{-0.048611in}{0.000000in}}{\pgfqpoint{-0.000000in}{0.000000in}}{%
\pgfpathmoveto{\pgfqpoint{-0.000000in}{0.000000in}}%
\pgfpathlineto{\pgfqpoint{-0.048611in}{0.000000in}}%
\pgfusepath{stroke,fill}%
}%
\begin{pgfscope}%
\pgfsys@transformshift{1.246638in}{9.518348in}%
\pgfsys@useobject{currentmarker}{}%
\end{pgfscope}%
\end{pgfscope}%
\begin{pgfscope}%
\definecolor{textcolor}{rgb}{0.000000,0.000000,0.000000}%
\pgfsetstrokecolor{textcolor}%
\pgfsetfillcolor{textcolor}%
\pgftext[x=0.581376in, y=9.351200in, left, base]{\color{textcolor}\rmfamily\fontsize{31.680000}{38.016000}\selectfont \(\displaystyle {0.7}\)}%
\end{pgfscope}%
\begin{pgfscope}%
\pgfsetbuttcap%
\pgfsetroundjoin%
\definecolor{currentfill}{rgb}{0.000000,0.000000,0.000000}%
\pgfsetfillcolor{currentfill}%
\pgfsetlinewidth{0.803000pt}%
\definecolor{currentstroke}{rgb}{0.000000,0.000000,0.000000}%
\pgfsetstrokecolor{currentstroke}%
\pgfsetdash{}{0pt}%
\pgfsys@defobject{currentmarker}{\pgfqpoint{-0.048611in}{0.000000in}}{\pgfqpoint{-0.000000in}{0.000000in}}{%
\pgfpathmoveto{\pgfqpoint{-0.000000in}{0.000000in}}%
\pgfpathlineto{\pgfqpoint{-0.048611in}{0.000000in}}%
\pgfusepath{stroke,fill}%
}%
\begin{pgfscope}%
\pgfsys@transformshift{1.246638in}{10.720338in}%
\pgfsys@useobject{currentmarker}{}%
\end{pgfscope}%
\end{pgfscope}%
\begin{pgfscope}%
\definecolor{textcolor}{rgb}{0.000000,0.000000,0.000000}%
\pgfsetstrokecolor{textcolor}%
\pgfsetfillcolor{textcolor}%
\pgftext[x=0.587096in, y=10.553190in, left, base]{\color{textcolor}\rmfamily\fontsize{31.680000}{38.016000}\selectfont \(\displaystyle {0.8}\)}%
\end{pgfscope}%
\begin{pgfscope}%
\definecolor{textcolor}{rgb}{0.000000,0.000000,0.000000}%
\pgfsetstrokecolor{textcolor}%
\pgfsetfillcolor{textcolor}%
\pgftext[x=0.525820in,y=6.389418in,,bottom,rotate=90.000000]{\color{textcolor}\rmfamily\fontsize{31.680000}{38.016000}\selectfont Sagat score average}%
\end{pgfscope}%
\begin{pgfscope}%
\pgfpathrectangle{\pgfqpoint{1.246638in}{1.104418in}}{\pgfqpoint{10.850000in}{10.570000in}}%
\pgfusepath{clip}%
\pgfsetrectcap%
\pgfsetroundjoin%
\pgfsetlinewidth{2.710125pt}%
\definecolor{currentstroke}{rgb}{0.260000,0.260000,0.260000}%
\pgfsetstrokecolor{currentstroke}%
\pgfsetdash{}{0pt}%
\pgfpathmoveto{\pgfqpoint{2.060388in}{5.611881in}}%
\pgfpathlineto{\pgfqpoint{2.060388in}{9.518348in}}%
\pgfusepath{stroke}%
\end{pgfscope}%
\begin{pgfscope}%
\pgfpathrectangle{\pgfqpoint{1.246638in}{1.104418in}}{\pgfqpoint{10.850000in}{10.570000in}}%
\pgfusepath{clip}%
\pgfsetrectcap%
\pgfsetroundjoin%
\pgfsetlinewidth{2.710125pt}%
\definecolor{currentstroke}{rgb}{0.260000,0.260000,0.260000}%
\pgfsetstrokecolor{currentstroke}%
\pgfsetdash{}{0pt}%
\pgfpathmoveto{\pgfqpoint{4.772888in}{5.605871in}}%
\pgfpathlineto{\pgfqpoint{4.772888in}{9.608498in}}%
\pgfusepath{stroke}%
\end{pgfscope}%
\begin{pgfscope}%
\pgfpathrectangle{\pgfqpoint{1.246638in}{1.104418in}}{\pgfqpoint{10.850000in}{10.570000in}}%
\pgfusepath{clip}%
\pgfsetrectcap%
\pgfsetroundjoin%
\pgfsetlinewidth{2.710125pt}%
\definecolor{currentstroke}{rgb}{0.260000,0.260000,0.260000}%
\pgfsetstrokecolor{currentstroke}%
\pgfsetdash{}{0pt}%
\pgfpathmoveto{\pgfqpoint{7.485388in}{4.701373in}}%
\pgfpathlineto{\pgfqpoint{7.485388in}{7.703343in}}%
\pgfusepath{stroke}%
\end{pgfscope}%
\begin{pgfscope}%
\pgfpathrectangle{\pgfqpoint{1.246638in}{1.104418in}}{\pgfqpoint{10.850000in}{10.570000in}}%
\pgfusepath{clip}%
\pgfsetrectcap%
\pgfsetroundjoin%
\pgfsetlinewidth{2.710125pt}%
\definecolor{currentstroke}{rgb}{0.260000,0.260000,0.260000}%
\pgfsetstrokecolor{currentstroke}%
\pgfsetdash{}{0pt}%
\pgfpathmoveto{\pgfqpoint{10.197888in}{6.738746in}}%
\pgfpathlineto{\pgfqpoint{10.197888in}{10.494965in}}%
\pgfusepath{stroke}%
\end{pgfscope}%
\begin{pgfscope}%
\pgfpathrectangle{\pgfqpoint{1.246638in}{1.104418in}}{\pgfqpoint{10.850000in}{10.570000in}}%
\pgfusepath{clip}%
\pgfsetrectcap%
\pgfsetroundjoin%
\pgfsetlinewidth{2.710125pt}%
\definecolor{currentstroke}{rgb}{0.260000,0.260000,0.260000}%
\pgfsetstrokecolor{currentstroke}%
\pgfsetdash{}{0pt}%
\pgfpathmoveto{\pgfqpoint{3.145388in}{6.800724in}}%
\pgfpathlineto{\pgfqpoint{3.145388in}{9.893970in}}%
\pgfusepath{stroke}%
\end{pgfscope}%
\begin{pgfscope}%
\pgfpathrectangle{\pgfqpoint{1.246638in}{1.104418in}}{\pgfqpoint{10.850000in}{10.570000in}}%
\pgfusepath{clip}%
\pgfsetrectcap%
\pgfsetroundjoin%
\pgfsetlinewidth{2.710125pt}%
\definecolor{currentstroke}{rgb}{0.260000,0.260000,0.260000}%
\pgfsetstrokecolor{currentstroke}%
\pgfsetdash{}{0pt}%
\pgfpathmoveto{\pgfqpoint{5.857888in}{5.897353in}}%
\pgfpathlineto{\pgfqpoint{5.857888in}{11.171085in}}%
\pgfusepath{stroke}%
\end{pgfscope}%
\begin{pgfscope}%
\pgfpathrectangle{\pgfqpoint{1.246638in}{1.104418in}}{\pgfqpoint{10.850000in}{10.570000in}}%
\pgfusepath{clip}%
\pgfsetrectcap%
\pgfsetroundjoin%
\pgfsetlinewidth{2.710125pt}%
\definecolor{currentstroke}{rgb}{0.260000,0.260000,0.260000}%
\pgfsetstrokecolor{currentstroke}%
\pgfsetdash{}{0pt}%
\pgfpathmoveto{\pgfqpoint{8.570388in}{4.785513in}}%
\pgfpathlineto{\pgfqpoint{8.570388in}{9.824480in}}%
\pgfusepath{stroke}%
\end{pgfscope}%
\begin{pgfscope}%
\pgfpathrectangle{\pgfqpoint{1.246638in}{1.104418in}}{\pgfqpoint{10.850000in}{10.570000in}}%
\pgfusepath{clip}%
\pgfsetrectcap%
\pgfsetroundjoin%
\pgfsetlinewidth{2.710125pt}%
\definecolor{currentstroke}{rgb}{0.260000,0.260000,0.260000}%
\pgfsetstrokecolor{currentstroke}%
\pgfsetdash{}{0pt}%
\pgfpathmoveto{\pgfqpoint{11.282888in}{7.264617in}}%
\pgfpathlineto{\pgfqpoint{11.282888in}{9.818846in}}%
\pgfusepath{stroke}%
\end{pgfscope}%
\begin{pgfscope}%
\pgfsetrectcap%
\pgfsetmiterjoin%
\pgfsetlinewidth{0.803000pt}%
\definecolor{currentstroke}{rgb}{0.000000,0.000000,0.000000}%
\pgfsetstrokecolor{currentstroke}%
\pgfsetdash{}{0pt}%
\pgfpathmoveto{\pgfqpoint{1.246638in}{1.104418in}}%
\pgfpathlineto{\pgfqpoint{1.246638in}{11.674418in}}%
\pgfusepath{stroke}%
\end{pgfscope}%
\begin{pgfscope}%
\pgfsetrectcap%
\pgfsetmiterjoin%
\pgfsetlinewidth{0.803000pt}%
\definecolor{currentstroke}{rgb}{0.000000,0.000000,0.000000}%
\pgfsetstrokecolor{currentstroke}%
\pgfsetdash{}{0pt}%
\pgfpathmoveto{\pgfqpoint{12.096638in}{1.104418in}}%
\pgfpathlineto{\pgfqpoint{12.096638in}{11.674418in}}%
\pgfusepath{stroke}%
\end{pgfscope}%
\begin{pgfscope}%
\pgfsetrectcap%
\pgfsetmiterjoin%
\pgfsetlinewidth{0.803000pt}%
\definecolor{currentstroke}{rgb}{0.000000,0.000000,0.000000}%
\pgfsetstrokecolor{currentstroke}%
\pgfsetdash{}{0pt}%
\pgfpathmoveto{\pgfqpoint{1.246638in}{1.104418in}}%
\pgfpathlineto{\pgfqpoint{12.096638in}{1.104418in}}%
\pgfusepath{stroke}%
\end{pgfscope}%
\begin{pgfscope}%
\pgfsetrectcap%
\pgfsetmiterjoin%
\pgfsetlinewidth{0.803000pt}%
\definecolor{currentstroke}{rgb}{0.000000,0.000000,0.000000}%
\pgfsetstrokecolor{currentstroke}%
\pgfsetdash{}{0pt}%
\pgfpathmoveto{\pgfqpoint{1.246638in}{11.674418in}}%
\pgfpathlineto{\pgfqpoint{12.096638in}{11.674418in}}%
\pgfusepath{stroke}%
\end{pgfscope}%
\begin{pgfscope}%
\definecolor{textcolor}{rgb}{0.000000,0.000000,0.000000}%
\pgfsetstrokecolor{textcolor}%
\pgfsetfillcolor{textcolor}%
\pgftext[x=6.671638in,y=11.791723in,,base]{\color{textcolor}\rmfamily\fontsize{38.016000}{45.619200}\selectfont Sagat score average for sight users}%
\end{pgfscope}%
\begin{pgfscope}%
\pgfsetbuttcap%
\pgfsetmiterjoin%
\definecolor{currentfill}{rgb}{1.000000,1.000000,1.000000}%
\pgfsetfillcolor{currentfill}%
\pgfsetfillopacity{0.800000}%
\pgfsetlinewidth{1.003750pt}%
\definecolor{currentstroke}{rgb}{0.800000,0.800000,0.800000}%
\pgfsetstrokecolor{currentstroke}%
\pgfsetstrokeopacity{0.800000}%
\pgfsetdash{}{0pt}%
\pgfpathmoveto{\pgfqpoint{9.984425in}{13.433979in}}%
\pgfpathlineto{\pgfqpoint{12.035527in}{13.433979in}}%
\pgfpathquadraticcurveto{\pgfqpoint{12.096638in}{13.433979in}}{\pgfqpoint{12.096638in}{13.495090in}}%
\pgfpathlineto{\pgfqpoint{12.096638in}{14.361507in}}%
\pgfpathquadraticcurveto{\pgfqpoint{12.096638in}{14.422618in}}{\pgfqpoint{12.035527in}{14.422618in}}%
\pgfpathlineto{\pgfqpoint{9.984425in}{14.422618in}}%
\pgfpathquadraticcurveto{\pgfqpoint{9.923314in}{14.422618in}}{\pgfqpoint{9.923314in}{14.361507in}}%
\pgfpathlineto{\pgfqpoint{9.923314in}{13.495090in}}%
\pgfpathquadraticcurveto{\pgfqpoint{9.923314in}{13.433979in}}{\pgfqpoint{9.984425in}{13.433979in}}%
\pgfpathclose%
\pgfusepath{stroke,fill}%
\end{pgfscope}%
\begin{pgfscope}%
\pgfsetbuttcap%
\pgfsetmiterjoin%
\definecolor{currentfill}{rgb}{0.651961,0.093137,0.093137}%
\pgfsetfillcolor{currentfill}%
\pgfsetlinewidth{0.000000pt}%
\definecolor{currentstroke}{rgb}{0.000000,0.000000,0.000000}%
\pgfsetstrokecolor{currentstroke}%
\pgfsetstrokeopacity{0.000000}%
\pgfsetdash{}{0pt}%
\pgfpathmoveto{\pgfqpoint{10.045536in}{14.068245in}}%
\pgfpathlineto{\pgfqpoint{10.656647in}{14.068245in}}%
\pgfpathlineto{\pgfqpoint{10.656647in}{14.282134in}}%
\pgfpathlineto{\pgfqpoint{10.045536in}{14.282134in}}%
\pgfpathclose%
\pgfusepath{fill}%
\end{pgfscope}%
\begin{pgfscope}%
\definecolor{textcolor}{rgb}{0.000000,0.000000,0.000000}%
\pgfsetstrokecolor{textcolor}%
\pgfsetfillcolor{textcolor}%
\pgftext[x=10.901092in,y=14.068245in,left,base]{\color{textcolor}\rmfamily\fontsize{22.000000}{26.400000}\selectfont First}%
\end{pgfscope}%
\begin{pgfscope}%
\pgfsetbuttcap%
\pgfsetmiterjoin%
\definecolor{currentfill}{rgb}{0.144608,0.218137,0.424020}%
\pgfsetfillcolor{currentfill}%
\pgfsetlinewidth{0.000000pt}%
\definecolor{currentstroke}{rgb}{0.000000,0.000000,0.000000}%
\pgfsetstrokecolor{currentstroke}%
\pgfsetstrokeopacity{0.000000}%
\pgfsetdash{}{0pt}%
\pgfpathmoveto{\pgfqpoint{10.045536in}{13.619759in}}%
\pgfpathlineto{\pgfqpoint{10.656647in}{13.619759in}}%
\pgfpathlineto{\pgfqpoint{10.656647in}{13.833648in}}%
\pgfpathlineto{\pgfqpoint{10.045536in}{13.833648in}}%
\pgfpathclose%
\pgfusepath{fill}%
\end{pgfscope}%
\begin{pgfscope}%
\definecolor{textcolor}{rgb}{0.000000,0.000000,0.000000}%
\pgfsetstrokecolor{textcolor}%
\pgfsetfillcolor{textcolor}%
\pgftext[x=10.901092in,y=13.619759in,left,base]{\color{textcolor}\rmfamily\fontsize{22.000000}{26.400000}\selectfont Return}%
\end{pgfscope}%
\end{pgfpicture}%
\makeatother%
\endgroup%
    
        %}
        \caption{Bar plot of the average Sagat score of the sighted participants on each method.}
        \label{fig:barplot_sagat_scene_sight}
    \end{minipage}
\end{figure}

The Figure \ref{fig:boxplot_sagat_scene} shows the Sagat score between the rounds of each participant. It is possible to assume only that some methods cause different Sagat scores than others, and in the global average the "blind" samnple had a better Sagat score than the "sight" sample, as also shown in the Figure \ref{fig:barplot_sagat}, which has the above data plotted, without considering the "Base" method.

%
\begin{table}[!htb]
\centering
\caption{Adapted Sagat average global score grouped by participant and guidance method.}
\label{tab:sagat_average}
\begin{tabular}{lrrrrrr}
\toprule
{} &   Base & Audio & \begin{tabular}[c]{@{}l@{}}Haptic\\ Belt\end{tabular} & \begin{tabular}[c]{@{}l@{}}Virtual\\ Cane\end{tabular} & Mixture & \begin{tabular}[c]{@{}l@{}}Visual\\ Condition\end{tabular} \\
Participant &        &       &                                                       &                                                        &         &                                                            \\
\midrule
001         &  10.00 &  5.25 &                                                  4.67 &                                                   3.83 &    5.50 &                                                      Sight \\
001C        &   6.25 &  6.00 &                                                  6.92 &                                                   5.67 &    4.50 &                                                      Blind \\
002C        &   6.00 &  4.75 &                                                  4.00 &                                                   5.50 &    7.38 &                                                      Blind \\
003         &  10.00 &  6.38 &                                                  6.62 &                                                   5.12 &    7.12 &                                                      Sight \\
003C        &   8.62 &  8.75 &                                                  8.00 &                                                   6.83 &    9.00 &                                                      Blind \\
004         &  10.00 &  7.50 &                                                  8.75 &                                                   7.12 &    7.62 &                                                      Sight \\
004C        &   8.25 &  6.00 &                                                  8.46 &                                                   6.12 &    7.75 &                                                      Blind \\
005         &  10.00 &  3.38 &                                                  3.08 &                                                   3.00 &    5.00 &                                                      Sight \\
\bottomrule
\end{tabular}
\end{table}



The Tables \ref{tab:sagat_average_group} and \ref{tab:sagat_var_average_group} shows the average and the variation between the rounds of the Sagat score grouped by visual condition, both of then present that the "blind" sample had a better performance than the "sight" sample.


\begin{table}[!htb]
\centering
\caption{Adapted Sagat average global score grouped by participant and visual Condition.}
\label{tab:sagat_average_group}
\begin{tabular}{lrrrrr}
\toprule
{} &  Base &  Audio &  Haptic Belt &  Virtual Cane &  Mixture \\
Visual Condition &       &        &              &               &          \\
\midrule
Blind            & 0.728 &  0.637 &        0.684 &         0.603 &    0.716 \\
Sight            & 1.000 &  0.562 &        0.578 &         0.477 &    0.631 \\
\bottomrule
\end{tabular}
\end{table}




\begin{table}[!htb]
\centering
\caption{Adapted Sagat global score variation grouped by participant and visual Condition}
\label{tab:sagat_var_average_group}
\begin{tabular}{llllll}
\toprule
{} &  Base &  Audio &  Haptic Belt &  Virtual Cane &  Mixture \\
Visual Condition &       &        &              &               &          \\
\midrule
Blind            & 8.927 & 15.657 &       23.492 &        44.302 &   32.901 \\
Sight            & 0.000 & 13.530 &       12.586 &        33.117 &    3.798 \\
\bottomrule
\end{tabular}
\end{table}



\begin{figure}[!htb]
    %\centering
    \begin{minipage}{.45\linewidth}
        \centering
        \includegraphics[width = \linewidth]{Resultados/Sagat/Figuras/png/boxplot_sagat_scene.png}
        %%% Creator: Matplotlib, PGF backend
%%
%% To include the figure in your LaTeX document, write
%%   \input{<filename>.pgf}
%%
%% Make sure the required packages are loaded in your preamble
%%   \usepackage{pgf}
%%
%% and, on pdftex
%%   \usepackage[utf8]{inputenc}\DeclareUnicodeCharacter{2212}{-}
%%
%% or, on luatex and xetex
%%   \usepackage{unicode-math}
%%
%% Figures using additional raster images can only be included by \input if
%% they are in the same directory as the main LaTeX file. For loading figures
%% from other directories you can use the `import` package
%%   \usepackage{import}
%%
%% and then include the figures with
%%   \import{<path to file>}{<filename>.pgf}
%%
%% Matplotlib used the following preamble
%%   \usepackage{url}
%%   \usepackage{unicode-math}
%%   \setmainfont{DejaVu Serif}
%%   \usepackage{fontspec}
%%
\begingroup%
\makeatletter%
\begin{pgfpicture}%
\pgfpathrectangle{\pgfpointorigin}{\pgfqpoint{12.196638in}{14.528848in}}%
\pgfusepath{use as bounding box, clip}%
\begin{pgfscope}%
\pgfsetbuttcap%
\pgfsetmiterjoin%
\pgfsetlinewidth{0.000000pt}%
\definecolor{currentstroke}{rgb}{1.000000,1.000000,1.000000}%
\pgfsetstrokecolor{currentstroke}%
\pgfsetstrokeopacity{0.000000}%
\pgfsetdash{}{0pt}%
\pgfpathmoveto{\pgfqpoint{0.000000in}{0.000000in}}%
\pgfpathlineto{\pgfqpoint{12.196638in}{0.000000in}}%
\pgfpathlineto{\pgfqpoint{12.196638in}{14.528848in}}%
\pgfpathlineto{\pgfqpoint{0.000000in}{14.528848in}}%
\pgfpathclose%
\pgfusepath{}%
\end{pgfscope}%
\begin{pgfscope}%
\pgfsetbuttcap%
\pgfsetmiterjoin%
\definecolor{currentfill}{rgb}{1.000000,1.000000,1.000000}%
\pgfsetfillcolor{currentfill}%
\pgfsetlinewidth{0.000000pt}%
\definecolor{currentstroke}{rgb}{0.000000,0.000000,0.000000}%
\pgfsetstrokecolor{currentstroke}%
\pgfsetstrokeopacity{0.000000}%
\pgfsetdash{}{0pt}%
\pgfpathmoveto{\pgfqpoint{1.246638in}{1.110648in}}%
\pgfpathlineto{\pgfqpoint{12.096638in}{1.110648in}}%
\pgfpathlineto{\pgfqpoint{12.096638in}{11.680648in}}%
\pgfpathlineto{\pgfqpoint{1.246638in}{11.680648in}}%
\pgfpathclose%
\pgfusepath{fill}%
\end{pgfscope}%
\begin{pgfscope}%
\pgfpathrectangle{\pgfqpoint{1.246638in}{1.110648in}}{\pgfqpoint{10.850000in}{10.570000in}}%
\pgfusepath{clip}%
\pgfsetbuttcap%
\pgfsetmiterjoin%
\definecolor{currentfill}{rgb}{0.651961,0.093137,0.093137}%
\pgfsetfillcolor{currentfill}%
\pgfsetlinewidth{1.505625pt}%
\definecolor{currentstroke}{rgb}{0.168627,0.168627,0.168627}%
\pgfsetstrokecolor{currentstroke}%
\pgfsetdash{}{0pt}%
\pgfpathmoveto{\pgfqpoint{1.797818in}{11.200194in}}%
\pgfpathlineto{\pgfqpoint{2.648458in}{11.200194in}}%
\pgfpathlineto{\pgfqpoint{2.648458in}{11.200194in}}%
\pgfpathlineto{\pgfqpoint{1.797818in}{11.200194in}}%
\pgfpathlineto{\pgfqpoint{1.797818in}{11.200194in}}%
\pgfpathclose%
\pgfusepath{stroke,fill}%
\end{pgfscope}%
\begin{pgfscope}%
\pgfpathrectangle{\pgfqpoint{1.246638in}{1.110648in}}{\pgfqpoint{10.850000in}{10.570000in}}%
\pgfusepath{clip}%
\pgfsetbuttcap%
\pgfsetmiterjoin%
\definecolor{currentfill}{rgb}{0.144608,0.218137,0.424020}%
\pgfsetfillcolor{currentfill}%
\pgfsetlinewidth{1.505625pt}%
\definecolor{currentstroke}{rgb}{0.168627,0.168627,0.168627}%
\pgfsetstrokecolor{currentstroke}%
\pgfsetdash{}{0pt}%
\pgfpathmoveto{\pgfqpoint{2.665818in}{4.041387in}}%
\pgfpathlineto{\pgfqpoint{3.516458in}{4.041387in}}%
\pgfpathlineto{\pgfqpoint{3.516458in}{6.613413in}}%
\pgfpathlineto{\pgfqpoint{2.665818in}{6.613413in}}%
\pgfpathlineto{\pgfqpoint{2.665818in}{4.041387in}}%
\pgfpathclose%
\pgfusepath{stroke,fill}%
\end{pgfscope}%
\begin{pgfscope}%
\pgfpathrectangle{\pgfqpoint{1.246638in}{1.110648in}}{\pgfqpoint{10.850000in}{10.570000in}}%
\pgfusepath{clip}%
\pgfsetbuttcap%
\pgfsetmiterjoin%
\definecolor{currentfill}{rgb}{0.823529,0.823529,0.823529}%
\pgfsetfillcolor{currentfill}%
\pgfsetlinewidth{1.505625pt}%
\definecolor{currentstroke}{rgb}{0.168627,0.168627,0.168627}%
\pgfsetstrokecolor{currentstroke}%
\pgfsetdash{}{0pt}%
\pgfpathmoveto{\pgfqpoint{3.533818in}{3.338366in}}%
\pgfpathlineto{\pgfqpoint{4.384458in}{3.338366in}}%
\pgfpathlineto{\pgfqpoint{4.384458in}{7.292428in}}%
\pgfpathlineto{\pgfqpoint{3.533818in}{7.292428in}}%
\pgfpathlineto{\pgfqpoint{3.533818in}{3.338366in}}%
\pgfpathclose%
\pgfusepath{stroke,fill}%
\end{pgfscope}%
\begin{pgfscope}%
\pgfpathrectangle{\pgfqpoint{1.246638in}{1.110648in}}{\pgfqpoint{10.850000in}{10.570000in}}%
\pgfusepath{clip}%
\pgfsetbuttcap%
\pgfsetmiterjoin%
\definecolor{currentfill}{rgb}{0.875000,0.419118,0.125000}%
\pgfsetfillcolor{currentfill}%
\pgfsetlinewidth{1.505625pt}%
\definecolor{currentstroke}{rgb}{0.168627,0.168627,0.168627}%
\pgfsetstrokecolor{currentstroke}%
\pgfsetdash{}{0pt}%
\pgfpathmoveto{\pgfqpoint{4.401818in}{2.450160in}}%
\pgfpathlineto{\pgfqpoint{5.252458in}{2.450160in}}%
\pgfpathlineto{\pgfqpoint{5.252458in}{5.191940in}}%
\pgfpathlineto{\pgfqpoint{4.401818in}{5.191940in}}%
\pgfpathlineto{\pgfqpoint{4.401818in}{2.450160in}}%
\pgfpathclose%
\pgfusepath{stroke,fill}%
\end{pgfscope}%
\begin{pgfscope}%
\pgfpathrectangle{\pgfqpoint{1.246638in}{1.110648in}}{\pgfqpoint{10.850000in}{10.570000in}}%
\pgfusepath{clip}%
\pgfsetbuttcap%
\pgfsetmiterjoin%
\definecolor{currentfill}{rgb}{0.696078,0.784314,0.872549}%
\pgfsetfillcolor{currentfill}%
\pgfsetlinewidth{1.505625pt}%
\definecolor{currentstroke}{rgb}{0.168627,0.168627,0.168627}%
\pgfsetstrokecolor{currentstroke}%
\pgfsetdash{}{0pt}%
\pgfpathmoveto{\pgfqpoint{5.269818in}{4.855862in}}%
\pgfpathlineto{\pgfqpoint{6.120458in}{4.855862in}}%
\pgfpathlineto{\pgfqpoint{6.120458in}{7.427888in}}%
\pgfpathlineto{\pgfqpoint{5.269818in}{7.427888in}}%
\pgfpathlineto{\pgfqpoint{5.269818in}{4.855862in}}%
\pgfpathclose%
\pgfusepath{stroke,fill}%
\end{pgfscope}%
\begin{pgfscope}%
\pgfpathrectangle{\pgfqpoint{1.246638in}{1.110648in}}{\pgfqpoint{10.850000in}{10.570000in}}%
\pgfusepath{clip}%
\pgfsetbuttcap%
\pgfsetmiterjoin%
\definecolor{currentfill}{rgb}{0.651961,0.093137,0.093137}%
\pgfsetfillcolor{currentfill}%
\pgfsetlinewidth{1.505625pt}%
\definecolor{currentstroke}{rgb}{0.168627,0.168627,0.168627}%
\pgfsetstrokecolor{currentstroke}%
\pgfsetdash{}{0pt}%
\pgfpathmoveto{\pgfqpoint{7.222818in}{5.970407in}}%
\pgfpathlineto{\pgfqpoint{8.073458in}{5.970407in}}%
\pgfpathlineto{\pgfqpoint{8.073458in}{8.928237in}}%
\pgfpathlineto{\pgfqpoint{7.222818in}{8.928237in}}%
\pgfpathlineto{\pgfqpoint{7.222818in}{5.970407in}}%
\pgfpathclose%
\pgfusepath{stroke,fill}%
\end{pgfscope}%
\begin{pgfscope}%
\pgfpathrectangle{\pgfqpoint{1.246638in}{1.110648in}}{\pgfqpoint{10.850000in}{10.570000in}}%
\pgfusepath{clip}%
\pgfsetbuttcap%
\pgfsetmiterjoin%
\definecolor{currentfill}{rgb}{0.144608,0.218137,0.424020}%
\pgfsetfillcolor{currentfill}%
\pgfsetlinewidth{1.505625pt}%
\definecolor{currentstroke}{rgb}{0.168627,0.168627,0.168627}%
\pgfsetstrokecolor{currentstroke}%
\pgfsetdash{}{0pt}%
\pgfpathmoveto{\pgfqpoint{8.090818in}{5.284533in}}%
\pgfpathlineto{\pgfqpoint{8.941458in}{5.284533in}}%
\pgfpathlineto{\pgfqpoint{8.941458in}{6.656280in}}%
\pgfpathlineto{\pgfqpoint{8.090818in}{6.656280in}}%
\pgfpathlineto{\pgfqpoint{8.090818in}{5.284533in}}%
\pgfpathclose%
\pgfusepath{stroke,fill}%
\end{pgfscope}%
\begin{pgfscope}%
\pgfpathrectangle{\pgfqpoint{1.246638in}{1.110648in}}{\pgfqpoint{10.850000in}{10.570000in}}%
\pgfusepath{clip}%
\pgfsetbuttcap%
\pgfsetmiterjoin%
\definecolor{currentfill}{rgb}{0.823529,0.823529,0.823529}%
\pgfsetfillcolor{currentfill}%
\pgfsetlinewidth{1.505625pt}%
\definecolor{currentstroke}{rgb}{0.168627,0.168627,0.168627}%
\pgfsetstrokecolor{currentstroke}%
\pgfsetdash{}{0pt}%
\pgfpathmoveto{\pgfqpoint{8.958818in}{5.966977in}}%
\pgfpathlineto{\pgfqpoint{9.809458in}{5.966977in}}%
\pgfpathlineto{\pgfqpoint{9.809458in}{8.607591in}}%
\pgfpathlineto{\pgfqpoint{8.958818in}{8.607591in}}%
\pgfpathlineto{\pgfqpoint{8.958818in}{5.966977in}}%
\pgfpathclose%
\pgfusepath{stroke,fill}%
\end{pgfscope}%
\begin{pgfscope}%
\pgfpathrectangle{\pgfqpoint{1.246638in}{1.110648in}}{\pgfqpoint{10.850000in}{10.570000in}}%
\pgfusepath{clip}%
\pgfsetbuttcap%
\pgfsetmiterjoin%
\definecolor{currentfill}{rgb}{0.875000,0.419118,0.125000}%
\pgfsetfillcolor{currentfill}%
\pgfsetlinewidth{1.505625pt}%
\definecolor{currentstroke}{rgb}{0.168627,0.168627,0.168627}%
\pgfsetstrokecolor{currentstroke}%
\pgfsetdash{}{0pt}%
\pgfpathmoveto{\pgfqpoint{9.826818in}{5.197084in}}%
\pgfpathlineto{\pgfqpoint{10.677458in}{5.197084in}}%
\pgfpathlineto{\pgfqpoint{10.677458in}{6.121299in}}%
\pgfpathlineto{\pgfqpoint{9.826818in}{6.121299in}}%
\pgfpathlineto{\pgfqpoint{9.826818in}{5.197084in}}%
\pgfpathclose%
\pgfusepath{stroke,fill}%
\end{pgfscope}%
\begin{pgfscope}%
\pgfpathrectangle{\pgfqpoint{1.246638in}{1.110648in}}{\pgfqpoint{10.850000in}{10.570000in}}%
\pgfusepath{clip}%
\pgfsetbuttcap%
\pgfsetmiterjoin%
\definecolor{currentfill}{rgb}{0.696078,0.784314,0.872549}%
\pgfsetfillcolor{currentfill}%
\pgfsetlinewidth{1.505625pt}%
\definecolor{currentstroke}{rgb}{0.168627,0.168627,0.168627}%
\pgfsetstrokecolor{currentstroke}%
\pgfsetdash{}{0pt}%
\pgfpathmoveto{\pgfqpoint{10.694818in}{6.613413in}}%
\pgfpathlineto{\pgfqpoint{11.545458in}{6.613413in}}%
\pgfpathlineto{\pgfqpoint{11.545458in}{8.542433in}}%
\pgfpathlineto{\pgfqpoint{10.694818in}{8.542433in}}%
\pgfpathlineto{\pgfqpoint{10.694818in}{6.613413in}}%
\pgfpathclose%
\pgfusepath{stroke,fill}%
\end{pgfscope}%
\begin{pgfscope}%
\pgfpathrectangle{\pgfqpoint{1.246638in}{1.110648in}}{\pgfqpoint{10.850000in}{10.570000in}}%
\pgfusepath{clip}%
\pgfsetbuttcap%
\pgfsetmiterjoin%
\definecolor{currentfill}{rgb}{0.651961,0.093137,0.093137}%
\pgfsetfillcolor{currentfill}%
\pgfsetlinewidth{0.752812pt}%
\definecolor{currentstroke}{rgb}{0.168627,0.168627,0.168627}%
\pgfsetstrokecolor{currentstroke}%
\pgfsetdash{}{0pt}%
\pgfpathmoveto{\pgfqpoint{3.959138in}{-2.517281in}}%
\pgfpathlineto{\pgfqpoint{3.959138in}{-2.517281in}}%
\pgfpathlineto{\pgfqpoint{3.959138in}{-2.517281in}}%
\pgfpathlineto{\pgfqpoint{3.959138in}{-2.517281in}}%
\pgfpathclose%
\pgfusepath{stroke,fill}%
\end{pgfscope}%
\begin{pgfscope}%
\pgfpathrectangle{\pgfqpoint{1.246638in}{1.110648in}}{\pgfqpoint{10.850000in}{10.570000in}}%
\pgfusepath{clip}%
\pgfsetbuttcap%
\pgfsetmiterjoin%
\definecolor{currentfill}{rgb}{0.144608,0.218137,0.424020}%
\pgfsetfillcolor{currentfill}%
\pgfsetlinewidth{0.752812pt}%
\definecolor{currentstroke}{rgb}{0.168627,0.168627,0.168627}%
\pgfsetstrokecolor{currentstroke}%
\pgfsetdash{}{0pt}%
\pgfpathmoveto{\pgfqpoint{3.959138in}{-2.517281in}}%
\pgfpathlineto{\pgfqpoint{3.959138in}{-2.517281in}}%
\pgfpathlineto{\pgfqpoint{3.959138in}{-2.517281in}}%
\pgfpathlineto{\pgfqpoint{3.959138in}{-2.517281in}}%
\pgfpathclose%
\pgfusepath{stroke,fill}%
\end{pgfscope}%
\begin{pgfscope}%
\pgfpathrectangle{\pgfqpoint{1.246638in}{1.110648in}}{\pgfqpoint{10.850000in}{10.570000in}}%
\pgfusepath{clip}%
\pgfsetbuttcap%
\pgfsetmiterjoin%
\definecolor{currentfill}{rgb}{0.823529,0.823529,0.823529}%
\pgfsetfillcolor{currentfill}%
\pgfsetlinewidth{0.752812pt}%
\definecolor{currentstroke}{rgb}{0.168627,0.168627,0.168627}%
\pgfsetstrokecolor{currentstroke}%
\pgfsetdash{}{0pt}%
\pgfpathmoveto{\pgfqpoint{3.959138in}{-2.517281in}}%
\pgfpathlineto{\pgfqpoint{3.959138in}{-2.517281in}}%
\pgfpathlineto{\pgfqpoint{3.959138in}{-2.517281in}}%
\pgfpathlineto{\pgfqpoint{3.959138in}{-2.517281in}}%
\pgfpathclose%
\pgfusepath{stroke,fill}%
\end{pgfscope}%
\begin{pgfscope}%
\pgfpathrectangle{\pgfqpoint{1.246638in}{1.110648in}}{\pgfqpoint{10.850000in}{10.570000in}}%
\pgfusepath{clip}%
\pgfsetbuttcap%
\pgfsetmiterjoin%
\definecolor{currentfill}{rgb}{0.875000,0.419118,0.125000}%
\pgfsetfillcolor{currentfill}%
\pgfsetlinewidth{0.752812pt}%
\definecolor{currentstroke}{rgb}{0.168627,0.168627,0.168627}%
\pgfsetstrokecolor{currentstroke}%
\pgfsetdash{}{0pt}%
\pgfpathmoveto{\pgfqpoint{3.959138in}{-2.517281in}}%
\pgfpathlineto{\pgfqpoint{3.959138in}{-2.517281in}}%
\pgfpathlineto{\pgfqpoint{3.959138in}{-2.517281in}}%
\pgfpathlineto{\pgfqpoint{3.959138in}{-2.517281in}}%
\pgfpathclose%
\pgfusepath{stroke,fill}%
\end{pgfscope}%
\begin{pgfscope}%
\pgfpathrectangle{\pgfqpoint{1.246638in}{1.110648in}}{\pgfqpoint{10.850000in}{10.570000in}}%
\pgfusepath{clip}%
\pgfsetbuttcap%
\pgfsetmiterjoin%
\definecolor{currentfill}{rgb}{0.696078,0.784314,0.872549}%
\pgfsetfillcolor{currentfill}%
\pgfsetlinewidth{0.752812pt}%
\definecolor{currentstroke}{rgb}{0.168627,0.168627,0.168627}%
\pgfsetstrokecolor{currentstroke}%
\pgfsetdash{}{0pt}%
\pgfpathmoveto{\pgfqpoint{3.959138in}{-2.517281in}}%
\pgfpathlineto{\pgfqpoint{3.959138in}{-2.517281in}}%
\pgfpathlineto{\pgfqpoint{3.959138in}{-2.517281in}}%
\pgfpathlineto{\pgfqpoint{3.959138in}{-2.517281in}}%
\pgfpathclose%
\pgfusepath{stroke,fill}%
\end{pgfscope}%
\begin{pgfscope}%
\pgfsetbuttcap%
\pgfsetroundjoin%
\definecolor{currentfill}{rgb}{0.000000,0.000000,0.000000}%
\pgfsetfillcolor{currentfill}%
\pgfsetlinewidth{0.803000pt}%
\definecolor{currentstroke}{rgb}{0.000000,0.000000,0.000000}%
\pgfsetstrokecolor{currentstroke}%
\pgfsetdash{}{0pt}%
\pgfsys@defobject{currentmarker}{\pgfqpoint{0.000000in}{-0.048611in}}{\pgfqpoint{0.000000in}{0.000000in}}{%
\pgfpathmoveto{\pgfqpoint{0.000000in}{0.000000in}}%
\pgfpathlineto{\pgfqpoint{0.000000in}{-0.048611in}}%
\pgfusepath{stroke,fill}%
}%
\begin{pgfscope}%
\pgfsys@transformshift{3.959138in}{1.110648in}%
\pgfsys@useobject{currentmarker}{}%
\end{pgfscope}%
\end{pgfscope}%
\begin{pgfscope}%
\definecolor{textcolor}{rgb}{0.000000,0.000000,0.000000}%
\pgfsetstrokecolor{textcolor}%
\pgfsetfillcolor{textcolor}%
\pgftext[x=3.959138in,y=1.013426in,,top]{\color{textcolor}\rmfamily\fontsize{31.680000}{38.016000}\selectfont Sight}%
\end{pgfscope}%
\begin{pgfscope}%
\pgfsetbuttcap%
\pgfsetroundjoin%
\definecolor{currentfill}{rgb}{0.000000,0.000000,0.000000}%
\pgfsetfillcolor{currentfill}%
\pgfsetlinewidth{0.803000pt}%
\definecolor{currentstroke}{rgb}{0.000000,0.000000,0.000000}%
\pgfsetstrokecolor{currentstroke}%
\pgfsetdash{}{0pt}%
\pgfsys@defobject{currentmarker}{\pgfqpoint{0.000000in}{-0.048611in}}{\pgfqpoint{0.000000in}{0.000000in}}{%
\pgfpathmoveto{\pgfqpoint{0.000000in}{0.000000in}}%
\pgfpathlineto{\pgfqpoint{0.000000in}{-0.048611in}}%
\pgfusepath{stroke,fill}%
}%
\begin{pgfscope}%
\pgfsys@transformshift{9.384138in}{1.110648in}%
\pgfsys@useobject{currentmarker}{}%
\end{pgfscope}%
\end{pgfscope}%
\begin{pgfscope}%
\definecolor{textcolor}{rgb}{0.000000,0.000000,0.000000}%
\pgfsetstrokecolor{textcolor}%
\pgfsetfillcolor{textcolor}%
\pgftext[x=9.384138in,y=1.013426in,,top]{\color{textcolor}\rmfamily\fontsize{31.680000}{38.016000}\selectfont Blind}%
\end{pgfscope}%
\begin{pgfscope}%
\definecolor{textcolor}{rgb}{0.000000,0.000000,0.000000}%
\pgfsetstrokecolor{textcolor}%
\pgfsetfillcolor{textcolor}%
\pgftext[x=6.671638in,y=0.525820in,,top]{\color{textcolor}\rmfamily\fontsize{31.680000}{38.016000}\selectfont Visual Condition}%
\end{pgfscope}%
\begin{pgfscope}%
\pgfsetbuttcap%
\pgfsetroundjoin%
\definecolor{currentfill}{rgb}{0.000000,0.000000,0.000000}%
\pgfsetfillcolor{currentfill}%
\pgfsetlinewidth{0.803000pt}%
\definecolor{currentstroke}{rgb}{0.000000,0.000000,0.000000}%
\pgfsetstrokecolor{currentstroke}%
\pgfsetdash{}{0pt}%
\pgfsys@defobject{currentmarker}{\pgfqpoint{-0.048611in}{0.000000in}}{\pgfqpoint{-0.000000in}{0.000000in}}{%
\pgfpathmoveto{\pgfqpoint{-0.000000in}{0.000000in}}%
\pgfpathlineto{\pgfqpoint{-0.048611in}{0.000000in}}%
\pgfusepath{stroke,fill}%
}%
\begin{pgfscope}%
\pgfsys@transformshift{1.246638in}{1.597962in}%
\pgfsys@useobject{currentmarker}{}%
\end{pgfscope}%
\end{pgfscope}%
\begin{pgfscope}%
\definecolor{textcolor}{rgb}{0.000000,0.000000,0.000000}%
\pgfsetstrokecolor{textcolor}%
\pgfsetfillcolor{textcolor}%
\pgftext[x=0.587096in, y=1.430813in, left, base]{\color{textcolor}\rmfamily\fontsize{31.680000}{38.016000}\selectfont \(\displaystyle {0.3}\)}%
\end{pgfscope}%
\begin{pgfscope}%
\pgfsetbuttcap%
\pgfsetroundjoin%
\definecolor{currentfill}{rgb}{0.000000,0.000000,0.000000}%
\pgfsetfillcolor{currentfill}%
\pgfsetlinewidth{0.803000pt}%
\definecolor{currentstroke}{rgb}{0.000000,0.000000,0.000000}%
\pgfsetstrokecolor{currentstroke}%
\pgfsetdash{}{0pt}%
\pgfsys@defobject{currentmarker}{\pgfqpoint{-0.048611in}{0.000000in}}{\pgfqpoint{-0.000000in}{0.000000in}}{%
\pgfpathmoveto{\pgfqpoint{-0.000000in}{0.000000in}}%
\pgfpathlineto{\pgfqpoint{-0.048611in}{0.000000in}}%
\pgfusepath{stroke,fill}%
}%
\begin{pgfscope}%
\pgfsys@transformshift{1.246638in}{2.969709in}%
\pgfsys@useobject{currentmarker}{}%
\end{pgfscope}%
\end{pgfscope}%
\begin{pgfscope}%
\definecolor{textcolor}{rgb}{0.000000,0.000000,0.000000}%
\pgfsetstrokecolor{textcolor}%
\pgfsetfillcolor{textcolor}%
\pgftext[x=0.587096in, y=2.802561in, left, base]{\color{textcolor}\rmfamily\fontsize{31.680000}{38.016000}\selectfont \(\displaystyle {0.4}\)}%
\end{pgfscope}%
\begin{pgfscope}%
\pgfsetbuttcap%
\pgfsetroundjoin%
\definecolor{currentfill}{rgb}{0.000000,0.000000,0.000000}%
\pgfsetfillcolor{currentfill}%
\pgfsetlinewidth{0.803000pt}%
\definecolor{currentstroke}{rgb}{0.000000,0.000000,0.000000}%
\pgfsetstrokecolor{currentstroke}%
\pgfsetdash{}{0pt}%
\pgfsys@defobject{currentmarker}{\pgfqpoint{-0.048611in}{0.000000in}}{\pgfqpoint{-0.000000in}{0.000000in}}{%
\pgfpathmoveto{\pgfqpoint{-0.000000in}{0.000000in}}%
\pgfpathlineto{\pgfqpoint{-0.048611in}{0.000000in}}%
\pgfusepath{stroke,fill}%
}%
\begin{pgfscope}%
\pgfsys@transformshift{1.246638in}{4.341457in}%
\pgfsys@useobject{currentmarker}{}%
\end{pgfscope}%
\end{pgfscope}%
\begin{pgfscope}%
\definecolor{textcolor}{rgb}{0.000000,0.000000,0.000000}%
\pgfsetstrokecolor{textcolor}%
\pgfsetfillcolor{textcolor}%
\pgftext[x=0.587096in, y=4.174308in, left, base]{\color{textcolor}\rmfamily\fontsize{31.680000}{38.016000}\selectfont \(\displaystyle {0.5}\)}%
\end{pgfscope}%
\begin{pgfscope}%
\pgfsetbuttcap%
\pgfsetroundjoin%
\definecolor{currentfill}{rgb}{0.000000,0.000000,0.000000}%
\pgfsetfillcolor{currentfill}%
\pgfsetlinewidth{0.803000pt}%
\definecolor{currentstroke}{rgb}{0.000000,0.000000,0.000000}%
\pgfsetstrokecolor{currentstroke}%
\pgfsetdash{}{0pt}%
\pgfsys@defobject{currentmarker}{\pgfqpoint{-0.048611in}{0.000000in}}{\pgfqpoint{-0.000000in}{0.000000in}}{%
\pgfpathmoveto{\pgfqpoint{-0.000000in}{0.000000in}}%
\pgfpathlineto{\pgfqpoint{-0.048611in}{0.000000in}}%
\pgfusepath{stroke,fill}%
}%
\begin{pgfscope}%
\pgfsys@transformshift{1.246638in}{5.713204in}%
\pgfsys@useobject{currentmarker}{}%
\end{pgfscope}%
\end{pgfscope}%
\begin{pgfscope}%
\definecolor{textcolor}{rgb}{0.000000,0.000000,0.000000}%
\pgfsetstrokecolor{textcolor}%
\pgfsetfillcolor{textcolor}%
\pgftext[x=0.587096in, y=5.546056in, left, base]{\color{textcolor}\rmfamily\fontsize{31.680000}{38.016000}\selectfont \(\displaystyle {0.6}\)}%
\end{pgfscope}%
\begin{pgfscope}%
\pgfsetbuttcap%
\pgfsetroundjoin%
\definecolor{currentfill}{rgb}{0.000000,0.000000,0.000000}%
\pgfsetfillcolor{currentfill}%
\pgfsetlinewidth{0.803000pt}%
\definecolor{currentstroke}{rgb}{0.000000,0.000000,0.000000}%
\pgfsetstrokecolor{currentstroke}%
\pgfsetdash{}{0pt}%
\pgfsys@defobject{currentmarker}{\pgfqpoint{-0.048611in}{0.000000in}}{\pgfqpoint{-0.000000in}{0.000000in}}{%
\pgfpathmoveto{\pgfqpoint{-0.000000in}{0.000000in}}%
\pgfpathlineto{\pgfqpoint{-0.048611in}{0.000000in}}%
\pgfusepath{stroke,fill}%
}%
\begin{pgfscope}%
\pgfsys@transformshift{1.246638in}{7.084952in}%
\pgfsys@useobject{currentmarker}{}%
\end{pgfscope}%
\end{pgfscope}%
\begin{pgfscope}%
\definecolor{textcolor}{rgb}{0.000000,0.000000,0.000000}%
\pgfsetstrokecolor{textcolor}%
\pgfsetfillcolor{textcolor}%
\pgftext[x=0.581376in, y=6.917803in, left, base]{\color{textcolor}\rmfamily\fontsize{31.680000}{38.016000}\selectfont \(\displaystyle {0.7}\)}%
\end{pgfscope}%
\begin{pgfscope}%
\pgfsetbuttcap%
\pgfsetroundjoin%
\definecolor{currentfill}{rgb}{0.000000,0.000000,0.000000}%
\pgfsetfillcolor{currentfill}%
\pgfsetlinewidth{0.803000pt}%
\definecolor{currentstroke}{rgb}{0.000000,0.000000,0.000000}%
\pgfsetstrokecolor{currentstroke}%
\pgfsetdash{}{0pt}%
\pgfsys@defobject{currentmarker}{\pgfqpoint{-0.048611in}{0.000000in}}{\pgfqpoint{-0.000000in}{0.000000in}}{%
\pgfpathmoveto{\pgfqpoint{-0.000000in}{0.000000in}}%
\pgfpathlineto{\pgfqpoint{-0.048611in}{0.000000in}}%
\pgfusepath{stroke,fill}%
}%
\begin{pgfscope}%
\pgfsys@transformshift{1.246638in}{8.456699in}%
\pgfsys@useobject{currentmarker}{}%
\end{pgfscope}%
\end{pgfscope}%
\begin{pgfscope}%
\definecolor{textcolor}{rgb}{0.000000,0.000000,0.000000}%
\pgfsetstrokecolor{textcolor}%
\pgfsetfillcolor{textcolor}%
\pgftext[x=0.587096in, y=8.289551in, left, base]{\color{textcolor}\rmfamily\fontsize{31.680000}{38.016000}\selectfont \(\displaystyle {0.8}\)}%
\end{pgfscope}%
\begin{pgfscope}%
\pgfsetbuttcap%
\pgfsetroundjoin%
\definecolor{currentfill}{rgb}{0.000000,0.000000,0.000000}%
\pgfsetfillcolor{currentfill}%
\pgfsetlinewidth{0.803000pt}%
\definecolor{currentstroke}{rgb}{0.000000,0.000000,0.000000}%
\pgfsetstrokecolor{currentstroke}%
\pgfsetdash{}{0pt}%
\pgfsys@defobject{currentmarker}{\pgfqpoint{-0.048611in}{0.000000in}}{\pgfqpoint{-0.000000in}{0.000000in}}{%
\pgfpathmoveto{\pgfqpoint{-0.000000in}{0.000000in}}%
\pgfpathlineto{\pgfqpoint{-0.048611in}{0.000000in}}%
\pgfusepath{stroke,fill}%
}%
\begin{pgfscope}%
\pgfsys@transformshift{1.246638in}{9.828446in}%
\pgfsys@useobject{currentmarker}{}%
\end{pgfscope}%
\end{pgfscope}%
\begin{pgfscope}%
\definecolor{textcolor}{rgb}{0.000000,0.000000,0.000000}%
\pgfsetstrokecolor{textcolor}%
\pgfsetfillcolor{textcolor}%
\pgftext[x=0.587096in, y=9.661298in, left, base]{\color{textcolor}\rmfamily\fontsize{31.680000}{38.016000}\selectfont \(\displaystyle {0.9}\)}%
\end{pgfscope}%
\begin{pgfscope}%
\pgfsetbuttcap%
\pgfsetroundjoin%
\definecolor{currentfill}{rgb}{0.000000,0.000000,0.000000}%
\pgfsetfillcolor{currentfill}%
\pgfsetlinewidth{0.803000pt}%
\definecolor{currentstroke}{rgb}{0.000000,0.000000,0.000000}%
\pgfsetstrokecolor{currentstroke}%
\pgfsetdash{}{0pt}%
\pgfsys@defobject{currentmarker}{\pgfqpoint{-0.048611in}{0.000000in}}{\pgfqpoint{-0.000000in}{0.000000in}}{%
\pgfpathmoveto{\pgfqpoint{-0.000000in}{0.000000in}}%
\pgfpathlineto{\pgfqpoint{-0.048611in}{0.000000in}}%
\pgfusepath{stroke,fill}%
}%
\begin{pgfscope}%
\pgfsys@transformshift{1.246638in}{11.200194in}%
\pgfsys@useobject{currentmarker}{}%
\end{pgfscope}%
\end{pgfscope}%
\begin{pgfscope}%
\definecolor{textcolor}{rgb}{0.000000,0.000000,0.000000}%
\pgfsetstrokecolor{textcolor}%
\pgfsetfillcolor{textcolor}%
\pgftext[x=0.587096in, y=11.033046in, left, base]{\color{textcolor}\rmfamily\fontsize{31.680000}{38.016000}\selectfont \(\displaystyle {1.0}\)}%
\end{pgfscope}%
\begin{pgfscope}%
\definecolor{textcolor}{rgb}{0.000000,0.000000,0.000000}%
\pgfsetstrokecolor{textcolor}%
\pgfsetfillcolor{textcolor}%
\pgftext[x=0.525820in,y=6.395648in,,bottom,rotate=90.000000]{\color{textcolor}\rmfamily\fontsize{31.680000}{38.016000}\selectfont sagat score}%
\end{pgfscope}%
\begin{pgfscope}%
\pgfpathrectangle{\pgfqpoint{1.246638in}{1.110648in}}{\pgfqpoint{10.850000in}{10.570000in}}%
\pgfusepath{clip}%
\pgfsetrectcap%
\pgfsetroundjoin%
\pgfsetlinewidth{1.505625pt}%
\definecolor{currentstroke}{rgb}{0.168627,0.168627,0.168627}%
\pgfsetstrokecolor{currentstroke}%
\pgfsetdash{}{0pt}%
\pgfpathmoveto{\pgfqpoint{2.223138in}{11.200194in}}%
\pgfpathlineto{\pgfqpoint{2.223138in}{11.200194in}}%
\pgfusepath{stroke}%
\end{pgfscope}%
\begin{pgfscope}%
\pgfpathrectangle{\pgfqpoint{1.246638in}{1.110648in}}{\pgfqpoint{10.850000in}{10.570000in}}%
\pgfusepath{clip}%
\pgfsetrectcap%
\pgfsetroundjoin%
\pgfsetlinewidth{1.505625pt}%
\definecolor{currentstroke}{rgb}{0.168627,0.168627,0.168627}%
\pgfsetstrokecolor{currentstroke}%
\pgfsetdash{}{0pt}%
\pgfpathmoveto{\pgfqpoint{2.223138in}{11.200194in}}%
\pgfpathlineto{\pgfqpoint{2.223138in}{11.200194in}}%
\pgfusepath{stroke}%
\end{pgfscope}%
\begin{pgfscope}%
\pgfpathrectangle{\pgfqpoint{1.246638in}{1.110648in}}{\pgfqpoint{10.850000in}{10.570000in}}%
\pgfusepath{clip}%
\pgfsetrectcap%
\pgfsetroundjoin%
\pgfsetlinewidth{1.505625pt}%
\definecolor{currentstroke}{rgb}{0.168627,0.168627,0.168627}%
\pgfsetstrokecolor{currentstroke}%
\pgfsetdash{}{0pt}%
\pgfpathmoveto{\pgfqpoint{2.010478in}{11.200194in}}%
\pgfpathlineto{\pgfqpoint{2.435798in}{11.200194in}}%
\pgfusepath{stroke}%
\end{pgfscope}%
\begin{pgfscope}%
\pgfpathrectangle{\pgfqpoint{1.246638in}{1.110648in}}{\pgfqpoint{10.850000in}{10.570000in}}%
\pgfusepath{clip}%
\pgfsetrectcap%
\pgfsetroundjoin%
\pgfsetlinewidth{1.505625pt}%
\definecolor{currentstroke}{rgb}{0.168627,0.168627,0.168627}%
\pgfsetstrokecolor{currentstroke}%
\pgfsetdash{}{0pt}%
\pgfpathmoveto{\pgfqpoint{2.010478in}{11.200194in}}%
\pgfpathlineto{\pgfqpoint{2.435798in}{11.200194in}}%
\pgfusepath{stroke}%
\end{pgfscope}%
\begin{pgfscope}%
\pgfpathrectangle{\pgfqpoint{1.246638in}{1.110648in}}{\pgfqpoint{10.850000in}{10.570000in}}%
\pgfusepath{clip}%
\pgfsetrectcap%
\pgfsetroundjoin%
\pgfsetlinewidth{1.505625pt}%
\definecolor{currentstroke}{rgb}{0.168627,0.168627,0.168627}%
\pgfsetstrokecolor{currentstroke}%
\pgfsetdash{}{0pt}%
\pgfpathmoveto{\pgfqpoint{3.091138in}{4.041387in}}%
\pgfpathlineto{\pgfqpoint{3.091138in}{2.112367in}}%
\pgfusepath{stroke}%
\end{pgfscope}%
\begin{pgfscope}%
\pgfpathrectangle{\pgfqpoint{1.246638in}{1.110648in}}{\pgfqpoint{10.850000in}{10.570000in}}%
\pgfusepath{clip}%
\pgfsetrectcap%
\pgfsetroundjoin%
\pgfsetlinewidth{1.505625pt}%
\definecolor{currentstroke}{rgb}{0.168627,0.168627,0.168627}%
\pgfsetstrokecolor{currentstroke}%
\pgfsetdash{}{0pt}%
\pgfpathmoveto{\pgfqpoint{3.091138in}{6.613413in}}%
\pgfpathlineto{\pgfqpoint{3.091138in}{7.770825in}}%
\pgfusepath{stroke}%
\end{pgfscope}%
\begin{pgfscope}%
\pgfpathrectangle{\pgfqpoint{1.246638in}{1.110648in}}{\pgfqpoint{10.850000in}{10.570000in}}%
\pgfusepath{clip}%
\pgfsetrectcap%
\pgfsetroundjoin%
\pgfsetlinewidth{1.505625pt}%
\definecolor{currentstroke}{rgb}{0.168627,0.168627,0.168627}%
\pgfsetstrokecolor{currentstroke}%
\pgfsetdash{}{0pt}%
\pgfpathmoveto{\pgfqpoint{2.878478in}{2.112367in}}%
\pgfpathlineto{\pgfqpoint{3.303798in}{2.112367in}}%
\pgfusepath{stroke}%
\end{pgfscope}%
\begin{pgfscope}%
\pgfpathrectangle{\pgfqpoint{1.246638in}{1.110648in}}{\pgfqpoint{10.850000in}{10.570000in}}%
\pgfusepath{clip}%
\pgfsetrectcap%
\pgfsetroundjoin%
\pgfsetlinewidth{1.505625pt}%
\definecolor{currentstroke}{rgb}{0.168627,0.168627,0.168627}%
\pgfsetstrokecolor{currentstroke}%
\pgfsetdash{}{0pt}%
\pgfpathmoveto{\pgfqpoint{2.878478in}{7.770825in}}%
\pgfpathlineto{\pgfqpoint{3.303798in}{7.770825in}}%
\pgfusepath{stroke}%
\end{pgfscope}%
\begin{pgfscope}%
\pgfpathrectangle{\pgfqpoint{1.246638in}{1.110648in}}{\pgfqpoint{10.850000in}{10.570000in}}%
\pgfusepath{clip}%
\pgfsetrectcap%
\pgfsetroundjoin%
\pgfsetlinewidth{1.505625pt}%
\definecolor{currentstroke}{rgb}{0.168627,0.168627,0.168627}%
\pgfsetstrokecolor{currentstroke}%
\pgfsetdash{}{0pt}%
\pgfpathmoveto{\pgfqpoint{3.959138in}{3.338366in}}%
\pgfpathlineto{\pgfqpoint{3.959138in}{1.707702in}}%
\pgfusepath{stroke}%
\end{pgfscope}%
\begin{pgfscope}%
\pgfpathrectangle{\pgfqpoint{1.246638in}{1.110648in}}{\pgfqpoint{10.850000in}{10.570000in}}%
\pgfusepath{clip}%
\pgfsetrectcap%
\pgfsetroundjoin%
\pgfsetlinewidth{1.505625pt}%
\definecolor{currentstroke}{rgb}{0.168627,0.168627,0.168627}%
\pgfsetstrokecolor{currentstroke}%
\pgfsetdash{}{0pt}%
\pgfpathmoveto{\pgfqpoint{3.959138in}{7.292428in}}%
\pgfpathlineto{\pgfqpoint{3.959138in}{9.478651in}}%
\pgfusepath{stroke}%
\end{pgfscope}%
\begin{pgfscope}%
\pgfpathrectangle{\pgfqpoint{1.246638in}{1.110648in}}{\pgfqpoint{10.850000in}{10.570000in}}%
\pgfusepath{clip}%
\pgfsetrectcap%
\pgfsetroundjoin%
\pgfsetlinewidth{1.505625pt}%
\definecolor{currentstroke}{rgb}{0.168627,0.168627,0.168627}%
\pgfsetstrokecolor{currentstroke}%
\pgfsetdash{}{0pt}%
\pgfpathmoveto{\pgfqpoint{3.746478in}{1.707702in}}%
\pgfpathlineto{\pgfqpoint{4.171798in}{1.707702in}}%
\pgfusepath{stroke}%
\end{pgfscope}%
\begin{pgfscope}%
\pgfpathrectangle{\pgfqpoint{1.246638in}{1.110648in}}{\pgfqpoint{10.850000in}{10.570000in}}%
\pgfusepath{clip}%
\pgfsetrectcap%
\pgfsetroundjoin%
\pgfsetlinewidth{1.505625pt}%
\definecolor{currentstroke}{rgb}{0.168627,0.168627,0.168627}%
\pgfsetstrokecolor{currentstroke}%
\pgfsetdash{}{0pt}%
\pgfpathmoveto{\pgfqpoint{3.746478in}{9.478651in}}%
\pgfpathlineto{\pgfqpoint{4.171798in}{9.478651in}}%
\pgfusepath{stroke}%
\end{pgfscope}%
\begin{pgfscope}%
\pgfpathrectangle{\pgfqpoint{1.246638in}{1.110648in}}{\pgfqpoint{10.850000in}{10.570000in}}%
\pgfusepath{clip}%
\pgfsetrectcap%
\pgfsetroundjoin%
\pgfsetlinewidth{1.505625pt}%
\definecolor{currentstroke}{rgb}{0.168627,0.168627,0.168627}%
\pgfsetstrokecolor{currentstroke}%
\pgfsetdash{}{0pt}%
\pgfpathmoveto{\pgfqpoint{4.827138in}{2.450160in}}%
\pgfpathlineto{\pgfqpoint{4.827138in}{1.591103in}}%
\pgfusepath{stroke}%
\end{pgfscope}%
\begin{pgfscope}%
\pgfpathrectangle{\pgfqpoint{1.246638in}{1.110648in}}{\pgfqpoint{10.850000in}{10.570000in}}%
\pgfusepath{clip}%
\pgfsetrectcap%
\pgfsetroundjoin%
\pgfsetlinewidth{1.505625pt}%
\definecolor{currentstroke}{rgb}{0.168627,0.168627,0.168627}%
\pgfsetstrokecolor{currentstroke}%
\pgfsetdash{}{0pt}%
\pgfpathmoveto{\pgfqpoint{4.827138in}{5.191940in}}%
\pgfpathlineto{\pgfqpoint{4.827138in}{7.249561in}}%
\pgfusepath{stroke}%
\end{pgfscope}%
\begin{pgfscope}%
\pgfpathrectangle{\pgfqpoint{1.246638in}{1.110648in}}{\pgfqpoint{10.850000in}{10.570000in}}%
\pgfusepath{clip}%
\pgfsetrectcap%
\pgfsetroundjoin%
\pgfsetlinewidth{1.505625pt}%
\definecolor{currentstroke}{rgb}{0.168627,0.168627,0.168627}%
\pgfsetstrokecolor{currentstroke}%
\pgfsetdash{}{0pt}%
\pgfpathmoveto{\pgfqpoint{4.614478in}{1.591103in}}%
\pgfpathlineto{\pgfqpoint{5.039798in}{1.591103in}}%
\pgfusepath{stroke}%
\end{pgfscope}%
\begin{pgfscope}%
\pgfpathrectangle{\pgfqpoint{1.246638in}{1.110648in}}{\pgfqpoint{10.850000in}{10.570000in}}%
\pgfusepath{clip}%
\pgfsetrectcap%
\pgfsetroundjoin%
\pgfsetlinewidth{1.505625pt}%
\definecolor{currentstroke}{rgb}{0.168627,0.168627,0.168627}%
\pgfsetstrokecolor{currentstroke}%
\pgfsetdash{}{0pt}%
\pgfpathmoveto{\pgfqpoint{4.614478in}{7.249561in}}%
\pgfpathlineto{\pgfqpoint{5.039798in}{7.249561in}}%
\pgfusepath{stroke}%
\end{pgfscope}%
\begin{pgfscope}%
\pgfpathrectangle{\pgfqpoint{1.246638in}{1.110648in}}{\pgfqpoint{10.850000in}{10.570000in}}%
\pgfusepath{clip}%
\pgfsetrectcap%
\pgfsetroundjoin%
\pgfsetlinewidth{1.505625pt}%
\definecolor{currentstroke}{rgb}{0.168627,0.168627,0.168627}%
\pgfsetstrokecolor{currentstroke}%
\pgfsetdash{}{0pt}%
\pgfpathmoveto{\pgfqpoint{5.695138in}{4.855862in}}%
\pgfpathlineto{\pgfqpoint{5.695138in}{4.341457in}}%
\pgfusepath{stroke}%
\end{pgfscope}%
\begin{pgfscope}%
\pgfpathrectangle{\pgfqpoint{1.246638in}{1.110648in}}{\pgfqpoint{10.850000in}{10.570000in}}%
\pgfusepath{clip}%
\pgfsetrectcap%
\pgfsetroundjoin%
\pgfsetlinewidth{1.505625pt}%
\definecolor{currentstroke}{rgb}{0.168627,0.168627,0.168627}%
\pgfsetstrokecolor{currentstroke}%
\pgfsetdash{}{0pt}%
\pgfpathmoveto{\pgfqpoint{5.695138in}{7.427888in}}%
\pgfpathlineto{\pgfqpoint{5.695138in}{7.942294in}}%
\pgfusepath{stroke}%
\end{pgfscope}%
\begin{pgfscope}%
\pgfpathrectangle{\pgfqpoint{1.246638in}{1.110648in}}{\pgfqpoint{10.850000in}{10.570000in}}%
\pgfusepath{clip}%
\pgfsetrectcap%
\pgfsetroundjoin%
\pgfsetlinewidth{1.505625pt}%
\definecolor{currentstroke}{rgb}{0.168627,0.168627,0.168627}%
\pgfsetstrokecolor{currentstroke}%
\pgfsetdash{}{0pt}%
\pgfpathmoveto{\pgfqpoint{5.482478in}{4.341457in}}%
\pgfpathlineto{\pgfqpoint{5.907798in}{4.341457in}}%
\pgfusepath{stroke}%
\end{pgfscope}%
\begin{pgfscope}%
\pgfpathrectangle{\pgfqpoint{1.246638in}{1.110648in}}{\pgfqpoint{10.850000in}{10.570000in}}%
\pgfusepath{clip}%
\pgfsetrectcap%
\pgfsetroundjoin%
\pgfsetlinewidth{1.505625pt}%
\definecolor{currentstroke}{rgb}{0.168627,0.168627,0.168627}%
\pgfsetstrokecolor{currentstroke}%
\pgfsetdash{}{0pt}%
\pgfpathmoveto{\pgfqpoint{5.482478in}{7.942294in}}%
\pgfpathlineto{\pgfqpoint{5.907798in}{7.942294in}}%
\pgfusepath{stroke}%
\end{pgfscope}%
\begin{pgfscope}%
\pgfpathrectangle{\pgfqpoint{1.246638in}{1.110648in}}{\pgfqpoint{10.850000in}{10.570000in}}%
\pgfusepath{clip}%
\pgfsetrectcap%
\pgfsetroundjoin%
\pgfsetlinewidth{1.505625pt}%
\definecolor{currentstroke}{rgb}{0.168627,0.168627,0.168627}%
\pgfsetstrokecolor{currentstroke}%
\pgfsetdash{}{0pt}%
\pgfpathmoveto{\pgfqpoint{7.648138in}{5.970407in}}%
\pgfpathlineto{\pgfqpoint{7.648138in}{5.713204in}}%
\pgfusepath{stroke}%
\end{pgfscope}%
\begin{pgfscope}%
\pgfpathrectangle{\pgfqpoint{1.246638in}{1.110648in}}{\pgfqpoint{10.850000in}{10.570000in}}%
\pgfusepath{clip}%
\pgfsetrectcap%
\pgfsetroundjoin%
\pgfsetlinewidth{1.505625pt}%
\definecolor{currentstroke}{rgb}{0.168627,0.168627,0.168627}%
\pgfsetstrokecolor{currentstroke}%
\pgfsetdash{}{0pt}%
\pgfpathmoveto{\pgfqpoint{7.648138in}{8.928237in}}%
\pgfpathlineto{\pgfqpoint{7.648138in}{9.314041in}}%
\pgfusepath{stroke}%
\end{pgfscope}%
\begin{pgfscope}%
\pgfpathrectangle{\pgfqpoint{1.246638in}{1.110648in}}{\pgfqpoint{10.850000in}{10.570000in}}%
\pgfusepath{clip}%
\pgfsetrectcap%
\pgfsetroundjoin%
\pgfsetlinewidth{1.505625pt}%
\definecolor{currentstroke}{rgb}{0.168627,0.168627,0.168627}%
\pgfsetstrokecolor{currentstroke}%
\pgfsetdash{}{0pt}%
\pgfpathmoveto{\pgfqpoint{7.435478in}{5.713204in}}%
\pgfpathlineto{\pgfqpoint{7.860798in}{5.713204in}}%
\pgfusepath{stroke}%
\end{pgfscope}%
\begin{pgfscope}%
\pgfpathrectangle{\pgfqpoint{1.246638in}{1.110648in}}{\pgfqpoint{10.850000in}{10.570000in}}%
\pgfusepath{clip}%
\pgfsetrectcap%
\pgfsetroundjoin%
\pgfsetlinewidth{1.505625pt}%
\definecolor{currentstroke}{rgb}{0.168627,0.168627,0.168627}%
\pgfsetstrokecolor{currentstroke}%
\pgfsetdash{}{0pt}%
\pgfpathmoveto{\pgfqpoint{7.435478in}{9.314041in}}%
\pgfpathlineto{\pgfqpoint{7.860798in}{9.314041in}}%
\pgfusepath{stroke}%
\end{pgfscope}%
\begin{pgfscope}%
\pgfpathrectangle{\pgfqpoint{1.246638in}{1.110648in}}{\pgfqpoint{10.850000in}{10.570000in}}%
\pgfusepath{clip}%
\pgfsetrectcap%
\pgfsetroundjoin%
\pgfsetlinewidth{1.505625pt}%
\definecolor{currentstroke}{rgb}{0.168627,0.168627,0.168627}%
\pgfsetstrokecolor{currentstroke}%
\pgfsetdash{}{0pt}%
\pgfpathmoveto{\pgfqpoint{8.516138in}{5.284533in}}%
\pgfpathlineto{\pgfqpoint{8.516138in}{3.998520in}}%
\pgfusepath{stroke}%
\end{pgfscope}%
\begin{pgfscope}%
\pgfpathrectangle{\pgfqpoint{1.246638in}{1.110648in}}{\pgfqpoint{10.850000in}{10.570000in}}%
\pgfusepath{clip}%
\pgfsetrectcap%
\pgfsetroundjoin%
\pgfsetlinewidth{1.505625pt}%
\definecolor{currentstroke}{rgb}{0.168627,0.168627,0.168627}%
\pgfsetstrokecolor{currentstroke}%
\pgfsetdash{}{0pt}%
\pgfpathmoveto{\pgfqpoint{8.516138in}{6.656280in}}%
\pgfpathlineto{\pgfqpoint{8.516138in}{6.656280in}}%
\pgfusepath{stroke}%
\end{pgfscope}%
\begin{pgfscope}%
\pgfpathrectangle{\pgfqpoint{1.246638in}{1.110648in}}{\pgfqpoint{10.850000in}{10.570000in}}%
\pgfusepath{clip}%
\pgfsetrectcap%
\pgfsetroundjoin%
\pgfsetlinewidth{1.505625pt}%
\definecolor{currentstroke}{rgb}{0.168627,0.168627,0.168627}%
\pgfsetstrokecolor{currentstroke}%
\pgfsetdash{}{0pt}%
\pgfpathmoveto{\pgfqpoint{8.303478in}{3.998520in}}%
\pgfpathlineto{\pgfqpoint{8.728798in}{3.998520in}}%
\pgfusepath{stroke}%
\end{pgfscope}%
\begin{pgfscope}%
\pgfpathrectangle{\pgfqpoint{1.246638in}{1.110648in}}{\pgfqpoint{10.850000in}{10.570000in}}%
\pgfusepath{clip}%
\pgfsetrectcap%
\pgfsetroundjoin%
\pgfsetlinewidth{1.505625pt}%
\definecolor{currentstroke}{rgb}{0.168627,0.168627,0.168627}%
\pgfsetstrokecolor{currentstroke}%
\pgfsetdash{}{0pt}%
\pgfpathmoveto{\pgfqpoint{8.303478in}{6.656280in}}%
\pgfpathlineto{\pgfqpoint{8.728798in}{6.656280in}}%
\pgfusepath{stroke}%
\end{pgfscope}%
\begin{pgfscope}%
\pgfpathrectangle{\pgfqpoint{1.246638in}{1.110648in}}{\pgfqpoint{10.850000in}{10.570000in}}%
\pgfusepath{clip}%
\pgfsetbuttcap%
\pgfsetmiterjoin%
\definecolor{currentfill}{rgb}{0.168627,0.168627,0.168627}%
\pgfsetfillcolor{currentfill}%
\pgfsetlinewidth{1.003750pt}%
\definecolor{currentstroke}{rgb}{0.168627,0.168627,0.168627}%
\pgfsetstrokecolor{currentstroke}%
\pgfsetdash{}{0pt}%
\pgfsys@defobject{currentmarker}{\pgfqpoint{-0.029463in}{-0.049105in}}{\pgfqpoint{0.029463in}{0.049105in}}{%
\pgfpathmoveto{\pgfqpoint{0.000000in}{-0.049105in}}%
\pgfpathlineto{\pgfqpoint{0.029463in}{0.000000in}}%
\pgfpathlineto{\pgfqpoint{0.000000in}{0.049105in}}%
\pgfpathlineto{\pgfqpoint{-0.029463in}{0.000000in}}%
\pgfpathclose%
\pgfusepath{stroke,fill}%
}%
\begin{pgfscope}%
\pgfsys@transformshift{8.516138in}{9.485510in}%
\pgfsys@useobject{currentmarker}{}%
\end{pgfscope}%
\end{pgfscope}%
\begin{pgfscope}%
\pgfpathrectangle{\pgfqpoint{1.246638in}{1.110648in}}{\pgfqpoint{10.850000in}{10.570000in}}%
\pgfusepath{clip}%
\pgfsetrectcap%
\pgfsetroundjoin%
\pgfsetlinewidth{1.505625pt}%
\definecolor{currentstroke}{rgb}{0.168627,0.168627,0.168627}%
\pgfsetstrokecolor{currentstroke}%
\pgfsetdash{}{0pt}%
\pgfpathmoveto{\pgfqpoint{9.384138in}{5.966977in}}%
\pgfpathlineto{\pgfqpoint{9.384138in}{2.962850in}}%
\pgfusepath{stroke}%
\end{pgfscope}%
\begin{pgfscope}%
\pgfpathrectangle{\pgfqpoint{1.246638in}{1.110648in}}{\pgfqpoint{10.850000in}{10.570000in}}%
\pgfusepath{clip}%
\pgfsetrectcap%
\pgfsetroundjoin%
\pgfsetlinewidth{1.505625pt}%
\definecolor{currentstroke}{rgb}{0.168627,0.168627,0.168627}%
\pgfsetstrokecolor{currentstroke}%
\pgfsetdash{}{0pt}%
\pgfpathmoveto{\pgfqpoint{9.384138in}{8.607591in}}%
\pgfpathlineto{\pgfqpoint{9.384138in}{9.080844in}}%
\pgfusepath{stroke}%
\end{pgfscope}%
\begin{pgfscope}%
\pgfpathrectangle{\pgfqpoint{1.246638in}{1.110648in}}{\pgfqpoint{10.850000in}{10.570000in}}%
\pgfusepath{clip}%
\pgfsetrectcap%
\pgfsetroundjoin%
\pgfsetlinewidth{1.505625pt}%
\definecolor{currentstroke}{rgb}{0.168627,0.168627,0.168627}%
\pgfsetstrokecolor{currentstroke}%
\pgfsetdash{}{0pt}%
\pgfpathmoveto{\pgfqpoint{9.171478in}{2.962850in}}%
\pgfpathlineto{\pgfqpoint{9.596798in}{2.962850in}}%
\pgfusepath{stroke}%
\end{pgfscope}%
\begin{pgfscope}%
\pgfpathrectangle{\pgfqpoint{1.246638in}{1.110648in}}{\pgfqpoint{10.850000in}{10.570000in}}%
\pgfusepath{clip}%
\pgfsetrectcap%
\pgfsetroundjoin%
\pgfsetlinewidth{1.505625pt}%
\definecolor{currentstroke}{rgb}{0.168627,0.168627,0.168627}%
\pgfsetstrokecolor{currentstroke}%
\pgfsetdash{}{0pt}%
\pgfpathmoveto{\pgfqpoint{9.171478in}{9.080844in}}%
\pgfpathlineto{\pgfqpoint{9.596798in}{9.080844in}}%
\pgfusepath{stroke}%
\end{pgfscope}%
\begin{pgfscope}%
\pgfpathrectangle{\pgfqpoint{1.246638in}{1.110648in}}{\pgfqpoint{10.850000in}{10.570000in}}%
\pgfusepath{clip}%
\pgfsetrectcap%
\pgfsetroundjoin%
\pgfsetlinewidth{1.505625pt}%
\definecolor{currentstroke}{rgb}{0.168627,0.168627,0.168627}%
\pgfsetstrokecolor{currentstroke}%
\pgfsetdash{}{0pt}%
\pgfpathmoveto{\pgfqpoint{10.252138in}{5.197084in}}%
\pgfpathlineto{\pgfqpoint{10.252138in}{5.027330in}}%
\pgfusepath{stroke}%
\end{pgfscope}%
\begin{pgfscope}%
\pgfpathrectangle{\pgfqpoint{1.246638in}{1.110648in}}{\pgfqpoint{10.850000in}{10.570000in}}%
\pgfusepath{clip}%
\pgfsetrectcap%
\pgfsetroundjoin%
\pgfsetlinewidth{1.505625pt}%
\definecolor{currentstroke}{rgb}{0.168627,0.168627,0.168627}%
\pgfsetstrokecolor{currentstroke}%
\pgfsetdash{}{0pt}%
\pgfpathmoveto{\pgfqpoint{10.252138in}{6.121299in}}%
\pgfpathlineto{\pgfqpoint{10.252138in}{6.851755in}}%
\pgfusepath{stroke}%
\end{pgfscope}%
\begin{pgfscope}%
\pgfpathrectangle{\pgfqpoint{1.246638in}{1.110648in}}{\pgfqpoint{10.850000in}{10.570000in}}%
\pgfusepath{clip}%
\pgfsetrectcap%
\pgfsetroundjoin%
\pgfsetlinewidth{1.505625pt}%
\definecolor{currentstroke}{rgb}{0.168627,0.168627,0.168627}%
\pgfsetstrokecolor{currentstroke}%
\pgfsetdash{}{0pt}%
\pgfpathmoveto{\pgfqpoint{10.039478in}{5.027330in}}%
\pgfpathlineto{\pgfqpoint{10.464798in}{5.027330in}}%
\pgfusepath{stroke}%
\end{pgfscope}%
\begin{pgfscope}%
\pgfpathrectangle{\pgfqpoint{1.246638in}{1.110648in}}{\pgfqpoint{10.850000in}{10.570000in}}%
\pgfusepath{clip}%
\pgfsetrectcap%
\pgfsetroundjoin%
\pgfsetlinewidth{1.505625pt}%
\definecolor{currentstroke}{rgb}{0.168627,0.168627,0.168627}%
\pgfsetstrokecolor{currentstroke}%
\pgfsetdash{}{0pt}%
\pgfpathmoveto{\pgfqpoint{10.039478in}{6.851755in}}%
\pgfpathlineto{\pgfqpoint{10.464798in}{6.851755in}}%
\pgfusepath{stroke}%
\end{pgfscope}%
\begin{pgfscope}%
\pgfpathrectangle{\pgfqpoint{1.246638in}{1.110648in}}{\pgfqpoint{10.850000in}{10.570000in}}%
\pgfusepath{clip}%
\pgfsetrectcap%
\pgfsetroundjoin%
\pgfsetlinewidth{1.505625pt}%
\definecolor{currentstroke}{rgb}{0.168627,0.168627,0.168627}%
\pgfsetstrokecolor{currentstroke}%
\pgfsetdash{}{0pt}%
\pgfpathmoveto{\pgfqpoint{11.120138in}{6.613413in}}%
\pgfpathlineto{\pgfqpoint{11.120138in}{6.613413in}}%
\pgfusepath{stroke}%
\end{pgfscope}%
\begin{pgfscope}%
\pgfpathrectangle{\pgfqpoint{1.246638in}{1.110648in}}{\pgfqpoint{10.850000in}{10.570000in}}%
\pgfusepath{clip}%
\pgfsetrectcap%
\pgfsetroundjoin%
\pgfsetlinewidth{1.505625pt}%
\definecolor{currentstroke}{rgb}{0.168627,0.168627,0.168627}%
\pgfsetstrokecolor{currentstroke}%
\pgfsetdash{}{0pt}%
\pgfpathmoveto{\pgfqpoint{11.120138in}{8.542433in}}%
\pgfpathlineto{\pgfqpoint{11.120138in}{9.828446in}}%
\pgfusepath{stroke}%
\end{pgfscope}%
\begin{pgfscope}%
\pgfpathrectangle{\pgfqpoint{1.246638in}{1.110648in}}{\pgfqpoint{10.850000in}{10.570000in}}%
\pgfusepath{clip}%
\pgfsetrectcap%
\pgfsetroundjoin%
\pgfsetlinewidth{1.505625pt}%
\definecolor{currentstroke}{rgb}{0.168627,0.168627,0.168627}%
\pgfsetstrokecolor{currentstroke}%
\pgfsetdash{}{0pt}%
\pgfpathmoveto{\pgfqpoint{10.907478in}{6.613413in}}%
\pgfpathlineto{\pgfqpoint{11.332798in}{6.613413in}}%
\pgfusepath{stroke}%
\end{pgfscope}%
\begin{pgfscope}%
\pgfpathrectangle{\pgfqpoint{1.246638in}{1.110648in}}{\pgfqpoint{10.850000in}{10.570000in}}%
\pgfusepath{clip}%
\pgfsetrectcap%
\pgfsetroundjoin%
\pgfsetlinewidth{1.505625pt}%
\definecolor{currentstroke}{rgb}{0.168627,0.168627,0.168627}%
\pgfsetstrokecolor{currentstroke}%
\pgfsetdash{}{0pt}%
\pgfpathmoveto{\pgfqpoint{10.907478in}{9.828446in}}%
\pgfpathlineto{\pgfqpoint{11.332798in}{9.828446in}}%
\pgfusepath{stroke}%
\end{pgfscope}%
\begin{pgfscope}%
\pgfpathrectangle{\pgfqpoint{1.246638in}{1.110648in}}{\pgfqpoint{10.850000in}{10.570000in}}%
\pgfusepath{clip}%
\pgfsetbuttcap%
\pgfsetmiterjoin%
\definecolor{currentfill}{rgb}{0.168627,0.168627,0.168627}%
\pgfsetfillcolor{currentfill}%
\pgfsetlinewidth{1.003750pt}%
\definecolor{currentstroke}{rgb}{0.168627,0.168627,0.168627}%
\pgfsetstrokecolor{currentstroke}%
\pgfsetdash{}{0pt}%
\pgfsys@defobject{currentmarker}{\pgfqpoint{-0.029463in}{-0.049105in}}{\pgfqpoint{0.029463in}{0.049105in}}{%
\pgfpathmoveto{\pgfqpoint{0.000000in}{-0.049105in}}%
\pgfpathlineto{\pgfqpoint{0.029463in}{0.000000in}}%
\pgfpathlineto{\pgfqpoint{0.000000in}{0.049105in}}%
\pgfpathlineto{\pgfqpoint{-0.029463in}{0.000000in}}%
\pgfpathclose%
\pgfusepath{stroke,fill}%
}%
\begin{pgfscope}%
\pgfsys@transformshift{11.120138in}{3.655583in}%
\pgfsys@useobject{currentmarker}{}%
\end{pgfscope}%
\end{pgfscope}%
\begin{pgfscope}%
\pgfpathrectangle{\pgfqpoint{1.246638in}{1.110648in}}{\pgfqpoint{10.850000in}{10.570000in}}%
\pgfusepath{clip}%
\pgfsetrectcap%
\pgfsetroundjoin%
\pgfsetlinewidth{1.505625pt}%
\definecolor{currentstroke}{rgb}{0.168627,0.168627,0.168627}%
\pgfsetstrokecolor{currentstroke}%
\pgfsetdash{}{0pt}%
\pgfpathmoveto{\pgfqpoint{1.797818in}{11.200194in}}%
\pgfpathlineto{\pgfqpoint{2.648458in}{11.200194in}}%
\pgfusepath{stroke}%
\end{pgfscope}%
\begin{pgfscope}%
\pgfpathrectangle{\pgfqpoint{1.246638in}{1.110648in}}{\pgfqpoint{10.850000in}{10.570000in}}%
\pgfusepath{clip}%
\pgfsetrectcap%
\pgfsetroundjoin%
\pgfsetlinewidth{1.505625pt}%
\definecolor{currentstroke}{rgb}{0.168627,0.168627,0.168627}%
\pgfsetstrokecolor{currentstroke}%
\pgfsetdash{}{0pt}%
\pgfpathmoveto{\pgfqpoint{2.665818in}{5.456001in}}%
\pgfpathlineto{\pgfqpoint{3.516458in}{5.456001in}}%
\pgfusepath{stroke}%
\end{pgfscope}%
\begin{pgfscope}%
\pgfpathrectangle{\pgfqpoint{1.246638in}{1.110648in}}{\pgfqpoint{10.850000in}{10.570000in}}%
\pgfusepath{clip}%
\pgfsetrectcap%
\pgfsetroundjoin%
\pgfsetlinewidth{1.505625pt}%
\definecolor{currentstroke}{rgb}{0.168627,0.168627,0.168627}%
\pgfsetstrokecolor{currentstroke}%
\pgfsetdash{}{0pt}%
\pgfpathmoveto{\pgfqpoint{3.533818in}{5.222804in}}%
\pgfpathlineto{\pgfqpoint{4.384458in}{5.222804in}}%
\pgfusepath{stroke}%
\end{pgfscope}%
\begin{pgfscope}%
\pgfpathrectangle{\pgfqpoint{1.246638in}{1.110648in}}{\pgfqpoint{10.850000in}{10.570000in}}%
\pgfusepath{clip}%
\pgfsetrectcap%
\pgfsetroundjoin%
\pgfsetlinewidth{1.505625pt}%
\definecolor{currentstroke}{rgb}{0.168627,0.168627,0.168627}%
\pgfsetstrokecolor{currentstroke}%
\pgfsetdash{}{0pt}%
\pgfpathmoveto{\pgfqpoint{4.401818in}{3.621289in}}%
\pgfpathlineto{\pgfqpoint{5.252458in}{3.621289in}}%
\pgfusepath{stroke}%
\end{pgfscope}%
\begin{pgfscope}%
\pgfpathrectangle{\pgfqpoint{1.246638in}{1.110648in}}{\pgfqpoint{10.850000in}{10.570000in}}%
\pgfusepath{clip}%
\pgfsetrectcap%
\pgfsetroundjoin%
\pgfsetlinewidth{1.505625pt}%
\definecolor{currentstroke}{rgb}{0.168627,0.168627,0.168627}%
\pgfsetstrokecolor{currentstroke}%
\pgfsetdash{}{0pt}%
\pgfpathmoveto{\pgfqpoint{5.269818in}{6.141875in}}%
\pgfpathlineto{\pgfqpoint{6.120458in}{6.141875in}}%
\pgfusepath{stroke}%
\end{pgfscope}%
\begin{pgfscope}%
\pgfpathrectangle{\pgfqpoint{1.246638in}{1.110648in}}{\pgfqpoint{10.850000in}{10.570000in}}%
\pgfusepath{clip}%
\pgfsetrectcap%
\pgfsetroundjoin%
\pgfsetlinewidth{1.505625pt}%
\definecolor{currentstroke}{rgb}{0.168627,0.168627,0.168627}%
\pgfsetstrokecolor{currentstroke}%
\pgfsetdash{}{0pt}%
\pgfpathmoveto{\pgfqpoint{7.222818in}{7.427888in}}%
\pgfpathlineto{\pgfqpoint{8.073458in}{7.427888in}}%
\pgfusepath{stroke}%
\end{pgfscope}%
\begin{pgfscope}%
\pgfpathrectangle{\pgfqpoint{1.246638in}{1.110648in}}{\pgfqpoint{10.850000in}{10.570000in}}%
\pgfusepath{clip}%
\pgfsetrectcap%
\pgfsetroundjoin%
\pgfsetlinewidth{1.505625pt}%
\definecolor{currentstroke}{rgb}{0.168627,0.168627,0.168627}%
\pgfsetstrokecolor{currentstroke}%
\pgfsetdash{}{0pt}%
\pgfpathmoveto{\pgfqpoint{8.090818in}{5.713204in}}%
\pgfpathlineto{\pgfqpoint{8.941458in}{5.713204in}}%
\pgfusepath{stroke}%
\end{pgfscope}%
\begin{pgfscope}%
\pgfpathrectangle{\pgfqpoint{1.246638in}{1.110648in}}{\pgfqpoint{10.850000in}{10.570000in}}%
\pgfusepath{clip}%
\pgfsetrectcap%
\pgfsetroundjoin%
\pgfsetlinewidth{1.505625pt}%
\definecolor{currentstroke}{rgb}{0.168627,0.168627,0.168627}%
\pgfsetstrokecolor{currentstroke}%
\pgfsetdash{}{0pt}%
\pgfpathmoveto{\pgfqpoint{8.958818in}{7.709097in}}%
\pgfpathlineto{\pgfqpoint{9.809458in}{7.709097in}}%
\pgfusepath{stroke}%
\end{pgfscope}%
\begin{pgfscope}%
\pgfpathrectangle{\pgfqpoint{1.246638in}{1.110648in}}{\pgfqpoint{10.850000in}{10.570000in}}%
\pgfusepath{clip}%
\pgfsetrectcap%
\pgfsetroundjoin%
\pgfsetlinewidth{1.505625pt}%
\definecolor{currentstroke}{rgb}{0.168627,0.168627,0.168627}%
\pgfsetstrokecolor{currentstroke}%
\pgfsetdash{}{0pt}%
\pgfpathmoveto{\pgfqpoint{9.826818in}{5.565741in}}%
\pgfpathlineto{\pgfqpoint{10.677458in}{5.565741in}}%
\pgfusepath{stroke}%
\end{pgfscope}%
\begin{pgfscope}%
\pgfpathrectangle{\pgfqpoint{1.246638in}{1.110648in}}{\pgfqpoint{10.850000in}{10.570000in}}%
\pgfusepath{clip}%
\pgfsetrectcap%
\pgfsetroundjoin%
\pgfsetlinewidth{1.505625pt}%
\definecolor{currentstroke}{rgb}{0.168627,0.168627,0.168627}%
\pgfsetstrokecolor{currentstroke}%
\pgfsetdash{}{0pt}%
\pgfpathmoveto{\pgfqpoint{10.694818in}{7.856560in}}%
\pgfpathlineto{\pgfqpoint{11.545458in}{7.856560in}}%
\pgfusepath{stroke}%
\end{pgfscope}%
\begin{pgfscope}%
\pgfsetrectcap%
\pgfsetmiterjoin%
\pgfsetlinewidth{0.803000pt}%
\definecolor{currentstroke}{rgb}{0.000000,0.000000,0.000000}%
\pgfsetstrokecolor{currentstroke}%
\pgfsetdash{}{0pt}%
\pgfpathmoveto{\pgfqpoint{1.246638in}{1.110648in}}%
\pgfpathlineto{\pgfqpoint{1.246638in}{11.680648in}}%
\pgfusepath{stroke}%
\end{pgfscope}%
\begin{pgfscope}%
\pgfsetrectcap%
\pgfsetmiterjoin%
\pgfsetlinewidth{0.803000pt}%
\definecolor{currentstroke}{rgb}{0.000000,0.000000,0.000000}%
\pgfsetstrokecolor{currentstroke}%
\pgfsetdash{}{0pt}%
\pgfpathmoveto{\pgfqpoint{12.096638in}{1.110648in}}%
\pgfpathlineto{\pgfqpoint{12.096638in}{11.680648in}}%
\pgfusepath{stroke}%
\end{pgfscope}%
\begin{pgfscope}%
\pgfsetrectcap%
\pgfsetmiterjoin%
\pgfsetlinewidth{0.803000pt}%
\definecolor{currentstroke}{rgb}{0.000000,0.000000,0.000000}%
\pgfsetstrokecolor{currentstroke}%
\pgfsetdash{}{0pt}%
\pgfpathmoveto{\pgfqpoint{1.246638in}{1.110648in}}%
\pgfpathlineto{\pgfqpoint{12.096638in}{1.110648in}}%
\pgfusepath{stroke}%
\end{pgfscope}%
\begin{pgfscope}%
\pgfsetrectcap%
\pgfsetmiterjoin%
\pgfsetlinewidth{0.803000pt}%
\definecolor{currentstroke}{rgb}{0.000000,0.000000,0.000000}%
\pgfsetstrokecolor{currentstroke}%
\pgfsetdash{}{0pt}%
\pgfpathmoveto{\pgfqpoint{1.246638in}{11.680648in}}%
\pgfpathlineto{\pgfqpoint{12.096638in}{11.680648in}}%
\pgfusepath{stroke}%
\end{pgfscope}%
\begin{pgfscope}%
\definecolor{textcolor}{rgb}{0.000000,0.000000,0.000000}%
\pgfsetstrokecolor{textcolor}%
\pgfsetfillcolor{textcolor}%
\pgftext[x=3.500415in, y=12.389169in, left, base]{\color{textcolor}\rmfamily\fontsize{38.016000}{45.619200}\selectfont Box plot comparison of }%
\end{pgfscope}%
\begin{pgfscope}%
\definecolor{textcolor}{rgb}{0.000000,0.000000,0.000000}%
\pgfsetstrokecolor{textcolor}%
\pgfsetfillcolor{textcolor}%
\pgftext[x=2.506418in, y=11.797953in, left, base]{\color{textcolor}\rmfamily\fontsize{38.016000}{45.619200}\selectfont sagat score between the users.}%
\end{pgfscope}%
\begin{pgfscope}%
\pgfsetbuttcap%
\pgfsetmiterjoin%
\definecolor{currentfill}{rgb}{1.000000,1.000000,1.000000}%
\pgfsetfillcolor{currentfill}%
\pgfsetfillopacity{0.800000}%
\pgfsetlinewidth{1.003750pt}%
\definecolor{currentstroke}{rgb}{0.800000,0.800000,0.800000}%
\pgfsetstrokecolor{currentstroke}%
\pgfsetstrokeopacity{0.800000}%
\pgfsetdash{}{0pt}%
\pgfpathmoveto{\pgfqpoint{9.108489in}{12.094751in}}%
\pgfpathlineto{\pgfqpoint{12.035527in}{12.094751in}}%
\pgfpathquadraticcurveto{\pgfqpoint{12.096638in}{12.094751in}}{\pgfqpoint{12.096638in}{12.155862in}}%
\pgfpathlineto{\pgfqpoint{12.096638in}{14.367737in}}%
\pgfpathquadraticcurveto{\pgfqpoint{12.096638in}{14.428848in}}{\pgfqpoint{12.035527in}{14.428848in}}%
\pgfpathlineto{\pgfqpoint{9.108489in}{14.428848in}}%
\pgfpathquadraticcurveto{\pgfqpoint{9.047378in}{14.428848in}}{\pgfqpoint{9.047378in}{14.367737in}}%
\pgfpathlineto{\pgfqpoint{9.047378in}{12.155862in}}%
\pgfpathquadraticcurveto{\pgfqpoint{9.047378in}{12.094751in}}{\pgfqpoint{9.108489in}{12.094751in}}%
\pgfpathclose%
\pgfusepath{stroke,fill}%
\end{pgfscope}%
\begin{pgfscope}%
\pgfsetbuttcap%
\pgfsetmiterjoin%
\definecolor{currentfill}{rgb}{0.651961,0.093137,0.093137}%
\pgfsetfillcolor{currentfill}%
\pgfsetlinewidth{0.752812pt}%
\definecolor{currentstroke}{rgb}{0.168627,0.168627,0.168627}%
\pgfsetstrokecolor{currentstroke}%
\pgfsetdash{}{0pt}%
\pgfpathmoveto{\pgfqpoint{9.169600in}{14.074476in}}%
\pgfpathlineto{\pgfqpoint{9.780712in}{14.074476in}}%
\pgfpathlineto{\pgfqpoint{9.780712in}{14.288365in}}%
\pgfpathlineto{\pgfqpoint{9.169600in}{14.288365in}}%
\pgfpathclose%
\pgfusepath{stroke,fill}%
\end{pgfscope}%
\begin{pgfscope}%
\definecolor{textcolor}{rgb}{0.000000,0.000000,0.000000}%
\pgfsetstrokecolor{textcolor}%
\pgfsetfillcolor{textcolor}%
\pgftext[x=10.025156in,y=14.074476in,left,base]{\color{textcolor}\rmfamily\fontsize{22.000000}{26.400000}\selectfont Base}%
\end{pgfscope}%
\begin{pgfscope}%
\pgfsetbuttcap%
\pgfsetmiterjoin%
\definecolor{currentfill}{rgb}{0.144608,0.218137,0.424020}%
\pgfsetfillcolor{currentfill}%
\pgfsetlinewidth{0.752812pt}%
\definecolor{currentstroke}{rgb}{0.168627,0.168627,0.168627}%
\pgfsetstrokecolor{currentstroke}%
\pgfsetdash{}{0pt}%
\pgfpathmoveto{\pgfqpoint{9.169600in}{13.625990in}}%
\pgfpathlineto{\pgfqpoint{9.780712in}{13.625990in}}%
\pgfpathlineto{\pgfqpoint{9.780712in}{13.839878in}}%
\pgfpathlineto{\pgfqpoint{9.169600in}{13.839878in}}%
\pgfpathclose%
\pgfusepath{stroke,fill}%
\end{pgfscope}%
\begin{pgfscope}%
\definecolor{textcolor}{rgb}{0.000000,0.000000,0.000000}%
\pgfsetstrokecolor{textcolor}%
\pgfsetfillcolor{textcolor}%
\pgftext[x=10.025156in,y=13.625990in,left,base]{\color{textcolor}\rmfamily\fontsize{22.000000}{26.400000}\selectfont Audio}%
\end{pgfscope}%
\begin{pgfscope}%
\pgfsetbuttcap%
\pgfsetmiterjoin%
\definecolor{currentfill}{rgb}{0.823529,0.823529,0.823529}%
\pgfsetfillcolor{currentfill}%
\pgfsetlinewidth{0.752812pt}%
\definecolor{currentstroke}{rgb}{0.168627,0.168627,0.168627}%
\pgfsetstrokecolor{currentstroke}%
\pgfsetdash{}{0pt}%
\pgfpathmoveto{\pgfqpoint{9.169600in}{13.177503in}}%
\pgfpathlineto{\pgfqpoint{9.780712in}{13.177503in}}%
\pgfpathlineto{\pgfqpoint{9.780712in}{13.391392in}}%
\pgfpathlineto{\pgfqpoint{9.169600in}{13.391392in}}%
\pgfpathclose%
\pgfusepath{stroke,fill}%
\end{pgfscope}%
\begin{pgfscope}%
\definecolor{textcolor}{rgb}{0.000000,0.000000,0.000000}%
\pgfsetstrokecolor{textcolor}%
\pgfsetfillcolor{textcolor}%
\pgftext[x=10.025156in,y=13.177503in,left,base]{\color{textcolor}\rmfamily\fontsize{22.000000}{26.400000}\selectfont Haptic Belt}%
\end{pgfscope}%
\begin{pgfscope}%
\pgfsetbuttcap%
\pgfsetmiterjoin%
\definecolor{currentfill}{rgb}{0.875000,0.419118,0.125000}%
\pgfsetfillcolor{currentfill}%
\pgfsetlinewidth{0.752812pt}%
\definecolor{currentstroke}{rgb}{0.168627,0.168627,0.168627}%
\pgfsetstrokecolor{currentstroke}%
\pgfsetdash{}{0pt}%
\pgfpathmoveto{\pgfqpoint{9.169600in}{12.729017in}}%
\pgfpathlineto{\pgfqpoint{9.780712in}{12.729017in}}%
\pgfpathlineto{\pgfqpoint{9.780712in}{12.942906in}}%
\pgfpathlineto{\pgfqpoint{9.169600in}{12.942906in}}%
\pgfpathclose%
\pgfusepath{stroke,fill}%
\end{pgfscope}%
\begin{pgfscope}%
\definecolor{textcolor}{rgb}{0.000000,0.000000,0.000000}%
\pgfsetstrokecolor{textcolor}%
\pgfsetfillcolor{textcolor}%
\pgftext[x=10.025156in,y=12.729017in,left,base]{\color{textcolor}\rmfamily\fontsize{22.000000}{26.400000}\selectfont Virtual Cane}%
\end{pgfscope}%
\begin{pgfscope}%
\pgfsetbuttcap%
\pgfsetmiterjoin%
\definecolor{currentfill}{rgb}{0.696078,0.784314,0.872549}%
\pgfsetfillcolor{currentfill}%
\pgfsetlinewidth{0.752812pt}%
\definecolor{currentstroke}{rgb}{0.168627,0.168627,0.168627}%
\pgfsetstrokecolor{currentstroke}%
\pgfsetdash{}{0pt}%
\pgfpathmoveto{\pgfqpoint{9.169600in}{12.280531in}}%
\pgfpathlineto{\pgfqpoint{9.780712in}{12.280531in}}%
\pgfpathlineto{\pgfqpoint{9.780712in}{12.494420in}}%
\pgfpathlineto{\pgfqpoint{9.169600in}{12.494420in}}%
\pgfpathclose%
\pgfusepath{stroke,fill}%
\end{pgfscope}%
\begin{pgfscope}%
\definecolor{textcolor}{rgb}{0.000000,0.000000,0.000000}%
\pgfsetstrokecolor{textcolor}%
\pgfsetfillcolor{textcolor}%
\pgftext[x=10.025156in,y=12.280531in,left,base]{\color{textcolor}\rmfamily\fontsize{22.000000}{26.400000}\selectfont Mixture}%
\end{pgfscope}%
\end{pgfpicture}%
\makeatother%
\endgroup%

        %}
        \caption{Boxplot of the average Sagat score of participant.}
        \label{fig:boxplot_sagat_scene}
    \end{minipage}
    \begin{minipage}{.1\linewidth}
        \hfill
    \end{minipage}
    \begin{minipage}{.45\linewidth}
        
        \vspace{1.8cm}
        \centering
        %\hspace{-4cm}
        \includegraphics[width = \linewidth]{Resultados/Sagat/Figuras/png/barplot_sagat_avg_global.png}
        %\resizebox{0.6\linewidth}{!}{
        %%% Creator: Matplotlib, PGF backend
%%
%% To include the figure in your LaTeX document, write
%%   \input{<filename>.pgf}
%%
%% Make sure the required packages are loaded in your preamble
%%   \usepackage{pgf}
%%
%% Figures using additional raster images can only be included by \input if
%% they are in the same directory as the main LaTeX file. For loading figures
%% from other directories you can use the `import` package
%%   \usepackage{import}
%%
%% and then include the figures with
%%   \import{<path to file>}{<filename>.pgf}
%%
%% Matplotlib used the following preamble
%%   \usepackage{url}
%%   \usepackage{unicode-math}
%%   \setmainfont{DejaVu Serif}
%%   \usepackage{fontspec}
%%
\begingroup%
\makeatletter%
\begin{pgfpicture}%
\pgfpathrectangle{\pgfpointorigin}{\pgfqpoint{14.205408in}{5.669751in}}%
\pgfusepath{use as bounding box, clip}%
\begin{pgfscope}%
\pgfsetbuttcap%
\pgfsetmiterjoin%
\pgfsetlinewidth{0.000000pt}%
\definecolor{currentstroke}{rgb}{1.000000,1.000000,1.000000}%
\pgfsetstrokecolor{currentstroke}%
\pgfsetstrokeopacity{0.000000}%
\pgfsetdash{}{0pt}%
\pgfpathmoveto{\pgfqpoint{0.000000in}{0.000000in}}%
\pgfpathlineto{\pgfqpoint{14.205408in}{0.000000in}}%
\pgfpathlineto{\pgfqpoint{14.205408in}{5.669751in}}%
\pgfpathlineto{\pgfqpoint{0.000000in}{5.669751in}}%
\pgfpathclose%
\pgfusepath{}%
\end{pgfscope}%
\begin{pgfscope}%
\pgfsetbuttcap%
\pgfsetmiterjoin%
\definecolor{currentfill}{rgb}{1.000000,1.000000,1.000000}%
\pgfsetfillcolor{currentfill}%
\pgfsetlinewidth{0.000000pt}%
\definecolor{currentstroke}{rgb}{0.000000,0.000000,0.000000}%
\pgfsetstrokecolor{currentstroke}%
\pgfsetstrokeopacity{0.000000}%
\pgfsetdash{}{0pt}%
\pgfpathmoveto{\pgfqpoint{4.777704in}{1.324875in}}%
\pgfpathlineto{\pgfqpoint{9.427704in}{1.324875in}}%
\pgfpathlineto{\pgfqpoint{9.427704in}{4.344875in}}%
\pgfpathlineto{\pgfqpoint{4.777704in}{4.344875in}}%
\pgfpathclose%
\pgfusepath{fill}%
\end{pgfscope}%
\begin{pgfscope}%
\pgfpathrectangle{\pgfqpoint{4.777704in}{1.324875in}}{\pgfqpoint{4.650000in}{3.020000in}}%
\pgfusepath{clip}%
\pgfsetbuttcap%
\pgfsetmiterjoin%
\definecolor{currentfill}{rgb}{0.651961,0.093137,0.093137}%
\pgfsetfillcolor{currentfill}%
\pgfsetlinewidth{0.000000pt}%
\definecolor{currentstroke}{rgb}{0.000000,0.000000,0.000000}%
\pgfsetstrokecolor{currentstroke}%
\pgfsetstrokeopacity{0.000000}%
\pgfsetdash{}{0pt}%
\pgfpathmoveto{\pgfqpoint{5.010204in}{1.324875in}}%
\pgfpathlineto{\pgfqpoint{6.870204in}{1.324875in}}%
\pgfpathlineto{\pgfqpoint{6.870204in}{3.565980in}}%
\pgfpathlineto{\pgfqpoint{5.010204in}{3.565980in}}%
\pgfpathclose%
\pgfusepath{fill}%
\end{pgfscope}%
\begin{pgfscope}%
\pgfpathrectangle{\pgfqpoint{4.777704in}{1.324875in}}{\pgfqpoint{4.650000in}{3.020000in}}%
\pgfusepath{clip}%
\pgfsetbuttcap%
\pgfsetmiterjoin%
\definecolor{currentfill}{rgb}{0.144608,0.218137,0.424020}%
\pgfsetfillcolor{currentfill}%
\pgfsetlinewidth{0.000000pt}%
\definecolor{currentstroke}{rgb}{0.000000,0.000000,0.000000}%
\pgfsetstrokecolor{currentstroke}%
\pgfsetstrokeopacity{0.000000}%
\pgfsetdash{}{0pt}%
\pgfpathmoveto{\pgfqpoint{7.335204in}{1.324875in}}%
\pgfpathlineto{\pgfqpoint{9.195204in}{1.324875in}}%
\pgfpathlineto{\pgfqpoint{9.195204in}{3.956632in}}%
\pgfpathlineto{\pgfqpoint{7.335204in}{3.956632in}}%
\pgfpathclose%
\pgfusepath{fill}%
\end{pgfscope}%
\begin{pgfscope}%
\pgfsetbuttcap%
\pgfsetroundjoin%
\definecolor{currentfill}{rgb}{0.000000,0.000000,0.000000}%
\pgfsetfillcolor{currentfill}%
\pgfsetlinewidth{0.803000pt}%
\definecolor{currentstroke}{rgb}{0.000000,0.000000,0.000000}%
\pgfsetstrokecolor{currentstroke}%
\pgfsetdash{}{0pt}%
\pgfsys@defobject{currentmarker}{\pgfqpoint{0.000000in}{-0.048611in}}{\pgfqpoint{0.000000in}{0.000000in}}{%
\pgfpathmoveto{\pgfqpoint{0.000000in}{0.000000in}}%
\pgfpathlineto{\pgfqpoint{0.000000in}{-0.048611in}}%
\pgfusepath{stroke,fill}%
}%
\begin{pgfscope}%
\pgfsys@transformshift{5.940204in}{1.324875in}%
\pgfsys@useobject{currentmarker}{}%
\end{pgfscope}%
\end{pgfscope}%
\begin{pgfscope}%
\definecolor{textcolor}{rgb}{0.000000,0.000000,0.000000}%
\pgfsetstrokecolor{textcolor}%
\pgfsetfillcolor{textcolor}%
\pgftext[x=5.940204in,y=1.227653in,,top]{\color{textcolor}\rmfamily\fontsize{38.016000}{45.619200}\selectfont Sight}%
\end{pgfscope}%
\begin{pgfscope}%
\pgfsetbuttcap%
\pgfsetroundjoin%
\definecolor{currentfill}{rgb}{0.000000,0.000000,0.000000}%
\pgfsetfillcolor{currentfill}%
\pgfsetlinewidth{0.803000pt}%
\definecolor{currentstroke}{rgb}{0.000000,0.000000,0.000000}%
\pgfsetstrokecolor{currentstroke}%
\pgfsetdash{}{0pt}%
\pgfsys@defobject{currentmarker}{\pgfqpoint{0.000000in}{-0.048611in}}{\pgfqpoint{0.000000in}{0.000000in}}{%
\pgfpathmoveto{\pgfqpoint{0.000000in}{0.000000in}}%
\pgfpathlineto{\pgfqpoint{0.000000in}{-0.048611in}}%
\pgfusepath{stroke,fill}%
}%
\begin{pgfscope}%
\pgfsys@transformshift{8.265204in}{1.324875in}%
\pgfsys@useobject{currentmarker}{}%
\end{pgfscope}%
\end{pgfscope}%
\begin{pgfscope}%
\definecolor{textcolor}{rgb}{0.000000,0.000000,0.000000}%
\pgfsetstrokecolor{textcolor}%
\pgfsetfillcolor{textcolor}%
\pgftext[x=8.265204in,y=1.227653in,,top]{\color{textcolor}\rmfamily\fontsize{38.016000}{45.619200}\selectfont Blind}%
\end{pgfscope}%
\begin{pgfscope}%
\definecolor{textcolor}{rgb}{0.000000,0.000000,0.000000}%
\pgfsetstrokecolor{textcolor}%
\pgfsetfillcolor{textcolor}%
\pgftext[x=7.102704in,y=0.653637in,,top]{\color{textcolor}\rmfamily\fontsize{38.016000}{45.619200}\selectfont Visual Condition}%
\end{pgfscope}%
\begin{pgfscope}%
\pgfsetbuttcap%
\pgfsetroundjoin%
\definecolor{currentfill}{rgb}{0.000000,0.000000,0.000000}%
\pgfsetfillcolor{currentfill}%
\pgfsetlinewidth{0.803000pt}%
\definecolor{currentstroke}{rgb}{0.000000,0.000000,0.000000}%
\pgfsetstrokecolor{currentstroke}%
\pgfsetdash{}{0pt}%
\pgfsys@defobject{currentmarker}{\pgfqpoint{-0.048611in}{0.000000in}}{\pgfqpoint{-0.000000in}{0.000000in}}{%
\pgfpathmoveto{\pgfqpoint{-0.000000in}{0.000000in}}%
\pgfpathlineto{\pgfqpoint{-0.048611in}{0.000000in}}%
\pgfusepath{stroke,fill}%
}%
\begin{pgfscope}%
\pgfsys@transformshift{4.777704in}{1.324875in}%
\pgfsys@useobject{currentmarker}{}%
\end{pgfscope}%
\end{pgfscope}%
\begin{pgfscope}%
\definecolor{textcolor}{rgb}{0.000000,0.000000,0.000000}%
\pgfsetstrokecolor{textcolor}%
\pgfsetfillcolor{textcolor}%
\pgftext[x=4.005698in, y=1.124297in, left, base]{\color{textcolor}\rmfamily\fontsize{38.016000}{45.619200}\selectfont \(\displaystyle {0.0}\)}%
\end{pgfscope}%
\begin{pgfscope}%
\pgfsetbuttcap%
\pgfsetroundjoin%
\definecolor{currentfill}{rgb}{0.000000,0.000000,0.000000}%
\pgfsetfillcolor{currentfill}%
\pgfsetlinewidth{0.803000pt}%
\definecolor{currentstroke}{rgb}{0.000000,0.000000,0.000000}%
\pgfsetstrokecolor{currentstroke}%
\pgfsetdash{}{0pt}%
\pgfsys@defobject{currentmarker}{\pgfqpoint{-0.048611in}{0.000000in}}{\pgfqpoint{-0.000000in}{0.000000in}}{%
\pgfpathmoveto{\pgfqpoint{-0.000000in}{0.000000in}}%
\pgfpathlineto{\pgfqpoint{-0.048611in}{0.000000in}}%
\pgfusepath{stroke,fill}%
}%
\begin{pgfscope}%
\pgfsys@transformshift{4.777704in}{3.318630in}%
\pgfsys@useobject{currentmarker}{}%
\end{pgfscope}%
\end{pgfscope}%
\begin{pgfscope}%
\definecolor{textcolor}{rgb}{0.000000,0.000000,0.000000}%
\pgfsetstrokecolor{textcolor}%
\pgfsetfillcolor{textcolor}%
\pgftext[x=4.005698in, y=3.118052in, left, base]{\color{textcolor}\rmfamily\fontsize{38.016000}{45.619200}\selectfont \(\displaystyle {0.5}\)}%
\end{pgfscope}%
\begin{pgfscope}%
\definecolor{textcolor}{rgb}{0.000000,0.000000,0.000000}%
\pgfsetstrokecolor{textcolor}%
\pgfsetfillcolor{textcolor}%
\pgftext[x=3.950142in,y=2.834875in,,bottom,rotate=90.000000]{\color{textcolor}\rmfamily\fontsize{38.016000}{45.619200}\selectfont Sagat score average}%
\end{pgfscope}%
\begin{pgfscope}%
\pgfpathrectangle{\pgfqpoint{4.777704in}{1.324875in}}{\pgfqpoint{4.650000in}{3.020000in}}%
\pgfusepath{clip}%
\pgfsetrectcap%
\pgfsetroundjoin%
\pgfsetlinewidth{2.710125pt}%
\definecolor{currentstroke}{rgb}{0.260000,0.260000,0.260000}%
\pgfsetstrokecolor{currentstroke}%
\pgfsetdash{}{0pt}%
\pgfpathmoveto{\pgfqpoint{5.940204in}{3.306406in}}%
\pgfpathlineto{\pgfqpoint{5.940204in}{3.827932in}}%
\pgfusepath{stroke}%
\end{pgfscope}%
\begin{pgfscope}%
\pgfpathrectangle{\pgfqpoint{4.777704in}{1.324875in}}{\pgfqpoint{4.650000in}{3.020000in}}%
\pgfusepath{clip}%
\pgfsetrectcap%
\pgfsetroundjoin%
\pgfsetlinewidth{2.710125pt}%
\definecolor{currentstroke}{rgb}{0.260000,0.260000,0.260000}%
\pgfsetstrokecolor{currentstroke}%
\pgfsetdash{}{0pt}%
\pgfpathmoveto{\pgfqpoint{8.265204in}{3.703926in}}%
\pgfpathlineto{\pgfqpoint{8.265204in}{4.201066in}}%
\pgfusepath{stroke}%
\end{pgfscope}%
\begin{pgfscope}%
\pgfsetrectcap%
\pgfsetmiterjoin%
\pgfsetlinewidth{0.803000pt}%
\definecolor{currentstroke}{rgb}{0.000000,0.000000,0.000000}%
\pgfsetstrokecolor{currentstroke}%
\pgfsetdash{}{0pt}%
\pgfpathmoveto{\pgfqpoint{4.777704in}{1.324875in}}%
\pgfpathlineto{\pgfqpoint{4.777704in}{4.344875in}}%
\pgfusepath{stroke}%
\end{pgfscope}%
\begin{pgfscope}%
\pgfsetrectcap%
\pgfsetmiterjoin%
\pgfsetlinewidth{0.803000pt}%
\definecolor{currentstroke}{rgb}{0.000000,0.000000,0.000000}%
\pgfsetstrokecolor{currentstroke}%
\pgfsetdash{}{0pt}%
\pgfpathmoveto{\pgfqpoint{9.427704in}{1.324875in}}%
\pgfpathlineto{\pgfqpoint{9.427704in}{4.344875in}}%
\pgfusepath{stroke}%
\end{pgfscope}%
\begin{pgfscope}%
\pgfsetrectcap%
\pgfsetmiterjoin%
\pgfsetlinewidth{0.803000pt}%
\definecolor{currentstroke}{rgb}{0.000000,0.000000,0.000000}%
\pgfsetstrokecolor{currentstroke}%
\pgfsetdash{}{0pt}%
\pgfpathmoveto{\pgfqpoint{4.777704in}{1.324875in}}%
\pgfpathlineto{\pgfqpoint{9.427704in}{1.324875in}}%
\pgfusepath{stroke}%
\end{pgfscope}%
\begin{pgfscope}%
\pgfsetrectcap%
\pgfsetmiterjoin%
\pgfsetlinewidth{0.803000pt}%
\definecolor{currentstroke}{rgb}{0.000000,0.000000,0.000000}%
\pgfsetstrokecolor{currentstroke}%
\pgfsetdash{}{0pt}%
\pgfpathmoveto{\pgfqpoint{4.777704in}{4.344875in}}%
\pgfpathlineto{\pgfqpoint{9.427704in}{4.344875in}}%
\pgfusepath{stroke}%
\end{pgfscope}%
\begin{pgfscope}%
\definecolor{textcolor}{rgb}{0.000000,0.000000,0.000000}%
\pgfsetstrokecolor{textcolor}%
\pgfsetfillcolor{textcolor}%
\pgftext[x=7.102704in,y=4.462180in,,base]{\color{textcolor}\rmfamily\fontsize{38.016000}{45.619200}\selectfont Global sagat score average for blind and sight users}%
\end{pgfscope}%
\begin{pgfscope}%
\pgfsetbuttcap%
\pgfsetmiterjoin%
\definecolor{currentfill}{rgb}{1.000000,1.000000,1.000000}%
\pgfsetfillcolor{currentfill}%
\pgfsetfillopacity{0.800000}%
\pgfsetlinewidth{1.003750pt}%
\definecolor{currentstroke}{rgb}{0.800000,0.800000,0.800000}%
\pgfsetstrokecolor{currentstroke}%
\pgfsetstrokeopacity{0.800000}%
\pgfsetdash{}{0pt}%
\pgfpathmoveto{\pgfqpoint{9.207704in}{4.836742in}}%
\pgfpathlineto{\pgfqpoint{9.354371in}{4.836742in}}%
\pgfpathquadraticcurveto{\pgfqpoint{9.427704in}{4.836742in}}{\pgfqpoint{9.427704in}{4.910075in}}%
\pgfpathlineto{\pgfqpoint{9.427704in}{5.056742in}}%
\pgfpathquadraticcurveto{\pgfqpoint{9.427704in}{5.130075in}}{\pgfqpoint{9.354371in}{5.130075in}}%
\pgfpathlineto{\pgfqpoint{9.207704in}{5.130075in}}%
\pgfpathquadraticcurveto{\pgfqpoint{9.134371in}{5.130075in}}{\pgfqpoint{9.134371in}{5.056742in}}%
\pgfpathlineto{\pgfqpoint{9.134371in}{4.910075in}}%
\pgfpathquadraticcurveto{\pgfqpoint{9.134371in}{4.836742in}}{\pgfqpoint{9.207704in}{4.836742in}}%
\pgfpathclose%
\pgfusepath{stroke,fill}%
\end{pgfscope}%
\end{pgfpicture}%
\makeatother%
\endgroup%

        %}
        \caption{Barplot of the average Sagat score of each group.}
        \label{fig:barplot_sagat}
    \end{minipage}
\end{figure}

The Shapiro–Wilk normality test on the Table \ref{tab:shapiro_sagat_score} shows that these data are normally distributed, with a p-value higher than 0.05, then it is possible to perform a T-Test to guarantee that the "blind" sample is different than the "sight" sample.

According to the T-Test presented in the Table \ref{tab:ttest_sagat_score}, the only method that showed a difference in the Sagat score between the "sight" sample and the "blind" sample is the "Base" method, which is expected. In the other methods both samples had a similar Sagat score.

\begin{table}[!htb]
    \begin{minipage}{.45\linewidth}
        
\centering
\begin{tabular}{lr}
\toprule
                    Method &  Shapiro P-Value \\
\midrule
        Base blinded users &            0.619 \\
        Base blinded users &            1.000 \\
       Audio blinded users &            0.202 \\
       Audio blinded users &            0.606 \\
 Haptic Belt blinded users &            0.137 \\
 Haptic Belt blinded users &            0.715 \\
Virtual Cane blinded users &            0.326 \\
Virtual Cane blinded users &            0.909 \\
     Mixture blinded users &            0.124 \\
     Mixture blinded users &            0.641 \\
\bottomrule
\end{tabular}

    \end{minipage}
    \hfill
    \begin{minipage}{.45\linewidth}
        \vspace{-2.75cm}
        
\centering
\begin{tabular}{lr}
\toprule
      Method & T-Test P-Value \\
\midrule
        Base &        0.000** \\
       Audio &          0.397 \\
 Haptic Belt &          0.355 \\
Virtual Cane &          0.182 \\
     Mixture &          0.357 \\
\bottomrule
\end{tabular}

    \end{minipage}
\end{table}

The Table \ref{tab:blocanova_sagat_avg} shows the Anova test p-value of the Sagat score of the "blind" sample between the guidance methods presented in the Table \ref{tab:sagat}. The p-value indicates that there is at least one method that is statistically equal to one of the other methods.


\begin{table}[!htb]
\centering
\caption{Anova p-value for the Sagat score on each method for blinded users.}
\label{tab:blocanova_sagat_avg}
\begin{tabular}{lrrrrr}
\toprule
            Source &  Squared sum &  DOF & Squared average &     F & \begin{tabular}[c]{@{}l@{}}P-Value \\ $(F_{0} > F)$\end{tabular} \\
\midrule
   Between factors &        0.045 &    4 &           0.011 & 0.966 &                                                            0.461 \\
    Between blocks &        0.241 &    3 &           0.080 & 6.962 &                                                          0.006** \\
Experimental error &        0.139 &   12 &           0.012 &       &                                                                  \\
             Total &        0.424 &   19 &                 &       &                                                                  \\
\bottomrule
\end{tabular}
\end{table}



The Table \ref{tab:lsd_sagat} presents the analysis of a pairwise Fisher LSD test of the blind average Sagat score between all the guidance methods. The results show that the "Audio" and "Virtual Cane" caused a different Sagat score than the one noticed on the "Base" Method. The rest of the methods did not significantly change it.


\begin{table}[!htb]
\centering
\caption{Cross validation p-value for the Sagat score on each method for blinded users.}
\label{tab:lsd_sagat_avg}
\begin{tabular}{rclr}
\toprule
      \multicolumn{3}{c}{Method} &                                       Analysis \\
\midrule
              Base & $X$ & Audio &               $H_0 : \mu_{Base} = \mu_{Audio}$ \\
        Base & $X$ & Haptic Belt &         $H_0 : \mu_{Base} = \mu_{Haptic Belt}$ \\
       Base & $X$ & Virtual Cane &    $H_1 : \mu_{Base} \ne \mu_{Virtual Cane}**$ \\
            Base & $X$ & Mixture &             $H_0 : \mu_{Base} = \mu_{Mixture}$ \\
       Audio & $X$ & Haptic Belt &        $H_0 : \mu_{Audio} = \mu_{Haptic Belt}$ \\
      Audio & $X$ & Virtual Cane &       $H_0 : \mu_{Audio} = \mu_{Virtual Cane}$ \\
           Audio & $X$ & Mixture &            $H_0 : \mu_{Audio} = \mu_{Mixture}$ \\
Haptic Belt & $X$ & Virtual Cane & $H_0 : \mu_{Haptic Belt} = \mu_{Virtual Cane}$ \\
     Haptic Belt & $X$ & Mixture &      $H_0 : \mu_{Haptic Belt} = \mu_{Mixture}$ \\
    Virtual Cane & $X$ & Mixture & $H_1 : \mu_{Virtual Cane} \ne \mu_{Mixture}**$ \\
\bottomrule
\end{tabular}
\end{table}



According to T-Test on Table \ref{tab:ttest_sagat_score}, there is no difference in Sagat score between the "sight" and the "blind" sample except the "Base" method, which is expected.

According to both Anova test at Table \ref{tab:blocanova_sagat_avg} and LSD test at Table \ref{tab:lsd_sagat} only the "Audio"  and the "Virtual Cane" provoked a similar situation awareness when compared to the "Base" method.

\FloatBarrier