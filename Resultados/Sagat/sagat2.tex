\subsubsection{Adapted SAGAT}
\label{subsubsec:results_adapted_sagat_2}

Table\ref{tab:sagat_table_noBase} presents the SAGAT score of all participants. The corresponding barplot is presented in Figure \ref{fig:barplot_sagat_avg_4_scene_blind_sight}.


\begin{table}[!htb]
\centering
\caption{SAGAT global score felled by the participants.}
\label{tab:sagat_table_noBase}
\begin{tabular}{lllrrrrr}
\toprule
    &       &        &  Audio & \begin{tabular}[c]{@{}l@{}}Haptic\\ Belt\end{tabular} & \begin{tabular}[c]{@{}l@{}}Virtual\\ Cane\end{tabular} & Mixture \\
Participant & \begin{tabular}[c]{@{}l@{}}Visual\\ Condition\end{tabular} & Round &        &                                                       &                                                        &         \\
\midrule
001C & Blind & First &  5.500 &                                                 5.330 &                                                  5.830 &   3.500 \\
    &       & Return &  6.500 &                                                 8.500 &                                                  5.500 &   5.500 \\
002C & Blind & First &  4.500 &                                                 3.990 &                                                  4.500 &   6.250 \\
    &       & Return &  5.000 &                                                 4.000 &                                                  6.500 &   8.500 \\
003C & Blind & First &  7.500 &                                                 7.490 &                                                  4.660 &   9.000 \\
    &       & Return & 10.000 &                                                 8.500 &                                                  9.000 &   9.000 \\
004C & Blind & First &  6.000 &                                                 7.660 &                                                  4.990 &   6.500 \\
    &       & Return &  6.000 &                                                 9.250 &                                                  7.250 &   9.000 \\
001 & Sight & First &  4.500 &                                                 4.330 &                                                  2.660 &   6.500 \\
    &       & Return &  6.000 &                                                 5.000 &                                                  5.000 &   4.500 \\
003 & Sight & First &  6.750 &                                                 5.990 &                                                  3.990 &   6.750 \\
    &       & Return &  6.000 &                                                 7.250 &                                                  6.250 &   7.500 \\
004 & Sight & First &  7.250 &                                                 7.990 &                                                  5.990 &   8.250 \\
    &       & Return &  7.750 &                                                 9.500 &                                                  8.250 &   7.000 \\
005 & Sight & First &  3.000 &                                                 3.160 &                                                  3.990 &   4.000 \\
    &       & Return &  3.750 &                                                 3.000 &                                                  2.000 &   6.000 \\
\bottomrule
\end{tabular}
\end{table}



Figure \ref{fig:barplot_sagat_avg_4_scene_blind_sight}. shows that the SAGAT score for sighted participants is on average lower than that of blind participants, which is expected as they are not used to navigate without vision. Also, the increase in situation awareness from the first to the return round is lower. In the case of the mixture method, the SAGAT score did not improve at all. For both groups, the ‘virtual cane’ was the method with lowest score in the first round.

\begin{figure}[!htb]
    \centering
    \begin{minipage}{\textwidth}
        \centering
        \includegraphics[width = 0.8\linewidth]{Resultados/Sagat/Figuras/png/barplot_sagat_avg_4_scene_blind.png}
        \subcaption{Blind participants.}
        \label{fig:barplot_sagat_avg_4_scene_blind}
    \end{minipage}
    \begin{minipage}{\textwidth}
        \centering
        \includegraphics[width = 0.8\linewidth]{Resultados/Sagat/Figuras/png/barplot_sagat_avg_4_scene_sight.png}
        \subcaption{Sight participants.}
        \label{fig:barplot_sagat_avg_4_scene_sight}
    \end{minipage}
    \caption{Barplot of the SAGAT score on each method and round.}
    \label{fig:barplot_sagat_avg_4_scene_blind_sight}
\end{figure}
%\begin{figure}[!htb]
%    \centering
%    \includegraphics[width = 0.8\linewidth]{Resultados/Sagat/Figuras/png/barplot_sagat_avg_4_scene.png}
%    \caption{Barplot of the SAGAT score of both participants on each method.}
%    \label{fig:barplot_sagat_avg_4_scene}
%\end{figure}

Figures \ref{fig:boxplot_sagat_4_scene} and \ref{fig:boxplot_sagat_4_rounds} bring the boxplots. According to Figure \ref{fig:boxplot_sagat_4_scene}, both groups presented a higher situation awareness with ‘mixture’ and ‘haptic’. On the other hand, Figure \ref{fig:boxplot_sagat_4_rounds} confirms that the difference between the rounds is greater for blind participants. 

\begin{figure}[!htb]
    \centering
    \begin{minipage}{0.45\textwidth}
        \centering
        \includegraphics[width = 0.8\linewidth]{Resultados/Sagat/Figuras/png/boxplot_sagat_4_scene.png}
        \caption{Boxplot of the Sagat score of the participants grouped by method.}
        \label{fig:boxplot_sagat_4_scene}
    \end{minipage}
    \begin{minipage}{0.45\textwidth}
        \centering
        \includegraphics[width = 0.8\linewidth]{Resultados/Sagat/Figuras/png/boxplot_sagat_4_rounds.png}
        \caption{Boxplot of the Sagat score of the participants grouped by round.}
        \label{fig:boxplot_sagat_4_rounds}
    \end{minipage}
\end{figure}

%The Table \ref{tab:sagat_average_group_noBase} shows the average SAGAT score of both samples and is possible to notice how the average score by the blind users was higher in every method.
%
%
\begin{table}[!htb]
\centering
\caption{Adapted Sagat average global score grouped by participant and visual Condition.}
\label{tab:sagat_average_group_noBase}
\begin{tabular}{lrrrrrr}
\toprule
{} & Audio & \begin{tabular}[c]{@{}l@{}}Haptic\\ Belt\end{tabular} & \begin{tabular}[c]{@{}l@{}}Virtual\\ Cane\end{tabular} &  Mixture \\
Visual Condition &       &                                                       &                                                        &          \\
\midrule
Blind            &  6.38 &                                                  6.84 &                                                   6.03 &    7.156 \\
Sight            &  5.62 &                                                  5.78 &                                                   4.77 &    6.312 \\
\bottomrule
\end{tabular}
\end{table}



Figures \ref{fig:qqplot_sagat_avg_two_way_sight} and \ref{fig:residplot_sagat_avg_two_way_sight} brings the QQ plot and residual distribution. It is clear that the residuals variance is not equal among the participants. Table \ref{tab:blocanova_sagat_avg_two_way_blind_sight} brings the p-value from ANOVA. While for blind participants the round is a significant factor and the method is not, for sighted participants the result is the opposite, showing a significant influence of the method and not of the round.

\begin{table}
    \caption{Anova p-value for the SAGAT score on each method}
    \label{tab:blocanova_sagat_avg_two_way_blind_sight}
\begin{minipage}{0.45\textwidth}
    \subcaption{Blind participants}
    
\centering
\begin{tabular}{ll}
\toprule
          Source & P-Value \\
\midrule
    \    Methods &   0.277 \\
     \    Rounds & 0.002** \\
\    Interaction &   0.834 \\
\bottomrule
\end{tabular}

\end{minipage}
\begin{minipage}{0.45\textwidth}
    \subcaption{Sight participants}
    
\centering
\begin{tabular}{ll}
\toprule
          Source & P-Value \\
\midrule
    \    Methods & 0.035** \\
     \    Rounds &   0.095 \\
\    Interaction &   0.578 \\
\bottomrule
\end{tabular}
    
\end{minipage}
\end{table}

\begin{figure}[!htb]
    \centering
    \begin{minipage}{0.45\textwidth}
        \centering
        \includegraphics[width = 0.8\linewidth]{Resultados/Sagat/Figuras/png/qqplot_sagat_avg_two_way_sight.png}
        \caption{QQ plot of the mental demand of the sight participants on each method.}
        \label{fig:qqplot_sagat_avg_two_way_sight}
    \end{minipage}
    \begin{minipage}{0.45\textwidth}
        \centering
        \includegraphics[width = 0.8\linewidth]{Resultados/Sagat/Figuras/png/residplot_sagat_avg_two_way_sight.png}
        \caption{Residual plot of the mental demand score the sight participants on each method.}
        \label{fig:residplot_sagat_avg_two_way_sight}
    \end{minipage}
\end{figure}

%%%%%%%%%%%%%%%%%%%%%%%%%%%%%%%%%%%%%%%%%%%%%%%%%%%%%

%The Table \ref{tab:lsd_sagat_avg_two_way_sight} presents the conclusion of a pairwise Fisher LSD test of the previous ANOVA test. The results show that only the "Audio" and the "Haptic Belt" had a similar SAGAT score. This is different than the result from the ANOVA of the blind users, which indicated for them that the rounds had an effect.

%\input{Resultados/Sagat/Tabelas/lsd_sagat_avg_two_way_sight.tex}

%The Table \ref{tab:sagat_var_group_noBase} shows the average of the SAGAT score variation between the rounds. This table and the Figure \ref{fig:boxplot_sagat_4_rounds} show that, besides the higher average score, the blind users also had a higher variation between the rounds.

%
\begin{table}[!htb]
\centering
\caption{Adapted Sagat global score variation grouped by participant and visual Condition}
\label{tab:sagat_var_group_noBase}
\begin{tabular}{lrrrrrr}
\toprule
{} &  Audio & \begin{tabular}[c]{@{}l@{}}Haptic\\ Belt\end{tabular} & \begin{tabular}[c]{@{}l@{}}Virtual\\ Cane\end{tabular} & Mixture \\
Visual Condition &        &                                                       &                                                        &         \\
\midrule
Blind            &  15.66 &                                                 23.49 &                                                  44.30 &   32.90 \\
Sight            &  13.53 &                                                 12.59 &                                                  33.12 &    3.80 \\
\bottomrule
\end{tabular}
\end{table}





%\begin{figure}[!htb]
%    \centering
%    \begin{minipage}{0.45\textwidth}
%        \centering
%        \includegraphics[width = 0.8\linewidth]{Resultados/Sagat/Figuras/png/qqplot_sagat_var_sight.png}
%        \caption{Residual plot of the variation SAGAT score of the blind participants on each method.}
%        \label{fig:qqplot_sagat_var_sight}
%    \end{minipage}
%    \begin{minipage}{0.45\textwidth}
%        \centering
%        \includegraphics[width = 0.8\linewidth]{Resultados/Sagat/Figuras/png/residplot_sagat_var_sight.png}
%        \caption{Residual plot of the variation SAGAT score of the sighted participants on each method.}
%        \label{fig:residplot_sagat_var_sight}
%    \end{minipage}
%\end{figure}

%The Table \ref{tab:lsdbloc_mental_demand_var} presents the conclusion of a pairwise Fisher LSD test of the blind mental demand between all the guidance methods. The results show that all methods have similar variations.

%\input{Resultados/Nasa/Tabelas/lsdbloc_mental_demand_var.tex}

%Figures \ref{tab:sagat_average_group_noBase} and \ref{tab:sagat_var_group_noBase} brings the QQ plot and residual distribution. It is clear that the residuals variance is not equal among the participants. Table \ref{tab:blocanova_sagat_avg_two_way} brings the p-value from ANOVA. While for blind participants the round is a significant factor and the method is not, for sighted participants the result is the opposite, showing a significant influence of the method and not of the round.
%
%To close up, according to the Tables  with the Figures \ref{fig:boxplot_sagat_4_scene} and \ref{fig:boxplot_sagat_4_rounds} the blind user scored a higher SAGAT score than the sight user with the same conditions and devices. Besides that, the ANOVA and the LSD Fisher Test at Tables \ref{tab:blocanova_sagat_avg_two_way_sight} and \ref{tab:lsd_sagat_avg_two_way_sight} show that for the sight user the methods impact more their score, whilst the blind user were affected more with the rounds.
%
%There is no influence in the tested methods in the participants mental demand variation, as shown in the Table \ref{tab:blocanova_sagat_var_sight}.

\FloatBarrier