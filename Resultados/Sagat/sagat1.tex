\subsubsection{Adapted SAGAT}
\label{subsubsec:results_adapted_sagat_1}

This section discusses the results of the adapted SAGAT questionnaire, which aims at assessing the participant situation awareness and mental map of the environment. 

For each question of the SAGAT questionnaire, the participant could score 1 point or a fraction of it. The total score achieved by each blind participant is presented in the Table \ref{tab:sagat_table_blind}. Figure  \ref{fig:barplot_sagat_avg_5_scene_blind} illustrates the corresponding bar plot, indicating the mean and standard deviation for each guidance method and each round. This figure shows clearly that the participants improved its situation awareness in the return round, when they already had some information about the environment. Also, it is possible to observe that the worst situation awareness is obtained in the ‘first’ round for the ‘virtual cane’. However, on the ‘return’ round, the SAGAT mean score becomes equivalent to that of the ‘audio’ method.


\begin{table}[!htb]
\centering
\caption{SAGAT global score felled by the blinded participants.}
\label{tab:sagat_table_blind}
\begin{tabular}{llrrrrr}
\toprule
     &        &   Base &  Audio & \begin{tabular}[c]{@{}l@{}}Haptic\\ Belt\end{tabular} & \begin{tabular}[c]{@{}l@{}}Virtual\\ Cane\end{tabular} & Mixture \\
Participant & Round &        &        &                                                       &                                                        &         \\
\midrule
001C & First &   6.25 &   5.50 &                                                  5.33 &                                                   5.83 &   3.500 \\
     & Return &   6.25 &   6.50 &                                                  8.50 &                                                   5.50 &   5.500 \\
002C & First &   6.75 &   4.50 &                                                  3.99 &                                                   4.50 &   6.250 \\
     & Return &   5.25 &   5.00 &                                                  4.00 &                                                   6.50 &   8.500 \\
003C & First &   7.25 &   7.50 &                                                  7.49 &                                                   4.66 &   9.000 \\
     & Return &  10.00 &  10.00 &                                                  8.50 &                                                   9.00 &   9.000 \\
004C & First &   7.50 &   6.00 &                                                  7.66 &                                                   4.99 &   6.500 \\
     & Return &   9.00 &   6.00 &                                                  9.25 &                                                   7.25 &   9.000 \\
\bottomrule
\end{tabular}
\end{table}



\begin{figure}[!htb]
    \centering
    \includegraphics[width = 0.8\linewidth]{Resultados/Sagat/Figuras/png/barplot_sagat_avg_5_scene_blind.png}
    \caption{Barplot of the average SAGAT score of the blind participants on each method.}
    \label{fig:barplot_sagat_avg_5_scene_blind}
\end{figure}

Figure \ref{fig:boxplot_sagat_blind_scene} brings the boxplot of the SAGAT score grouped by guidance method. It shows that the methods can be divided in two groups. The first one is composed of ‘base’, ‘haptic belt’ and the ‘mixture’. This group received scores higher than the second group, composed of ‘audio’ and ‘virtual cane’. Following, Figure \ref{fig:boxplot_sagat_blind_rounds} shows the boxplot of the data grouped by round and confirms the general improvement of situation awareness from the ‘first’ to the ‘return’ round. 

\begin{figure}[!htb]
    \centering
    \begin{minipage}{0.45\textwidth}
        \centering
        \includegraphics[width = 0.8\linewidth]{Resultados/Sagat/Figuras/png/boxplot_sagat_blind_scene.png}
        \caption{Boxplot of the SAGAT score of the blind participants grouped by method.}
        \label{fig:boxplot_sagat_blind_scene}
    \end{minipage}
    \begin{minipage}{0.45\textwidth}
        \centering
        \includegraphics[width = 0.8\linewidth]{Resultados/Sagat/Figuras/png/boxplot_sagat_blind_rounds.png}
        \caption{Boxplot of the SAGAT score of the blind participants grouped by round.}
        \label{fig:boxplot_sagat_blind_rounds}
    \end{minipage}
\end{figure}

%The Table \ref{tab:sagat_average_group_blind} shows the average SAGAT score in the “blind” sample and is possible to notice how the average score by the “blind” sample was lower during the “Audio” and the “Base” methods.
%
%
\begin{table}[!htb]
\centering
\caption{SAGAT score average grouped by participant and visual condition}
\label{tab:sagat_average_group_blind}
\begin{tabular}{lrrrrrr}
\toprule
{} &  Base & Audio & \begin{tabular}[c]{@{}l@{}}Haptic\\ Belt\end{tabular} & \begin{tabular}[c]{@{}l@{}}Virtual\\ Cane\end{tabular} &  Mixture \\
Visual Condition &       &       &                                                       &                                                        &          \\
\midrule
Blind            &  7.28 &  6.38 &                                                  6.84 &                                                   6.03 &    7.156 \\
\bottomrule
\end{tabular}
\end{table}



Proceeding to the statistical analysis of the data, Figures \ref{fig:qqplot_sagat_avg_two_way_blind} and \ref{fig:residplot_sagat_avg_two_way_blind} present the QQ plot and the residual distribution, which confirms the normal distribution assumption and the homogeneity of variances

\begin{figure}[!htb]
    \centering
    \begin{minipage}{0.45\textwidth}
        \centering
        \includegraphics[width = 0.8\linewidth]{Resultados/Sagat/Figuras/png/qqplot_sagat_avg_two_way_blind.png}
        \caption{QQ plot of the SAGAT score of the blind participants on each method.}
        \label{fig:qqplot_sagat_avg_two_way_blind}
    \end{minipage}
    \begin{minipage}{0.45\textwidth}
        \centering
        \includegraphics[width = 0.8\linewidth]{Resultados/Sagat/Figuras/png/residplot_sagat_avg_two_way_blind.png}
        \caption{Residual plot of the SAGAT score the blind participants on each method.}
        \label{fig:residplot_sagat_avg_two_way_blind}
    \end{minipage}
\end{figure}

Finally, Table \ref{tab:blocanova_sagat_avg_two_way_blind} shows the Anova test p-value of the SAGAT score. It indicates that the round is a significant variable that influences the value of the SAGAT score. The same cannot be said for the method, which, apparently, has no significant influence.


\begin{table}[!htb]
\centering
\caption{Anova p-value for the SAGAT score on each method for blinded users.}
\label{tab:blocanova_sagat_avg_two_way_blind}
\begin{tabular}{lrrrrr}
\toprule
          Source & P-Value \\
\midrule
    \    Methods &   0.277 \\
     \    Rounds & 0.002** \\
\    Interaction &   0.834 \\
\bottomrule
\end{tabular}
\end{table}



%The Table \ref{tab:lsd_sagat_avg_two_way} presents the conclusion of a pairwise Fisher LSD test of the blind NASA-TLX score between all the guidance methods. The results show that only the "Audio" has a similar NASA-TLX score as the "Base" method, as it was also posible to notice at Figure \ref{fig:boxplot_sagat_blind_scene}.

%\input{Resultados/Sagat/Tabelas/lsd_sagat_avg_two_way}

Finally, Table \ref{tab:sagat_var_group_blind} brings the mean difference in the SAGAT score between the first and return round for each guidance method. It shows that the ‘base’ and ‘audio’ methods have the lowest difference, while the highest one was obtained for the ‘virtual cane’.


\begin{table}[!htb]
\centering
\caption{Adapted Sagat global score variation grouped by participant and visual Condition}
\label{tab:sagat_var_group_blind}
\begin{tabular}{lrrrrrr}
\toprule
{} &  Base &  Audio & \begin{tabular}[c]{@{}l@{}}Haptic\\ Belt\end{tabular} & \begin{tabular}[c]{@{}l@{}}Virtual\\ Cane\end{tabular} & Mixture \\
Visual Condition &       &        &                                                       &                                                        &         \\
\midrule
Blind            &  8.93 &  15.66 &                                                 23.49 &                                                  44.30 &   32.90 \\
\bottomrule
\end{tabular}
\end{table}



%The Figures \ref{fig:qqplot_sagat_var_blind} and \ref{fig:residplot_sagat_var_blind} shows the distribution and variance of the SAGAT score variation of the Table \ref{tab:sagat_table_blind}. These Figures shows that the data are normally distributed and that the methods have a similar variance.
%The Table \ref{tab:blocanova_sagat_var_blind} shows the Anova test p-value of the SAGAT score of the "blind" sample between the guidance methods. The p-value indicates that there are no difference between the variation in any method. 
%
%
\begin{table}[!htb]
\centering
\caption{Anova p-value for the SAGAT score variation on each method for blinded users.}
\label{tab:blocanova_sagat_var_blind}
\begin{tabular}{lrrrrr}
\toprule
               Source &  Squared sum &  DOF & Squared average &     F & \begin{tabular}[c]{@{}l@{}}P-Value \\ $(F_{0} > F)$\end{tabular} \\
\midrule
Participants (blocks) &     1176.902 &    3 &         782.885 & 0.473 &                                                                  \\
               Method &     3131.542 &    4 &         392.301 & 0.944 &                                                            0.472 \\
   Experimental error &     9956.458 &   12 &         829.705 &       &                                                                  \\
                Total &    14264.902 &   19 &                 &       &                                                                  \\
\bottomrule
\end{tabular}
\end{table}


%
%\begin{figure}[!htb]
%    \centering
%    \begin{minipage}{0.45\textwidth}
%        \centering
%        \includegraphics[width = 0.8\linewidth]{Resultados/Sagat/Figuras/png/qqplot_sagat_var_sight.png}
%        \caption{QQ plot of the SAGAT score variation of the blind participants on each method.}
%        \label{fig:qqplot_sagat_var_blind}
%    \end{minipage}
%    \begin{minipage}{0.45\textwidth}
%        \centering
%        \includegraphics[width = 0.8\linewidth]{Resultados/Sagat/Figuras/png/residplot_sagat_var_sight.png}
%        \caption{Residual plot of the SAGAT score variation of the blind participants on each method.}
%        \label{fig:residplot_sagat_var_blind}
%    \end{minipage}
%\end{figure}
%
%%The Table \ref{tab:lsdbloc_nasa_var} presents the conclusion of a pairwise Fisher LSD test of the blind NASA-TLX score between all the guidance methods. The results show that all methods have similar variations.
%
%%\input{Resultados/Nasa/Tabelas/lsdbloc_nasa_var}
%
%To close up, according to the ANOVA test at Table \ref{tab:blocanova_sagat_avg_two_way_blind} the methods caused no reaction on the SAGAT score, but the rounds did. That means that the participants were able in all methods to learn a little about their environment and that learning impacted their environmental perception in the next round. The fact that the test has not found any influence of the methods on the SAGAT score may be because of the small sample size, since it is posible to notice a difference between the methods at Figure \ref{fig:boxplot_sagat_blind_scene}. Also the interaction between method and round caused no influence in the Sagat score. According to the ANOVA test at Table \ref{tab:blocanova_sagat_var_blind}, the methods did not influenced the SAGAT score.
%
\FloatBarrier