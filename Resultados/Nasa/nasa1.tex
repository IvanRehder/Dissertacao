\subsubsection{NASA-TLX}
\label{subsubsec:results_nasa_tlx_1}

The NASA-TLX provides two relevant pieces of information to the workload analysis. The first is the score attributed to the "mental demand" dimension and the second is the average obtained from NASA-TLX's six dimensions. The two analyses are presented in the next subsections.

\paragraph{Analysis of the mental demand scale}\mbox{}\\

Table \ref{tab:md_table_blind} presents the "mental demand" score of each blind participant to each guidance method. The higher the value, the higher the mental demand is. The base method refers to the guidance method that the person uses in his/her daily life (e.g., white cane). 


\begin{table}[!htb]
\centering
\caption{Score of NASA-TLX mental demand for the blind participants.}
\label{tab:md_table_blind}
\begin{tabular}{llrrrrr}
\toprule
     &        & Base & Audio & \begin{tabular}[c]{@{}l@{}}Haptic\\ Belt\end{tabular} & \begin{tabular}[c]{@{}l@{}}Virtual\\ Cane\end{tabular} & Mixture \\
Participant & Round &      &       &                                                       &                                                        &         \\
\midrule
001C & First &    3 &     1 &                                                    14 &                                                      3 &       6 \\
     & Return &    1 &     1 &                                                    10 &                                                      2 &       6 \\
002C & First &    5 &     1 &                                                     1 &                                                     10 &      12 \\
     & Return &    1 &     1 &                                                     1 &                                                     10 &       3 \\
003C & First &    5 &     5 &                                                     5 &                                                      8 &       1 \\
     & Return &    3 &     1 &                                                     1 &                                                      2 &       1 \\
004C & First &    9 &    10 &                                                    15 &                                                     10 &      10 \\
     & Return &    7 &    10 &                                                    14 &                                                      8 &      10 \\
\bottomrule
\end{tabular}
\end{table}



The mean value obtained for each guidance method is illustrated in Figure \ref{fig:barplot_md_avg_5_scene_blind}. It shows a systematic reduction in the perceived mental workload between the rounds for all methods, confirming that the participants get familiar with the devices after the first use. It also shows that although the haptic belt obtained the most considerable mean, it also had the most significant variation, showing that the effort required from the user may vary significantly.

\begin{figure}[!htb]
    \centering
    \includegraphics[width = \textwidth]{Resultados/Nasa/Figuras/pdf/barplot_md_avg_5_scene_blind.pdf}
    \caption{Mean and standard deviation of mental demand of blind participants for each method.}
    \label{fig:barplot_md_avg_5_scene_blind}
\end{figure}


Figure \ref{fig:boxplot_md_blind_scene}  presents a boxplot of the mental demand score grouped by the methods. This figure shows that there may be two groups: one associated with lower demand, composed of base and audio, and another with higher demand, composed of haptic belt, virtual cane and mixture. It indicates that maybe a guidance method that uses vibration as input is not intuitive. Figure \ref{fig:boxplot_md_blind_rounds} presents a boxplot of the mental demand grouped by the rounds, confirming the general tendency to reduce the required "mental demand". 

\begin{figure}[!htb]
    \centering
    \begin{minipage}{0.45\textwidth}
        \centering
        \includegraphics[width = \textwidth]{Resultados/Nasa/Figuras/pdf/boxplot_md_blind_scene.pdf}
        \caption{Boxplot of the mental demand of the blind participants grouped by the methods.}
        \label{fig:boxplot_md_blind_scene}
    \end{minipage}
    \begin{minipage}{0.075\textwidth}
        \hfill
    \end{minipage}
    \begin{minipage}{0.45\textwidth}
        \centering
        %\vspace{3ex}
        \includegraphics[width = \textwidth]{Resultados/Nasa/Figuras/pdf/boxplot_md_blind_rounds.pdf}
        \caption{Boxplot of the mental demand of the blind participants grouped by the rounds.}
        \label{fig:boxplot_md_blind_rounds}
    \end{minipage}
\end{figure}

In order to support the statistical analysis, Figures \ref{fig:qqplot_md_avg_two_way_blind} and \ref{fig:residplot_md_avg_two_way_blind} presents the QQ-plot and the residual plot of the "mental demand" data, confirming that the data follow a normal distribution and the residues are homogenous.

Figures \ref{fig:qqplot_md_avg_two_way_blind} and \ref{fig:residplot_md_avg_two_way_blind} show the distribution and variance of Table \ref{tab:md_table_blind}. These figures show that the data are normally distributed and that the methods have a similar variance. Table \ref{tab:blocanova_md_avg_two_way_blind} shows the ANOVA test p-values of the mental demand of the "blind” sample between the guidance methods. The methods' and the rounds' p-values indicate that there is no influence from them in the mental demand. The interaction between the methods and the round also does not influence the mental demand.

\begin{figure}[!htb]
    \centering
    \begin{minipage}{0.45\textwidth}
        \centering
        %\vspace{1ex}
        \includegraphics[width = \textwidth]{Resultados/Nasa/Figuras/pdf/qqplot_md_avg_two_way_blind.pdf}
        \caption{QQ plot of the mental demand of the blind participants on each method.}
        \label{fig:qqplot_md_avg_two_way_blind}
    \end{minipage}
    \begin{minipage}{0.075\textwidth}
        \hfill
    \end{minipage}
    \begin{minipage}{0.45\textwidth}
        \centering
        %\vspace{1ex}
        \includegraphics[width = \textwidth]{Resultados/Nasa/Figuras/pdf/residplot_md_avg_two_way_blind.pdf}
        \caption{Residual plot of the mental demand score the blind participants on each method.}
        \label{fig:residplot_md_avg_two_way_blind}
    \end{minipage}
\end{figure}

Following, the statistical model of Equation 5.1 is used for the analysis of variance (ANOVA). The ANOVA analyses the influence of the used method, the rounds and the interaction of those two in the analysed variable: 

\begin{equation}
    \label{eq:statistical_model}
    y_{ijk} = \mu + \tau_i + \beta_j + \msout{\omega_k} + (\tau\beta_{ij}) + e
\end{equation}

where:

\begin{itemize}
    \item $y_{ij}$ - output variable for method $i$, round $j$ and participant $k$;
    \item $\mu$ - mean of all the observations;
    \item $\tau_i$ - variance from method $i$;
    \item $\beta_j$ - variance from round $j$;
    \item \sout{$\omega_k$} - variance from participant k, which is treated as a block;
    \item $\tau\beta_{ij}$ - combined variance from the interaction between method i and round j;
    \item $e$ - residual error.
\end{itemize}

The results of ANOVA are presented in Table \ref{tab:blocanova_md_avg_two_way_blind}. ANOVA tests the hypothesis that the means of independent data groups are equal or not. In the literature, a p-value of 0.05 is commonly adopted as a threshold to confirm the hypothesis. A p-value < 0.05 indicates that the means of the groups are statistically different with 95\% of confidence. According to this criterion, neither method or round have a significant influence on the mental demand.

However, due to the low number of participants, the threshold of 0.1 could also be considered. In this case, it indicates, with 90\% confidence, that the mean of the first and return rounds are different. For the guidance method, the p-value of 0.170 is close to the threshold but slightly higher, suggesting that the means may be different. However, this hypothesis is not statistically confirmed with the current data. 


\begin{table}[!htb]
\centering
\caption{Anova p-value for the mental demand average on each method for blinded users.}
\label{tab:blocanova_md_avg_two_way_blind}
\begin{tabular}{lrrrrl}
\toprule
               Source &  Squared sum &  DOF & Squared average &     F & \begin{tabular}[c]{@{}l@{}}P-Value \\ $(F_{0} > F)$\end{tabular} \\
\midrule
Participants (Blocks) &      298.475 &    3 &          99.492 & 8.133 &                                                                  \\
         \    Methods &       85.150 &    4 &          21.288 & 1.740 &                                                            0.170 \\
          \    Rounds &       42.025 &    1 &          42.025 & 3.436 &                                                            0.075 \\
     \    Interaction &        2.850 &    4 &           0.712 & 0.058 &                                                            0.993 \\
   Experimental Error &      330.275 &   27 &          12.232 &       &                                                                  \\
                Total &      758.775 &   39 &                 &       &                                                                  \\
\bottomrule
\end{tabular}
\end{table}



In order to conclude the analysis of the NASA-TLX mental demand, Table \ref{tab:md_var_average_group_blind} brings the average difference between the mental demand of the first and return rounds. If the variation is positive, it means that the user had an increase on his/her mental demand. Unexpectedly, it shows that the most significant variation is obtained to the base, i.e., the guidance method the participant uses and, therefore, should not present a significant variation. The methods with the lower variation was audio, probably because it already had a shallow score in the first round. 


\begin{table}[!htb]
\centering
\caption{Mental demand variation grouped by participant and visual condition}
\label{tab:md_var_average_group_blind}
\begin{tabular}{lrrrrrr}
\toprule
{} &  Base & Audio & \begin{tabular}[c]{@{}l@{}}Haptic\\ Belt\end{tabular} & \begin{tabular}[c]{@{}l@{}}Virtual\\ Cane\end{tabular} & Mixture \\
Visual Condition &       &       &                                                       &                                                        &         \\
\midrule
Blind            &  -2.5 &  -1.0 &                                                  -2.2 &                                                   -2.2 &    -2.2 \\
\bottomrule
\end{tabular}
\end{table}



\FloatBarrier

%%%%%%%%%%%%%%%%%%%%%%%%%%%%%%%%%%%%%%%%%%%%%%%%%%%%%%%%%%%%%%%%%%%%%%%%%%%
%%%%%%%%%%%%%%%%%%%%%%%%%%%%%%%%%%%%%%%%%%%%%%%%%%%%%%%%%%%%%%%%%%%%%%%%%%%
%%%%%%%%%%%%%%%%%%%%%%%%%%%%%%%%%%%%%%%%%%%%%%%%%%%%%%%%%%%%%%%%%%%%%%%%%%%
%%%%%%%%%%%%%%%%%%%%%%%%%%%%%%%%%%%%%%%%%%%%%%%%%%%%%%%%%%%%%%%%%%%%%%%%%%%


\paragraph{Analysis of the NASA-TLX score}\mbox{}\\

This section repeats the analysis steps of the previous section but now considers the mean value of all dimensions of NASA-TLX, referred to in this text as the global score. Again, the higher the value of the average, higher the mental workload perceived by the user. Table \ref{tab:nasa_table_blind} presents the global score of each blind participant. 


\begin{table}[!htb]
\centering
\caption{NASA-TLX score felled by the blinded participants.}
\label{tab:nasa_table_blind}
\begin{tabular}{llrrrrr}
\toprule
     &        &  Base &  Audio & \begin{tabular}[c]{@{}l@{}}Haptic\\ Belt\end{tabular} & \begin{tabular}[c]{@{}l@{}}Virtual\\ Cane\end{tabular} & Mixture \\
Participant & Round &       &        &                                                       &                                                        &         \\
\midrule
001C & First & 4.833 &  4.000 &                                                 8.833 &                                                  5.167 &   6.333 \\
     & Return & 4.167 &  4.000 &                                                 6.667 &                                                  4.500 &   6.167 \\
002C & First & 6.333 &  4.833 &                                                 4.833 &                                                  9.000 &   7.000 \\
     & Return & 4.500 &  4.833 &                                                 4.833 &                                                  7.000 &   5.167 \\
003C & First & 4.000 &  4.000 &                                                 5.333 &                                                  6.667 &   3.500 \\
     & Return & 4.000 &  3.833 &                                                 3.667 &                                                  3.500 &   3.500 \\
004C & First & 9.833 & 10.000 &                                                12.667 &                                                  9.667 &  11.000 \\
     & Return & 8.667 &  9.167 &                                                11.667 &                                                  9.333 &  10.833 \\
\bottomrule
\end{tabular}
\end{table}



Figure \ref{fig:barplot_nasa_avg_5_scene_blind} brings the corresponding barplot with the mean value and standard deviation for each guidance method and each round. In a qualitative comparison with Figure \ref{fig:barplot_md_avg_5_scene_blind}, the differences between the methods are confirmed but softened. It is possible to notice that the mean score of audio and base are still lower than that of the other methods. The differences between first and return rounds are also reduced. However, the standard deviation is also considerably reduced for all methods, and especially for the haptic belt.

\begin{figure}[!htb]
    \centering
    \includegraphics[width = \textwidth]{Resultados/Nasa/Figuras/pdf/barplot_nasa_avg_5_scene_blind.pdf}
    \caption{Barplot of the average NASA-TLX score of the blind participants on each method.}
    \label{fig:barplot_nasa_avg_5_scene_blind}
\end{figure}

Figure \ref{fig:boxplot_nasa_blind_scene} presents the boxplot with the NASA-TLX global score grouped by the methods. Similar to what happened for the "mental demand", it is possible to split the methods into two different groups: base and audio, which require a lower level of workload, and another group, which requires a higher level. Figure \ref{fig:boxplot_nasa_blind_rounds} presents a boxplot with the NASA-TLX global score grouped by the rounds, showing that the two groups are still different. 

\begin{figure}[!htb]
    \centering
    \begin{minipage}{0.45\textwidth}
        \centering
        \includegraphics[width = \textwidth]{Resultados/Nasa/Figuras/pdf/boxplot_nasa_blind_scene.pdf}
        \caption{Boxplot of the NASA-TLX of the blind participants grouped by the methods.}
        \label{fig:boxplot_nasa_blind_scene}
    \end{minipage}
    \begin{minipage}{0.075\textwidth}
        \hfill
    \end{minipage}
    \begin{minipage}{0.45\textwidth}
        \centering
        \includegraphics[width = \textwidth]{Resultados/Nasa/Figuras/pdf/boxplot_nasa_blind_rounds.pdf}
        \caption{Boxplot of the NASA-TLX demand of the blind participants grouped by the rounds.}
        \label{fig:boxplot_nasa_blind_rounds}
    \end{minipage}
\end{figure}

Figures \ref{fig:qqplot_nasa_avg_two_way_blind} and \ref{fig:residplot_nasa_avg_two_way_blind} presents the QQ plot and residual distribution of the NASA-TLX global score, showing that the data are normally distributed. However, the residuals are not so homogeneous as in the previous case, showing that the participants have different variability among them.

\begin{figure}[!htb]
    \centering
    \begin{minipage}{0.45\textwidth}
        \centering
        \includegraphics[width = \textwidth]{Resultados/Nasa/Figuras/pdf/qqplot_nasa_avg_two_way_blind.pdf}
        \caption{QQ plot of the NASA-TLX score of the blind participants on each method.}
        \label{fig:qqplot_nasa_avg_two_way_blind}
    \end{minipage}
    \begin{minipage}{0.075\textwidth}
        \hfill
    \end{minipage}
    \begin{minipage}{0.45\textwidth}
        %\vspace{2ex}
        \centering
        \includegraphics[width = \textwidth]{Resultados/Nasa/Figuras/pdf/residplot_nasa_avg_two_way_blind.pdf}
        \caption{Residual plot of the NASA-TLX score the blind participants on each method.}
        \label{fig:residplot_nasa_avg_two_way_blind}
    \end{minipage}
\end{figure}

Table \ref{tab:blocanova_nasa_avg_two_way_blind} brings the p-value resulting from ANOVA. In this case, both the methods and the rounds were appointed as significant variables that influence the mean value of the NASA-TLX global score. 


\begin{table}[!htb]
\centering
\caption{Anova p-value for the NASA-TLX score on each method for blinded users.}
\label{tab:blocanova_nasa_avg_two_way_blind}
\begin{tabular}{lrrrrl}
\toprule
          Source & P-Value \\
\midrule
    \    Methods & 0.029** \\
     \    Rounds & 0.022** \\
\    Interaction &   0.814 \\
\bottomrule
\end{tabular}
\end{table}



Finally, Table \ref{tab:lsd_nasa_avg_two_way_blind} presents the results of a pairwise Fisher LSD test comparing each pair of guidance methods. The results show that only audio is similar base. All the other methods are different from each other.

\input{Resultados/Nasa/Tabelas/lsd_nasa_avg_two_way_blind}

Table \ref{tab:nasa_var_group_blind} shows the difference in the NASA-TLX global score between the first and return rounds. As before, if the variation is positive, it means that the user had an increase on his/her Mental Workload. It shows that the audio difference is the lowest among all methods, while the highest difference is for the virtual cane.

\input{Resultados/Nasa/Tabelas/nasa_var_group_blind}

\FloatBarrier