\subsubsection{NASA-TLX}
\label{subsubsec:results_nasa_tlx_2}

\paragraph{Analysis of the mental demand scale}\mbox{}\\

The Table \ref{tab:md_table_noBase} presents the mental demand averages of all participants on each scene and their average are plotted in the Figures from \ref{fig:barplot_md_avg_4_scene_blind} to \ref{fig:barplot_md_avg_4_scene}. 


\begin{table}[!htb]
\centering
\caption{Mental demand felled by the participants.}
\label{tab:md_table_noBase}
\begin{tabular}{lllrrrrr}
\toprule
    &       &        & Audio & \begin{tabular}[c]{@{}l@{}}Haptic\\ Belt\end{tabular} & \begin{tabular}[c]{@{}l@{}}Virtual\\ Cane\end{tabular} & Mixture \\
Participant & \begin{tabular}[c]{@{}l@{}}Visual\\ Condition\end{tabular} & Round &       &                                                       &                                                        &         \\
\midrule
001C & Blind & First &     1 &                                                    14 &                                                      3 &       6 \\
    &       & Return &     1 &                                                    10 &                                                      2 &       6 \\
002C & Blind & First &     1 &                                                     1 &                                                     10 &      12 \\
    &       & Return &     1 &                                                     1 &                                                     10 &       3 \\
003C & Blind & First &     5 &                                                     5 &                                                      8 &       1 \\
    &       & Return &     1 &                                                     1 &                                                      2 &       1 \\
004C & Blind & First &    10 &                                                    15 &                                                     10 &      10 \\
    &       & Return &    10 &                                                    14 &                                                      8 &      10 \\
001 & Sight & First &    12 &                                                    11 &                                                      5 &       9 \\
    &       & Return &    13 &                                                    13 &                                                      5 &      10 \\
003 & Sight & First &    18 &                                                    18 &                                                     16 &      10 \\
    &       & Return &    12 &                                                    15 &                                                     11 &       8 \\
004 & Sight & First &    17 &                                                    20 &                                                     12 &      20 \\
    &       & Return &    12 &                                                    15 &                                                     10 &      15 \\
005 & Sight & First &     4 &                                                    12 &                                                     10 &      13 \\
    &       & Return &     6 &                                                    10 &                                                      6 &      12 \\
\bottomrule
\end{tabular}
\end{table}



The Figures \ref{fig:barplot_md_avg_4_scene_blind} and \ref{fig:barplot_md_avg_4_scene_sight} show a systematic reduction on the perceived mental demand in all methods between the rounds for both groups. But the Figure \ref{fig:barplot_md_avg_4_scene} shows that the average of each method was very different between the two groups.

\begin{figure}[!htb]
    \centering
    \begin{minipage}{\textwidth}
        \centering
        \includegraphics[width = 0.8\linewidth]{Resultados/Nasa/Figuras/png/barplot_md_avg_4_scene_blind.png}
        \caption{Barplot of the average mental demand of the blind participants on each method and round.}
        \label{fig:barplot_md_avg_4_scene_blind}
    \end{minipage}
    \begin{minipage}{\textwidth}
        \centering
        \includegraphics[width = 0.8\linewidth]{Resultados/Nasa/Figuras/png/barplot_md_avg_4_scene_sight.png}
        \caption{Barplot of the average mental demand of the sight participants on each method and round.}
        \label{fig:barplot_md_avg_4_scene_sight}
    \end{minipage}
\end{figure}
\begin{figure}[!htb]
    \centering
    \includegraphics[width = 0.8\linewidth]{Resultados/Nasa/Figuras/png/barplot_md_avg_4_scene.png}
    \caption{Barplot of the average mental demand of both participants on each method.}
    \label{fig:barplot_md_avg_4_scene}
\end{figure}

The Figure \ref{fig:boxplot_noBase_md_4_scene} and \ref{fig:boxplot_noBase_md_4_rounds} presents a box plot with the mental demand of both groups by method. These figures show that the reaction of each group is completelly different between the methods and rounds.

\begin{figure}[!htb]
    \centering
    \begin{minipage}{0.45\textwidth}
        \centering
        \includegraphics[width = 0.8\linewidth]{Resultados/Nasa/Figuras/png/boxplot_noBase_md_4_scene.png}
        \caption{Boxplot of the mental demand of the participants grouped by method.}
        \label{fig:boxplot_noBase_md_4_scene}
    \end{minipage}
    \begin{minipage}{0.45\textwidth}
        \centering
        \includegraphics[width = 0.8\linewidth]{Resultados/Nasa/Figuras/png/boxplot_noBase_md_4_rounds.png}
        \caption{Boxplot of the mental demand of the participants grouped by round.}
        \label{fig:boxplot_noBase_md_4_rounds}
    \end{minipage}
\end{figure}

The Table \ref{tab:md_average_group_noBase} shows the average mental demand of both samples and is possible to notice how the average perceived mental demand by the sight sample was higher in every method.


\begin{table}[!htb]
\centering
\caption{Mental demand average grouped by participant and visual condition}
\label{tab:md_average_group_noBase}
\begin{tabular}{lrrrrrr}
\toprule
{} &  Audio & \begin{tabular}[c]{@{}l@{}}Haptic\\ Belt\end{tabular} & \begin{tabular}[c]{@{}l@{}}Virtual\\ Cane\end{tabular} &  Mixture \\
Visual Condition &        &                                                       &                                                        &          \\
\midrule
Blind            &   3.75 &                                                  7.62 &                                                   6.62 &    6.125 \\
Sight            &  11.75 &                                                 14.25 &                                                   9.38 &   12.125 \\
\bottomrule
\end{tabular}
\end{table}



The Figures \ref{fig:qqplot_md_avg_two_way_sight} and \ref{fig:residplot_md_avg_two_way_sight} shows the distribution and variance of sighted participants in the Table \ref{tab:md_table_noBase}. These Figures shows that the data are normally distributed and that the methods have a similar variance.
The Table \ref{tab:blocanova_md_avg_two_way_sight} shows the Anova test p-values of the mental demand of the "sight" sample between the guidance methods. The method's and the round's p-values indicates that there is some influence of the method in the mental demand but no influence from the round or the interaction between them.


\begin{table}[!htb]
\centering
\caption{Anova p-value for the mental demand average on each method for sighted users.}
\label{tab:blocanova_md_avg_two_way_sight}
\begin{tabular}{lrrrrl}
\toprule
          Source & P-Value \\
\midrule
    \    Methods & 0.049** \\
     \    Rounds &   0.075 \\
\    Interaction &   0.990 \\
\bottomrule
\end{tabular}
\end{table}



\begin{figure}[!htb]
    \centering
    \begin{minipage}{0.45\textwidth}
        \centering
        \includegraphics[width = 0.8\linewidth]{Resultados/Nasa/Figuras/png/qqplot_md_avg_two_way_sight.png}
        \caption{QQ plot of the mental demand of the sight participants on each method.}
        \label{fig:qqplot_md_avg_two_way_sight}
    \end{minipage}
    \begin{minipage}{0.45\textwidth}
        \centering
        \includegraphics[width = 0.8\linewidth]{Resultados/Nasa/Figuras/png/residplot_md_avg_two_way_sight.png}
        \caption{Residual plot of the mental demand score the sighted participants on each method.}
        \label{fig:residplot_md_avg_two_way_sight}
    \end{minipage}
\end{figure}

The Table \ref{tab:lsd_md_avg_two_way_sight} presents the conclusion of a pairwise Fisher LSD test of the previous ANOVA test. The results show that only the "Audio" has a similar mental demand as the "Mixture" method.


\begin{table}[!htb]
\centering
\caption{Cross validation p-value for the mental demand average on each method for sighted users.}
\label{tab:lsd_md_avg_two_way_sight}
\begin{tabular}{rcllr}
\toprule
      \multicolumn{3}{c}{Method} &                          \multicolumn{2}{c}{Analysis} \\
\midrule
       Audio & $X$ & Haptic Belt &        $H_1 : \mu_{Audio} \ne \mu_{Haptic Belt}$ & ** \\
      Audio & $X$ & Virtual Cane &       $H_1 : \mu_{Audio} \ne \mu_{Virtual Cane}$ & ** \\
           Audio & $X$ & Mixture &                $H_0 : \mu_{Audio} = \mu_{Mixture}$ &  \\
Haptic Belt & $X$ & Virtual Cane & $H_1 : \mu_{Haptic Belt} \ne \mu_{Virtual Cane}$ & ** \\
     Haptic Belt & $X$ & Mixture &      $H_1 : \mu_{Haptic Belt} \ne \mu_{Mixture}$ & ** \\
    Virtual Cane & $X$ & Mixture &     $H_1 : \mu_{Virtual Cane} \ne \mu_{Mixture}$ & ** \\
\bottomrule
\end{tabular}
\end{table}



The Table \ref{tab:md_var_average_group} shows the average of the mental demand variation between the rounds. This table shows that the mental demand variation from the “Audio” has the lower variation, and the rest are similar variations.


\begin{table}[!htb]
\centering
\caption{Mental demand variation grouped by participant and visual condition}
\label{tab:md_var_average_group}
\begin{tabular}{lrrrrrr}
\toprule
{} &  Base & Audio & \begin{tabular}[c]{@{}l@{}}Haptic\\ Belt\end{tabular} & \begin{tabular}[c]{@{}l@{}}Virtual\\ Cane\end{tabular} & Mixture \\
Visual Condition &       &       &                                                       &                                                        &         \\
\midrule
Blind            &  -2.5 &  -1.0 &                                                  -2.2 &                                                   -2.2 &    -2.2 \\
Sight            &  -1.0 &  -2.0 &                                                  -2.0 &                                                   -2.8 &    -1.8 \\
\bottomrule
\end{tabular}
\end{table}



The Figures \ref{fig:qqplot_md_var_sight} and \ref{fig:residplot_md_var_sight} shows the distribution and variance of the mental demand variation of the Table \ref{tab:md_table_blind}. These Figures shows that the data are normally distributed and that the methods have a similar variance.
The Table \ref{tab:blocanova_md_var_sight} shows the Anova test p-value of the mental demand of the "sight" sample between the guidance methods. The p-value indicates that there is no influence of the methods in the variation of mental demand between the rounds. 


\begin{table}[!htb]
\centering
\caption{Anova p-value for the mental demand variation on each method for sighted users.}
\label{tab:blocanova_md_var_sight}
\begin{tabular}{lrrrrr}
\toprule
Source & P-Value \\
\midrule
Method &   0.900 \\
\bottomrule
\end{tabular}
\end{table}



\begin{figure}[!htb]
    \centering
    \begin{minipage}{0.45\textwidth}
        \centering
        \includegraphics[width = 0.8\linewidth]{Resultados/Nasa/Figuras/png/qqplot_md_var_sight.png}
        \caption{Residual plot of the mental demand variation of the blind participants on each method.}
        \label{fig:qqplot_md_var_sight}
    \end{minipage}
    \begin{minipage}{0.45\textwidth}
        \centering
        \includegraphics[width = 0.8\linewidth]{Resultados/Nasa/Figuras/png/residplot_md_var_sight.png}
        \caption{Residual plot of the mental demand variation of the sighted participants on each method.}
        \label{fig:residplot_md_var_sight}
    \end{minipage}
\end{figure}

%The Table \ref{tab:lsdbloc_mental_demand_var} presents the conclusion of a pairwise Fisher LSD test of the blind mental demand between all the guidance methods. The results show that all methods have similar variations.

%\input{Resultados/Nasa/Tabelas/lsdbloc_mental_demand_var.tex}

To close up, according to the ANOVA test at Table \ref{tab:lsd_md_avg_two_way_sight} the method do have influence on the mental demand of the sighted participant and that the "Audio" and the "Mixture" method have the same mental demand for them. This differs from the result of the previous section that was the ANOVA did not prove any effect and that the "Audio" and "Mixture" methods could not be said to be similar. Although for the "blind" users, they were also the methods that caused the lowest mental demand.

There is no influence in the tested methods in the participants mental demand variation, as shown in the Table \ref{tab:blocanova_md_var_sight}.

\FloatBarrier

%%%%%%%%%%%%%%%%%%%%%%%%%%%%%%%%%%%%%%%%%%%%%%%%%%%%%%%%%%%%%%%%%%%%%%%%%%%
%%%%%%%%%%%%%%%%%%%%%%%%%%%%%%%%%%%%%%%%%%%%%%%%%%%%%%%%%%%%%%%%%%%%%%%%%%%
%%%%%%%%%%%%%%%%%%%%%%%%%%%%%%%%%%%%%%%%%%%%%%%%%%%%%%%%%%%%%%%%%%%%%%%%%%%
%%%%%%%%%%%%%%%%%%%%%%%%%%%%%%%%%%%%%%%%%%%%%%%%%%%%%%%%%%%%%%%%%%%%%%%%%%%


\paragraph{Analysis of the NASA-TLX score}\mbox{}\\

The Table \ref{tab:nasa_table_noBase} presents the NASA-TLX score of all participants on each scene and their average are plotted in the Figures from \ref{fig:barplot_nasa_avg_4_scene_blind} to \ref{fig:barplot_nasa_avg_4_scene}. 


\begin{table}[!htb]
\centering
\caption{NASA-TLX score felled by the participants.}
\label{tab:nasa_table_noBase}
\begin{tabular}{lllrrrrr}
\toprule
    &       &        &  Audio & \begin{tabular}[c]{@{}l@{}}Haptic\\ Belt\end{tabular} & \begin{tabular}[c]{@{}l@{}}Virtual\\ Cane\end{tabular} & Mixture \\
Participant & \begin{tabular}[c]{@{}l@{}}Visual\\ Condition\end{tabular} & Round &        &                                                       &                                                        &         \\
\midrule
001C & Blind & First &  4.000 &                                                 8.833 &                                                  5.167 &   6.333 \\
    &       & Return &  4.000 &                                                 6.667 &                                                  4.500 &   6.167 \\
002C & Blind & First &  4.833 &                                                 4.833 &                                                  9.000 &   7.000 \\
    &       & Return &  4.833 &                                                 4.833 &                                                  7.000 &   5.167 \\
003C & Blind & First &  4.000 &                                                 5.333 &                                                  6.667 &   3.500 \\
    &       & Return &  3.833 &                                                 3.667 &                                                  3.500 &   3.500 \\
004C & Blind & First & 10.000 &                                                12.667 &                                                  9.667 &  11.000 \\
    &       & Return &  9.167 &                                                11.667 &                                                  9.333 &  10.833 \\
001 & Sight & First & 10.167 &                                                 9.833 &                                                  7.000 &   9.000 \\
    &       & Return & 11.000 &                                                10.833 &                                                  6.167 &   9.333 \\
003 & Sight & First &  9.833 &                                                10.167 &                                                  9.500 &   6.500 \\
    &       & Return &  6.667 &                                                 9.667 &                                                  7.833 &   4.833 \\
004 & Sight & First & 14.833 &                                                13.667 &                                                 11.500 &  15.833 \\
    &       & Return & 11.833 &                                                11.833 &                                                 10.833 &  12.167 \\
005 & Sight & First &  7.667 &                                                 9.000 &                                                  8.000 &   9.667 \\
    &       & Return &  7.667 &                                                 8.667 &                                                  7.667 &   6.000 \\
\bottomrule
\end{tabular}
\end{table}



The Figures \ref{fig:barplot_nasa_avg_4_scene_blind} and \ref{fig:barplot_nasa_avg_4_scene_sight} also show a decrease of the score on in all methods between the rounds for both groups and the Figure \ref{fig:barplot_nasa_avg_4_scene} shows that the average of each method was very different between the two groups, as it was with the mental demand.

\begin{figure}[!htb]
    \centering
    \begin{minipage}{\textwidth}
        \centering
        \includegraphics[width = 0.8\linewidth]{Resultados/Nasa/Figuras/png/barplot_nasa_avg_4_scene_blind.png}
        \caption{Barplot of the NASA-TLX score of the blind participants on each method and round.}
        \label{fig:barplot_nasa_avg_4_scene_blind}
    \end{minipage}
    \begin{minipage}{\textwidth}
        \centering
        \includegraphics[width = 0.8\linewidth]{Resultados/Nasa/Figuras/png/barplot_nasa_avg_4_scene_sight.png}
        \caption{Barplot of the NASA-TLX score of the sight participants on each method and round.}
        \label{fig:barplot_nasa_avg_4_scene_sight}
    \end{minipage}
\end{figure}
\begin{figure}[!htb]
    \centering
    \includegraphics[width = 0.8\linewidth]{Resultados/Nasa/Figuras/png/barplot_nasa_avg_4_scene.png}
    \caption{Barplot of the NASA-TLX score of both participants on each method.}
    \label{fig:barplot_nasa_avg_4_scene}
\end{figure}

The Figure \ref{fig:boxplot_noBase_nasa_4_scene} and \ref{fig:boxplot_noBase_nasa_4_rounds} presents a box plot with the NASA-TLX score of both groups by method. These figures show that the reaction of each group is also different between the methods and rounds.

\begin{figure}[!htb]
    \centering
    \begin{minipage}{0.45\textwidth}
        \centering
        \includegraphics[width = 0.8\linewidth]{Resultados/Nasa/Figuras/png/boxplot_noBase_nasa_4_scene.png}
        \caption{Boxplot of the NASA-TLX score of the participants grouped by method.}
        \label{fig:boxplot_noBase_nasa_4_scene}
    \end{minipage}
    \begin{minipage}{0.45\textwidth}
        \centering
        \includegraphics[width = 0.8\linewidth]{Resultados/Nasa/Figuras/png/boxplot_noBase_nasa_4_rounds.png}
        \caption{Boxplot of the NASA-TLX score of the participants grouped by round.}
        \label{fig:boxplot_noBase_nasa_4_rounds}
    \end{minipage}
\end{figure}

The Table \ref{tab:nasa_average_group_noBase} shows the average NASA-TLX score of both samples and is possible to notice how the average perceived NASA-TXL average by the sight sample was also higher in every method.


\begin{table}[!htb]
\centering
\caption{Average NASA-TLX score grouped by participant and visual condition}
\label{tab:nasa_average_group_noBase}
\begin{tabular}{lrrrrrr}
\toprule
{} & Audio & \begin{tabular}[c]{@{}l@{}}Haptic\\ Belt\end{tabular} & \begin{tabular}[c]{@{}l@{}}Virtual\\ Cane\end{tabular} &  Mixture \\
Visual Condition &       &                                                       &                                                        &          \\
\midrule
Blind            &  5.58 &                                                  7.31 &                                                   6.85 &    6.688 \\
Sight            &  9.96 &                                                 10.46 &                                                   8.56 &    9.167 \\
\bottomrule
\end{tabular}
\end{table}



The Figures \ref{fig:qqplot_nasa_avg_two_way_sight} and \ref{fig:residplot_nasa_avg_two_way_sight} shows the distribution and variance of sighted participants in the Table \ref{tab:nasa_table_noBase}. These Figures shows that the data are normally distributed and that the methods have a similar variance.
The Table \ref{tab:blocanova_nasa_avg_two_way_sight} shows the ANOVA test p-values of the NASA-TLX score of the "sight" sample between the guidance methods and they show that the round has an effect on the score.


\begin{table}[!htb]
\centering
\caption{Anova p-value for the NASA-TLX score on each method for sighted users.}
\label{tab:blocanova_nasa_avg_two_way_sight}
\begin{tabular}{lrrrrl}
\toprule
               Source &  Squared sum &  DOF & Squared average &      F & \begin{tabular}[c]{@{}l@{}}P-Value \\ $(F_{0} > F)$\end{tabular} \\
\midrule
Participants (Blocks) &      120.766 &    3 &          40.255 & 17.948 &                                                                  \\
         \    Methods &       16.905 &    3 &           5.635 &  2.512 &                                                            0.086 \\
          \    Rounds &       11.480 &    1 &          11.480 &  5.118 &                                                          0.034** \\
     \    Interaction &        3.343 &    3 &           1.114 &  0.497 &                                                            0.688 \\
   Experimental Error &       47.102 &   21 &           2.243 &        &                                                                  \\
                Total &      199.596 &   31 &                 &        &                                                                  \\
\bottomrule
\end{tabular}
\end{table}



\begin{figure}[!htb]
    \centering
    \begin{minipage}{0.45\textwidth}
        \centering
        \includegraphics[width = 0.8\linewidth]{Resultados/Nasa/Figuras/png/qqplot_nasa_avg_two_way_sight.png}
        \caption{QQ plot of the NASA-TLX score of the sight participants on each method.}
        \label{fig:qqplot_nasa_avg_two_way_sight}
    \end{minipage}
    \begin{minipage}{0.45\textwidth}
        \centering
        \includegraphics[width = 0.8\linewidth]{Resultados/Nasa/Figuras/png/residplot_nasa_avg_two_way_sight.png}
        \caption{Residual plot of the NASA-TLX score the sight participants on each method.}
        \label{fig:residplot_nasa_avg_two_way_sight}
    \end{minipage}
\end{figure}

The Table \ref{tab:lsd_nasa_avg_two_way_sight} presents the conclusion of a pairwise Fisher LSD test of the previous ANOVA test and it shows that all the method had an effect in the NASA-TLX score.

\input{Resultados/Nasa/Tabelas/lsd_nasa_avg_two_way_sight.tex}

The Table \ref{tab:nasa_var_group} shows the average of the NASA-TLX score variation between the rounds. This table shows that the score variation from the “Audio” has the lower variation, and the rest are similar variations.


\begin{table}[!htb]
\centering
\caption{NASA-TLX score grouped by participant and visual Condition.}
\label{tab:nasa_var_group}
\begin{tabular}{lrrrrrr}
\toprule
{} &     Base &    Audio & \begin{tabular}[c]{@{}l@{}}Haptic\\ Belt\end{tabular} & \begin{tabular}[c]{@{}l@{}}Virtual\\ Cane\end{tabular} &  Mixture \\
Visual Condition &          &          &                                                       &                                                        &          \\
\midrule
Blind            &  -13.7\% &   -3.1\% &                                               -15.9\% &                                                -21.5\% &   -7.6\% \\
Sight            &   -1.4\% &  -11.1\% &                                                -3.0\% &                                                 -9.9\% &  -20.8\% \\
\bottomrule
\end{tabular}
\end{table}



The Figures \ref{fig:qqplot_nasa_var_sight} and \ref{fig:residplot_nasa_var_sight} shows the distribution and variance of the NASA-TLX score variation of the Table \ref{tab:md_table_blind}. These Figures shows that the data are normally distributed and that the methods have a similar variance.
The Table \ref{tab:blocanova_nasa_var_sight} shows the Anova test p-value of the NASA-TLX score of the "sight" sample between the guidance methods and it proves that there is no influence of the methods in the variation of score between the rounds. 


\begin{table}[!htb]
\centering
\caption{Anova p-value for the NASA score variation on each method for sighted users.}
\label{tab:blocanova_nasa_var_sight}
\begin{tabular}{lrrrrr}
\toprule
               Source &  Squared sum &  DOF & Squared average &     F & \begin{tabular}[c]{@{}l@{}}P-Value \\ $(F_{0} > F)$\end{tabular} \\
\midrule
Participants (blocks) &       15.436 &    3 &           2.229 & 3.517 &                                                                  \\
               Method &        6.686 &    3 &           5.145 & 1.523 &                                                            0.274 \\
   Experimental error &       13.168 &    9 &           1.463 &       &                                                                  \\
                Total &       35.290 &   15 &                 &       &                                                                  \\
\bottomrule
\end{tabular}
\end{table}



\begin{figure}[!htb]
    \centering
    \begin{minipage}{0.45\textwidth}
        \centering
        \includegraphics[width = 0.8\linewidth]{Resultados/Nasa/Figuras/png/qqplot_nasa_var_sight.png}
        \caption{Residual plot of the variation NASA-TLX score of the blind participants on each method.}
        \label{fig:qqplot_nasa_var_sight}
    \end{minipage}
    \begin{minipage}{0.45\textwidth}
        \centering
        \includegraphics[width = 0.8\linewidth]{Resultados/Nasa/Figuras/png/residplot_nasa_var_sight.png}
        \caption{Residual plot of the variation NASA-TLX score of the sighted participants on each method.}
        \label{fig:residplot_nasa_var_sight}
    \end{minipage}
\end{figure}

%The Table \ref{tab:lsdbloc_mental_demand_var} presents the conclusion of a pairwise Fisher LSD test of the blind mental demand between all the guidance methods. The results show that all methods have similar variations.

%\input{Resultados/Nasa/Tabelas/lsdbloc_mental_demand_var.tex}

To close up, according to the ANOVA test at Table \ref{tab:blocanova_nasa_avg_two_way_sight} and the LSD Fisher Test at Table \ref{tab:lsd_nasa_avg_two_way_sight} all the methods do have a particular effect on the NASA-TLX score of the sighted participant, very different than what happened with the blind users. Despite that, both groups had the lowest NASA-TLX score with "Audio" and "Mixture" method.

There is no influence in the tested methods in the participants NASA-TLX score variation, as shown in the Table \ref{tab:blocanova_nasa_var_sight}.

\FloatBarrier