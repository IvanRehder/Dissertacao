Bradley and Dunlop published two works (\citeyear{bradley2002investigating,bradley2005experimental}) about how BVI navigates and how much it is similar or different to how a sighted person navigates. 

The first work of Bradley and Dunlop was published in \citeyear{bradley2002investigating} and discussed which type of information BVI uses to navigate in an environment and how it compares to sighted people. The data were collected during structured interviews where the participant had to explain how to arrive at two different locations as if they were talking to someone with the same vision condition \cite{bradley2002investigating}.

Based on the answers, the authors defined 11 categories of information: 1) directional (e.g. left/right, north/south); 2) structural (e.g. road, monument, church); 3) environmental (e.g. hill, river, tree); 4) textual-structural (e.g. name of shops, places, restaurants); 5) textual-area/street-based (e.g. name of street, neighbourhoods, squares); 6) numerical (e.g. first, second, 100m); 7) descriptive (e.g. steep, tall); 8) temporal/distance based (e.g. ”walk until you reach...” or ”before you get to”); 9) sensory (e.g. the sound of engines, the smell of bread from a bakery); 10) motion (e.g. cars passing by, doors opening); 11) social contact (e.g. asking people or using a guide dog for help) \cite{bradley2002investigating}.

As an output from the interviews, the authors provided the average number which each category was used by each group and is reproduced in Figure \ref{fig:bradley_2002}. From the results, the researchers observed that BVI participants used less text-based information than the sighted participants. However BVI participants used more words to describe a path than the sighted participants. Another essential result was that visually impaired people used, on average, 9 to 10 categories to describe a route, while sighted people used around 6 categories.

\begin{figure}[!htb]
    \centering
    \tikzstyle{barraVI} = [fill = cor1]
    \tikzstyle{barraS} = [fill = cor2]
    \tikzstyle{legenda} = [fill = white, line width = 0.25mm]
    \tikzstyle{--} = [line width = 0.25mm]
    
    \resizebox{0.8\linewidth}{!}{
    \begin{tikzpicture}[node distance=0cm]
        % Fundo do gráfico
    
        \renewcommand{\tamX}{13.875cm}
        \renewcommand{\tamY}{5.4cm}
        
        \node (origin) {};
        \node (endX) [xshift = \tamX] {};
        \node (endY) [yshift = \tamY] {};
        \node (endXY) [above of = endX, yshift = \tamY] {};
        
        %Título
        \node (titulo) [xshift = \tamX*0.5, yshift = \tamY + 1.25cm] {\textbf{Average nº of utterances used within each contextual category}}
        node [below of = titulo, yshift = -0.5cm] {\textbf{between sighted and visually impaired participants}};
        
        \draw[--] (origin.west) node[anchor = east]{\footnotesize 0} to (endX.center) node[anchor = north, xshift = -7.5cm, yshift = -2.0cm]{\textbf{Type of contextual categories}};
        \draw[--] (origin.south) to (endY.center) node[anchor = east, xshift = -1.0cm,rotate = 90]{\textbf{Average nº of utterance}};
        \draw[--] (endX.center) to (endXY.center);
        
        \foreach \r/\n in        {1/5,2/10,3/15,4/20,5/25,6/30,7/35,8/40,9/45,10/50,11/55,12/60,13/65,14/70,15/75,16/80,17/85,18/90}
        {
            \draw [--] (-0.15,0.30cm*\r) node[anchor = east]{\footnotesize \n} to (\tamX,0.30cm*\r);
        }
        
        \renewcommand{\largX}{0.25}
        \renewcommand{\altY}{0}
        \renewcommand{\distX}{0.5}
        %Direct
        \draw[barraVI] (\distX,0) node[yshift = -1.0cm, rotate = 45]{Direct} rectangle ++(\largX,5.3);
        \draw[barraS] (\distX+\largX,0) rectangle ++(\largX,1.6);
        \draw[--] (\distX+3.5*\largX,0) to ++(0,-0.2);
        
        \renewcommand{\distX}{1.75}
        %Struct
        \draw[barraVI] (\distX,0) node[yshift = -1.0cm, rotate = 45]{Struct} rectangle ++(\largX,3.6);
        \draw[barraS] (\distX+\largX,0) rectangle ++(\largX,0.5);
        \draw[--] (\distX+3.5*\largX,0) to ++(0,-0.2);
        
        \renewcommand{\distX}{3.0}
        %Struct
        \draw[barraVI] (\distX,0) node[yshift = -1.0cm, rotate = 45]{Environ} rectangle ++(\largX,0.5);
        \draw[barraS] (\distX+\largX,0) rectangle ++(\largX,0.2);
        \draw[--] (\distX+3.5*\largX,0) to ++(0,-0.2);
        
        \renewcommand{\distX}{4.25}
        %Struct
        \draw[barraVI] (\distX,0) node[yshift = -1.0cm, rotate = 45]{Text-struct} rectangle ++(\largX,0.2);
        \draw[barraS] (\distX+\largX,0) rectangle ++(\largX,0.4);
        \draw[--] (\distX+3.5*\largX,0) to ++(0,-0.2);
        
        \renewcommand{\distX}{5.5}
        %Struct
        \draw[barraVI] (\distX,0) node[yshift = -1.0cm, rotate = 45]{Text-area/st} rectangle ++(\largX,0.5);
        \draw[barraS] (\distX+\largX,0) rectangle ++(\largX,0.7);
        \draw[--] (\distX+3.5*\largX,0) to ++(0,-0.2);
        
        \renewcommand{\distX}{6.75}
        %Struct
        \draw[barraVI] (\distX,0) node[yshift = -1.0cm, rotate = 45]{Numer} rectangle ++(\largX,1.3);
        \draw[barraS] (\distX+\largX,0) rectangle ++(\largX,0.2);
        \draw[--] (\distX+3.5*\largX,0) to ++(0,-0.2);
        
        \renewcommand{\distX}{8.0}
        %Struct
        \draw[barraVI] (\distX,0) node[yshift = -1.0cm, rotate = 45]{Desc} rectangle ++(\largX, 4.2);
        \draw[barraS] (\distX+\largX,0) rectangle ++(\largX,0.45);
        \draw[--] (\distX+3.5*\largX,0) to ++(0,-0.2);
        
        \renewcommand{\distX}{9.25}
        %Struct
        \draw[barraVI] (\distX,0) node[yshift = -1.0cm, rotate = 45]{Sensory} rectangle ++(\largX,0.8);
        \draw[barraS] (\distX+\largX,0) rectangle ++(\largX,0.0);
        \draw[--] (\distX+3.5*\largX,0) to ++(0,-0.2);
        
        \renewcommand{\distX}{10.5}
        %Struct
        \draw[barraVI] (\distX,0) node[yshift = -1.0cm, rotate = 45]{Tem/Dist} rectangle ++(\largX,0.95);
        \draw[barraS] (\distX+\largX,0) rectangle ++(\largX,0.35);
        \draw[--] (\distX+3.5*\largX,0) to ++(0,-0.2);
        
        \renewcommand{\distX}{11.75}
        %Struct
        \draw[barraVI] (\distX,0) node[yshift = -1.0cm, rotate = 45]{Motion} rectangle ++(\largX,0.1);
        \draw[barraS] (\distX+\largX,0) rectangle ++(\largX,0.0);
        \draw[--] (\distX+3.5*\largX,0) to ++(0,-0.2);
        
        \renewcommand{\distX}{13.0}
        %Struct
        \draw[barraVI] (\distX,0) node[yshift = -1.0cm, rotate = 45]{Social} rectangle ++(\largX,0.2);
        \draw[barraS] (\distX+\largX,0) rectangle ++(\largX,0.0);
        \draw[--] (\distX+3.5*\largX,0) to ++(0,-0.2);
        
        %Legenda
        \draw[legenda] (\tamX-3.0cm,\tamY-1.5cm) rectangle (\tamX+1.5cm, \tamY+0.25cm);
        \draw[barraVI] (\tamX-2.5cm,\tamY-0.25cm) rectangle ++(0.25cm,0.25cm) node[anchor = west, xshift = 0.15cm, yshift = -0.15cm]{Visualy Impaired};
        \draw[barraS] (\tamX-2.5cm,\tamY-1.0cm) rectangle ++(0.25cm,0.25cm) node[anchor = west, xshift = 0.15cm, yshift = -0.15cm]{Sighted};
        
    \end{tikzpicture}
    }
    \centering
    \caption{Comparison between sighted participants with BVI participants (Adapted from \citeonline{bradley2002investigating}).}
    \label{fig:bradley_2002}
\end{figure}

Among the comments provided by BVI participants, a common one was about the limitations of available navigation methods, such as white canes and guide dogs. They also emphasize that, when navigating, using different senses is essential for confirming one piece of information. 

To extend the findings of their previous work, Bradley and Dunlop designed an experiment to investigate if there is a difference between the perceived workload of BVI participants and sighted participants when they navigate using user-tailored information created with the results of the previous experiments \cite{bradley2005experimental}.

The experiment was performed with 16 participants, 8 sighted and 8 BVI, who were recruited to walk to four pre-determined landmarks in the centre of Glasgow. They followed the orientations recorded during the interviews from their previous work. For each participant, orientations for 2 of the 4 landmarks were made using sighted users' interviews, while the other 2 used data from BVI interviews. The results showed that BVI users reached landmarks significantly quicker when given the information made for that group, but still longer than sighted users. 

Another issue analysed during the experiment was the perceived workload. After each landmark, the participant was asked to complete the NASA-TLX questionnaire. The average score for each dimension of the NASA-TLX is reproduced in Figure \ref{fig:bradley_2005_participants}. As expected, it shows that BVI participants systematically have a higher workload than sighted participants. It also confirms that BVI did have a higher workload when guided by orientations provided by sighted people, as well as the sighted participants did with orientations from BVI. Another essential piece of information that stands out is the high frustration score given by the BVI users when they were guided by the orientations of sighted people.

\begin{figure}[htbp]
    \centering
    \begin{subfigure}{.49\textwidth}
        \centering
        \resizebox{\linewidth}{!}{
        \tikzstyle{barraVI} = [fill = cor1]
\tikzstyle{barraS} = [fill = cor2]
\tikzstyle{legenda} = [fill = white, line width = 0.25mm]
\tikzstyle{--} = [line width = 0.25mm]

%\resizebox{0.8\linewidth}{!}{
\begin{tikzpicture}[node distance=0cm]
    % Fundo do gráfico

    \renewcommand{\tamX}{12.0cm}
    \renewcommand{\tamY}{7.0cm}
    
    \node (origin) {};
    \node (endX) [xshift = \tamX] {};
    \node (endY) [yshift = \tamY] {};
    \node (endXY) [above of = endX, yshift = \tamY] {};
    
    %Título
    \node (titulo) [xshift = \tamX*0.5, yshift = \tamY + 1cm] {\textbf{\LARGE Comparing group scores for condition 1}};
    \draw[--] (origin.west) node[anchor = east]{\Large 0} to (endX.center) node[anchor = north, xshift = -\tamX*0.5, yshift = -1.0cm]{\textbf{\LARGE Workload dimensions}};
    \draw[--] (origin.south) to (endY.center) 
    node(eixoY)[anchor = east, xshift = -2.5cm, yshift = 0.5cm, rotate = 90]{\textbf{\LARGE Average weighted score}};
    \draw[--] (endX.center) to (endXY.center);
    
   \foreach \r/\n in {1/50,2/100,3/150,4/200,5/250,6/300, 7/350}
    {
        \draw [--] (-0.15,1cm*\r) node[anchor = east]{\Large \n} to (\tamX,1cm*\r);
    }
    
    \renewcommand{\largX}{0.5}
    \renewcommand{\altY}{0}
    \renewcommand{\distX}{0.5}
    
    %Mental Demand
    \draw[barraVI] (\distX,0) node[xshift = \largX*1cm, yshift = -0.5cm]{\textbf{\LARGE MD}} rectangle ++(\largX,6.1);
    \draw[barraS] (\distX+\largX,0) rectangle ++(\largX,2.8);
    \draw[--] (\distX+3*\largX,0) to ++(0,-0.2);
    
    \renewcommand{\distX}{2.5}
    %Physical Demand
    \draw[barraVI] (\distX,0) node[xshift = \largX*1cm, yshift = -0.5cm]{\textbf{\LARGE PD}} rectangle ++(\largX,0.3);
    \draw[barraS] (\distX+\largX,0) rectangle ++(\largX,0.5);
    \draw[--] (\distX+3*\largX,0) to ++(0,-0.2);
    
    \renewcommand{\distX}{4.5}
    %Temporal demand
    \draw[barraVI] (\distX,0) node[xshift = \largX*1cm, yshift = -0.5cm]{\textbf{\LARGE TD}} rectangle ++(\largX,0.9);
    \draw[barraS] (\distX+\largX,0) rectangle ++(\largX,0.7);
    \draw[--] (\distX+3*\largX,0) to ++(0,-0.2);
    
    \renewcommand{\distX}{6.5}
    %Performance
    \draw[barraVI] (\distX,0) node[xshift = \largX*1cm, yshift = -0.5cm]{\textbf{\LARGE OP}} rectangle ++(\largX,2.2);
    \draw[barraS] (\distX+\largX,0) rectangle ++(\largX,0.6);
    \draw[--] (\distX+3*\largX,0) to ++(0,-0.2);
    
    \renewcommand{\distX}{8.5}
    %Effort
    \draw[barraVI] (\distX,0) node[xshift = \largX*1cm, yshift = -0.5cm]{\textbf{\LARGE EF}} rectangle ++(\largX,4.1);
    \draw[barraS] (\distX+\largX,0) rectangle ++(\largX,2.3);
    \draw[--] (\distX+3*\largX,0) to ++(0,-0.2);
    
    \renewcommand{\distX}{10.5}
    %Frustation
    \draw[barraVI] (\distX,0) node[xshift = \largX*1cm, yshift = -0.5cm]{\textbf{\LARGE FR}} rectangle ++(\largX,3.7);
    \draw[barraS] (\distX+\largX,0) rectangle ++(\largX,0.2);
    \draw[--] (\distX+3*\largX,0) to ++(0,-0.2);
    
    
    %Legenda
    \draw[legenda] (\tamX-3.0cm,\tamY-1.5cm) rectangle ++(6.5cm,2cm);
    \draw[barraVI] (\tamX-2.75cm,\tamY-0.25cm) rectangle ++(0.25cm,0.25cm) 
    node[anchor = west, xshift = 0.15cm, yshift = -0.15cm]{\LARGE Visually impaired};
    \draw[barraS] (\tamX-2.75cm,\tamY-1.0cm) rectangle ++(0.25cm,0.25cm) 
    node[anchor = west, xshift = 0.15cm, yshift = -0.15cm]{\LARGE Sighted};
    
\end{tikzpicture}
%}
        }
        \caption{Condition 1.}
        \label{fig:bradley_2005_nasa_participants_1}
    \end{subfigure}
    \hfill
    \begin{subfigure}{.49\textwidth}
        \centering
        \resizebox{\linewidth}{!}{
        \tikzstyle{barraVI} = [fill = cor1]
\tikzstyle{barraS} = [fill = cor2]
\tikzstyle{legenda} = [fill = white, line width = 0.25mm]
\tikzstyle{--} = [line width = 0.25mm]

%\resizebox{0.8\linewidth}{!}{
\begin{tikzpicture}[node distance=0cm]
    % Fundo do gráfico

    \renewcommand{\tamX}{12.0cm}
    \renewcommand{\tamY}{6.0cm}
    
    \node (origin) {};
    \node (endX) [xshift = \tamX] {};
    \node (endY) [yshift = \tamY] {};
    \node (endXY) [above of = endX, yshift = \tamY] {};
    
    %Título
    \node (titulo) [xshift = \tamX*0.5, yshift = \tamY + 1cm] {\textbf{\LARGE Orientation from BVI people}};
    \draw[--] (origin.west) node[anchor = east]{\Large 0} to (endX.center) node[anchor = north, xshift = -\tamX*0.5, yshift = -5.0cm]{\textbf{\LARGE Workload dimensions}};
    \draw[--] (origin.south) to (endY.center) 
    node(eixoY)[anchor = east, xshift = -2.5cm, yshift = 0.5cm, rotate = 90]{\textbf{\LARGE Average weighted score}};
    \draw[--] (endX.center) to (endXY.center);
    
    \foreach \r/\n in {1/50,2/100,3/150,4/200,5/250,6/300}
    {
        \draw [--] (-0.15,1cm*\r) node[anchor = east]{\Large \n} to (\tamX,1cm*\r);
    }
    
    \renewcommand{\largX}{0.5}
    \renewcommand{\altY}{0}
    \renewcommand{\distX}{0.5}
    
    %Mental Demand
    %\draw[barraVI] (\distX,0) node[xshift = \largX*1cm, yshift = -0.5cm]{\textbf{\LARGE MD}} rectangle ++(\largX,5.6);
    \draw[barraVI] (\distX,0) node[xshift = \largX*-1.65cm, yshift = -2.25cm]{\rotatebox{50}{\textbf{\LARGE Mental Demand}}} rectangle ++(\largX,5.6);
    \draw[barraS] (\distX+\largX,0) rectangle ++(\largX,3.1);
    \draw[--] (\distX+3*\largX,0) to ++(0,-0.2);
    
    \renewcommand{\distX}{2.5}
    %Physical Demand
    %\draw[barraVI] (\distX,0) node[xshift = \largX*1cm, yshift = -0.5cm]{\textbf{\LARGE PD}} rectangle ++(\largX,0.7);
    \draw[barraVI] (\distX,0) node[xshift = \largX*-1.75cm, yshift = -2.25cm]{\rotatebox{50}{\textbf{\LARGE Physical Demand}}} rectangle ++(\largX,0.7);
    \draw[barraS] (\distX+\largX,0) rectangle ++(\largX,0.5);
    \draw[--] (\distX+3*\largX,0) to ++(0,-0.2);
    
    \renewcommand{\distX}{4.5}
    %Temporal demand
    %\draw[barraVI] (\distX,0) node[xshift = \largX*1cm, yshift = -0.5cm]{\textbf{\LARGE TD}} rectangle ++(\largX,0.9);
    \draw[barraVI] (\distX,0) node[xshift = \largX*-2cm, yshift = -2.5cm]{\rotatebox{50}{\textbf{\LARGE Temporal Demand}}} rectangle ++(\largX,0.9);
    \draw[barraS] (\distX+\largX,0) rectangle ++(\largX,0.7);
    \draw[--] (\distX+3*\largX,0) to ++(0,-0.2);
    
    \renewcommand{\distX}{6.5}
    %Performance
    %\draw[barraVI] (\distX,0) node[xshift = \largX*1cm, yshift = -0.5cm]{\textbf{\LARGE OP}} rectangle ++(\largX,2.1);
    \draw[barraVI] (\distX,0) node[xshift = \largX*-1.25cm, yshift = -1.65cm]{\rotatebox{50}{\textbf{\LARGE Performance}}} rectangle ++(\largX,2.1);
    \draw[barraS] (\distX+\largX,0) rectangle ++(\largX,0.5);
    \draw[--] (\distX+3*\largX,0) to ++(0,-0.2);
    
    \renewcommand{\distX}{8.5}
    %Effort
    %\draw[barraVI] (\distX,0) node[xshift = \largX*1cm, yshift = -0.5cm]{\textbf{\LARGE EF}} rectangle ++(\largX,3.8);
    \draw[barraVI] (\distX,0) node[xshift = \largX*-0.0cm, yshift = -0.850cm]{\rotatebox{50}{\textbf{\LARGE Effort}}} rectangle ++(\largX,3.8);
    \draw[barraS] (\distX+\largX,0) rectangle ++(\largX,2.5);
    \draw[--] (\distX+3*\largX,0) to ++(0,-0.2);
    
    \renewcommand{\distX}{10.5}
    %Frustation
    %\draw[barraVI] (\distX,0) node[xshift = \largX*1cm, yshift = -0.5cm]{\textbf{\LARGE FR}} rectangle ++(\largX,1.5);
    \draw[barraVI] (\distX,0) node[xshift = \largX*-1.0cm, yshift = -1.35cm]{\rotatebox{50}{\textbf{\LARGE Frustation}}} rectangle ++(\largX,1.5);
    \draw[barraS] (\distX+\largX,0) rectangle ++(\largX,1.1);
    \draw[--] (\distX+3*\largX,0) to ++(0,-0.2);
    
    
    %Legenda
    \draw[legenda] (\tamX-3.0cm,\tamY-1.5cm) rectangle ++(6.5cm,2cm);
    \draw[barraVI] (\tamX-2.75cm,\tamY-0.25cm) rectangle ++(0.25cm,0.25cm) 
    node[anchor = west, xshift = 0.15cm, yshift = -0.15cm]{\LARGE BVI};
    \draw[barraS] (\tamX-2.75cm,\tamY-1.0cm) rectangle ++(0.25cm,0.25cm) 
    node[anchor = west, xshift = 0.15cm, yshift = -0.15cm]{\LARGE Sighted};
    
\end{tikzpicture}
%}
        }
        \caption{Condition 2.}
        \label{fig:bradley_2005_nasa_participants_2}
    \end{subfigure}
\caption{Comparison of the NASA-TLX between the participants (Adapted from \citeonline{bradley2005experimental}).}
\label{fig:bradley_2005_participants}
\end{figure}

The work of \citeonline{bradley2002investigating} brings some relevant information for developing this work. Firstly, it shows the differences between the way sighted and BVI people navigate, highlighting the importance of including BVI in the design process of assistive technologies. It confirms the limitations of the current solutions. It brings essential insights on what type of information to include in developing audio systems, which is one of the assistive devices evaluated in this work. Finally, it shows the importance of using different workload assessment methods when evaluating assistive technologies.