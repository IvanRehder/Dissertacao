\citeonline{kirner2011using} raised two questions, "How can blind people learn 3D concepts aiming to be able to convert explored 3D environments into pictures?" and "How can we develop a spatial audio tutor with augmented reality technology to make easy the understanding of 3D concepts by blind people?" and used not using virtual reality technology but augmented reality to answer them. They developed a augmented reality application to be a tutor for BVI users. The application used allowed BVI users to play audio streams that were associeated with spatial positions.

The application had 4 different environments and they for played in sequence to teach BVI users the idea of perspective in images and each environment had. These environments were designed as a sequence which would make easy for the user to understand the concept of depth in a 2D image. These environment also had virtual points and a physical board to be interacted. These virtual points, which were coincidente with specific physical locations of each environment, allowed the user to hear audible information when using the application. When the BVI user explored the environment, s/he used different marker. These marker allow the user to execute commands on the virtual points like inspect, erase, copy. The users were allowed to explore the environments by themselves

To test the application, ten BVI users have used it, making comments about the learning of perspective concepts. They explored the four environments in the same sequence. The users learned 3D concepts and also were able to  perceive, understand and produce embossed pictures representing real and imaginary 3D scenes. Also they were able to understand descriptions of 3D scenes described by non-BVI people. The authors believe that this application can be evolved to explain other concepts such as colors, transparency, shades, etc. 

This work uses sounds and touch (haptic) to teach BVI users 3D concepts that it would be hard for them to learn on their own. It is a good application of augmented reality and uses the same principles of communicating information for BVI users as the one of this master thesis.