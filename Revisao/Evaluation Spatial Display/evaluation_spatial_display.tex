\citeonline{marston2006evaluation} compare assistive devices for BVI. Two guidance displays were evaluated, one based on haptics and another based on sound. They were tested in two scenarios: a busy street block with a variety of street furniture, parked bicycles and people, and a park, with paths made of concrete, crushed gravel and paver blocks. 

The experiment was performed with 8 BVI participants. As output, the authors collected the time to reach a set of waypoints, the errors made by the participants, the travelled distance and the percentage of the total time that the users accessed the guidance device. All participants were able to complete the task with both devices. However, the configuration (audio x haptic, street x park) that resulted in the best performance varied among the participants. One relevant consideration is that the haptic device caused strain on the participants' arm and was considered less acceptable when compared to the sound device, which required no use of the arms.

Similar to the study of \citeonline{marston2006evaluation}, this work also compares different devices based on haptics and sound. But, complementary to \citeonline{marston2006evaluation}, this work also evaluates the combined use of devices, as well as a guidance system currently familiar by the BVI participant.