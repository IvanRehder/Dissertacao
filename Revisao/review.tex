This chapter discusses a set of selected works from the literature that are related to different aspects of this work. Their selection was performed in the Scopus and Web of Science databases, using keywords such as “human factors”, “virtual reality” and “blindness”. From an initial set of 344 papers, a set of seven were selected as more relevant to this work and are detailed in the next sections.

\section{Virtual reality for BVI users}
\label{sec:vr_without_vision}
Motivated by the popularization of virtual reality technology, \citeonline{siu2020virtual} developed a white cane to be used by BVI users in a virtual environment. Their purpose was to make virtual reality applications available for BVI users. 

The traditional white cane transmits three sources of information to the user: detection of obstacles, surface topography and foot placement preview. In their work, these sources of information were transmitted through sounds or haptics \cite{siu2020virtual}, which would be defined based on the cane position in the virtual environment. For obstacle detection, the cane was built with a three-degree-of-freedom brake mechanism that would stop the movement when the cane hit an obstacle. A coil actuator was used to simulate surface properties. Lastly, a wave-based acoustic simulation was used to render geometry-aware sound effects in other to give the user a sense of the surroundings (echo localization).

In order to evaluate their proposal, the authors performed an experiment where the participants had to play a “scavenger hunt” using an HTC Vive system. During the experiment, each participant had two tasks: collect targets along the way (primary task) and avoid virtual obstacles and walls (secondary task). The targets appeared, one at a time, once the previous target was collected, and they emitted a sound that acted like an audio beacon for the participant. The obstacles did not emit any sound as a beacon, but the participant could detect it by the shape and the noise it emitted when in contact with the cane. The experiment was performed with 8 blind users (4 female, 4 male) from 25 to 70 years old. All of them did a training section where the virtual environment was presented. 

Among the relevant findings of \citeonline{siu2020virtual} is that not all the participants reacted the same to a particular stimulus. The vibration of the cane was considered confusing by some participants, while others were familiar with it. This difference affected the performance of the participants. The ones that had already used vibrating devices performed better. It shows that user's previous experiences can impact their performance in the virtual environment.

Another interesting observation was that, similar to what happens in the real world, it was easier for the participants to navigate in larger areas than in tight spaces. Moreover, the authors observed that the participants focused their attention on the primary task, without freely exploring the environment, which might have impacted the low time to achieve the goal and the low number of obstacle hits. 

Among the limitations pointed out by the authors is the lack of feedback possibilities for situations such as when the obstacle contacts a point along with the cane, not the tip of it, and the fact that the brake system did not stop the participant when he/she walked forwards toward a wall.

Comparing the work of \citeonline{siu2020virtual} to this work, \citeonline{siu2020virtual} were focused on providing mechanisms for a BVI user to navigate inside virtual environments. In this work, the purpose is to use the virtual environment to collect data about how the BVI user would navigate in a real environment. Another difference is in the functioning of the virtual cane, which in this work is limited to vibration, with no brake system, as the BVI user does not need to touch the environment with it. One common observation of both works is the sound importance for the BVI guidance and the need to use high-quality spatialized audio to increase the realism of the virtual environment.

\section{Feeling of presence in virtual reality}
\label{sec:emotion_presence_vr}
One of the many feelings that flourish during the use of a VR is the feeling of presence. This feeling, inside the virtuality context, is when someone feels draw into a VE and starts to occupy the VE instead of the real one \cite{cummings2016immersive}.

\citeonline{jicol2021effects} explores this feeling in its work. The authors aim to correlate the feeling of presence with one's agency (which is the self perception that the user is in control of a situation or some actions \cite{farrer2002experiencing}) and emotion, both of these in a VE context. Besides assessing this correlation, the author also did a structural equation model (SEM) based on their findings. The author did this by creating two different VE, one that would trigger happy emotions, and another that would trigger fear. For each VE there was two different variations of it, one that the user could interact with it's elements and another that it could not. So at the end, four different VE were designed as the Figures \ref{fig:happy_without} to \ref{fig:fear_with} show.

\begin{figure}[!h] 
    \begin{subfigure}{0.45\textwidth}
    \centering
    \includegraphics[width=\textwidth]{Revisao/Emotion Presence/Scene Happy No.png} 
    \caption{Without agency.} 
    \label{fig:happy_without}
  \end{subfigure}
  \hfill
  \begin{subfigure}{0.45\textwidth}
    \centering
    \includegraphics[width=\textwidth]{Revisao/Emotion Presence/Scene Happy Yes.png}
    \caption{With agency.} 
    \label{fig:happy_with}
  \end{subfigure}
\caption{Happy environment \cite{jicol2021effects}.} 
\label{fig:happy_environment}
\end{figure}

\begin{figure}[!h] 
    \begin{subfigure}{0.45\textwidth}
    \centering
    \includegraphics[width=\textwidth]{Revisao/Emotion Presence/Scene Fear No.png}
    \caption{Without agency.} 
    \label{fig:fear_without}
  \end{subfigure}
  \hfill
  \begin{subfigure}{0.45\textwidth}
    \centering
    \includegraphics[width=\textwidth]{Revisao/Emotion Presence/Scene Fear Yes.png}
    \caption{With agency.} 
    \label{fig:fear_with}
  \end{subfigure}
\caption{Fear environment \cite{jicol2021effects}.} 
\label{fig:fear_environment}
\end{figure}

This experiment had 121 participants and they were randomly assigned to one of the four VE. Participants with a neurological disease, fear of dogs, psychological or emotional issues, epilepsy or use of medical device were excluded.

The authors had three hypothesis about their experiment:
\begin{enumerate}
    \item The intensity of the dominant emotion in each VE will correlate positively with the presence
    \item Presence will be significantly higher in environments where participants have agency
    \item Agency will moderate the effect of the emotion on the presence
\end{enumerate}

The first hypothesis was confirmed. No matter if the feeling is positive (happiness) or negative (fear), the users did felt a stronger presence when the positive or negative feeling were more intense.

The second hypothesis was partially correct. In the VE that provoked fear, agency did make a difference and induced a higher feeling of presence, whilst in the VE that provoked happiness, agency did not affected the presence. The same could be said about the third hypothesis.

This is an important work for its findings about the user's presence feeling. Inside a VE, users that have a direct interaction inside it do find a bigger feeling of presence. This is important for this master's thesis experiment. It is possible that, if the participant did not feel "present" inside the VE, the gathered data could be less sensitive to the experiment's goals.

This experiment did not assess directly the feeling of presence, but the feeling of presence inside a VE with BVI users could be a suggestion for future works or even a base study.



\section{Information for BVI navigation}
\label{sec:bradley_dunlop}
Bradley and Dunlop published two works (\citeyear{bradley2002investigating,bradley2005experimental}) about how BVI navigates and how much it is similar or different to how a sighted person navigates. 

The first work of Bradley and Dunlop was published in \citeyear{bradley2002investigating} and discussed which type of information BVI uses to navigate in an environment and how it compares to sighted people. The data were collected during structured interviews where the participant had to explain how to arrive at two different locations as if they were talking to someone with the same vision condition \cite{bradley2002investigating}.

Based on the answers, the authors defined 11 categories of information: 1) directional (e.g. left/right, north/south); 2) structural (e.g. road, monument, church); 3) environmental (e.g. hill, river, tree); 4) textual-structural (e.g. name of shops, places, restaurants); 5) textual-area/street-based (e.g. name of street, neighbourhoods, squares); 6) numerical (e.g. first, second, 100m); 7) descriptive (e.g. steep, tall); 8) temporal/distance based (e.g. ”walk until you reach...” or ”before you get to”); 9) sensory (e.g. the sound of engines, the smell of bread from a bakery); 10) motion (e.g. cars passing by, doors opening); 11) social contact (e.g. asking people or using a guide dog for help) \cite{bradley2002investigating}.

As an output from the interviews, the authors provided the average number which each category was used by each group and is reproduced in Figure \ref{fig:bradley_2002}. From the results, the researchers observed that BVI participants used less text-based information than the sighted participants. However BVI participants used more words to describe a path than the sighted participants. Another essential result was that visually impaired people used, on average, 9 to 10 categories to describe a route, while sighted people used around 6 categories.

\begin{figure}[!htb]
    \centering
    \tikzstyle{barraVI} = [fill = cor1]
    \tikzstyle{barraS} = [fill = cor2]
    \tikzstyle{legenda} = [fill = white, line width = 0.25mm]
    \tikzstyle{--} = [line width = 0.25mm]
    
    \resizebox{0.8\linewidth}{!}{
    \begin{tikzpicture}[node distance=0cm]
        % Fundo do gráfico
    
        \renewcommand{\tamX}{13.875cm}
        \renewcommand{\tamY}{5.4cm}
        
        \node (origin) {};
        \node (endX) [xshift = \tamX] {};
        \node (endY) [yshift = \tamY] {};
        \node (endXY) [above of = endX, yshift = \tamY] {};
        
        %Título
        \node (titulo) [xshift = \tamX*0.5, yshift = \tamY + 1.25cm] {\textbf{Average nº of utterances used within each contextual category}}
        node [below of = titulo, yshift = -0.5cm] {\textbf{between sighted and visually impaired participants}};
        
        \draw[--] (origin.west) node[anchor = east]{\footnotesize 0} to (endX.center) node[anchor = north, xshift = -7.5cm, yshift = -2.0cm]{\textbf{Type of contextual categories}};
        \draw[--] (origin.south) to (endY.center) node[anchor = east, xshift = -1.0cm,rotate = 90]{\textbf{Average nº of utterance}};
        \draw[--] (endX.center) to (endXY.center);
        
        \foreach \r/\n in        {1/5,2/10,3/15,4/20,5/25,6/30,7/35,8/40,9/45,10/50,11/55,12/60,13/65,14/70,15/75,16/80,17/85,18/90}
        {
            \draw [--] (-0.15,0.30cm*\r) node[anchor = east]{\footnotesize \n} to (\tamX,0.30cm*\r);
        }
        
        \renewcommand{\largX}{0.25}
        \renewcommand{\altY}{0}
        \renewcommand{\distX}{0.5}
        %Direct
        \draw[barraVI] (\distX,0) node[yshift = -1.0cm, rotate = 45]{Direct} rectangle ++(\largX,5.3);
        \draw[barraS] (\distX+\largX,0) rectangle ++(\largX,1.6);
        \draw[--] (\distX+3.5*\largX,0) to ++(0,-0.2);
        
        \renewcommand{\distX}{1.75}
        %Struct
        \draw[barraVI] (\distX,0) node[yshift = -1.0cm, rotate = 45]{Struct} rectangle ++(\largX,3.6);
        \draw[barraS] (\distX+\largX,0) rectangle ++(\largX,0.5);
        \draw[--] (\distX+3.5*\largX,0) to ++(0,-0.2);
        
        \renewcommand{\distX}{3.0}
        %Struct
        \draw[barraVI] (\distX,0) node[yshift = -1.0cm, rotate = 45]{Environ} rectangle ++(\largX,0.5);
        \draw[barraS] (\distX+\largX,0) rectangle ++(\largX,0.2);
        \draw[--] (\distX+3.5*\largX,0) to ++(0,-0.2);
        
        \renewcommand{\distX}{4.25}
        %Struct
        \draw[barraVI] (\distX,0) node[yshift = -1.0cm, rotate = 45]{Text-struct} rectangle ++(\largX,0.2);
        \draw[barraS] (\distX+\largX,0) rectangle ++(\largX,0.4);
        \draw[--] (\distX+3.5*\largX,0) to ++(0,-0.2);
        
        \renewcommand{\distX}{5.5}
        %Struct
        \draw[barraVI] (\distX,0) node[yshift = -1.0cm, rotate = 45]{Text-area/st} rectangle ++(\largX,0.5);
        \draw[barraS] (\distX+\largX,0) rectangle ++(\largX,0.7);
        \draw[--] (\distX+3.5*\largX,0) to ++(0,-0.2);
        
        \renewcommand{\distX}{6.75}
        %Struct
        \draw[barraVI] (\distX,0) node[yshift = -1.0cm, rotate = 45]{Numer} rectangle ++(\largX,1.3);
        \draw[barraS] (\distX+\largX,0) rectangle ++(\largX,0.2);
        \draw[--] (\distX+3.5*\largX,0) to ++(0,-0.2);
        
        \renewcommand{\distX}{8.0}
        %Struct
        \draw[barraVI] (\distX,0) node[yshift = -1.0cm, rotate = 45]{Desc} rectangle ++(\largX, 4.2);
        \draw[barraS] (\distX+\largX,0) rectangle ++(\largX,0.45);
        \draw[--] (\distX+3.5*\largX,0) to ++(0,-0.2);
        
        \renewcommand{\distX}{9.25}
        %Struct
        \draw[barraVI] (\distX,0) node[yshift = -1.0cm, rotate = 45]{Sensory} rectangle ++(\largX,0.8);
        \draw[barraS] (\distX+\largX,0) rectangle ++(\largX,0.0);
        \draw[--] (\distX+3.5*\largX,0) to ++(0,-0.2);
        
        \renewcommand{\distX}{10.5}
        %Struct
        \draw[barraVI] (\distX,0) node[yshift = -1.0cm, rotate = 45]{Tem/Dist} rectangle ++(\largX,0.95);
        \draw[barraS] (\distX+\largX,0) rectangle ++(\largX,0.35);
        \draw[--] (\distX+3.5*\largX,0) to ++(0,-0.2);
        
        \renewcommand{\distX}{11.75}
        %Struct
        \draw[barraVI] (\distX,0) node[yshift = -1.0cm, rotate = 45]{Motion} rectangle ++(\largX,0.1);
        \draw[barraS] (\distX+\largX,0) rectangle ++(\largX,0.0);
        \draw[--] (\distX+3.5*\largX,0) to ++(0,-0.2);
        
        \renewcommand{\distX}{13.0}
        %Struct
        \draw[barraVI] (\distX,0) node[yshift = -1.0cm, rotate = 45]{Social} rectangle ++(\largX,0.2);
        \draw[barraS] (\distX+\largX,0) rectangle ++(\largX,0.0);
        \draw[--] (\distX+3.5*\largX,0) to ++(0,-0.2);
        
        %Legenda
        \draw[legenda] (\tamX-3.0cm,\tamY-1.5cm) rectangle (\tamX+1.5cm, \tamY+0.25cm);
        \draw[barraVI] (\tamX-2.5cm,\tamY-0.25cm) rectangle ++(0.25cm,0.25cm) node[anchor = west, xshift = 0.15cm, yshift = -0.15cm]{Visualy Impaired};
        \draw[barraS] (\tamX-2.5cm,\tamY-1.0cm) rectangle ++(0.25cm,0.25cm) node[anchor = west, xshift = 0.15cm, yshift = -0.15cm]{Sighted};
        
    \end{tikzpicture}
    }
    \centering
    \caption{Comparison between sighted participants with BVI participants (Adapted from \citeonline{bradley2002investigating}).}
    \label{fig:bradley_2002}
\end{figure}

Among the comments provided by BVI participants, a common one was about the limitations of available navigation options, such as white canes and guide dogs. They also emphasize that, when navigating, using different senses is essential for confirming one piece of information. 

To extend the findings of their previous work, Bradley and Dunlop designed an experiment to investigate if there is a difference between the perceived workload of BVI participants and sighted participants when they navigate using user-tailored information created with the results of the previous experiments \cite{bradley2005experimental}.

The experiment was performed with 16 participants, 8 sighted and 8 BVI, who were recruited to walk to four pre-determined landmarks in the centre of Glasgow. They followed the orientations recorded during the interviews from their previous work. For each participant, orientations for 2 of the 4 landmarks were made using sighted users' interviews, while the other 2 used data from BVI interviews. The results showed that BVI users reached landmarks significantly quicker when given the information made for that group, but still longer than sighted users. 

Another issue analysed during the experiment was the perceived workload. After each landmark, the participant was asked to complete the NASA-TLX questionnaire. The average score for each dimension of the NASA-TLX is reproduced in Figure \ref{fig:bradley_2005_participants}. As expected, it shows that BVI participants systematically have a higher workload than sighted participants. It also confirms that BVI did have a higher workload when guided by orientations provided by sighted people, as well as the sighted participants did with orientations from BVI. Another essential piece of information that stands out is the high frustration score given by the BVI users when they were guided by the orientations of sighted people.

\begin{figure}[htbp]
    \centering
    \begin{subfigure}{.49\textwidth}
        \centering
        \resizebox{\linewidth}{!}{
        \input{Revisao/Bradley Dunlop/Grafico_2005_Nasa_7}
        }
        \caption{Condition 1.}
        \label{fig:bradley_2005_nasa_participants_1}
    \end{subfigure}
    \hfill
    \begin{subfigure}{.49\textwidth}
        \centering
        \resizebox{\linewidth}{!}{
        \tikzstyle{barraVI} = [fill = cor1]
\tikzstyle{barraS} = [fill = cor2]
\tikzstyle{legenda} = [fill = white, line width = 0.25mm]
\tikzstyle{--} = [line width = 0.25mm]

%\resizebox{0.8\linewidth}{!}{
\begin{tikzpicture}[node distance=0cm]
    % Fundo do gráfico

    \renewcommand{\tamX}{12.0cm}
    \renewcommand{\tamY}{6.0cm}
    
    \node (origin) {};
    \node (endX) [xshift = \tamX] {};
    \node (endY) [yshift = \tamY] {};
    \node (endXY) [above of = endX, yshift = \tamY] {};
    
    %Título
    \node (titulo) [xshift = \tamX*0.5, yshift = \tamY + 1cm] {\textbf{\LARGE Orientation from BVI people}};
    \draw[--] (origin.west) node[anchor = east]{\Large 0} to (endX.center) node[anchor = north, xshift = -\tamX*0.5, yshift = -5.0cm]{\textbf{\LARGE Workload dimensions}};
    \draw[--] (origin.south) to (endY.center) 
    node(eixoY)[anchor = east, xshift = -2.5cm, yshift = 0.5cm, rotate = 90]{\textbf{\LARGE Average weighted score}};
    \draw[--] (endX.center) to (endXY.center);
    
    \foreach \r/\n in {1/50,2/100,3/150,4/200,5/250,6/300}
    {
        \draw [--] (-0.15,1cm*\r) node[anchor = east]{\Large \n} to (\tamX,1cm*\r);
    }
    
    \renewcommand{\largX}{0.5}
    \renewcommand{\altY}{0}
    \renewcommand{\distX}{0.5}
    
    %Mental Demand
    %\draw[barraVI] (\distX,0) node[xshift = \largX*1cm, yshift = -0.5cm]{\textbf{\LARGE MD}} rectangle ++(\largX,5.6);
    \draw[barraVI] (\distX,0) node[xshift = \largX*-1.65cm, yshift = -2.25cm]{\rotatebox{50}{\textbf{\LARGE Mental Demand}}} rectangle ++(\largX,5.6);
    \draw[barraS] (\distX+\largX,0) rectangle ++(\largX,3.1);
    \draw[--] (\distX+3*\largX,0) to ++(0,-0.2);
    
    \renewcommand{\distX}{2.5}
    %Physical Demand
    %\draw[barraVI] (\distX,0) node[xshift = \largX*1cm, yshift = -0.5cm]{\textbf{\LARGE PD}} rectangle ++(\largX,0.7);
    \draw[barraVI] (\distX,0) node[xshift = \largX*-1.75cm, yshift = -2.25cm]{\rotatebox{50}{\textbf{\LARGE Physical Demand}}} rectangle ++(\largX,0.7);
    \draw[barraS] (\distX+\largX,0) rectangle ++(\largX,0.5);
    \draw[--] (\distX+3*\largX,0) to ++(0,-0.2);
    
    \renewcommand{\distX}{4.5}
    %Temporal demand
    %\draw[barraVI] (\distX,0) node[xshift = \largX*1cm, yshift = -0.5cm]{\textbf{\LARGE TD}} rectangle ++(\largX,0.9);
    \draw[barraVI] (\distX,0) node[xshift = \largX*-2cm, yshift = -2.5cm]{\rotatebox{50}{\textbf{\LARGE Temporal Demand}}} rectangle ++(\largX,0.9);
    \draw[barraS] (\distX+\largX,0) rectangle ++(\largX,0.7);
    \draw[--] (\distX+3*\largX,0) to ++(0,-0.2);
    
    \renewcommand{\distX}{6.5}
    %Performance
    %\draw[barraVI] (\distX,0) node[xshift = \largX*1cm, yshift = -0.5cm]{\textbf{\LARGE OP}} rectangle ++(\largX,2.1);
    \draw[barraVI] (\distX,0) node[xshift = \largX*-1.25cm, yshift = -1.65cm]{\rotatebox{50}{\textbf{\LARGE Performance}}} rectangle ++(\largX,2.1);
    \draw[barraS] (\distX+\largX,0) rectangle ++(\largX,0.5);
    \draw[--] (\distX+3*\largX,0) to ++(0,-0.2);
    
    \renewcommand{\distX}{8.5}
    %Effort
    %\draw[barraVI] (\distX,0) node[xshift = \largX*1cm, yshift = -0.5cm]{\textbf{\LARGE EF}} rectangle ++(\largX,3.8);
    \draw[barraVI] (\distX,0) node[xshift = \largX*-0.0cm, yshift = -0.850cm]{\rotatebox{50}{\textbf{\LARGE Effort}}} rectangle ++(\largX,3.8);
    \draw[barraS] (\distX+\largX,0) rectangle ++(\largX,2.5);
    \draw[--] (\distX+3*\largX,0) to ++(0,-0.2);
    
    \renewcommand{\distX}{10.5}
    %Frustation
    %\draw[barraVI] (\distX,0) node[xshift = \largX*1cm, yshift = -0.5cm]{\textbf{\LARGE FR}} rectangle ++(\largX,1.5);
    \draw[barraVI] (\distX,0) node[xshift = \largX*-1.0cm, yshift = -1.35cm]{\rotatebox{50}{\textbf{\LARGE Frustation}}} rectangle ++(\largX,1.5);
    \draw[barraS] (\distX+\largX,0) rectangle ++(\largX,1.1);
    \draw[--] (\distX+3*\largX,0) to ++(0,-0.2);
    
    
    %Legenda
    \draw[legenda] (\tamX-3.0cm,\tamY-1.5cm) rectangle ++(6.5cm,2cm);
    \draw[barraVI] (\tamX-2.75cm,\tamY-0.25cm) rectangle ++(0.25cm,0.25cm) 
    node[anchor = west, xshift = 0.15cm, yshift = -0.15cm]{\LARGE BVI};
    \draw[barraS] (\tamX-2.75cm,\tamY-1.0cm) rectangle ++(0.25cm,0.25cm) 
    node[anchor = west, xshift = 0.15cm, yshift = -0.15cm]{\LARGE Sighted};
    
\end{tikzpicture}
%}
        }
        \caption{Condition 2.}
        \label{fig:bradley_2005_nasa_participants_2}
    \end{subfigure}
\caption{Comparison of the NASA-TLX between the participants (Adapted from \citeonline{bradley2005experimental}).}
\label{fig:bradley_2005_participants}
\end{figure}

The work of \citeonline{bradley2002investigating} brings some relevant information for developing this work. Firstly, it shows the differences between the way sighted and BVI people navigate, highlighting the importance of including BVI in the design process of assistive technologies. It confirms the limitations of the current solutions. It brings essential insights on what type of information to include in developing audio systems, which is one of the assistive devices evaluated in this work. Finally, it shows the importance of using different workload assessment techniques when evaluating assistive technologies.

\section{Audio navigation for BVI}
\label{sec:auditory_navigation}
Despite the several existing navigation systems for BVI users, their limitations have been pointed out in many works. \citeonline{yang2014design} explored the effect of two factors when BVIs use a standard GPS navigation system. The first factor is the amount of detail of the provided information (information completeness). The second one is the distance between the information reproduction and the object referred by that same information (broadcasting timing).

In order to evaluate the impact of these two factors, \citeonline{yang2014design} experimented with BVI users where each factor had two levels. The completeness of the information could be “complete” and “simple” and the broadcast timing could be 5m and 7m. As outputs the authors evaluated the participants' performance by their precision and time in finding a goal, and evaluated their perceived workload with NASA-TLX. 

The independent variables were analysed by a two-way ANOVA hypothesis test. They found that the precision in finding the goal was only influenced by the broadcasting timing. The time in finding the goal was influenced by both variables. The task's workload was influenced by the broadcasting timing and the interaction between it and the information completeness.

The work of \citeonline{yang2014design} shows the importance of synchronizing the information provided by the audio system with the current position of the BVI user – a point to be taken into account when developing the audio solution used in this work. However, concerning the lack of influence of information completeness, it is relevant to observe that, in a certain way, this result contradicts the conclusions of Bradley and Dunlop of \citeyear{bradley2002investigating,bradley2005experimental} and, therefore, should be considered with caution. It may be due to the difference between the two levels (complete and simple) adopted in the experiment. Finally, \citeonline{yang2014design} confirms the NASA-TLX as a feasible tool to evaluate workload in experiments with BVI participants.


\section{Comparison of assistive devices}
\label{sec:evaluation_spatial_display}
In the work of \cite{marston2006evaluation}, the author wanted to test a prototype developed in previous research on the street and in a park with a blind user. This experiment would also compare two different guidance displays, one based on haptics transmission and another based on sounds.

8 BVI participants attended the experiment, which was divided into one training set and two test sites. The first was in a busy block that had a variety of street furniture, parked bicycles and people and the participant needed to pass through 4 waypoints for a total of 244m. The second site was inside a park, with paths made of concrete, crushed gravel and paver blocks, with 7 waypoints for a total of 187m. Each participant did each route with both guidance displays.

The researchers collected the time to collect all waypoints, the errors made, the distance traveled and the percentage of the total time that the users accessed the guidance device. All participants were able to complete all routes and collect all waypoints with both devices. This shows that they were able to be guided by new sound or haptic devices. The mean time to collect all the waypoints using the sound device was lower than with the haptic device, as shown in the Figure \ref{fig:evaluation_mean_time}. This Figure shows a standardized time made based on the time that two researchers took to complete the route, both of them blindfolded and with a cane, but already had made the same route many times before and during the experiment.

%\begin{figure}[htbp]
%    \centering
%    \includegraphics[width = \linewidth]{Revisao/Evaluation Spatial Display/Evaluation mean time.png}
%    \caption{Standardize mean completion time for each subject with each device in each route \cite{marston2006evaluation}.}
%    \label{fig:evaluation_mean_time}
%\end{figure}

\begin{figure}[htbp]
    \centering    
    \tikzstyle{barraHPIRua} = [fill = cor1]
    \tikzstyle{barraSomRua} = [fill = cor2]
    \tikzstyle{barraHPIParque} = [fill = cor3]
    \tikzstyle{barraSomParque} = [fill = cor4]
    \tikzstyle{legenda} = [fill = white, line width = 0.25mm]
    \tikzstyle{--} = [line width = 0.25mm]
    
    \resizebox{\linewidth}{!}{
    \begin{tikzpicture}[node distance=0cm]
        \centering    
        % Fundo do gráfico
        \renewcommand{\tamX}{16.0cm}
        \renewcommand{\tamY}{6.0cm}
        
        \node (origin) {};
        \node (endX) [xshift = \tamX] {};
        \node (endY) [yshift = \tamY] {};
        \node (endXY) [above of = endX, yshift = \tamY] {};
        
        %Título
        \node (titulo) [xshift = \tamX*0.5, yshift = \tamY + 1cm] {};
        \draw[--] (origin.west) node[anchor = east]{ 1} to (endX.center) node[anchor = north, xshift = -\tamX*0.5, yshift = -1.0cm]{\textbf{\Large Subjects}};
        \draw[--] (origin.south) to (endY.center) 
        node(eixoY)[anchor = east, xshift = -2.5cm, yshift = 0.5cm, rotate = 90]{\textbf{\Large Relative Access Measure}}
        node[right of = eixoY, anchor = west, xshift = 1.0cm, yshift = -0.75cm, rotate = 90]{\textbf{\Large (RAM)}};
        \draw[--] (endX.center) to (endXY.center);
        
       \foreach \r/\n in {1/1.5, 2/2.0, 3/2.5, 4/3.0, 5/3.5, 6/4.0}
        {
            \draw [--] (-0.15,1cm*\r) node[anchor = east]{\n} to (\tamX,1cm*\r);
        }
        
        \renewcommand{\largX}{0.25}
        \renewcommand{\altY}{0}
        
        \renewcommand{\distX}{0.5}
        %S1
        \draw[barraHPIRua] (\distX,0) rectangle ++(\largX,2.25);
        \draw[barraSomRua] (\distX+\largX,0) node[xshift = \largX*1cm, yshift = -0.5cm]{\textbf{\Large S1}} rectangle ++(\largX,2.1);
        \draw[barraHPIParque] (\distX+2*\largX,0) rectangle ++(\largX,3.9);
        \draw[barraSomParque] (\distX+3*\largX,0) rectangle ++(\largX,0);
        \draw[--] (\distX+6*\largX,0) to ++(0,-0.2);
        
        \renewcommand{\distX}{2.5}
        %S2
        \draw[barraHPIRua] (\distX,0) rectangle ++(\largX,0.6);
        \draw[barraSomRua] (\distX+\largX,0) node[xshift = \largX*1cm, yshift = -0.5cm]{\textbf{\Large S2}} rectangle ++(\largX,0.5);
        \draw[barraHPIParque] (\distX+2*\largX,0) rectangle ++(\largX,0.4);
        \draw[barraSomParque] (\distX+3*\largX,0) rectangle ++(\largX,0.3);
        \draw[--] (\distX+6*\largX,0) to ++(0,-0.2);
        
        \renewcommand{\distX}{4.5}
        %S3
        \draw[barraHPIRua] (\distX,0) rectangle ++(\largX,2.1);
        \draw[barraSomRua] (\distX+\largX,0) node[xshift = \largX*1cm, yshift = -0.5cm]{\textbf{\Large S3}} rectangle ++(\largX,1.8);
        \draw[barraHPIParque] (\distX+2*\largX,0) rectangle ++(\largX,1.9);
        \draw[barraSomParque] (\distX+3*\largX,0) rectangle ++(\largX,1.7);
        \draw[--] (\distX+6*\largX,0) to ++(0,-0.2);
        
        \renewcommand{\distX}{6.5}
        %S4
        \draw[barraHPIRua] (\distX,0) rectangle ++(\largX,1.6);
        \draw[barraSomRua] (\distX+\largX,0) node[xshift = \largX*1cm, yshift = -0.5cm]{\textbf{\Large S4}} rectangle ++(\largX,1.5);
        \draw[barraHPIParque] (\distX+2*\largX,0) rectangle ++(\largX,2.0);
        \draw[barraSomParque] (\distX+3*\largX,0) rectangle ++(\largX,0.7);
        \draw[--] (\distX+6*\largX,0) to ++(0,-0.2);
        
        \renewcommand{\distX}{8.5}
        %S5
        \draw[barraHPIRua] (\distX,0) rectangle ++(\largX,1.1);
        \draw[barraSomRua] (\distX+\largX,0) node[xshift = \largX*1cm, yshift = -0.5cm]{\textbf{\Large S5}} rectangle ++(\largX,1.7);
        \draw[barraHPIParque] (\distX+2*\largX,0) rectangle ++(\largX,1.6);
        \draw[barraSomParque] (\distX+3*\largX,0) rectangle ++(\largX,0);
        \draw[--] (\distX+6*\largX,0) to ++(0,-0.2);
        
        \renewcommand{\distX}{10.5}
        %S6
        \draw[barraHPIRua] (\distX,0) rectangle ++(\largX,3.1);
        \draw[barraSomRua] (\distX+\largX,0) node[xshift = \largX*1cm, yshift = -0.5cm]{\textbf{\Large S6}} rectangle ++(\largX,4.9);
        \draw[barraHPIParque] (\distX+2*\largX,0) rectangle ++(\largX,4.8);
        \draw[barraSomParque] (\distX+3*\largX,0) rectangle ++(\largX,2.4);
        \draw[--] (\distX+6*\largX,0) to ++(0,-0.2);
        
        \renewcommand{\distX}{12.5}
        %S7
        \draw[barraHPIRua] (\distX,0) rectangle ++(\largX,3.5);
        \draw[barraSomRua] (\distX+\largX,0) node[xshift = \largX*1cm, yshift = -0.5cm]{\textbf{\Large S7}} rectangle ++(\largX,3.6);
        \draw[barraHPIParque] (\distX+2*\largX,0) rectangle ++(\largX,3.0);
        \draw[barraSomParque] (\distX+3*\largX,0) rectangle ++(\largX,2.8);
        \draw[--] (\distX+6*\largX,0) to ++(0,-0.2);
        
        \renewcommand{\distX}{14.5}
        %S8
        \draw[barraHPIRua] (\distX,0) rectangle ++(\largX,2.9);
        \draw[barraSomRua] (\distX+\largX,0) node[xshift = \largX*1cm, yshift = -0.5cm]{\textbf{\Large S8}} rectangle ++(\largX,1.8);
        \draw[barraHPIParque] (\distX+2*\largX,0) rectangle ++(\largX,4.5);
        \draw[barraSomParque] (\distX+3*\largX,0) rectangle ++(\largX,3.2);
        \draw[--] (\distX+6*\largX,0) to ++(0,-0.2);
        
        %Legenda
        \draw[legenda] (\tamX-6.0cm,\tamY-0.75cm) rectangle ++(6.5cm,3.25cm);
        \draw[barraHPIRua] (\tamX-5.75cm,\tamY+2.0cm) rectangle ++(0.25cm,0.25cm) 
        node[anchor = west, xshift = 0.15cm, yshift = -0.15cm]{\Large Street HPI};
        \draw[barraSomRua] (\tamX-5.75cm,\tamY+1.25cm) rectangle ++(0.25cm,0.25cm) 
        node[anchor = west, xshift = 0.15cm, yshift = -0.15cm]{\Large Street Virtual Sound};
        \draw[barraHPIParque] (\tamX-5.75cm,\tamY+0.5cm) rectangle ++(0.25cm,0.25cm) 
        node[anchor = west, xshift = 0.15cm, yshift = -0.15cm]{\Large Park HPI};
        \draw[barraSomParque] (\tamX-5.75cm,\tamY-0.25cm) rectangle ++(0.25cm,0.25cm) 
        node[anchor = west, xshift = 0.15cm, yshift = -0.15cm]{\Large Park Virtual Sound};
        
    \end{tikzpicture}
    }
    \caption{Standardize mean completion time for each subject with each device in each route (Adapted from \citeonline{marston2006evaluation}).}
    \label{fig:evaluation_mean_time}
\end{figure}

Another finding from this work is about the use of the haptic device caused some strain on the arm and was less acceptable as compared to the sound device, which required no use of the arms.

This study was relevant for current work because it also compares the same types of guidance devices. The participants were asked to score both devices in three questions from 1 = very unacceptable to 5 = very acceptable. These scores are presented in the Table \ref{tab:evaluation_table}. As said above, the participants were able to perform the full experiment with both devices, but there seems to be a preference for the sound-based device.

\begin{table}[htbp]
    \centering
    \caption{Scores of the device}
    \label{tab:evaluation_table}
    \begin{tabular}{|l|l|l|l|l|}
        \hline
        \multirow{2}{*}{\textbf{Statement}} &
        \multirow{2}{*}{\textbf{\begin{tabular}[c]{@{}l@{}}Haptic device's\\ mean score \end{tabular}}} & \multirow{2}{*}{\textbf{SD}} &
        \multirow{2}{*}{\textbf{\begin{tabular}[c]{@{}l@{}}Sound device's \\ mean score\end{tabular}}} & 
        \multirow{2}{*}{\textbf{SD}} \\
        &&&& \\ \hline
        Precision of the directiona information & 4.0 & 0    & 4.1 & 0.83 \\ \hline
        Personal safety while using the device  & 4.1 & 0.35 & 4.0 & 0.76 \\ \hline
        Ease of use                             & 3.5 & 0.53 & 4.6 & 0.52 \\ \hline
    \end{tabular}
\end{table}

But what about being able to use both devices? That's one of the questions that the experiment of this master's thesis aims to answer.

\section{Virtual reality in the design process}
\label{sec:vr_cabin}
% \lipsum[2-4]

VR is also being studied by the aeronautics and aerospace industries. \citeauthor{moerland2021application} proposed the use of the VR during the aircraft cabin's design procedure. The idea is to create easy communication between the development actors and its clients.

The cabin design procedure is often said to be a complex product because it involves a lot of users and stakeholder and each of them have their own set of preferences and requirements \citeauthor{moerland2021application}. The time needed to attend to all of these demands tends to be long and expensive. To understand better the design process of an aircraft cabin, \citeauthor{moerland2021application} interviewed a cabin designer and this interview concluded that the cabin design needed in general 2 years to be concluded. As an example, the interviewees cited a design, in which multiple mock-ups and more than ten meetings with the stakeholder were needed while designing the cabin. The Figure \ref{fig:simplified_cabin_process} illustrates a simplified cabin design process and shows that the traditional process has a high chance to return to initials phases even in the final phases.

%\begin{figure}[h]
%\centering
%\begin{minipage}{\textwidth}
%    \centering
%    \includegraphics[width = 0.5\textwidth]{Revisao/VR Cabin/Process ilustration.png}
%    \caption{Simplified cabin design process \cite{moerland2021application}.}
%    \label{fig:simplified_cabin_process}
%\end{minipage}
%\hfil
%\begin{minipage}{\textwidth}
%    \centering
%    \includegraphics[width = 0.5\textwidth]{Revisao/VR Cabin/Design User involment.png}
%    \caption{Best moments for user involvement \cite{moerland2021application}.}
%    \label{fig:user_involvement}
%\end{minipage}
%\end{figure}

\begin{figure}[!htbp]
    \centering    
    \tikzstyle{hexag} = [regular polygon, regular polygon sides=6, minimum size = 3cm, inner sep = 0cm, xshift = 1.5cm]
    \tikzstyle{hexagText} = [text = white, xshift = 1.5cm]
    \tikzstyle{legenda} = [fill = white, line width = 0.25mm]
    \tikzstyle{--} = [line width = 0.25mm]
    
    \tikzstyle{arrow} = [rounded corners, line width = 1mm, -to]
    \tikzstyle{arrow_blue} = [cor5, arrow]
    
    \resizebox{\linewidth}{!}{
    \begin{tikzpicture}[node distance=1.75cm]
        \centering    

        \node [hexag, draw = cor1, fill = cor1] (emphatize) {};
        \node [hexagText] (emphatizeText) {};
        
        \node [hexag, draw = cor2, fill = cor2, right of = emphatize] (define) {};
        \node [hexagText, right of = emphatize] (defineText) {DEFINE};
        
        \node(inicioFlecha1) [xshift = 3.75cm, yshift = 2.25cm] {};
        \node(fimFlecha1) [right of = emphatize, xshift = 1cm, yshift = 1.2cm] {};
        \node(inicioFlecha2) [xshift = 2.75cm, yshift = 1.5cm] {};
        \node(fimFlecha2) [right of = emphatize, xshift = 0.5cm, yshift = 0.8cm] {};
        \node(inicioFlecha3) [xshift = 1.7cm, yshift = 0.8cm] {};
        \node(fimFlecha3) [right of = emphatize, xshift = 0.25cm, yshift = 0.4cm] {};
        \node(inicioFlecha4) [xshift = 2cm, yshift = -0.5cm] {};
        \node(fimFlecha4) [right of = emphatize, xshift = 0.15cm, yshift = 0cm] {};
        \node(inicioFlecha5) [xshift = 2.77cm, yshift = -1.6cm] {};
        \node(fimFlecha5) [right of = emphatize, xshift = 0.25cm, yshift = -0.5cm] {};
        \node(inicioFlecha6) [xshift = 3.4cm, yshift = -2.2cm] {};
        \node(fimFlecha6) [right of = emphatize, xshift = 0.8cm, yshift = -1.2cm] {};
        
        \draw[arrow] (inicioFlecha1) .. controls ++(0.3,-0.5) .. (fimFlecha1);
        \draw[arrow] (inicioFlecha2) .. controls ++(0.5,-0.15) .. (fimFlecha2);
        \draw[arrow] (inicioFlecha3) .. controls ++(0.75,0) .. (fimFlecha3);
        \draw[arrow] (inicioFlecha4) .. controls ++(0.5,0.25) .. (fimFlecha4);
        \draw[arrow] (inicioFlecha5) .. controls ++(0.25,0.5) .. (fimFlecha5);
        \draw[arrow] (inicioFlecha6) .. controls ++(0.25,0.5) .. (fimFlecha6);
        
        \node [hexag, draw = cor3, fill = cor3, right of = define] (ideate) {};
        \node [hexagText, right of = define] (ideateText) {IDEATE};
        
        \node(inicioDFlecha1) [above of = define, xshift = 0.25cm, yshift = -0.1cm] {};
        \node(fimDFlecha1) [above of = ideate, xshift = -0.25cm, yshift = -0.1cm] {};
        \node(inicioIFlecha1) [below of = ideate, xshift = -0.25cm, yshift = 0.1cm] {};
        \node(fimIFlecha1) [below of = define, xshift = 0.25cm, yshift = 0.1cm] {};
        
        \draw[arrow, draw = cor3] (inicioDFlecha1.south) .. controls ++(0.25,0.5) and ++(-0.75,0.5).. (fimDFlecha1.south);
        \draw[arrow, draw = cor3] (inicioIFlecha1.north) .. controls ++(-0.25,-0.5) and ++(0.75,-0.5).. (fimIFlecha1.north);
        
        \node [hexag, draw = cor4, fill = cor4, right of = ideate] (prototype) {};
        \node [hexagText, right of = ideate] (prototypeText) {PROTOTYPE};
        
        \node(inicioIFlecha2) [above of = ideate, xshift = 0.25cm, yshift = -0.2cm] {};
        \node(fimIFlecha2) [above of = prototype, xshift = -0.25cm, yshift = -0.2cm] {};
        \node(inicioPFlecha1) [below of = prototype, xshift = -0.25cm, yshift = 0.2cm] {};
        \node(fimPFlecha1) [below of = ideate, xshift = 0.25cm, yshift = 0.2cm] {};
        \node(inicioPFlecha2) [above of = prototype, xshift = 0.25cm, yshift = -0.2cm] {};
        \node(fimPFlecha2) [above of = define, xshift = -0.25cm, yshift = -0.2cm] {};
        
        \draw[arrow, draw = cor4] (inicioIFlecha2.center) .. controls ++(0.25,0.5) and ++(-0.75,0.5).. (fimIFlecha2.east);
        \draw[arrow, draw = cor4] (inicioPFlecha1.east) .. controls ++(-0.25,-0.5) and ++(0.75,-0.5).. (fimPFlecha1.center);
        \draw[arrow, draw = cor1] (inicioPFlecha2.west) .. controls ++(-1,1.75) and ++(1.2,1.5).. (fimPFlecha2);
        
        
        \node [hexag, draw = cor5, fill = cor5, right of = prototype] (assess) {};
        
        \node(inicioPFlecha3) [above of = prototype, xshift = 0.25cm, yshift = -0.2cm] {};
        \node(fimPFlecha3) [above of = assess, xshift = -0.25cm, yshift = -0.2cm] {};
        \node(inicioAFlecha1) [below of = assess, xshift = -0.25cm, yshift = 0.2cm] {};
        \node(fimAFlecha1) [below of = prototype, xshift = 0.25cm, yshift = 0.2cm] {};
        \node(inicioAFlecha2) [below of = assess, xshift = 0cm, yshift = 0.2cm] {};
        \node(fimAFlecha2) [below of = ideate, xshift = -0.25cm, yshift = 0.1cm] {};
        \node(inicioAFlecha3) [below of = assess, xshift = 0.25cm, yshift = 0.2cm] {};
        \node(fimAFlecha3) [below of = define, xshift = -0.25cm, yshift = 0.1cm] {};
        
        \draw[arrow, draw = cor5] (inicioPFlecha3.center) .. controls ++(0.5,1) and ++(-0.75,1).. (fimPFlecha3.center);
        \draw[arrow, draw = cor5] (inicioAFlecha1.center) .. controls ++(-0.25,-0.5) and ++(0.75,-1).. (fimAFlecha1.west);
        \draw[arrow, draw = cor2] (inicioAFlecha2.center) .. controls ++(-0.25,-1.25) and ++(0.75,-1).. (fimAFlecha2.south east);
        \draw[arrow, draw = cor1] (inicioAFlecha3.center) .. controls ++(-0.25,-1.75) and ++(1.2,-1.5).. (fimAFlecha3.north);
        
        \node [hexagText, right of = prototype] (assessText) {ASSESS};
        

        
        
    \end{tikzpicture}
    }
    \caption{Simplified cabin design process (Adapted from \citeonline{moerland2021application}).}
    \label{fig:simplified_cabin_process}
\end{figure}
\begin{figure}[!htbp]
    \centering    
    \tikzstyle{hexag} = [regular polygon, regular polygon sides=6, minimum size = 3cm, inner sep = 0cm, xshift = 1.5cm]
    \tikzstyle{hexagText} = [text = white, xshift = 1.5cm]
    \tikzstyle{legenda} = [fill = white, line width = 0.25mm]
    \tikzstyle{--} = [line width = 0.25mm]
    
    \resizebox{\linewidth}{!}{
    \begin{tikzpicture}[node distance=1.75cm]
        \centering    

        \node [hexag, draw = cor1, fill = cor1] {};
        \node [hexagText] (emphatize) {EMPHATIZE};
        %\node [regular polygon, regular polygon sides=6, text = white, draw = cor1, minimum size = 5cm, fill = cor1] (emphatize) {EMPHATIZE};
        \node [hexag, draw = cor2, fill = cor2, right of = emphatize] (define) {};
        \node [hexagText, right of = emphatize] (defineText) {DEFINE};
        %\node [regular polygon, regular polygon sides=6, text = white, draw = cor2, minimum size = 5cm, fill = cor2, right of = emphatize] (define) {DEFINE};
        \node [hexag, draw = cor3, fill = cor3, right of = define] (ideate) {};
        \node [hexagText, right of = define] (ideateText) {IDEATE};
        %\node [regular polygon, regular polygon sides=6, text = white, draw = cor3, minimum size = 5cm, fill = cor3, right of = define] (ideate) {IDEATE};
        \node [hexag, draw = cor4, fill = cor4, right of = ideate] (prototype) {};
        \node [hexagText, right of = ideate] (prototypeText) {PROTOTYPE};
        %\node [regular polygon, regular polygon sides=6, text = white, draw = cor4, minimum size = 5cm, fill = cor4, right of = ideate] (prototype) {PROTOTYPE};
        \node [hexag, draw = cor5, fill = cor5, right of = prototype] (assess) {};
        \node [hexagText, right of = prototype] (assessText) {ASSESS};
        %\node [regular polygon, regular polygon sides=6, text = white, draw = cor5, minimum size = 5cm, fill = cor5, right of = prototype] (assess) {ASSESS};

        \node(insertNorth1) [right of = emphatize, xshift = -0.125cm, yshift = 0.25cm] {};
        \node(insertNorthWest1) [above of = insertNorth1, left of = insertNorth1, xshift = 0.75cm] {};
        \node(insertNorthEast1) [above of = insertNorth1, right of = insertNorth1, xshift = -0.75cm] {};
        \draw[--] (insertNorthWest1.center) to (insertNorth1.center) to (insertNorthEast1.center);
        
        \node(insertSouth1) [right of = emphatize, xshift = -0.125cm, yshift = -0.25cm] {};
        \node(insertSouthWest1) [below of = insertSouth1, left of = insertSouth1, xshift = 0.75cm] {};
        \node(insertSouthEast1) [below of = insertSouth1, right of = insertSouth1, xshift = -0.75cm] {};
        \draw[--] (insertSouthWest1.center) to (insertSouth1.center) to (insertSouthEast1.center);
        
        \node(insertNorth2) [right of = define, xshift = -0.125cm, yshift = 0.25cm] {};
        \node(insertNorthWest2) [above of = insertNorth2, left of = insertNorth2, xshift = 0.75cm] {};
        \node(insertNorthEast2) [above of = insertNorth2, right of = insertNorth2, xshift = -0.75cm] {};
        \draw[--] (insertNorthWest2.center) to (insertNorth2.center) to (insertNorthEast2.center);
        
        \node(insertSouth2) [right of = define, xshift = -0.125cm, yshift = -0.25cm] {};
        \node(insertSouthWest2) [below of = insertSouth2, left of = insertSouth2, xshift = 0.75cm] {};
        \node(insertSouthEast2) [below of = insertSouth2, right of = insertSouth2, xshift = -0.75cm] {};
        \draw[--] (insertSouthWest2.center) to (insertSouth2.center) to (insertSouthEast2.center);


    \end{tikzpicture}
    }
    \caption{Best moments for user involvement (Adapted from \citeonline{moerland2021application}).}
    \label{fig:user_involvement}
\end{figure}


\citeauthor{moerland2021application} are inside the German Aerospace Center (DLR, \textit{Deutsch Zentrum für Luft- Raumfahrt}) and decided to study a new procedure that could bring the involvement of the final users in the design process. This procedure is based on co-design, where the users can influence the product's development from the beginning until the end. The Figure \ref{fig:user_involvement} shows the best moments to bring the users to the process. But for the involvement to happen, a communication channel needed to be established. The authors choose to use \textit{Reality Works} and test on a DLR's inside project. This project's goal was to design a new cabin that would be incorporated into a large workflow, but the design process was to be completely made in a digital environment. This was the perfect test case for the VR use in the cabin's design procedure.

A pilot use case was made with the members of this project. Three different designers (two with around 5 years of experience and another with more than 35 years of experience) initiated a cabin design. 
The Figures \ref{fig:cabin_sketch} and \ref{fig:cabin_3d_model} show the results using the traditional method. The sketch can only present a glance of what the cabin will be. The 3D model has more details, but any change to this representation needs a new rendering session and this can take hours, or even days, to be made.

\begin{figure}[h]
\centering
\begin{minipage}{.45\textwidth}
    \centering
    \includegraphics[width = \linewidth]{Revisao/VR Cabin/Sketch.png}
    \vspace{0.4cm}
    \caption{Cabin sketch made with Adobe Photoshop \cite{moerland2021application}.}
    \label{fig:cabin_sketch}
\end{minipage}
\hfil
\begin{minipage}{.45\textwidth}
    \centering
    \includegraphics[width = \linewidth]{Revisao/VR Cabin/3D Model.png}
    \caption{Cabin 3D model made with Rhyno \cite{moerland2021application}.}
    \label{fig:cabin_3d_model}
\end{minipage}
\end{figure}

The Figures \ref{fig:vr_sketch} and \ref{fig:vr_3d_model} show the same representation but made in a VR environment. The sketch was made inside the aircraft cabin and this could have been done with a client or a stakeholder and they could also draw and give their opinions from the beginning. The 3D models can be imported to increase the sketch's level of detail.

\begin{figure}[h]
\centering
\begin{minipage}{.45\textwidth}
    \centering
    \includegraphics[width = \linewidth]{Revisao/VR Cabin/VR sketch.png}
    \caption{VR navigation with sketching \cite{moerland2021application}.}
    \label{fig:vr_sketch}
\end{minipage}
\hfil
\begin{minipage}{.45\textwidth}
    \centering
    \includegraphics[width = \linewidth]{Revisao/VR Cabin/VR 3D Model.png}
    \caption{VR navigation with imported 3D models \cite{moerland2021application}.}
    \label{fig:vr_3d_model}
\end{minipage}
\end{figure}

This case was well received by the design team and they have chosen to continue to use the VR tool. The benefits disadvantages pointed by \citeauthor{moerland2021application} are listed in the Table \ref{tab:benefits_disvantages_vr_cabin}. The VR helps to bring the clients closer to the design team, allows them to draw quick sketches in brainstorming gatherings and has a steep learning curve for the designers. On the other hand, is its a high-cost tool, the use for a long time can cause nausea and maybe other health implications, even though the learning curve is steep, there is still a learning curve and the user needs to get used to the exposure to others that can see the user from outside the virtual environment (some find this situation uncomfortable).

\begin{table}[h]
    \centering
    \caption{The benefits and disadvantages noted by the authors \citeauthor{moerland2021application}.}
    \label{tab:benefits_disvantages_vr_cabin}
    \begin{tabular}{|l|l|}
        \hline
        \textbf{Benefits}                         & \textbf{Disadvantages}                                  \\ \hline
        Bottleneck at early concept design stages & High cost                                               \\ \hline
        Quick sketchs during brainstorming        & Nausea and other health implacations                    \\ \hline
        Steep learning curve                      & \begin{tabular}[c]{@{}l@{}}There is a learning and personal\\ adaptation to exposure\end{tabular} \\ \hline
    \end{tabular}
\end{table}

The current master's thesis isn't about designing or aircraft cabins, but this research shows that VR is being studied to be implemented inside industries. The current research could be done by any product industry that wanted to create a test environment for their clients to increase the user's approval or to bring other teams close to reducing the full design time.

\section{Final Remarks}
\label{sec:final_remarks3}

In this chapter, 7 papers were reviewed with the current thesis in mind. These 7 papers are related to one of the three keywords, or a combination of them: "human factors", "virtual reality" and "blindness".

These papers pointed that BVI users are sensible to sensorial inputs. \ref{sec:vr_without_vision} showed that they were sensible with the vibration of the cane. This sensibility was based on their past experience with canes made from different materials and with vibration devices, but this paper was more about designing a navigation tool, not a framework for testing human factors.

When designing an virtual environment, it is important to consider the feeling of presence, and in the Section \ref{sec:emotion_presence_vr} presented a paper that concluded that to increase this feeling one should allow the user agency inside the environment. That means that the user will feel more present (will forget that he/she is inside the virtual environment) when he/she is induced to interact with the elements from the virtual environment. The current experiment used real furniture and real actors mixed with virtual sounds to increase the sensation that the user was not inside a laboratory, but inside a medical clinic reception.

The Section \ref{sec:bradley_dunlop} was about to papers from the same authors that studied the differences between the information used by BVI and sighted users during the navigation. They found out that the last group used more text-based information, while the first group used information with more words and from a wider variety of categories. Another conclusion an information that was made for BVI users was more mentally demandful for sighted users, and vice-verse. This conclusion was importante for two of the used methods in the thesis, which used audio information for the navigation.

But when is the perfect moment to announce a audio information for BVI users? The Section \ref{sec:auditory_navigation} showed the the moment that an information is transmitted has affect on the accuracy they reach a goal. This conclusion was made by analysing their navigation task with a NASA-TLX and the results confirmed that this tool can be used for their tasks.

One of the evaluations that this work does is comparing different BVI devices. This evalutation was presented in \ref{sec:evaluation_spatial_display} and the author made a similar comparison. They compared an haptic and sound information in different environments. Their conclusion was that the haptic were less acceptable. This work employs two models of haptic devices and a audio method. Is likely that the conclusion will be comparable.

Finally, the last section of this chapter, Section \ref{sec:vr_cabin}, showed the benefits and disavantages of implementing virtual reality with co-design on a design team. Similarly the ideia of this work is to used virtual reality to allow the users to test new concepts of products and give the designer their feedbacks and impressions.

All these papers contributed to the designing, improvement, and decision making of the current experiment. The next chapter will describe all these steps with more details.

