This chapter discusses a set of selected works from the literature related to different aspects of this work. Their selection was performed in the Scopus and Web of Science databases, using keywords such as “human factors”, “virtual reality” and “blindness”. From an initial set of 344 papers, a set of seven were selected as more relevant to this work and are detailed in the following sections.

\section{Virtual reality in the design process}
\label{sec:vr_cabin}
% \lipsum[2-4]

The use of virtual reality for design purposes is not new. It has been studied and evaluated in several areas, including Aeronautics. \citeonline{moerland2021application} investigated using virtual reality during the aircraft cabin’s design process with the purpose of facilitating the communication between the design team and the client. 

The cabin design process (Figure \ref{fig:simplified_cabin_process}) is often said to be complex because it involves several stakeholders, each with his/her own set of preferences and requirements. According to the authors, the time needed to satisfy the multiple demands tends to be long, and the process usually requires building multiple mock-ups and attending many meetings with the stakeholders. 

\begin{figure}[!htbp]
    \centering    
    \tikzstyle{hexag} = [regular polygon, regular polygon sides=6, minimum size = 3cm, inner sep = 0cm, xshift = 1.5cm]
    \tikzstyle{hexagText} = [text = white, xshift = 1.5cm]
    \tikzstyle{legenda} = [fill = white, line width = 0.25mm]
    \tikzstyle{--} = [line width = 0.25mm]
    
    \tikzstyle{arrow} = [rounded corners, line width = 1mm, -to]
    \tikzstyle{arrow_blue} = [cor5, arrow]
    
    \resizebox{\linewidth}{!}{
    \begin{tikzpicture}[node distance=1.75cm]
        \centering    

        \node [hexag, draw = cor1, fill = cor1] (emphatize) {};
        \node [hexagText] (emphatizeText) {};
        
        \node [hexag, draw = cor2, fill = cor2, right of = emphatize] (define) {};
        \node [hexagText, right of = emphatize] (defineText) {DEFINE};
        
        \node(inicioFlecha1) [xshift = 3.75cm, yshift = 2.25cm] {};
        \node(fimFlecha1) [right of = emphatize, xshift = 1cm, yshift = 1.2cm] {};
        \node(inicioFlecha2) [xshift = 2.75cm, yshift = 1.5cm] {};
        \node(fimFlecha2) [right of = emphatize, xshift = 0.5cm, yshift = 0.8cm] {};
        \node(inicioFlecha3) [xshift = 1.7cm, yshift = 0.8cm] {};
        \node(fimFlecha3) [right of = emphatize, xshift = 0.25cm, yshift = 0.4cm] {};
        \node(inicioFlecha4) [xshift = 2cm, yshift = -0.5cm] {};
        \node(fimFlecha4) [right of = emphatize, xshift = 0.15cm, yshift = 0cm] {};
        \node(inicioFlecha5) [xshift = 2.77cm, yshift = -1.6cm] {};
        \node(fimFlecha5) [right of = emphatize, xshift = 0.25cm, yshift = -0.5cm] {};
        \node(inicioFlecha6) [xshift = 3.4cm, yshift = -2.2cm] {};
        \node(fimFlecha6) [right of = emphatize, xshift = 0.8cm, yshift = -1.2cm] {};
        
        \draw[arrow] (inicioFlecha1) .. controls ++(0.3,-0.5) .. (fimFlecha1);
        \draw[arrow] (inicioFlecha2) .. controls ++(0.5,-0.15) .. (fimFlecha2);
        \draw[arrow] (inicioFlecha3) .. controls ++(0.75,0) .. (fimFlecha3);
        \draw[arrow] (inicioFlecha4) .. controls ++(0.5,0.25) .. (fimFlecha4);
        \draw[arrow] (inicioFlecha5) .. controls ++(0.25,0.5) .. (fimFlecha5);
        \draw[arrow] (inicioFlecha6) .. controls ++(0.25,0.5) .. (fimFlecha6);
        
        \node [hexag, draw = cor3, fill = cor3, right of = define] (ideate) {};
        \node [hexagText, right of = define] (ideateText) {IDEATE};
        
        \node(inicioDFlecha1) [above of = define, xshift = 0.25cm, yshift = -0.1cm] {};
        \node(fimDFlecha1) [above of = ideate, xshift = -0.25cm, yshift = -0.1cm] {};
        \node(inicioIFlecha1) [below of = ideate, xshift = -0.25cm, yshift = 0.1cm] {};
        \node(fimIFlecha1) [below of = define, xshift = 0.25cm, yshift = 0.1cm] {};
        
        \draw[arrow, draw = cor3] (inicioDFlecha1.south) .. controls ++(0.25,0.5) and ++(-0.75,0.5).. (fimDFlecha1.south);
        \draw[arrow, draw = cor3] (inicioIFlecha1.north) .. controls ++(-0.25,-0.5) and ++(0.75,-0.5).. (fimIFlecha1.north);
        
        \node [hexag, draw = cor4, fill = cor4, right of = ideate] (prototype) {};
        \node [hexagText, right of = ideate] (prototypeText) {PROTOTYPE};
        
        \node(inicioIFlecha2) [above of = ideate, xshift = 0.25cm, yshift = -0.2cm] {};
        \node(fimIFlecha2) [above of = prototype, xshift = -0.25cm, yshift = -0.2cm] {};
        \node(inicioPFlecha1) [below of = prototype, xshift = -0.25cm, yshift = 0.2cm] {};
        \node(fimPFlecha1) [below of = ideate, xshift = 0.25cm, yshift = 0.2cm] {};
        \node(inicioPFlecha2) [above of = prototype, xshift = 0.25cm, yshift = -0.2cm] {};
        \node(fimPFlecha2) [above of = define, xshift = -0.25cm, yshift = -0.2cm] {};
        
        \draw[arrow, draw = cor4] (inicioIFlecha2.center) .. controls ++(0.25,0.5) and ++(-0.75,0.5).. (fimIFlecha2.east);
        \draw[arrow, draw = cor4] (inicioPFlecha1.east) .. controls ++(-0.25,-0.5) and ++(0.75,-0.5).. (fimPFlecha1.center);
        \draw[arrow, draw = cor1] (inicioPFlecha2.west) .. controls ++(-1,1.75) and ++(1.2,1.5).. (fimPFlecha2);
        
        
        \node [hexag, draw = cor5, fill = cor5, right of = prototype] (assess) {};
        
        \node(inicioPFlecha3) [above of = prototype, xshift = 0.25cm, yshift = -0.2cm] {};
        \node(fimPFlecha3) [above of = assess, xshift = -0.25cm, yshift = -0.2cm] {};
        \node(inicioAFlecha1) [below of = assess, xshift = -0.25cm, yshift = 0.2cm] {};
        \node(fimAFlecha1) [below of = prototype, xshift = 0.25cm, yshift = 0.2cm] {};
        \node(inicioAFlecha2) [below of = assess, xshift = 0cm, yshift = 0.2cm] {};
        \node(fimAFlecha2) [below of = ideate, xshift = -0.25cm, yshift = 0.1cm] {};
        \node(inicioAFlecha3) [below of = assess, xshift = 0.25cm, yshift = 0.2cm] {};
        \node(fimAFlecha3) [below of = define, xshift = -0.25cm, yshift = 0.1cm] {};
        
        \draw[arrow, draw = cor5] (inicioPFlecha3.center) .. controls ++(0.5,1) and ++(-0.75,1).. (fimPFlecha3.center);
        \draw[arrow, draw = cor5] (inicioAFlecha1.center) .. controls ++(-0.25,-0.5) and ++(0.75,-1).. (fimAFlecha1.west);
        \draw[arrow, draw = cor2] (inicioAFlecha2.center) .. controls ++(-0.25,-1.25) and ++(0.75,-1).. (fimAFlecha2.south east);
        \draw[arrow, draw = cor1] (inicioAFlecha3.center) .. controls ++(-0.25,-1.75) and ++(1.2,-1.5).. (fimAFlecha3.north);
        
        \node [hexagText, right of = prototype] (assessText) {ASSESS};
        

        
        
    \end{tikzpicture}
    }
    \caption{Simplified cabin design process (Adapted from \citeonline{moerland2021application}).}
    \label{fig:simplified_cabin_process}
\end{figure}

\citeonline{moerland2021application} proposed to anticipate the involvement of the final users based on co-design. In their proposal, the users can influence the product's development from the beginning, as shown in Figure \ref{fig:user_involvement}. However, for the involvement to happen, a communication channel needed to be established, and it was done using virtual reality.

\begin{figure}[!htbp]
    \centering    
    \tikzstyle{hexag} = [regular polygon, regular polygon sides=6, minimum size = 3cm, inner sep = 0cm, xshift = 1.5cm]
    \tikzstyle{hexagText} = [text = white, xshift = 1.5cm]
    \tikzstyle{legenda} = [fill = white, line width = 0.25mm]
    \tikzstyle{--} = [line width = 0.25mm]
    
    \resizebox{\linewidth}{!}{
    \begin{tikzpicture}[node distance=1.75cm]
        \centering    

        \node [hexag, draw = cor1, fill = cor1] {};
        \node [hexagText] (emphatize) {EMPHATIZE};
        %\node [regular polygon, regular polygon sides=6, text = white, draw = cor1, minimum size = 5cm, fill = cor1] (emphatize) {EMPHATIZE};
        \node [hexag, draw = cor2, fill = cor2, right of = emphatize] (define) {};
        \node [hexagText, right of = emphatize] (defineText) {DEFINE};
        %\node [regular polygon, regular polygon sides=6, text = white, draw = cor2, minimum size = 5cm, fill = cor2, right of = emphatize] (define) {DEFINE};
        \node [hexag, draw = cor3, fill = cor3, right of = define] (ideate) {};
        \node [hexagText, right of = define] (ideateText) {IDEATE};
        %\node [regular polygon, regular polygon sides=6, text = white, draw = cor3, minimum size = 5cm, fill = cor3, right of = define] (ideate) {IDEATE};
        \node [hexag, draw = cor4, fill = cor4, right of = ideate] (prototype) {};
        \node [hexagText, right of = ideate] (prototypeText) {PROTOTYPE};
        %\node [regular polygon, regular polygon sides=6, text = white, draw = cor4, minimum size = 5cm, fill = cor4, right of = ideate] (prototype) {PROTOTYPE};
        \node [hexag, draw = cor5, fill = cor5, right of = prototype] (assess) {};
        \node [hexagText, right of = prototype] (assessText) {ASSESS};
        %\node [regular polygon, regular polygon sides=6, text = white, draw = cor5, minimum size = 5cm, fill = cor5, right of = prototype] (assess) {ASSESS};

        \node(insertNorth1) [right of = emphatize, xshift = -0.125cm, yshift = 0.25cm] {};
        \node(insertNorthWest1) [above of = insertNorth1, left of = insertNorth1, xshift = 0.75cm] {};
        \node(insertNorthEast1) [above of = insertNorth1, right of = insertNorth1, xshift = -0.75cm] {};
        \draw[--] (insertNorthWest1.center) to (insertNorth1.center) to (insertNorthEast1.center);
        
        \node(insertSouth1) [right of = emphatize, xshift = -0.125cm, yshift = -0.25cm] {};
        \node(insertSouthWest1) [below of = insertSouth1, left of = insertSouth1, xshift = 0.75cm] {};
        \node(insertSouthEast1) [below of = insertSouth1, right of = insertSouth1, xshift = -0.75cm] {};
        \draw[--] (insertSouthWest1.center) to (insertSouth1.center) to (insertSouthEast1.center);
        
        \node(insertNorth2) [right of = define, xshift = -0.125cm, yshift = 0.25cm] {};
        \node(insertNorthWest2) [above of = insertNorth2, left of = insertNorth2, xshift = 0.75cm] {};
        \node(insertNorthEast2) [above of = insertNorth2, right of = insertNorth2, xshift = -0.75cm] {};
        \draw[--] (insertNorthWest2.center) to (insertNorth2.center) to (insertNorthEast2.center);
        
        \node(insertSouth2) [right of = define, xshift = -0.125cm, yshift = -0.25cm] {};
        \node(insertSouthWest2) [below of = insertSouth2, left of = insertSouth2, xshift = 0.75cm] {};
        \node(insertSouthEast2) [below of = insertSouth2, right of = insertSouth2, xshift = -0.75cm] {};
        \draw[--] (insertSouthWest2.center) to (insertSouth2.center) to (insertSouthEast2.center);


    \end{tikzpicture}
    }
    \caption{Best moments for user involvement \cite{moerland2021application}.}
    \label{fig:user_involvement}
\end{figure}

The authors described the application of the proposed approach to a use case. Three different designers initiated a cabin design. In the traditional method, the results were illustrated in a sketch, which could only present a glance of what the cabin would be, and in a 3D model, which had more details. However, any modification required a new rendering session and this could take hours, or even days. The same solution was also illustrated in a virtual reality environment. The sketch was inside the aircraft cabin, where the client or the stakeholder could draw and give their opinions from the beginning of the design process. The 3D models could be imported to increase the sketch's level of detail.

The use case showed some benefits and disadvantages of using virtual reality. The virtual reality helped to bring the client closer to the design team, allowing them to draw quick sketches in brainstorming gatherings. It was associated with a steep learning curve for the designers. Among the disadvantages, it was considered a high-cost tool, and its use for a long time was associated with nausea. 

The work of \citeonline{moerland2021application} is an example of how virtual reality can be used to bring the user into the design process. Similarly, in this work, virtual reality is explored to create a test environment where BVI users can try out the device under development, contributing to improving its usability. Differently from the work of \citeonline{moerland2021application}, in this work, users are not expected to feel sick, as it is usually associated with discrepancies between the motion of the image in the virtual environment and the motion perceived by the vestibular system of the user.

\section{Virtual reality for BVI users}
\label{sec:vr_without_vision}
Motivated by the popularization of virtual reality technology, \citeonline{siu2020virtual} developed a white cane to be used by BVI users in a virtual environment. Their purpose was to make virtual reality applications available for BVI users. 

The traditional white cane transmits three sources of information to the user: detection of obstacles, surface topography and foot placement preview. In their work, these sources of information were transmitted through sounds or haptics \cite{siu2020virtual}, which would be defined based on the cane position in the virtual environment. For obstacle detection, the cane was built with a three-degree-of-freedom brake mechanism that would stop the movement when the cane hit an obstacle. A coil actuator was used to simulate surface properties. Lastly, a wave-based acoustic simulation was used to render geometry-aware sound effects in other to give the user a sense of the surroundings (echo localization).

In order to evaluate their proposal, the authors performed an experiment where the participants had to play a “scavenger hunt” using an HTC Vive system. During the experiment, each participant had two tasks: collect targets along the way (primary task) and avoid virtual obstacles and walls (secondary task). The targets appeared, one at a time, once the previous target was collected, and they emitted a sound that acted like an audio beacon for the participant. The obstacles did not emit any sound as a beacon, but the participant could detect it by the shape and the noise it emitted when in contact with the cane. The experiment was performed with 8 blind users (4 female, 4 male) from 25 to 70 years old. All of them did a training section where the virtual environment was presented. 

Among the relevant findings of \citeonline{siu2020virtual} is that not all the participants reacted the same to a particular stimulus. The vibration of the cane was considered confusing by some participants, while others were familiar with it. This difference affected the performance of the participants. The ones that had already used vibrating devices performed better. It shows that user's previous experiences can impact their performance in the virtual environment.

Another interesting observation was that, similar to what happens in the real world, it was easier for the participants to navigate in larger areas than in tight spaces. Moreover, the authors observed that the participants focused their attention on the primary task, without freely exploring the environment, which might have impacted the low time to achieve the goal and the low number of obstacle hits. 

Among the limitations pointed out by the authors is the lack of feedback possibilities for situations such as when the obstacle contacts a point along with the cane, not the tip of it, and the fact that the brake system did not stop the participant when he/she walked forwards toward a wall.

Comparing the work of \citeonline{siu2020virtual} to this work, \citeonline{siu2020virtual} were focused on providing mechanisms for a BVI user to navigate inside virtual environments. In this work, the purpose is to use the virtual environment to collect data about how the BVI user would navigate in a real environment. Another difference is in the functioning of the virtual cane, which in this work is limited to vibration, with no brake system, as the BVI user does not need to touch the environment with it. One common observation of both works is the sound importance for the BVI guidance and the need to use high-quality spatialized audio to increase the realism of the virtual environment.

\section{Augmented reality for BVI users}
\label{sec:ar_without_vision}
\citeonline{kirner2011using} raised two questions, "How can blind people learn 3D concepts aiming to be able to convert explored 3D environments into pictures?" and "How can we develop a spatial audio tutor with augmented reality technology to make easy the understanding of 3D concepts by blind people?" and used not using virtual reality technology but augmented reality to answer them. They developed a augmented reality application to be a tutor for BVI users. The application used allowed BVI users to play audio streams that were associeated with spatial positions.

The application had 4 different environments and they for played in sequence to teach BVI users the idea of perspective in images and each environment had .

To test the application, ten congenitally people have used it, making comments about the learning of perspective concepts.



\section{Information for BVI navigation}
\label{sec:bradley_dunlop}
Bradley and Dunlop published two works (\citeyear{bradley2002investigating,bradley2005experimental}) about how BVI navigates and how much it is similar or different to how a sighted person navigates. 

The first work of Bradley and Dunlop was published in \citeyear{bradley2002investigating} and discussed which type of information BVI uses to navigate in an environment and how it compares to sighted people. The data were collected during structured interviews where the participant had to explain how to arrive at two different locations as if they were talking to someone with the same vision condition \cite{bradley2002investigating}.

Based on the answers, the authors defined 11 categories of information: 1) directional (e.g. left/right, north/south); 2) structural (e.g. road, monument, church); 3) environmental (e.g. hill, river, tree); 4) textual-structural (e.g. name of shops, places, restaurants); 5) textual-area/street-based (e.g. name of street, neighbourhoods, squares); 6) numerical (e.g. first, second, 100m); 7) descriptive (e.g. steep, tall); 8) temporal/distance based (e.g. ”walk until you reach...” or ”before you get to”); 9) sensory (e.g. the sound of engines, the smell of bread from a bakery); 10) motion (e.g. cars passing by, doors opening); 11) social contact (e.g. asking people or using a guide dog for help) \cite{bradley2002investigating}.

As an output from the interviews, the authors provided the average number which each category was used by each group and is reproduced in Figure \ref{fig:bradley_2002}. From the results, the researchers observed that BVI participants used less text-based information than the sighted participants. However BVI participants used more words to describe a path than the sighted participants. Another essential result was that visually impaired people used, on average, 9 to 10 categories to describe a route, while sighted people used around 6 categories.

\begin{figure}[!htb]
    \centering
    \tikzstyle{barraVI} = [fill = cor1]
    \tikzstyle{barraS} = [fill = cor2]
    \tikzstyle{legenda} = [fill = white, line width = 0.25mm]
    \tikzstyle{--} = [line width = 0.25mm]
    
    \resizebox{0.8\linewidth}{!}{
    \begin{tikzpicture}[node distance=0cm]
        % Fundo do gráfico
    
        \renewcommand{\tamX}{13.875cm}
        \renewcommand{\tamY}{5.4cm}
        
        \node (origin) {};
        \node (endX) [xshift = \tamX] {};
        \node (endY) [yshift = \tamY] {};
        \node (endXY) [above of = endX, yshift = \tamY] {};
        
        %Título
        \node (titulo) [xshift = \tamX*0.5, yshift = \tamY + 1.25cm] {\textbf{Average nº of utterances used within each contextual category}}
        node [below of = titulo, yshift = -0.5cm] {\textbf{between sighted and visually impaired participants}};
        
        \draw[--] (origin.west) node[anchor = east]{\footnotesize 0} to (endX.center) node[anchor = north, xshift = -7.5cm, yshift = -2.0cm]{\textbf{Type of contextual categories}};
        \draw[--] (origin.south) to (endY.center) node[anchor = east, xshift = -1.0cm,rotate = 90]{\textbf{Average nº of utterance}};
        \draw[--] (endX.center) to (endXY.center);
        
        \foreach \r/\n in        {1/5,2/10,3/15,4/20,5/25,6/30,7/35,8/40,9/45,10/50,11/55,12/60,13/65,14/70,15/75,16/80,17/85,18/90}
        {
            \draw [--] (-0.15,0.30cm*\r) node[anchor = east]{\footnotesize \n} to (\tamX,0.30cm*\r);
        }
        
        \renewcommand{\largX}{0.25}
        \renewcommand{\altY}{0}
        \renewcommand{\distX}{0.5}
        %Direct
        \draw[barraVI] (\distX,0) node[yshift = -1.0cm, rotate = 45]{Direct} rectangle ++(\largX,5.3);
        \draw[barraS] (\distX+\largX,0) rectangle ++(\largX,1.6);
        \draw[--] (\distX+3.5*\largX,0) to ++(0,-0.2);
        
        \renewcommand{\distX}{1.75}
        %Struct
        \draw[barraVI] (\distX,0) node[yshift = -1.0cm, rotate = 45]{Struct} rectangle ++(\largX,3.6);
        \draw[barraS] (\distX+\largX,0) rectangle ++(\largX,0.5);
        \draw[--] (\distX+3.5*\largX,0) to ++(0,-0.2);
        
        \renewcommand{\distX}{3.0}
        %Struct
        \draw[barraVI] (\distX,0) node[yshift = -1.0cm, rotate = 45]{Environ} rectangle ++(\largX,0.5);
        \draw[barraS] (\distX+\largX,0) rectangle ++(\largX,0.2);
        \draw[--] (\distX+3.5*\largX,0) to ++(0,-0.2);
        
        \renewcommand{\distX}{4.25}
        %Struct
        \draw[barraVI] (\distX,0) node[yshift = -1.0cm, rotate = 45]{Text-struct} rectangle ++(\largX,0.2);
        \draw[barraS] (\distX+\largX,0) rectangle ++(\largX,0.4);
        \draw[--] (\distX+3.5*\largX,0) to ++(0,-0.2);
        
        \renewcommand{\distX}{5.5}
        %Struct
        \draw[barraVI] (\distX,0) node[yshift = -1.0cm, rotate = 45]{Text-area/st} rectangle ++(\largX,0.5);
        \draw[barraS] (\distX+\largX,0) rectangle ++(\largX,0.7);
        \draw[--] (\distX+3.5*\largX,0) to ++(0,-0.2);
        
        \renewcommand{\distX}{6.75}
        %Struct
        \draw[barraVI] (\distX,0) node[yshift = -1.0cm, rotate = 45]{Numer} rectangle ++(\largX,1.3);
        \draw[barraS] (\distX+\largX,0) rectangle ++(\largX,0.2);
        \draw[--] (\distX+3.5*\largX,0) to ++(0,-0.2);
        
        \renewcommand{\distX}{8.0}
        %Struct
        \draw[barraVI] (\distX,0) node[yshift = -1.0cm, rotate = 45]{Desc} rectangle ++(\largX, 4.2);
        \draw[barraS] (\distX+\largX,0) rectangle ++(\largX,0.45);
        \draw[--] (\distX+3.5*\largX,0) to ++(0,-0.2);
        
        \renewcommand{\distX}{9.25}
        %Struct
        \draw[barraVI] (\distX,0) node[yshift = -1.0cm, rotate = 45]{Sensory} rectangle ++(\largX,0.8);
        \draw[barraS] (\distX+\largX,0) rectangle ++(\largX,0.0);
        \draw[--] (\distX+3.5*\largX,0) to ++(0,-0.2);
        
        \renewcommand{\distX}{10.5}
        %Struct
        \draw[barraVI] (\distX,0) node[yshift = -1.0cm, rotate = 45]{Tem/Dist} rectangle ++(\largX,0.95);
        \draw[barraS] (\distX+\largX,0) rectangle ++(\largX,0.35);
        \draw[--] (\distX+3.5*\largX,0) to ++(0,-0.2);
        
        \renewcommand{\distX}{11.75}
        %Struct
        \draw[barraVI] (\distX,0) node[yshift = -1.0cm, rotate = 45]{Motion} rectangle ++(\largX,0.1);
        \draw[barraS] (\distX+\largX,0) rectangle ++(\largX,0.0);
        \draw[--] (\distX+3.5*\largX,0) to ++(0,-0.2);
        
        \renewcommand{\distX}{13.0}
        %Struct
        \draw[barraVI] (\distX,0) node[yshift = -1.0cm, rotate = 45]{Social} rectangle ++(\largX,0.2);
        \draw[barraS] (\distX+\largX,0) rectangle ++(\largX,0.0);
        \draw[--] (\distX+3.5*\largX,0) to ++(0,-0.2);
        
        %Legenda
        \draw[legenda] (\tamX-3.0cm,\tamY-1.5cm) rectangle (\tamX+1.5cm, \tamY+0.25cm);
        \draw[barraVI] (\tamX-2.5cm,\tamY-0.25cm) rectangle ++(0.25cm,0.25cm) node[anchor = west, xshift = 0.15cm, yshift = -0.15cm]{Visualy Impaired};
        \draw[barraS] (\tamX-2.5cm,\tamY-1.0cm) rectangle ++(0.25cm,0.25cm) node[anchor = west, xshift = 0.15cm, yshift = -0.15cm]{Sighted};
        
    \end{tikzpicture}
    }
    \centering
    \caption{Comparison between sighted participants with BVI participants (Adapted from \citeonline{bradley2002investigating}).}
    \label{fig:bradley_2002}
\end{figure}

Among the comments provided by BVI participants, a common one was about the limitations of available navigation options, such as white canes and guide dogs. They also emphasize that, when navigating, using different senses is essential for confirming one piece of information. 

To extend the findings of their previous work, Bradley and Dunlop designed an experiment to investigate if there is a difference between the perceived workload of BVI participants and sighted participants when they navigate using user-tailored information created with the results of the previous experiments \cite{bradley2005experimental}.

The experiment was performed with 16 participants, 8 sighted and 8 BVI, who were recruited to walk to four pre-determined landmarks in the centre of Glasgow. They followed the orientations recorded during the interviews from their previous work. For each participant, orientations for 2 of the 4 landmarks were made using sighted users' interviews, while the other 2 used data from BVI interviews. The results showed that BVI users reached landmarks significantly quicker when given the information made for that group, but still longer than sighted users. 

Another issue analysed during the experiment was the perceived workload. After each landmark, the participant was asked to complete the NASA-TLX questionnaire. The average score for each dimension of the NASA-TLX is reproduced in Figure \ref{fig:bradley_2005_participants}. As expected, it shows that BVI participants systematically have a higher workload than sighted participants. It also confirms that BVI did have a higher workload when guided by orientations provided by sighted people, as well as the sighted participants did with orientations from BVI. Another essential piece of information that stands out is the high frustration score given by the BVI users when they were guided by the orientations of sighted people.

\begin{figure}[htbp]
    \centering
    \begin{subfigure}{.49\textwidth}
        \centering
        \resizebox{\linewidth}{!}{
        \tikzstyle{barraVI} = [fill = cor1]
\tikzstyle{barraS} = [fill = cor2]
\tikzstyle{legenda} = [fill = white, line width = 0.25mm]
\tikzstyle{--} = [line width = 0.25mm]

%\resizebox{0.8\linewidth}{!}{
\begin{tikzpicture}[node distance=0cm]
    % Fundo do gráfico

    \renewcommand{\tamX}{12.0cm}
    \renewcommand{\tamY}{7.0cm}
    
    \node (origin) {};
    \node (endX) [xshift = \tamX] {};
    \node (endY) [yshift = \tamY] {};
    \node (endXY) [above of = endX, yshift = \tamY] {};
    
    %Título
    \node (titulo) [xshift = \tamX*0.5, yshift = \tamY + 1cm] {\textbf{\LARGE Comparing group scores for condition 1}};
    \draw[--] (origin.west) node[anchor = east]{\Large 0} to (endX.center) node[anchor = north, xshift = -\tamX*0.5, yshift = -1.0cm]{\textbf{\LARGE Workload dimensions}};
    \draw[--] (origin.south) to (endY.center) 
    node(eixoY)[anchor = east, xshift = -2.5cm, yshift = 0.5cm, rotate = 90]{\textbf{\LARGE Average weighted score}};
    \draw[--] (endX.center) to (endXY.center);
    
   \foreach \r/\n in {1/50,2/100,3/150,4/200,5/250,6/300, 7/350}
    {
        \draw [--] (-0.15,1cm*\r) node[anchor = east]{\Large \n} to (\tamX,1cm*\r);
    }
    
    \renewcommand{\largX}{0.5}
    \renewcommand{\altY}{0}
    \renewcommand{\distX}{0.5}
    
    %Mental Demand
    \draw[barraVI] (\distX,0) node[xshift = \largX*1cm, yshift = -0.5cm]{\textbf{\LARGE MD}} rectangle ++(\largX,6.1);
    \draw[barraS] (\distX+\largX,0) rectangle ++(\largX,2.8);
    \draw[--] (\distX+3*\largX,0) to ++(0,-0.2);
    
    \renewcommand{\distX}{2.5}
    %Physical Demand
    \draw[barraVI] (\distX,0) node[xshift = \largX*1cm, yshift = -0.5cm]{\textbf{\LARGE PD}} rectangle ++(\largX,0.3);
    \draw[barraS] (\distX+\largX,0) rectangle ++(\largX,0.5);
    \draw[--] (\distX+3*\largX,0) to ++(0,-0.2);
    
    \renewcommand{\distX}{4.5}
    %Temporal demand
    \draw[barraVI] (\distX,0) node[xshift = \largX*1cm, yshift = -0.5cm]{\textbf{\LARGE TD}} rectangle ++(\largX,0.9);
    \draw[barraS] (\distX+\largX,0) rectangle ++(\largX,0.7);
    \draw[--] (\distX+3*\largX,0) to ++(0,-0.2);
    
    \renewcommand{\distX}{6.5}
    %Performance
    \draw[barraVI] (\distX,0) node[xshift = \largX*1cm, yshift = -0.5cm]{\textbf{\LARGE OP}} rectangle ++(\largX,2.2);
    \draw[barraS] (\distX+\largX,0) rectangle ++(\largX,0.6);
    \draw[--] (\distX+3*\largX,0) to ++(0,-0.2);
    
    \renewcommand{\distX}{8.5}
    %Effort
    \draw[barraVI] (\distX,0) node[xshift = \largX*1cm, yshift = -0.5cm]{\textbf{\LARGE EF}} rectangle ++(\largX,4.1);
    \draw[barraS] (\distX+\largX,0) rectangle ++(\largX,2.3);
    \draw[--] (\distX+3*\largX,0) to ++(0,-0.2);
    
    \renewcommand{\distX}{10.5}
    %Frustation
    \draw[barraVI] (\distX,0) node[xshift = \largX*1cm, yshift = -0.5cm]{\textbf{\LARGE FR}} rectangle ++(\largX,3.7);
    \draw[barraS] (\distX+\largX,0) rectangle ++(\largX,0.2);
    \draw[--] (\distX+3*\largX,0) to ++(0,-0.2);
    
    
    %Legenda
    \draw[legenda] (\tamX-3.0cm,\tamY-1.5cm) rectangle ++(6.5cm,2cm);
    \draw[barraVI] (\tamX-2.75cm,\tamY-0.25cm) rectangle ++(0.25cm,0.25cm) 
    node[anchor = west, xshift = 0.15cm, yshift = -0.15cm]{\LARGE Visually impaired};
    \draw[barraS] (\tamX-2.75cm,\tamY-1.0cm) rectangle ++(0.25cm,0.25cm) 
    node[anchor = west, xshift = 0.15cm, yshift = -0.15cm]{\LARGE Sighted};
    
\end{tikzpicture}
%}
        }
        \caption{Condition 1.}
        \label{fig:bradley_2005_nasa_participants_1}
    \end{subfigure}
    \hfill
    \begin{subfigure}{.49\textwidth}
        \centering
        \resizebox{\linewidth}{!}{
        \tikzstyle{barraVI} = [fill = cor1]
\tikzstyle{barraS} = [fill = cor2]
\tikzstyle{legenda} = [fill = white, line width = 0.25mm]
\tikzstyle{--} = [line width = 0.25mm]

%\resizebox{0.8\linewidth}{!}{
\begin{tikzpicture}[node distance=0cm]
    % Fundo do gráfico

    \renewcommand{\tamX}{12.0cm}
    \renewcommand{\tamY}{6.0cm}
    
    \node (origin) {};
    \node (endX) [xshift = \tamX] {};
    \node (endY) [yshift = \tamY] {};
    \node (endXY) [above of = endX, yshift = \tamY] {};
    
    %Título
    \node (titulo) [xshift = \tamX*0.5, yshift = \tamY + 1cm] {\textbf{\LARGE Orientation from BVI people}};
    \draw[--] (origin.west) node[anchor = east]{\Large 0} to (endX.center) node[anchor = north, xshift = -\tamX*0.5, yshift = -5.0cm]{\textbf{\LARGE Workload dimensions}};
    \draw[--] (origin.south) to (endY.center) 
    node(eixoY)[anchor = east, xshift = -2.5cm, yshift = 0.5cm, rotate = 90]{\textbf{\LARGE Average weighted score}};
    \draw[--] (endX.center) to (endXY.center);
    
    \foreach \r/\n in {1/50,2/100,3/150,4/200,5/250,6/300}
    {
        \draw [--] (-0.15,1cm*\r) node[anchor = east]{\Large \n} to (\tamX,1cm*\r);
    }
    
    \renewcommand{\largX}{0.5}
    \renewcommand{\altY}{0}
    \renewcommand{\distX}{0.5}
    
    %Mental Demand
    %\draw[barraVI] (\distX,0) node[xshift = \largX*1cm, yshift = -0.5cm]{\textbf{\LARGE MD}} rectangle ++(\largX,5.6);
    \draw[barraVI] (\distX,0) node[xshift = \largX*-1.65cm, yshift = -2.25cm]{\rotatebox{50}{\textbf{\LARGE Mental Demand}}} rectangle ++(\largX,5.6);
    \draw[barraS] (\distX+\largX,0) rectangle ++(\largX,3.1);
    \draw[--] (\distX+3*\largX,0) to ++(0,-0.2);
    
    \renewcommand{\distX}{2.5}
    %Physical Demand
    %\draw[barraVI] (\distX,0) node[xshift = \largX*1cm, yshift = -0.5cm]{\textbf{\LARGE PD}} rectangle ++(\largX,0.7);
    \draw[barraVI] (\distX,0) node[xshift = \largX*-1.75cm, yshift = -2.25cm]{\rotatebox{50}{\textbf{\LARGE Physical Demand}}} rectangle ++(\largX,0.7);
    \draw[barraS] (\distX+\largX,0) rectangle ++(\largX,0.5);
    \draw[--] (\distX+3*\largX,0) to ++(0,-0.2);
    
    \renewcommand{\distX}{4.5}
    %Temporal demand
    %\draw[barraVI] (\distX,0) node[xshift = \largX*1cm, yshift = -0.5cm]{\textbf{\LARGE TD}} rectangle ++(\largX,0.9);
    \draw[barraVI] (\distX,0) node[xshift = \largX*-2cm, yshift = -2.5cm]{\rotatebox{50}{\textbf{\LARGE Temporal Demand}}} rectangle ++(\largX,0.9);
    \draw[barraS] (\distX+\largX,0) rectangle ++(\largX,0.7);
    \draw[--] (\distX+3*\largX,0) to ++(0,-0.2);
    
    \renewcommand{\distX}{6.5}
    %Performance
    %\draw[barraVI] (\distX,0) node[xshift = \largX*1cm, yshift = -0.5cm]{\textbf{\LARGE OP}} rectangle ++(\largX,2.1);
    \draw[barraVI] (\distX,0) node[xshift = \largX*-1.25cm, yshift = -1.65cm]{\rotatebox{50}{\textbf{\LARGE Performance}}} rectangle ++(\largX,2.1);
    \draw[barraS] (\distX+\largX,0) rectangle ++(\largX,0.5);
    \draw[--] (\distX+3*\largX,0) to ++(0,-0.2);
    
    \renewcommand{\distX}{8.5}
    %Effort
    %\draw[barraVI] (\distX,0) node[xshift = \largX*1cm, yshift = -0.5cm]{\textbf{\LARGE EF}} rectangle ++(\largX,3.8);
    \draw[barraVI] (\distX,0) node[xshift = \largX*-0.0cm, yshift = -0.850cm]{\rotatebox{50}{\textbf{\LARGE Effort}}} rectangle ++(\largX,3.8);
    \draw[barraS] (\distX+\largX,0) rectangle ++(\largX,2.5);
    \draw[--] (\distX+3*\largX,0) to ++(0,-0.2);
    
    \renewcommand{\distX}{10.5}
    %Frustation
    %\draw[barraVI] (\distX,0) node[xshift = \largX*1cm, yshift = -0.5cm]{\textbf{\LARGE FR}} rectangle ++(\largX,1.5);
    \draw[barraVI] (\distX,0) node[xshift = \largX*-1.0cm, yshift = -1.35cm]{\rotatebox{50}{\textbf{\LARGE Frustation}}} rectangle ++(\largX,1.5);
    \draw[barraS] (\distX+\largX,0) rectangle ++(\largX,1.1);
    \draw[--] (\distX+3*\largX,0) to ++(0,-0.2);
    
    
    %Legenda
    \draw[legenda] (\tamX-3.0cm,\tamY-1.5cm) rectangle ++(6.5cm,2cm);
    \draw[barraVI] (\tamX-2.75cm,\tamY-0.25cm) rectangle ++(0.25cm,0.25cm) 
    node[anchor = west, xshift = 0.15cm, yshift = -0.15cm]{\LARGE BVI};
    \draw[barraS] (\tamX-2.75cm,\tamY-1.0cm) rectangle ++(0.25cm,0.25cm) 
    node[anchor = west, xshift = 0.15cm, yshift = -0.15cm]{\LARGE Sighted};
    
\end{tikzpicture}
%}
        }
        \caption{Condition 2.}
        \label{fig:bradley_2005_nasa_participants_2}
    \end{subfigure}
\caption{Comparison of the NASA-TLX between the participants (Adapted from \citeonline{bradley2005experimental}).}
\label{fig:bradley_2005_participants}
\end{figure}

The work of \citeonline{bradley2002investigating} brings some relevant information for developing this work. Firstly, it shows the differences between the way sighted and BVI people navigate, highlighting the importance of including BVI in the design process of assistive technologies. It confirms the limitations of the current solutions. It brings essential insights on what type of information to include in developing audio systems, which is one of the assistive devices evaluated in this work. Finally, it shows the importance of using different workload assessment techniques when evaluating assistive technologies.

\section{Audio navigation for BVI}
\label{sec:auditory_navigation}
Despite the several existing navigation systems for BVI users, their limitations have been pointed in many works. The work of \citeonline{yang2014design} explored the effect of two factors when BVIs use a common GPS navigation system. The first factor is the amount of detail of the provided information (information completeness). The second one is the distance between the information reproduction and the object referred by that same information (broadcasting timing).

In order to evaluate the impact of these two factors, \citeonline{yang2014design} performed an experiment with BVI users where each factor had two levels. The completeness of the information could be “complete” and “simple” and the broadcast timing could be 5m and 7m. As output the authors evaluated the participants’ performance by their precision and time in finding a goal, and evaluated their perceived workload with NASA-TLX. 

The independent variables were analysed by a two-way ANOVA hypothesis test. They found out that the precision in finding the goal was only influenced by the broadcasting timing. The time in finding the goal was influenced by both variables. The task’s workload was influenced by the broadcasting timing and the interaction between it and the information completeness.

The work of \citeonline{yang2014design} shows the importance of synchronizing the information provided by the audio system with the current position of the BVI user – a point to be taken into account when developing the audio solution used in this work. However, concerning the lack of influence of information completeness, it is important to observe that, in a certain way, this result contradicts the conclusions of Bradley and Dunlop of \citeyear{bradley2002investigating,bradley2005experimental} and, therefore, should be considered with caution. It may be due to the difference between the two levels (complete and simple) adopted in the experiment. Finally, the work of \citeonline{yang2014design} confirms the NASA-TLX as a feasible tool to evaluate workload in experiments with BVI participants.


\section{Comparison of assistive devices}
\label{sec:evaluation_spatial_display}
\citeonline{marston2006evaluation} presents a comparison of assistive devices for BVI. Two guidance displays were evaluated, one based on haptics and another based on sound. They were tested in two different scenarios: a busy street block, with a variety of street furniture, parked bicycles and people, and a park, with paths made of concrete, crushed gravel and paver blocks. 

The experiment was performed with 8 BVI participants. As output, the authors collected the time to reach a set of waypoints, the errors made by the participants, the travelled distance and the percentage of the total time that the users accessed the guidance device. All participants were able to complete the task with both devices. However, the configuration (audio x haptic, street x park) that resulted in the best performance varied among the participants. One relevant consideration is that the haptic device caused strain on the participants’ arm and was considered less acceptable, when compared to the sound device, which required no use of the arms.

Similar to the study of \citeonline{marston2006evaluation}, this work also compares different devices, based on haptics and sound. But, complementary to \citeonline{marston2006evaluation}, this work also evaluates the combined use of devices, as well as how they compare to the guidance system current in use by the BVI participant. 

\section{Feeling of presence in virtual reality}
\label{sec:emotion_presence_vr}
The second work that discussed in this literature review is an evaluation of what affects the user’s feeling of presence in virtual reality, i.e., when the user feels drawn into the virtual environment and starts to occupy it instead of the real one \cite{cummings2016immersive}.


One of the many feelings that flourish during the use of a VR is the feeling of presence. This feeling, inside the virtuality context, is when someone feels drawn into a VE and starts to occupy the VE instead of the real one \cite{cummings2016immersive}.

\citeonline{jicol2021effects} aim to correlate the feeling of presence with one’s agency (which is the self-perception that the user is in control of a situation or some actions \cite{farrer2002experiencing} and emotion. For this purpose, the authors created two different virtual environment, one that would trigger happy emotions, and another that would trigger fear. For each of them, two variations were provided: one that the user could interact with (with agency) and another that it could not (without agency).

Following, they performed an experiment where 121 participants were randomly assigned to one of the four virtual environments. The purpose was to evaluate three hypotheses: 1) The intensity of the dominant emotion correlates positively with the presence; 2) Presence is significantly higher in environments where participants have agency; and 3) Agency moderates the effect of the emotion on the presence.

The results of the experiment confirmed the first hypothesis: no matter if the feeling was positive (happiness) or negative (fear), the users did feel a stronger presence when the positive or negative feelings were more intense. The second hypothesis was only partially confirmed. In the virtual environment that induced fear, agency did make a difference and induced a higher feeling of presence, whilst in the environment that induced happiness, agency did not affect the presence. The same could be said about the third hypothesis.

Although the study of \citeonline{jicol2021effects} was limited to sighted people, it highlights the importance of the feeling of presence for virtual reality. It provides important inputs to this work, such as the need of including mechanisms for the user to interact with the virtual environment in order to increase the feeling of presence. 


\section{Final Remarks}
\label{sec:final_remarks3}

This chapter discussed published works that are related to this dissertation. These works combine at least two of the following concepts: ”human factors”, ”virtual reality” and ”blindness”. 

Some of the works discuss the design of navigation tools, investigating different aspects of their solutions, while others focus on providing a virtual reality experiment to BVI users. Among the assessment techniques used in the literature, the NASA-TLX is a recurrent option, as well as the proposal of performance metrics.

Generally, all the surveyed works bring exciting conclusions about the designing of BVI devices and contributed in some way to the proposal of next chapter.

