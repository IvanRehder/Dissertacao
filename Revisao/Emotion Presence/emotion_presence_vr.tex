The second work discussed in this literature review is an evaluation of what affects the user's feeling of presence in virtual reality, i.e. when the user feels drawn into the virtual environment and starts to occupy it instead of the real one \cite{cummings2016immersive}.

One of the many feelings that flourish during the use of a VR is the feeling of presence. This feeling, inside the virtuality context, is when someone feels drawn into a VE and starts to occupy the VE instead of the real one \cite{cummings2016immersive}.

\citeonline{jicol2021effects} aim to correlate the feeling of presence with one's agency (which is the self-perception that the user is in control of a situation or some actions \cite{farrer2002experiencing} and emotion. For this purpose, the authors created two virtual environment, one that would trigger happy emotions, and other that would trigger fear. For each, two variations were provided: one that the user could interact with (with agency) and another that it could not (without agency).

Following, they performed an experiment where 121 participants were randomly assigned to one of the four virtual environments. The purpose was to evaluate three hypotheses: 1) The intensity of the dominant emotion correlates positively with the presence; 2) Presence is significantly higher in environments where participants have agency; and 3) Agency moderates the effect of the emotion on the presence.

The experiment's results confirmed the first hypothesis: no matter if the feeling was positive (happiness) or negative (fear), the users did feel a more substancial presence when the positive or negative feelings were more intense. The second hypothesis was only partially confirmed. In the virtual environment that induced fear, agency did make a difference and induced a higher feeling of presence, while in the environment that induced happiness, agency did not affect the presence. The same could be said about the third hypothesis.

Although the study of \citeonline{jicol2021effects} was limited to sighted people, it highlights the importance of the feeling of presence in virtual reality. It provides important inputs to this work, such as the need of including mechanisms for the user to interact with the virtual environment in order to increase the feeling of presence. 
