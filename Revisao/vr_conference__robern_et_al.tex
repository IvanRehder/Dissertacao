At the moment that this work was being written, the article from \citeauthor{robern2021work} was a work-in-progress. Their work was also related to the COVID-19 pandemic, but it was more concerned with the navigation of BVI users inside VE.
The experiment simulated a virtual conference and presented two moments for the participantes:

\begin{itemize}
    \item Registration
    
        Here the participant should pick up it registration pass and bring it to the registration desk.
        
    \item Conference
    
        After the registration, the participant should enter the conference room and seat in a chair. After that the participant could choose if wants to leave or stay at the room and while in the room it could ask questions to a presenter.
        
\end{itemize}

The author chosen the System Usability Scale (SUS), with 10 questions in a 5-point Likert Scale, and the Game Engagement Questionnaire (GEQ), which rates the experience in 7 areas: "Competence", "Immersion", "Flow", "Tension" , "Challenge", "Negative Affect" and "Positive Affect". Each participant did twice the experiment, on without the VR HMD (non-immersive VR) and othe with it. The results from each trial was compared and the authors noticed an increase in both SUS and GEQ when the participants used the VR.

Besides being a "work-in-progress" the author used audio cues for the navigation.