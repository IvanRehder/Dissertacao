Despite the several existing navigation systems for BVI users, their limitations have been pointed in many works. The work of \citeonline{yang2014design} explored the effect of two factors when BVIs use a common GPS navigation system. The first factor is the amount of detail of the provided information (information completeness). The second one is the distance between the information reproduction and the object referred by that same information (broadcasting timing).

In order to evaluate the impact of these two factors, \citeonline{yang2014design} performed an experiment with BVI users where each factor had two levels. The completeness of the information could be “complete” and “simple” and the broadcast timing could be 5m and 7m. As output the authors evaluated the participants’ performance by their precision and time in finding a goal, and evaluated their perceived workload with NASA-TLX. 

The independent variables were analysed by a two-way ANOVA hypothesis test. They found out that the precision in finding the goal was only influenced by the broadcasting timing. The time in finding the goal was influenced by both variables. The task’s workload was influenced by the broadcasting timing and the interaction between it and the information completeness.

The work of \citeonline{yang2014design} shows the importance of synchronizing the information provided by the audio system with the current position of the BVI user -– a point to be taken into account when developing the audio solution used in this work. However, concerning the lack of influence of information completeness, it important to observe that, in a certain way, this result contradicts the conclusions of Bradley and Dunlop of \citeyear{bradley2002investigating,bradley2005experimental} and, therefore, should be considered with caution. It may be due to the difference between the two levels (complete and simple) adopted in the experiment. Finally, the work of \citeonline{yang2014design} confirms the NASA-TLX as a feasible tool to evaluate workload in experiments with BVI participants.
