Despite existing several types of navigation systems for BVI users, they suffer from the same unsatisfaction from their users mentioned at the beginning of this dissertation. \citeonline{yang2014design} desired to explore the use of the common navigation GPS by BVI users. Their objectives were:

\begin{itemize}
    \item Verify the effects on performance caused by the amount of detail of the Information (Information completeness);
    \item Verify the effects on performance caused by the distance between the information reproduction and the object referred by that same information (Broadcasting timing).
\end{itemize}

The participant of \citeauthor{yang2014design} experiment performed a way-finding test based on information from a GPS navigation system. There were two independent variables and each of them had 2 options, the completeness of the information (it could be "complete" and "simple") and the broadcast timing (it could be 5m and 7m).

\citeauthor{yang2014design} evaluated the participants' performance by their precision and the time in finding the goal and evaluated their subjective workload with a NASA-TLX. The independent variables were analyzed by a Two-way ANOVA hypothesis test. They found out that the precision in finding the goal was influenced, and only, by the broadcasting timing. The time in finding the goal was influenced by both variables, but not by their interaction. The task's workload was influenced by the broadcasting timing and by the interaction between it and the information completeness.

Despite \citeauthor{yang2014design} findings, that BVI users are more influenced by the broadcasting timing than the information completeness, the described work used the same subjective test used by this dissertation with the same target group and it showed good results.