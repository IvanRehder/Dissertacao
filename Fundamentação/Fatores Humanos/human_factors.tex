Studies started during the Second World War because of the performance shortfalls and failures noted in manned equipment. These studies showed that these problems could diminish when engineering, psychology and physiology were gathered when designing a system that was to be handled by a human being \cite{sandom2004human}.

This area of study was named "Human Factors" in the United States and "Ergonomics" in Europe. Despite this difference in the names, today they are considered the same field of study. The International Ergonomics Association (IEA) defines Human Factors, and Ergonomics, as the following:

\begin{quote}
    Ergonomics (or human factors) is the scientific discipline concerned with the understanding of interactions among humans and other elements of a system, and the profession that applies theory, principles, data and methods to design in order to optimize human well-being and overall system performance. Human Factors professionals contribute to the design and evaluation of tasks, jobs, products, environments and systems in order to make them compatible with the needs, abilities and limitations of people \cite{karwowski2012discipline}.
\end{quote}

Besides being synonyms, this definition shows that humans are a variable inside a system and their interactions should be studied and that is the focus of Human Factors \cite{sandom2004human, sanders1998human, dul2003ergonomics}. 

Humans handle devices, machines and equipment during their daily activities and all of these manipulations are susceptible to accidents or failures that can happen because of the interaction between operator, equipment and environment. Each interface with the operator can be a factor, for example:

\begin{itemize}
    \item The operator's body position during an activity;
    
    The position can impact the comfort felled by the operator and this impacts its concentration throughout the activity, therefore, impacting the success rate or the chance of some accident happening \cite{sanders1998human}.
    
    \item The environment's lighting;
    
    The illumination can make details easier to be noted without provoking discomfort or distraction to the user and even increase productivity \cite{sanders1998human}.
    
    \item The information displayed and manipulation of the device.
    
    The way information is displayed on a screen, figure or text impacts how efficiently it will be understood by the operator. If this takes too long it can draw the operator's attention for too long and compromise his/her reaction time.
    
\end{itemize}

Taking humans into account when designing a product or a system is one of the principles for human factors \cite{sandom2004human} and the results of this human-centred design are already an ISO Standard (BS EN ISO 13407 "Human-centred design processes for interactive systems"). This standard was originally written for computer-based-systems, but is easily applicable in other scenarios and areas \cite{sandom2004human}.

It is important to say that when it is said "User", it doesn't mean that one needs to design a product specifically for an individual. The design has to be suited to everyone \cite{dul2003ergonomics}.

"Human-Machine systems" (on this thesis, for now on, called simply "Systems"), are interactions between humans and machines. These systems are designed to have an input, or demand, and an output, or product. Here, "machine" can be any manipulated object, from a simple screwdriver to a car, or some machine operated by more than one human, like a cargo ship for example. The Figure \ref{fig:human_machine_representaion} represent a general human-system machine interaction.

\begin{figure}[!htb]
    \centering

    \tikzstyle{arrow} = [rounded corners, line width = 1mm, bend left = 15, ->]
    
    \resizebox{0.85\width}{!}{
    \begin{tikzpicture}[node distance=1cm]
        
        \node (information) {\includegraphics[width=.15\textwidth]{Fundamentação/Fatores Humanos/thinking.png}} 
        node(t_information)[below of = information,yshift=-0.75cm] {Information}
        node(t_information2)[below of = t_information,yshift=0.25cm] {processing};
        
        \node (controlling) [right of=information, xshift=5cm, yshift=-3cm] {\includegraphics[width=.15\textwidth]{Fundamentação/Fatores Humanos/slider.png}}
        node(t_controlling)[below of = controlling,yshift = -0.75cm] {Controlling};
        
        \node (controls) [below of=controlling, yshift=-5cm,] {\includegraphics[width=.15\textwidth,angle=90,origin=c]{Fundamentação/Fatores Humanos/control.png}} 
        node(t_controls) [below of = controls, yshift = -0.75cm]{Controls};
        
        \node (machine) [left of=controls, xshift=-5cm, yshift=-2cm] {\includegraphics[width=.15\textwidth]{Fundamentação/Fatores Humanos/machine.png}} 
        node(t_machine) [below of = machine, yshift = -0.75cm]{Operation};
        
        \node (display) [left of=machine, xshift=-5cm, yshift=2cm] {\includegraphics[width=.15\textwidth]{Fundamentação/Fatores Humanos/monitor.png}} 
        node(t_display) [below of = display, yshift = -0.75cm]{Display};
        
        \node (senses) [left of=information, xshift=-5cm, yshift=-2cm,] {\begin{tikzpicture}[node distance=1cm]
    \centering
    
    \node (eye) {\includegraphics[width=.075\textwidth]{Fundamentação/Fatores Humanos/eye.png}};
    
    \node (ear) [right of=eye, yshift=-0.65cm] {\includegraphics[width=.075\textwidth]{Fundamentação/Fatores Humanos/ear.png}};
    
    \node (nose) [left of=ear, yshift=-0.65cm] {\includegraphics[width=.075\textwidth]{Fundamentação/Fatores Humanos/nose.png}};
    
    \node (hand) [right of=nose, yshift=-0.85cm] {\includegraphics[width=.075\textwidth]{Fundamentação/Fatores Humanos/hand.png}};

\end{tikzpicture}} 
        node(t_senses) [below of = senses, yshift = -1.25cm]{Senses};
        
        \node (human) [below of=information, yshift=-4.75cm] {\Large{Human}};
        \node (human) [above of=machine, yshift=3.25cm] {\Large{Machine}};
        \node (human) [above of=information, yshift=1cm] {\Large{Work Environment}};
        
        \node (left_point) [left of=display, xshift=-2, yshift=2.75cm] {};
        \node (right_point) [right of=left_point, xshift=14cm] {};
        
        \node (input) [left of=machine, xshift=-5cm, yshift=-2cm] {\Large{Input}};
        \node (output) [right of=machine, xshift=5cm, yshift=-2cm] {\Large{Output}};
        
    
        \draw [arrow] (information.east) to (controlling.north west);
        \draw [arrow] (t_controlling.south) to (controls.north);
        \draw [arrow] (controls.west) to (machine.east);
        \draw [arrow] (machine.west) to (display.east);
        \draw [arrow] (display.north) to (t_senses.south);
        \draw [arrow] (senses.east) to (information.west);
        \draw [arrow] (input.east) -- (machine);
        \draw [arrow] (machine) -- (output.west);
        \draw [dashed,gray] (left_point) to (right_point);
        
        \draw (-8,-14) rectangle(8cm,1.5cm);
        
    \end{tikzpicture}
    }
    \caption{Human-Machine system representation. Adapted from \cite{sanders1998human}.}
    \label{fig:human_machine_representaion}
\end{figure}