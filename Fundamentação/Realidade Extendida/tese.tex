%%% Exemplo de utilização da classe ITA
%%%
%%%   por        Fábio Fagundes Silveira   -  ffs [at] ita [dot] br
%%%              Benedito C. O. Maciel     -  bcmaciel [at] ita [dot] br
%%%              Giovani Volnei Meinertz   -  giovani [at] ita [dot] br
%%%    	         Hudson Alberto Bode       -  bode [at] ita [dot]br
%%%    	         P. I. Braga de Queiroz    -  pi [at] ita [dot] br
%%%    	         Jorge A. B. Gripp         -  gripp [at] ita [dot] br
%%%    	         Juliano Monte-Mor         -  jamontemor [at] yahoo [dot] com [dot] br
%%%    	         Tarcisio A. B. Gripp      -  tarcisio.gripp [at] gmail [dot] com
%%%    	         
%%%   Versão para overleaf:
%%%   por           Alejandro A. Rios Cruz - aarc.88@gmail.com 	         
%%%                 Saulo Gómez            - sagomezs@unal.edu.co 
%%%  IMPORTANTE: O texto contido neste exemplo nao significa absolutamente nada.  :-)
%%%              O intuito aqui eh demonstrar os comandos criados na classe e suas
%%%              respectivas utilizacoes.
%%%
%%%  Tese.tex  2016-08-25
%%%  $HeadURL: http://www.apgita.org.br/apgita/teses-e-latex.php $
%%%
%%% ITALUS
%%% Instituto Tecnológico de Aeronáutica --- ITA, Sao Jose dos Campos, Brasil
%%%                   http://groups.yahoo.com/group/italus/
%%% Discussion list: italus {at} yahoogroups.com
%%%
%++++++++++++++++++++++++++++++++++++++++++++++++++++++++++++++++++++++++++++++
% Para alterar o TIPO DE DOCUMENTO, preencher a linha abaixo \documentclass[?]{?}
%   \documentclass[tg]{ita}			= Trabalho de Graduacao
%   \documentclass[tgfem]{ita}	= Para Engenheiras
%   								msc     		= Dissertacao de Mestrado
%   								mscfem   		= Para Mestras
%   								dsc      		= Tese de Doutorado
%   								dscfem   		= Para Doutoras
%   								quali    		= Exame de Qualificacao
%   								qualifem 		= Exame de Qualificacao para Doutoras
% Para 'Draft Version'/'Versao Preliminar' com data no rodape, adicionar 'dv':
%   \documentclass[dsc, dv]{ita} 
% Para trabalhos em Inglês, adicionar 'eng':
%   \documentclass[dsc, eng]{ita}
%		\documentclass[dsc, eng, dv]{ita}
%++++++++++++++++++++++++++++++++++++++++++++++++++++++++++++++++++++++++++++++
\documentclass[msc, dv, eng]{ita}    % ITA.cls based on standard book.cls 
% Quando alterar a classe, por exemplo de [msc] para [msc, eng]) rode mais uma vez o botão BUILD OUTPUT caso haja erro

\hfuzz=5.002pt 

\usepackage{ae}
\usepackage{graphicx}
%\usepackage{array}
\usepackage{epsfig}
\usepackage{amsmath}
\usepackage{amssymb} 
%\usepackage{subfig}
\usepackage{multirow}
\usepackage{float}
\usepackage{amsthm}
\usepackage{url}         % formats URL addresses properly
\usepackage{appendix}    % allows appendix section to be included
\usepackage{lscape}      % allows a page to be rendered in landscape mode
\usepackage{multicol}    % allows text in multi columns
\usepackage{cancel}      % needed to show canceled terms in equations
\usepackage{lettrine}
\usepackage{float}
\usepackage{placeins}
\usepackage{pdfpages}
\usepackage{listings}   % Code formating
\usepackage[dvipsnames]{xcolor}   % Code formating
    \definecolor{ccmBlue}{RGB}{20, 80, 200}
    \definecolor{ccmDBlue}{RGB}{25, 50, 120}
    \definecolor{ccmLBlue}{RGB}{170, 200, 230}
    \definecolor{ccmOrange}{RGB}{255, 100, 0}
    \definecolor{ccmRed}{RGB}{190, 0, 0}
    \definecolor{ccmLGray}{RGB}{210, 210, 210}
    \definecolor{ccmDGray}{RGB}{85, 85, 85}
    
    % Ordem Paleta
    \definecolor{cor1}{RGB}{190, 000, 000}  %ccmRed
    \definecolor{cor2}{RGB}{025, 050, 120}  %ccmDBlue
    \definecolor{cor3}{RGB}{210, 210, 210}  %ccmLGray
    \definecolor{cor4}{RGB}{255, 100, 000}  %ccmOrange
    \definecolor{cor5}{RGB}{170, 200, 230}  %ccmLBlue
    \definecolor{cor6}{RGB}{085, 085, 085}  %ccmDGray
    \definecolor{cor7}{RGB}{020, 080, 200}  %ccmBlue

%\usepackage{pgfplots}
%\DeclareUnicodeCharacter{2212}{−}
%\usepgfplotslibrary{groupplots,dateplot}
%\usetikzlibrary{patterns,shapes.arrows}
%\pgfplotsset{compat=newest}


\usepackage{boldline}
\usepackage{blindtext}
\usepackage{lipsum}
\usepackage{pgf}
\usepackage{tikz}
    \usetikzlibrary{matrix,shapes.geometric,shapes.symbols,arrows.meta,positioning}
    \tikzset{>={Latex[round]}}
    
\usepackage{caption}
\usepackage{subcaption}

% Varíaveis para plotar gráficos em barr
\newcommand{\tamX}{0}
\newcommand{\tamY}{0}
\newcommand{\distX}{0}
\newcommand{\distY}{0}
\newcommand{\largX}{0}
\newcommand{\largY}{0}
\newcommand{\altX}{0}
\newcommand{\altY}{0}

\lstdefinestyle{sharpc}{language=[Sharp]C, frame=lr, rulecolor=\color{blue!80!black}}

\newcolumntype{Y}{>{\centering\arraybackslash}X}

%HHHHHHHHHHHHHHHHHHHHHHHHHHHHHHHHHHHHHHHHHHHHHHHHHHHHHHHHHHHHHHHHHHHHHHHHHHHHHHHHHHHHHHHHHHHHHHHHHHHHHHHHHHHH
%\usepackage{subfigure}
%\usepackage{subfigmat}
%PACOTEFIGURAS_SE _ERRADO_ESXCLUIR_ACIMA
\usepackage{booktabs}
%PACOTETABELAS_SE _ERRADO_ESXCLUIR_ACIMA
%HHHHHHHHHHHHHHHHHHHHHHHHHHHHHHHHHHHHHHHHHHHHHHHHHHHHHHHHHHHHHHHHHHHHHHHHHHHHHHHHHHHHHHHHHHHHHHHHHHHHHHHHHHHH

%++++++++++++++++++++++++++++++++++++++++++++++++++++++++++++++++++++++++++++++
% Espaçamento padrão de todo o documento
%++++++++++++++++++++++++++++++++++++++++++++++++++++++++++++++++++++++++++++++
\onehalfspacing

%singlespacing Para um espaçamento simples
%onehalfspacing Para um espaçamento de 1,5
%doublespacing Para um espaçamento duplo

%++++++++++++++++++++++++++++++++++++++++++++++++++++++++++++++++++++++++++++++
% Identificacoes (se o trabalho for em inglês, insira os dados em inglês)
% Para entradas abreviadas de Professora (Profa.) em português escreva: Prof$^\textnormal{a}$.
%++++++++++++++++++++++++++++++++++++++++++++++++++++++++++++++++++++++++++++++
\course{Aeronautical and Mechanical Engineering}  % Programa de PG ou Curso de Graduação
\area{Automation} % Área de concentração na PG (Não utilizado no caso de TG)

% Autor do trabalho: Nome Sobrenome
\authorgender{masc}                     %sexo: masc ou fem
\author{Ivan de Souza}{Rehder}
\itaauthoraddress{Av. Francisco José Longo, 633}{12.245-906}{São José dos Campos--SP}

% Titulo da Tese/Dissertação
% \title{Simulação em realidade virtual para identificar fatores humanos}
\title{Virtual reality simulation for human factors assessment}

% Orientador
\advisorgender{fem}                    % masc ou fem
\advisor{Prof$^\textnormal{a}$.~Dr$^\textnormal{a}$.}{Emilia Villani}{ITA}

% Coorientador (Caso não haja coorientador, colocar ambas as variáveis \coadvisorgender e \coadvisor comentadas, com um % na frente)
\coadvisorgender{fem}									% masc ou fem
\coadvisor{Prof.~Dr.}{Edmar Thomaz da Silva}{ITA}

% Pró-reitor da Pós-graduação
\bossgender{fem}                    % masc ou fem
\boss{Prof$^\textnormal{a}$.~Dr$^\textnormal{a}$.}{Emilia Villani}

%Coordenador do curso no caso de TG
%\bosscoursegender{masc}									% masc ou fem
%\bosscourse{Prof.~Dr.}{John Walker}

% Palavras-Chaves informadas pela Biblioteca -> utilizada na CIP
\kwcip{Virtual Reality}
\kwcip{Human Factors}
\kwcip{Collaborative Design}

% membros da banca examinadora

%\examiner{Prof. Dr.}{Alan Turing}{Presidente}{ITA}
\examiner{Prof. Dr.}{--}{--}{--}
\examiner{Prof. Dr.}{--}{--}{--}
\examiner{Prof. Dr.}{--}{--}{--}
\examiner{Prof. Dr.}{--}{--}{--}
\examiner{Prof. Dr.}{--}{--}{--}

% Data da defesa (mês em maiúsculo, se trabalho em inglês, e minúsculo se trabalho em português) 
\date{}{}{}

% Número CDU - (somente para TG)
\cdu{621.38}

% Glossario
\makeglossary
\frontmatter

%\DeclareUnicodeCharacter{2212}{-}

\begin{document}
% Folha de Rosto e Capa para o caso do TG
\maketitle

% Dedicatoria: Nao esqueca essa secao  ... :-)
\begin{itadedication}
    “If we believe most people can't be trusted, that's how we'll treat each other, to everyone's detriment. Few ideas have as much power to shape the world as our view of other people.”\\
    --- \textsc{Rutger Bregman}
\end{itadedication}

% Agradecimentos
\begin{itathanks}
%% Agradecimentos

I would like to thank my advisor Profª Emilia Vilani and co-advisor Dr. Edmar Thomaz da Silva for their guidance and trust throughout this project.

To thank my parents, for all of the support that was given to me all of these years and that I will always be thankful for having.

To thank Sidney, that guided me in this universe of blind and visually impaired users. Without him, the devices and the results would be very different, and probably less efficient, than the final ones.

To thank Gabriel "Boi", Santiago, André, Wellington and every one of the Competence Center in Manufacturing (and all my other friends that I could not mention here) that shared experiences and knowledge of the daily life of a master's student and also helped me on some of the solutions used in the work.

To thank all of the participants that gave a little bit of their time and patience to provide me with data for this work.

\end{itathanks}

% Epígrafe
\thispagestyle{empty}
\ifhyperref\pdfbookmark[0]{\nameepigraphe}{epigrafe}\fi
\begin{flushright}
\begin{spacing}{1}
\mbox{}\vfill
{\sffamily\itshape
"As a blind man has no idea of colors, so have we no idea of the manner by which they feel the world.”\\}
--- \textsc{Isaac Newton (adapted).}
\end{spacing}
\end{flushright}

% % Resumo
% \begin{abstract}
% \noindent
% %% Resumo

%A sociedade alcançou tecnologia para criar veículos autônomos e conectar diferentes aparelhos e máquinas umas às outras a fim de trocar informações e otimizar a eficiência de produção. Com essa tecnologia, logo será possível obter melhores métodos para orientar usuários cegos e deficientes visuais (CDV) nas suas atividades diárias. Os produtos que estão disponíveis no mercado hoje em dia possuem um número de limitações e não agradam os usuários CDV. Acredita-se que uma das razões desse problema é a ausência do envolvimento de indivíduos CDV no desenvolvimento desses produtos. A falta de uma solução eficiente para a navegação desse público tornou-se mais grave com a pandemia da SARS-CoV 2, quando pessoas eram instruídas a praticar isolamento social e evitar contato em superfícies que possam estar contaminadas. O objetivo desse trabalho é propor um método para avaliação de opções de design para produtos assistivos para CDV baseados em Realidade Virtual (RV). A ideia é usar o RV como um campo de teste, onde o usuário pode experimentar diferentes soluções em diferentes cenários. Com isso, ele se torna integrante do design e da avaliação, resultando em um produto melhor e com uma interface mais simples. O método proposto inclui, além da montagem do ambiente virtual, o uso de sensores fisiológicos e testes subjetivos que aferem a carga mental e a consciência situacional nas diferentes situações e produtos que estão em desenvolvimento. Para ilustrar o método proposto, é estudado a navegação de indivíduos CDV em um hospital que usa protocolos COVID-19. Esse estudo de caso foi escolhido devido a ocorrente pandemia e a situação crítica que ela causa à população CDV. O cenário virtual foi feito usando Unity3D, uma plataforma de desenvolvimento de aplicações para realidade virtual largamente utilizada. O aparelho RV é o Tobbi Eye Tracking VR. São óculos que foram desenvolvidos usando o HTC VIVE. Esses óculos são utilizados para definir a posição e orientação do usuário no ambiente virtual do Unity. Para inferir a carga mental, foram utilizados os sensores fisiológicos da TEA Capitv T-Sens. Eles são o eletrocardiograma (ECG), usado para coletar a frequência cardíaca e a variância cardíaca, e o GSR (Galvanic skin reaction, reação galvânica da pele), para captar a condutância da pele. Além desses sensores, os voluntários também responderam os testes NASA-TLX, também para verificar a carga mental, e uma versão adaptada do SAGAT, para determinar a consciência situacional. Entre os benefícios esperados pelo método é a flexibilidade e a agilidade para se criar diferentes cenários e também a possibilidade de testar eles no mesmo espaço físico. Isso pode acelerar o design de novas soluções e melhorar a qualidade dos produtos. Outro resultado esperado da pesquisa é a identificação de características chaves dos produtos que causam o aumento ou diminuição da carga mental ou da consciência situacional nos usuários CDV.

Society has developed technology to create autonomous vehicles and to connect different devices and machinery to exchange data and optimize production efficiency.  With this technology, soon, it will be possible to achieve better methods to guide blind and visually impaired (BVI) users in their daily activities. The available products in the market have several limitations and do not satisfy BVI users. We believe that one of the reasons behind this problem is that they are not members of the development team or are not consulted by these. 
The lack of an efficient solution for BVI users' navigation became even more significant with the SARS-CoV2 pandemic, in which people had to avoid contact with one another and not touch another surface.
The purpose of this paper is to use virtual reality (VR) to test and evaluate different designs of BVI products. Also to verify if BVI and non-BVI users have the same mental demand and situation awareness when using assistive products. The idea is to use VR as a testing ground where a BVI user can try different assistive solutions in different scenarios. By doing so, the user becomes part of the product design and evaluation, resulting in better and more user-friendly products. The proposed method includes not only the setup of the virtual environment but also the use of physiological sensors and subjective tests to assess the mental workload and situational awareness in different situations.
To illustrate the proposed method, a case study is proposed, in which the navigation of BVI users inside a medical clinic is studied. This case study is chosen due to the current undergoing SARS-CoV-2 pandemic and the impact on BVI people, so the simulated clinic is also applying COVID health protocols.
The scenes were made using Unity3D, a widely used development platform for virtual reality applications. The VR device was the Tobii Eye Tracking VR, a head-mounted display for virtual reality developed using the HTC VIVE. This VR device is used for defining the user position and orientation inside the virtual environment. Based on the current situation in the virtual environment, inputs are provided to the user using aural commands and haptics devices. To assess the mental workload, physiological sensors, from TEA Captiv T-Sens, are used. Among them, are an electrocardiogram sensor (ECG), to gather heart-rate and heart-rate variance data, and a galvanic skin response sensor (GSR), to collect skin conductance. Besides these sensors, the users are also expected to answer mental workload assessment tests and situation awareness questionnaires.
Among the proposed method's expected benefits are the flexibility and agility to create different scenarios, and also the possibility to test all of them in the same physical room. The method could not only speed the design of new solutions but also improve the overall quality of the products and verify the need of a BVI user in the development team of an assistive product.
% \end{abstract}

% Abstract
\begin{englishabstract}
\noindent
%% Abstract (Resumo em inglês)

Society has developed technology to create autonomous vehicles and to connect different devices and machinery to exchange data and optimize production efficiency.  With this technology, soon, it will be possible to achieve better methods to guide blind and visually impaired (BVI) users in their daily activities. We believe that the available products in the market have several limitations and do not satisfy BVI users and that one of the reasons behind this problem is that they are not members of the development team or are not consulted by these.

The lack of an efficient solution for BVI users' navigation became even more significant with the SARS-CoV2 pandemic, in which people had to avoid contact with one another and not touch another surface.

The purpose of this master's thesis is to use virtual reality (VR) to test and evaluate different designs of BVI products. Also to verify if BVI and non-BVI users have the same mental demand and situation awareness when using assistive products. The idea is to use VR as a testing ground where a BVI user can try different assistive solutions in different scenarios. By doing so, the user becomes part of the product design and evaluation, resulting in better and more user-friendly products. The proposed method includes not only the setup of the virtual environment but also the use of physiological sensors and subjective tests to assess the mental workload and situational awareness in different situations.

To illustrate the proposed method, a case study of navigation of BVI users inside a medical clinic is performed. This case study is chosen due to the current undergoing SARS-CoV-2 pandemic and the impact on BVI people, so the simulated clinic is also applying COVID health protocols.

The scenes were made using Unity3D, a widely used development platform for virtual reality applications. The VR device was the Tobii Eye Tracking VR, a head-mounted display for virtual reality developed using the HTC VIVE. This VR device is used for defining the user position and orientation inside the virtual environment. Based on the current situation in the virtual environment, inputs are provided to the user using aural commands and haptics devices. To assess the mental workload, physiological sensors, from TEA Captiv T-Sens, are used. Among them, are an electrocardiogram sensor (ECG), to gather heart-rate and heart-rate variance data, and a galvanic skin response sensor (GSR), to collect skin conductance. Besides these sensors, the users are also expected to answer mental workload assessment tests and situation awareness questionnaires.

Among the proposed method's expected benefits are the flexibility and agility to create different scenarios, and also the possibility to test all of them in the same physical room. The method could not only speed the design of new solutions but also improve the overall quality of the products and verify the need of a BVI user in the development team of an assistive product.
\end{englishabstract}

% Lista de figuras
\listoffigures %opcional

% Lista de tabelas
\listoftables %opcional

% Lista de abreviaturas
\listofabbreviations
%% Lista de abreviaturas

\begin{longtable}{ll}

AR & Augmented Reality \\
AV & Augmented Virtuality \\
BVI & Blind and Visually Impaired \\
CCM & Competence Center of Manufacturing \\
DLR & German Aerospace Center (\textit{Deutsch Zentrum für Luft- Raumfahrt}) \\
ECG & Electrocardiogram \\
EF & Effort \\
EDA & Electrodermal Activity \\
FR & Frustation \\
GSR & Galvanic Skin Response \\
HR & Heart Rate \\
HRV & Heart Rate Variance \\
IEA & International Ergonomics Association \\
MR & Mixed Reality \\
MD & Mental Demand \\
MWL & Mental Workload \\
NASA-TLX & NASA Task Load Index \\
PCB & Printed Circuit Board \\
PD & Physical Demand \\
PE & Performance \\
RE & Real Environment \\
SA & Situation Awareness \\
SAGAT & Situation Awareness Global Assessment Technique \\
SWAT & Subjective Workload Assessment Technique \\
TD & Temporal Demand \\
UN & United Nations \\
VE & Virtual Environment \\
VR & Virtual Reality \\
WHO & World Health Organization \\
XR & Extended Reality \\


\end{longtable}

 %opcional

% Lista de simbolos
% \listofsymbol
% \input{Chap0/listasimbolos} %opcional

% Sumario
\tableofcontents

\mainmatter
% Os capitulos comecam aqui

\chapter{Introduction}
\label{ch:introducao}
%\section{Motivation}

% Motivação: contextualização do problema ou da necessidade que motivou o trabalho. Essa seção deve preparar o leitor para a <Pergunta da Pesquisa>

According to the World Health Organisation (WHO), there are at least 2.2 billion people with some visual impairment degree \cite{world2019world}. Among them, 43,3 million are classified as blind and 295 million have moderate or severe vision impairment. In order to be fully integrated into our society, they rely on assistive devices, such as canes, braille speakers, among others \cite{bourne2021trends}. 

Although a range of products has already been proposed, incorporating different features, they do not entirely fulfill their aim. Among the problems, of the solutions available in the market, are the lack of practicality and portability, invasive and requiring too much effort to learn \cite{lozano2009electrotactile}.

The difficulty of using or learn how to use a device could be avoided if concepts from Human Factors, or Ergonomics, were analysed during the product’s development, using appropriate methods. The early application of these methods and tests could be a gamechanger for the success of the product's user experience \cite{wolf2019towards}.

Motivated by the dissatisfaction of blind people with the currently available products, this dissertation starts from the hypothesis that a human-factors-centred design of assistive devices for blind and visually impaired people (BVIs) requires the involvement of BVIs in the design process in order to evaluate the product under design. The user has to test the product under development to provide feedback for the design team to improve the product.

In order to approach this problem, this work proposes using virtual reality (VR) as a tool for creating virtual environments, where proof of concepts or prototypes of assistive devices could be tested by BVIs. VR can be used to create specific, immersive and interactive situations that could help the user to learn and train \cite{farrell2018learning}, and the the developers to create more user-friendly products.

In a virtual environment, as long as the BVI is wearing a locating system, s/he can navigate the environment. Any information about the scenario, such as the position of objects and their distances to the user, is known and could be extracted from the virtual platform. As a consequence the designer can test different ways of translating this information into inputs before actually implementing a prototype of the assistive device, providing a flexible, safe and easy way to have it evaluated by different users.

As a second motivation, this dissertation considers the COVID-19 pandemic scenario that dominated the world during the last two years. In order to try to slow the rate at which the virus spread, WHO recommended strategies such as wearing face masks, washing hands regularly, social distancing, and avoiding touching surfaces that have not been disinfected \cite{who_2020}. The recommendations bring additional difficulties for BVI people as the touch is one of the senses they rely on to compensate for the visual impairment. The BVI depends on others to do daily activities \cite{jondani2021strategies}. The development of solutions that can guide a BVI in an environment respecting social distance and other recommendations is also considered in this work.

\section{Objectives}
\label{sec:objetivos}

    

 This dissertation's main goal is the use of virtual reality as a tool for evaluating proofs of concept of assistive devices for blind and visually impaired people from a human-factors perspective. The purpose is to provide a flexible and easily configured way of testing different concepts of assistive devices in order to support an agile and user-centered development.

 This goal is related to the following research questions, which are investigated in this work:

 \begin{itemize}
    \item Is it possible to evaluate and compare concepts of assistive devices from a human factors perspective in a virtual environment? What are the main limitations of the use of a virtual reality environment? \label{itm:obj_first}
    \item Do non-BVI users, when deprived of their vision, similarly evaluate assistive devices as BVI users? \label{itm:obj_second}
\end{itemize}
  
 To investigate the main goal and answer the research questions, the following specific objectives are defined:
 
 \begin{itemize}
     \item Select a scenario for testing assistive devices and develop it in a virtual environment; \label{itm:subobj_first}
     
     This is done during the Steps 2 and 3 of the proposed approach using Unity3D.

     \item Develop three concepts of assistive devices that use different senses to provide input to the BVI; \label{itm:subobj_second}
     
     The step 2 and 5 define and develop those assistive devices

     \item Propose a set of techniques for BVI to evaluate assistive devices from human factors perspective; \label{itm:subobj_third}
     
    They are specified in Step 2 and the techniques and tools used to their assessment will be defined in step 4

     \item Design and execute an experiment to evaluate the concepts of assistive devices in the virtual environment using the proposed  methods. \label{itm:subobj_forth}
    
    Both design and execution are concluded during step 6

 \end{itemize}

 A summary of these topics is displayed in Figure \ref{fig:intro_diagram}.     

 \begin{figure}[!htb]
    \centering
    %\tikzstyle{every node}=[font=\Large]
    
    %\tikzstyle{task} = [rectangle, rounded corners, minimum width=4cm, minimum height=1.5cm,text centered, draw=black, fill=ccmWhite, text width=3cm]
    \tikzstyle{phase}   = [rectangle, minimum width=2cm, minimum height=1.5cm,text centered, text width=5.0cm]

    \tikzstyle{probl}   = [rectangle, rounded corners, minimum width=2cm, minimum height=3cm,   text centered, draw=black, fill=ccmWhite, text width=2.5cm]
    
    \tikzstyle{hypot}   = [rectangle, rounded corners, minimum width=2cm, minimum height=3cm,   text centered, draw=black, fill=ccmWhite, text width=3cm]
    \tikzstyle{task_h}  = [rectangle, rounded corners, minimum width=2cm, minimum height=1.5cm, text centered, draw=black, fill=ccmWhite, text width=2.5cm]
    
    \tikzstyle{prop}    = [rectangle, rounded corners, minimum width=2cm, minimum height=3cm,   text centered, draw=black, fill=ccmWhite, text width=3cm]
    \tikzstyle{task_p}  = [rectangle, rounded corners, minimum width=2cm, minimum height=1.5cm, text centered, draw=black, fill=ccmWhite, text width=2.5cm]
    
    \tikzstyle{spec}    = [rectangle, rounded corners, minimum width=2cm, minimum height=3cm,   text centered, draw=black, fill=ccmWhite, text width=3cm]
    \tikzstyle{task_s}  = [rectangle, rounded corners, minimum width=2cm, minimum height=1.5cm, text centered, draw=black, fill=ccmWhite, text width=2.5cm]
    
    \tikzstyle{--gray} = [ccmLGray, dashed, dash pattern=on 1cm off 1cm , rounded corners, line width = 1.5mm]
    \tikzstyle{--red} = [ccmRed, rounded corners, line width = 2mm]
    \tikzstyle{--blue} = [ccmDBlue, rounded corners, line width = 2mm]
    \tikzstyle{--black} = [black, rounded corners, line width = 2mm]
    
    \tikzstyle{arrow_blue} = [ccmDBlue, rounded corners, line width = 1.25mm, ->]
    \tikzstyle{arrow_black} = [black, rounded corners, line width = 1.25mm, ->]
    \tikzstyle{d--blue} = [ccmDBlue, dashed, dash pattern=on 1.0cm off 0.65cm, rounded corners, line width = 2mm]
    \tikzstyle{arrow_--_blue} = [ccmDBlue, dashed, dash pattern=on 1.0cm off 0.65cm, rounded corners, line width = 2mm, ->]
    \tikzstyle{arrow_red} = [ccmRed, rounded corners, line width = 2mm, ->]
    
    %\resizebox{\linewidth}{!}{
    \begin{tikzpicture}[node distance = 1cm]
        \centering
        \node (Problem) [phase] {Problem:};
        \node (problem) [probl, below of = Problem, yshift = -1.25cm] {Devices do not please the BVI community};
        
        \node (a1) [right of = Problem, above of = Problem, xshift = 0.65cm] {};
        \node (a2) [below of = a1, yshift = -10.75cm] {};
        \draw [--gray] (a1) to (a2);
        
        \node (Hypothesis) [phase] [phase, right of = Problem, xshift = 2.5cm] {Hypothesis:};
        \node (hypothesis) [hypot, right of = problem, xshift = 2.5cm] {There is no BVI user inside the device's design team};
        \node (hypothesis_1) [task_h, below of = hypothesis, yshift = -1.75cm, xshift = 0.25cm] {Make the BVI user part of the design team};
        \node (hypothesis_2) [task_h, below of = hypothesis_1, yshift = -1.75cm] {non-BVI users do not evaluate assistive devices as BVI users};
        
        \draw[arrow_blue] (problem.east) to (hypothesis.west);
        \draw[arrow_black] (hypothesis.south west) to ++(0,-1.25cm) to (hypothesis_1.west);
        \draw[arrow_black] (hypothesis.south west) to ++(0,-4.0cm) to (hypothesis_2.west);
        
        \node (b1) [right of = Hypothesis, above of = Hypothesis, xshift = 0.9cm] {};
        \node (b2) [below of = b1, yshift = -11.0cm] {};
        \draw [--gray] (b1) to (b2);
        
        \node (Proposal) [phase] [phase, right of = Hypothesis, xshift = 2.75cm] {Proposal:};
        \node (proposal) [prop, right of = hypothesis, xshift = 2.75cm, yshift = 0cm] {Create a method to test products in the design process's early stages};
        \node (proposal_1) [task_p, below of = proposal, xshift = 0.25cm, yshift = -1.75cm] {Consult BVI users about assitive devices};
        \node (proposal_2) [task_p, below of = proposal_1, xshift = 0cm, yshift = -1.25cm] {Invite them to test the devices};
        
        \node (c1) [right of = Proposal, above of = Proposal, xshift = 0.9cm] {};
        \node (c2) [below of = c1, yshift = -11.0cm] {};
        \draw [--gray] (c1) to (c2);
        
        \draw[arrow_blue] (hypothesis.east) to (proposal.west);
        \draw[arrow_black] (proposal.south west) to ++(0,-1.25cm) to (proposal_1.west);
        \draw[arrow_black] (proposal.south west) to ++(0,-3.45cm) to (proposal_2.west);
        
        \node (Specific) [phase] [phase, right of = Proposal, xshift = 2.75cm] {Specific \\ Objectives:};
        \node (specific) [spec, right of = proposal, xshift = 2.75cm] {Use VR to create a virtual testing ground};
        \node (develop_ve) [task_s, below of = specific, yshift = -1.5cm, xshift = 0.25cm] {Develop the VE};
        \node (develop_devices) [task_s, below of = develop_ve, yshift = -0.75cm] {Develop the devices};
        \node (develop_evaluation) [task_s, below of = develop_devices, yshift = -0.75cm] {Develop the evaluation methods};
        \node (develop_experiment) [task_s, below of = develop_evaluation, yshift = -0.75cm] {Develop the experiment};
        
        \draw[arrow_blue] (proposal.east) to (specific.west);
        \draw[arrow_black] (specific.south west) to ++(0,-1.0cm) to (develop_ve.west);
        \draw[arrow_black] (specific.south west) to ++(0,-2.75cm) to (develop_devices.west);
        \draw[arrow_black] (specific.south west) to ++(0,-4.5cm) to (develop_evaluation.west);
        \draw[arrow_black] (specific.south west) to ++(0,-6.25cm) to (develop_experiment.west);
    
    \end{tikzpicture}
    %}
    
    \centering
    \caption{Problem, hypothesis, proposal and objectives of this thesis.}
    \label{fig:intro_diagram}
\end{figure}

\section{Resources and methods} 

This work adopts an experimental approach to evaluate the proposal of this dissertation and to investigate the questions stated in Section \ref{sec:objetivos}. 
The work is organized in the following steps, illustrated in Figure \ref{fig:steps_work}:

\begin{enumerate}[leftmargin = 6em, label = Step \arabic* -- ]
    \item Literature review 
    
    It is composed of two parts. The first is to review the fundamental concepts related to the topics covered in this work: human factors and virtual reality. The second part aims at contextualizing the dissertation’s proposal. It reviews recently published works on the development and evaluation of assistive devices for BVI people.
    
    \item Specification of examples of the virtual environment, the evaluated human factors and assistive devices.

    This step consists of specifying one example of a virtual environment, the proposed evaluated human factors and a few examples of assistive devices to test the proposed approach of using virtual reality for evaluating purposes. Considering the above-mentioned motivation related to the covid-19 pandemic, the chosen virtual environment is the reception of a health clinic. The analysed human factors are mental workload, situation awareness. The satifisfaction of the users with the devices are also evaluated. The assistive devices used as examples are: an audio system, a haptic belt and a virtual cane, which could be used as stand-alone devices or combined.
    
    \item Development of the specified virtual environment.
    
    The virtual environment of a health clinic reception is developed in the Unit3D environment. The HTC VIVE VR Head Mounted Device (HMD) is used as a localizing system to define the user's position inside the virtual environment.

    \item Development of the assessment techniques and tools.
    
    The techniques used to assess the Mental Workload are the number of collisions, as a performance evaluation, the heartrate frequency, heartrate variance and skin conductance, as a physiological measurement, and the NASA-TLX (National Aeronautics and Space Administration Task Load Index) questionnaire, as a subjective measurement. For Situation awareness, the SAGAT (Situation Awareness Global Assessment Technique) is assessed. For the satisfaction of the users, a questionnaire for each device was created with the help of the BVI consultants.
    
    \item Development of proofs of concept of the specified assistive devices
    
    The three examples of devices are developed using low-cost and available laboratory equipment. The audio guide is developed using the audio system of HTV VIVE HMD, while the virtual cane is developed using the HTC VIVE VR hand controller. Finally, the haptic belt is developed using an ESP32 microcontroller, eight vibrating motors 1027 and 3D printed pieces.
    
    \item Design and execution of the experiment.
    
    The proposed experiment is based on the best practices and principles of the Design of Experiment (DoE) discipline.
    %The following techniques and tools are used for evaluating human factors:

    %\begin{enumerate}[label = \alph*)]
    %    \item Questionnaires adapted from the literature, such as NASA-TLX and SAGAT (Situation Awareness Global Assessment Technique), or explicitly proposed for this work;
    %    \item Physiological sensors, such as GSR (Galvanic Skin Response) and ECG, to capture the body's response.
    %\end{enumerate}
    
    \item Analysis of results
    
    The results of the experiment are graphically and statistically analysed to estimate the user's mental workload and situation awareness. In the statistical analysis, the outputs are verified for normality distribution, then pairwise compared using the Student's T-Test, and their variance inside the group is verified using an ANOVA. Finally, when needed, a Fisher's Least Squared Difference Test is done to verify similarities between pairs.

\end{enumerate}

\begin{figure}[!htb]
    \centering
    %\tikzstyle{every node}=[font=\large]

    \tikzstyle{start} = [rectangle, rounded corners, minimum width=3cm, minimum height=1.0cm,text centered, draw=black, fill=white!30, text width=3cm]
    \tikzstyle{steps} = [rectangle, minimum width=3cm, minimum height=1.5cm, text centered, draw=black, fill=white!30, text width=6.50cm]
    \tikzstyle{paral_steps} = [rectangle, minimum width=3cm, minimum height=1.5cm, text centered, draw=black, fill=white!30, text width=3.25cm]
    
    \tikzstyle{arrow} = [ccmDBlue, rounded corners, line width = 1mm, ->]
    
    \begin{tikzpicture}[node distance = 3cm]
        \centering
        \node (start) [start] {Start};

        \node (review) [steps, below of=start, yshift = 0.5cm] {Step 1 - Literature Review};
        \draw [arrow] (start) to node[midway,right]{} (review);

        \node (examples) [steps, below of=review, yshift = 0.5cm] {Step 2 - Specification of examples};
        \draw [arrow] (review) to node[midway,right]{} (examples);
        
        \node (ve) [paral_steps, below of=examples, left of = examples, xshift = -1cm] {Step 3 - Development of the VE};
        \draw [arrow] (examples) to ++(0,-1.5cm) to ++(-4.0cm,0) to node[midway,right]{} (ve.north);
        
        \node (assess) [paral_steps, below of=examples] {Step 4 - Development of the assessment techniques and tools};
        \draw [arrow] (examples) to node[midway,right]{} (assess.north);

        \node (devices) [paral_steps, below of=examples, right of = examples, xshift = 1cm] {Step 5 - Development of the assistive devices};
        \draw [arrow] (examples) to ++(0,-1.5cm) to ++(4.0cm,0) to node[midway,right]{} (devices.north);

        \node (experiment) [steps, below of=devices, left of = devices, xshift = -1cm] {Step 6 - Design and execution of the experiment};
        \draw [arrow] (ve) to ++(0,-1.5cm) to ++(4.0cm,0) to node[midway,right]{} (experiment.north);
        \draw [arrow] (assess) to node[midway,right]{} (experiment.north);
        \draw [arrow] (devices) to ++(0,-1.5cm) to ++(-4.0cm,0) to node[midway,right]{} (experiment.north);

        \node (analysis) [steps, below of=experiment, yshift = 0.5cm] {Step 7 – Analysis of results};
        \draw [arrow] (experiment) to node[midway,right]{} (analysis);
    \end{tikzpicture}
        \centering
        \caption{Steps of this work}
        \label{fig:steps_work}
\end{figure}
\FloatBarrier

\section{Research boundaries}

% Delimitação da pesquisa: é o recorte do seu trabalho

The concepts of assistive devices presented as part of this work are used only as examples for investigating the research questions presented in Section \ref{sec:objetivos}. The challenges related to their full development up to high Technology Readiness Levels (TRLs), as well as their feasibility as commercial products, are out of the scope of this work.

% Estrutura do texto

\section{Structure of the text}

This dissertation is organized into seven additional chapters as follows.

Chapter \ref{ch:fundamentacao} introduces the concepts and techniques that are used in this work. It starts with a review of human factors, emphasizing mental workload and situational awareness, and introduces some human factors' evaluation tools and techniques. Then, it presents the definitions of virtual reality (VR) and Extended Reality (XR) and, to conclude, discusses the concept of co-design.

Chapter \ref{ch:revisao} is dedicated to the state of the art. It brings a review of the literature and discusses published research works that are related to this dissertation. It covers the proposal and evaluation of BVI devices with emphasis on human factors analysis or virtual reality.

Chapter \ref{ch:metodologia} details the proposal of this dissertation describing how virtual reality could be used to integrate BVI users into the design process of assistive design. It illustrates the proposed method by applying it to evaluate three different assistive devices (audio guide, virtual cane and haptic belt), as well as their mixed-use, in the environment of a hospital reception. 

Chapter \ref{ch:resultados} describes the experiment designed to evaluate the dissertation's proposal and analyses the results in order to investigate the research questions of Section \ref{sec:objetivos}

Finally, Chapter \ref{ch:conclusao} summarizes the main conclusions of this work and discusses future work.


\chapter{Theoretical Foundation}
\label{ch:fundamentacao}
% CAPÍTULO 2: FUNDAMENTAÇÃO TEÓRICA
%   DEVE SER UM CAPÍTULO ENXUTO (LEAN)
%1. Elabore um parágrafo que introduz o capítulo: Este capítulo apresenta (descreva o objetivo do capítulo...). É constituído de N seções a saber...
%2. Desenvolvimento do capítulo: os conteúdos devem ser somente os necessários para o leitor entender a sua contribuição.
%3. Trabalhos relacionados: a última seção do Capítulo 2 deve apresentar o posicionamento da sua contribuição em relação à literatura. É um detalhamento da Justificativa apresentada na Introdução.
%4. Elabore um parágrafo que conclui o capítulo e introduz o capítulo seguinte.

The proposal of this work combines the concepts of co-design and extended reality (XR) with techniques from the area of Human Factors. 

In order to facilitate the understanding of this dissertation, this chapter introduces these concepts and techniques. It starts with the definition of Human Factors, also known as Ergonomics, and describes mental workload and situational awareness, as well as the corresponding assessment methods that are used in this work. 


\section{Human Factor or Ergonomics}
\label{sec:human_factors}

    Studies in the area of Human Factors started during the Second World War, motivated by performance shortfalls and failures related to the operation of equipments used by humans. The studies showed that these problems could diminish when, other than engineering, psychology and physiology were also considered when designing systems that would be handled by human beings \cite{sandom2004human}.

This study area was named "Human Factors" in the United States and "Ergonomics" in Europe. Despite this difference in the names, today they are considered the same field of study. The International Ergonomics Association (IEA) defines Human Factors, therefore Ergonomics, as the following:

\begin{quote}
    \textit{"Ergonomics (or human factors) is the scientific discipline concerned with the understanding of interactions among humans and other elements of a system, and the profession that applies theory, principles, data and methods to design in order to optimize human well-being and overall system performance. Human Factors professionals contribute to the design and evaluation of tasks, jobs, products, environments and systems in order to make them compatible with the needs, abilities and limitations of people"} \cite{karwowski2012discipline}.
\end{quote}

This definition shows that humans and their interaction with systems and devices should be considered during the design process \cite{sandom2004human, sanders1998human, dul2003ergonomics}. This need resulted in the proposal of an ISO Standard: BS EN ISO 13407 "Human-centred design processes for interactive systems". It is essencial to highlight that human-centered design does not mean that the product is designed specifically for an individual. The design has to be suited to everyone, i.e., anyone that may interact with the system \cite{dul2003ergonomics}.

The interaction between humans and machines can be abstracted as illustrated in Figure \ref{fig:human_machine_representaion}. The machine receives inputs from its environment and provides information to the human operator through displays and other monitoring devices. The operator perceives the available information, process it and decides on his/her control actions. Based on the environment's inputs and operator's commands, the machine defines its outputs to the environment.

\begin{figure}[!htb]
    \centering

    \tikzstyle{arrow} = [rounded corners, line width = 1mm, bend left = 15, ->]
    
    \resizebox{0.85\width}{!}{
    \begin{tikzpicture}[node distance=1cm]
        
        \node (information) {\includegraphics[width=.15\textwidth]{Fundamentação/Fatores Humanos/thinking.png}} 
        node(t_information)[below of = information,yshift=-0.75cm] {Information}
        node(t_information2)[below of = t_information,yshift=0.25cm] {processing};
        
        \node (controlling) [right of=information, xshift=5cm, yshift=-3cm] {\includegraphics[width=.15\textwidth]{Fundamentação/Fatores Humanos/slider.png}}
        node(t_controlling)[below of = controlling,yshift = -0.75cm] {Controlling};
        
        \node (controls) [below of=controlling, yshift=-5cm,] {\includegraphics[width=.15\textwidth,angle=90,origin=c]{Fundamentação/Fatores Humanos/control.png}} 
        node(t_controls) [below of = controls, yshift = -0.75cm]{Controls};
        
        \node (machine) [left of=controls, xshift=-5cm, yshift=-2cm] {\includegraphics[width=.15\textwidth]{Fundamentação/Fatores Humanos/machine.png}} 
        node(t_machine) [below of = machine, yshift = -0.75cm]{Operation};
        
        \node (display) [left of=machine, xshift=-5cm, yshift=2cm] {\includegraphics[width=.15\textwidth]{Fundamentação/Fatores Humanos/monitor.png}} 
        node(t_display) [below of = display, yshift = -0.75cm]{Display};
        
        \node (senses) [left of=information, xshift=-5cm, yshift=-2cm,] {\input{Fundamentação/Fatores Humanos/senses}} 
        node(t_senses) [below of = senses, yshift = -1.25cm]{Senses};
        
        \node (human) [below of=information, yshift=-4.75cm] {\Large{Human}};
        \node (human) [above of=machine, yshift=3.25cm] {\Large{Machine}};
        \node (human) [above of=information, yshift=1cm] {\Large{Work Environment}};
        
        \node (left_point) [left of=display, xshift=-2, yshift=2.75cm] {};
        \node (right_point) [right of=left_point, xshift=14cm] {};
        
        \node (input) [left of=machine, xshift=-5cm, yshift=-2cm] {\Large{Input}};
        \node (output) [right of=machine, xshift=5cm, yshift=-2cm] {\Large{Output}};
        
    
        \draw [arrow] (information.east) to (controlling.north west);
        \draw [arrow] (t_controlling.south) to (controls.north);
        \draw [arrow] (controls.west) to (machine.east);
        \draw [arrow] (machine.west) to (display.east);
        \draw [arrow] (display.north) to (t_senses.south);
        \draw [arrow] (senses.east) to (information.west);
        \draw [arrow] (input.east) -- (machine);
        \draw [arrow] (machine) -- (output.west);
        \draw [dashed,gray] (left_point) to (right_point);
        
        \draw (-8,-14) rectangle(8cm,1.5cm);
        
    \end{tikzpicture}
    }
    \caption{Human-Machine system representation. Adapted from \cite{sanders1998human}.}
    \label{fig:human_machine_representaion}
\end{figure}

Humans handle devices, machines and equipment during their daily activities. All of these manipulations are susceptible to accidents or failures that can happen because of the interaction between operator, equipment and environment. Each interface with the operator can be a factor. For example:

\begin{itemize}
    \item The operator's body position during an activity: the position can impact the operator's comfort and concentration throughout the activity, therefore, impacting the success rate or the chance of some accident happening  \cite{sanders1998human}.
    
    \item The environment's lighting: illumination can make details more noticeable without provoking discomfort or distraction and even increase productivity  \cite{sanders1998human}.
    
    \item The information displayed and manipulation of the device: the way a information is displayed on a screen, figure or text impacts how efficiently it will be understood by the operator. If this takes too long, it can draw the operator's attention for too long and compromise his/her reaction time.
    
\end{itemize}

Among the various human factors related to human-machine interaction, this work considers mental workload and situation awareness, which are explained in detail in the following sections.

\section{Mental Workload (MWL)}
\label{sec:mental_workload}

    
Mental workload (MWL) is one of the main concepts studied in Human Factors  and is not a familiar concept to the most people \cite{stanton2004handbook}. A good way to explain it is with a analogy with physical workload \cite{stanton2004handbook}. When a athlete must lift a dumbbell (one of those gym's weights bars). The strength's demand from the athlete will be proportional with the the dumbbell's mass he/she is lifting. If the dumbbell is lighter than the athlete's capability, then it will be easy enough for him/her to lift it. So if the athlete is strong enough to carry the dumbbell, he/she will not feel a physical demand bigger than his/her capabilities. So the physical workload of this activity is properly fitted for this athlete.

If the dumbbell is heavier than the athlete can lift then two things can happen:
\begin{itemize}
    \item Or the athlete adapts to lift that dumbbell using tools (adjust the strategy)
    \item Or the athlete will not be able lift completely the dumbbell (performance degrades)
\end{itemize}

This is a scenario that represent an user, or operator, executing a task that is not fitted for their capabilities.

It is the same with MWL. Each human being has a finite mental capacity and can only use it with a limited number of tasks at the same time. If the sum of these mental demands are higher than the user's capacity, the user will need to adapt in order to finish those task, otherwise he/she will compromise the overall performance of those tasks.

Although, if the workload is too low, the same operator may get bored and easily distracted and so could also fail or not process the task's information.  

It's important to say that MWL is unique within each individual and is influenced by his/her perception of the task`s workspace but is also impacted by other factors outside the task itself and more related to the operator (like it's skill, age, education, training) or to the environment (like noise, heat and toxicity) \cite{cain2007review, fallahi2016effects, cardoso2012evaluation}.

MWL is not a quantitative resource or something that one can directly measure, but is has methods to infer it. The Figure \ref{fig:mwl_overview} has an overview of MWL and its measureament methods.
        
    \begin{figure}[!htb]
    \centering

    \tikzstyle{arrow} = [rounded corners, line width = 1mm, ->]
    
    \resizebox{0.85\width}{!}{
    \begin{tikzpicture}[node distance=1cm]
        
        \node (mwl) {\includegraphics[width=.15\textwidth]{Fundamentação/Carga Mental/mwl.png}}
        node(t_mwl)[below of = mwl,yshift=-0.75cm] {Mental}
        node(t_mwl2)[below of = t_mwl,yshift=0.5cm] {workload};
        
        \node (demand) [above of=mwl, xshift=3cm, yshift=2.5cm] {\input{Fundamentação/Carga Mental/demand}}
        node(t_demand)[below of = demand,yshift=-0.75cm] {Task}
        node(t_demand2)[below of = t_demand,yshift=0.5cm] {Demand};
        
        \node (capacity) [above of=mwl, xshift=-3cm, yshift=2.5cm] {\includegraphics[width=.15\textwidth]{Fundamentação/Carga Mental/full-battery.png}}
        node(t_capacity)[below of = capacity,yshift=-0.75cm] {Mental}
        node(t_capacity2)[below of = t_capacity,yshift=0.5cm] {Capacity};
        
        \node (tasks) [left of=mwl, xshift=-4cm, yshift=-6cm] {\includegraphics[width=.15\textwidth]{Fundamentação/Carga Mental/multitasking.png}}
        node(t_tasks)[below of = tasks,yshift = -0.75cm] {Primary and} 
        node(t_tasks2)[below of = t_tasks,yshift = 0.5cm] {secondary tasks};
        
        \node (physiological) [below of=mwl, yshift=-5cm,] {\includegraphics[width=.15\textwidth]{Fundamentação/Carga Mental/physiological.png}} 
        node(t_physiological) [below of = physiological, yshift = -0.75cm]{Physiological}
        node(t_physiological2) [below of = t_physiological, yshift = 0.5cm]{measurements};
        
        \node (subjective) [right of=mwl, xshift=4cm, yshift=-6cm] {\includegraphics[width=.15\textwidth]{Fundamentação/Carga Mental/subjective.png}}
        node(t_subjective) [below of = subjective, yshift = -0.75cm]{Subjective} 
        node(t_subjective2) [below of = t_subjective, yshift = 0.5cm]{measurements};
    
    
        \draw [arrow] (t_mwl2.south) to ++(0,-0.75) to +(-5,0) to (tasks.north);
        \draw [arrow] (t_mwl2.south) to (physiological.north);
        \draw [arrow] (t_mwl2.south) to ++(0,-0.75) to +(5,0) to (subjective.north);
        \draw [arrow, sharp corners] (capacity.east) to +(1.6,0) to (mwl.north);
        \draw [arrow, sharp corners] (demand.west) -- +(-1.35,0) to (mwl.north);
    
        
    \end{tikzpicture}
    }
    \caption{A overview of mental workload and the techniques to infer it.}
    \label{fig:mwl_overview}
\end{figure}    
    
    \subsection{Task Performance}
    \label{subsec:task_performance}
    
        If the MWL influences on the task perfomance, then it would be possible to infer it using the performance's variation of a task. Because there are cases that the user's mental capacity is too high for only one task, two tasks are designed. In these evaluations, the user is asked to maintain a good perfomance level and still try to execute both tasks. Both tasks are similar and use the same kind of skill. \cite{stanton2004handbook, sanders1998human}.
        
        For example, an experiment to assess MWL in a flight simulator that uses two tasks:
        \begin{itemize}
            \item Fly a fighter aircraft and maintain a good performance level;
            \item Mentally sum two random numbers that appear on the screen. If the numbers' sum is odd, then the pilot should press left on the keyboard, if the result is even the he/she should press right.
        \end{itemize} 
    
        If the pilot's performance at the second task is too low, it means that the demand from the first task is too high for him/her to be able to pay attention on it, than it means that the MWL at the flight was high \cite{mohanavelu2020cognitive}.
        
%    \item Physiological measures;
    \subsection{Physiological measures}
    \label{subsec:physiological_measures}
    
        There are many physiological reflexes that one can use to assess MWL. These measures are a good, unbiased method to assess MWL \cite{fallahi2016effects}, but, still, it is recommended that they are evaluated alongside other method. It is possible to extract MWL information from the heart and brain activity \cite{chakladar2020eeg, orlandi2018measuring}, skin conductance, eye movement, pupillary contraction \cite{stanton2004handbook, rodriguez2015pupillometry} This master's thesis it is used heart activity and skin conductance.
        
        \subsubsection{Heart rate and heart variability with electrocardiogram (ECG)}
        \label{subsubsec:ecg}
        
            Electrocardiogram is a recording of the heart's electrical activity. With this recording one can verify the heart's interval between heartbeats and frequency (heart rate, HR), and other statistical parameters such as the standard deviation and the mean error (heart rate variability, HRV) and these are a good way to assess MWL \cite{cain2007review}. This is a simple and non-invasive method used in many human factors' experiments \cite{mohanavelu2020cognitive, mansikka2016fighter, zhang2014detection}.
        
            The heart activity is controlled by the sympathetic and parasympathetic nervous systems. These systems are responsible to control many of the body's autonomous activities \cite{stanton2004handbook}. (DEFINIR MELHOR)
        
            During a task that has a mental demand the user's heart activity changes with MWL. The higher the MWL, higher the HR and lower the HRV. This happens because of the mechanism that controls our heart activity. These are consequences of two reactions in our system when in a mental demand situation \cite{stanton2004handbook}.:
            
            \begin{itemize}
                \item A decreased parasympathetic nervous system activity and;
                \item An increase sympathetic nervous system activity.
            \end{itemize}
    
        \subsubsection{Electrodermal response with galvanic skin reaction (GSR)}
        \label{subsubsec:gsr}
        
        One of the electrodermal activity that can happen in our skin is controlled by the the sweating and the moisture level and both can be used to reveal changes in our sympathetic system \cite{nourbakhsh2012using, shi2007galvanic}. So its origin lies solely in the sympathetic branch of the autonomic nervous system as is MWL \cite{stanton2004handbook}. EDA is being used to assess stress, emotion, arousal, mental strain and cognitive activity \cite{nourbakhsh2012using, stanton2004handbook, shi2007galvanic}m also used to evaluate the usability of HCI systems \cite{shi2007galvanic} and some are to assess the mental workload \cite{zhang2014detection, borghini2014measuring}.
    
    \subsection{Subjective measures}
    \label{subsec:subjective_measures}    

        It is discussed if one should only use subjective measures to measure MWL \cite{sanders1998human, stanton2004handbook}. They are sensitive to perceived difficulty, automation, concurrent activities and demand for multiple resources. These test can be unidimensional, that are simplier but has only a general workload score \cite{stanton2004handbook}, or multidimensional. Some example of the latter is the Subjective Workload Assessment Technique (SWAT) and the NASA Task Load Index (NASA-TLX), both multidimensional tests. SWAT treats MWL as a load defined by three dimensions: time load; mental effort load; and psychological stress. In this test the user score each of these dimensions based on a 3-point scale while NASA-TLX uses 6 different dimensions.
        
        \subsubsection{NASA-TLX}
        \label{subsec:nasa_tlx}
        
            NASA-TLX is a questionnaire created by \citeonline{hart1988development}. It is answered by an user who has just completed a task/activity that someone wish to infer its MWL. This questionnaire will assess the task's MWL felt by that user with 6 rating scales and each of these is explained, ideally, at the experiment's briefing. The Table \ref{tab:nasa_dimensions} presents each scale with a description of it.
        
            \begin{table}[htb]
                \centering
                \caption{NASA-TLX dimensions and the description of each dimension. \cite{stanton2004handbook}.}
                \label{tab:nasa_dimensions}
                    \begin{tabular}{|l|l|}

                        \hline
                       \textbf{Dimension}   & \textbf{Explanation}                                                                                                                                                   \\ \hline
                        Mental demand (MD)   & \begin{tabular}[c]{@{}l@{}}The mental and perceptive activity\\ demanded by the task (chose, decide,\\ think, calculate, search, etc.).\end{tabular}                       \\ \hline
                        Physical demand (PD) & \begin{tabular}[c]{@{}l@{}}The physical activity demanded by\\ the task (pull, lift, spin, drag, etc.).\end{tabular}                                                       \\ \hline
                        Temporal demand (TD) & \begin{tabular}[c]{@{}l@{}}The time pressure felt by the user.\\ A rating the leverages the time \\ available and the time necessary to\\ completed the task.\end{tabular} \\ \hline
                        Performance (PE)     & \begin{tabular}[c]{@{}l@{}}The user's satisfaction with it's \\ perfomance or result the task.\end{tabular}                                                                \\ \hline
                        Effort (EF)          & \begin{tabular}[c]{@{}l@{}}A rating of the effort necessary \\ to achieve that perfomance felt by\\ the user.\end{tabular}                                                 \\ \hline
                        Frustration (FR)     & \begin{tabular}[c]{@{}l@{}}A rating of stress, annoy or irritation\\ felt by the user throughout the task.\end{tabular}                                                    \\ \hline
                    \end{tabular}
            \end{table}
            
            This questionnaires evaluate only one task/activity. So if the user executed two tasks (like a primary and secondary tasks), he/she should be oriented to answer about primary task only, not a combination of both of them \cite{sanders1998human}.

        To measure mental workload, it is recommended not to chose only one measuring method, but more. MWL is multidimensional and can reflect partially or differently in each of the methods \cite{sanders1998human}.


            
    

   

    


\section{Situation Awareness (SA)}
\label{sec:situation_awareness}

    Situation awareness can be defined as “the perception of the elements within a volume of time and space (Level 1), the comprehension of their meaning (Level 2), and the projection of their status in the near future (Level 3)” as illustrated in Figure \ref{fig:sa_overview}. One example is when an air traffic controller looks at a radar display (Level 1). He/she seeks to understand the aircraft’s position and speed (Level 2) and then predict its position in the near future, 5, 10 or 15 minutes after (Level 3) \cite{sanders1998human}. Similarly, when a pilot reads the cockpit panel (Level 1), and understands their data (Level 2) then he/she can predict the next reading of that same instrument or some other status of the aircraft after a couple of minutes (Level 3).

The term “situation awareness” was first proposed for the Aeronautics domain and today is considered a key factor for designing complex and dynamic systems also from other domains, such as automotive, medical and nuclear \cite{endsley1995measurement}. It is an important factor to make sure that the user will be capable to make important decisions correctly and achieve high-performance \cite{endsley1988design, endsley2018automation}.

\begin{figure}[!htb]
    \centering
    \tikzstyle{arrow} = [rounded corners, line width = 1mm, ->]

    \resizebox{0.85\textwidth}{!}{
    \begin{tikzpicture}[node distance=3.3cm]
        \centering
        
        \node (info) [fill=white] {\includegraphics[width = 0.20\textwidth]{Fundamentação/Percepção situacional/information.png}}
        node(t_info)[below of = info,yshift=1.3cm] {\Large Information};
        
        \node (arrow) [right of=info, xshift=7.0cm, yshift = -0.5cm] {\input{Fundamentação/Percepção situacional/seta}};
        
        \node (perception) [right of=info, xshift=2.5cm, draw, line width=2mm] {\input{Fundamentação/Percepção situacional/perception}}
        node(t_perception)[below of = perception, yshift=0.9cm] {\Large 1st Level}
        node(t_perception2)[below of = t_perception, yshift=2.5cm] {\Large Perception};

        \node (comprehension) [right of=perception, xshift=0.25cm, draw, line width=2mm] {\input{Fundamentação/Percepção situacional/comprehension}}
        node(t_comprehension)[below of = comprehension, yshift=0.9cm] {\Large 2nd Level}
        node(t_comprehension2)[below of = t_comprehension, yshift=2.5cm] {\Large Comprehension};

        \node (projection) [right of=perception, xshift=3.75cm, draw, line width=2mm] {\input{Fundamentação/Percepção situacional/projection}}
        node(t_projection)[below of = projection, yshift=0.9cm] {\Large 3rd Level}
        node(t_projection2)[below of = t_projection, yshift=2.5cm] {\Large Projection};

        \node (decision) [right of=projection, xshift=5.0cm] {\includegraphics[width = 0.30\textwidth]{Fundamentação/Percepção situacional/decision.png}}
        node(t_decision)[below of = decision, yshift=0.5cm] {\Large Decision}
        node(t_decision2)[below of = t_decision, yshift=2.5cm] {\Large making};

    \end{tikzpicture}
    }
    \caption{An overview of situation awareness and the SAGAT.} %and the methods to infer it
    \label{fig:sa_overview}
\end{figure}

As it is for the mental workload, situation awareness is not a quantitative subject. The most common way to measure it is using subjective methods, among which one of the most famous is the Situation Awareness Global Assessment Technique (SAGAT). It was proposed by \cite{endsley1988design} and is based on how the information is processed inside the user’s mind. The test application is made by freezing the operator activity, usually made in a simulation environment, and, then, asking the user some questions that were previously defined based on the user’s activity. These questions should be as similar as possible to how the person thinks when reasoning about the situation, in order to avoid extra effort in understanding it \cite{stanton2004handbook}.  Although freezing the activity may sound troublesome, empirical work has shown that it doesn’t interfere with the user performance and the user memory can withstand a break as long as 5 to 6 min \cite{endsley1988design}.


\section{Extended Reality (XR)}
\label{sec:extended_reality}

    Extended reality is a broad term that refers to all different ways of combining virtual and real entities in a human-machine interface system. It is usually decomposed into four classes (augmented reality, augmented virtuality, mixed reality, and virtual reality) that differ on the level of reality and virtuality involved in the interface system. 

\citeonline{milgram1994taxonomy} organized these classes and created the concept of the “virtuality continuum”, as illustrated in Figure, as illustrated in Figure \ref{fig:virtuality_continuum}. On the extreme left, the real environment represents the cases where the user operates physical elements inside that environment. Along the path to the right, the environment starts to incorporate digital elements until it reaches the far right, where all the elements in the environment are virtual and have a digital origin \cite{nijholt2005virtuality, doolani2020review}. The first step from the "Real Environment" to "Virtual Reality" is the augmented reality.

\begin{figure}[!htb]
    \tikzstyle{arrow} = [ccmDBlue, rounded corners, line width = 2mm, ->]
    \tikzstyle{--blue} = [ccmDBlue, rounded corners, line width = 2mm]
    \tikzstyle{--black} = [rounded corners, line width = 1mm]
    
    %\tikzstyle{arrow_flow} = [ccmblue, rounded corners, line width = 2mm, ->]
    %\tikzstyle{arrow_return} = [ccmred, rounded corners, line width = 2mm, ->]
    
    \resizebox{0.80\width}{!}{
    \begin{tikzpicture}[node distance=1cm]
        \centering
    
        \node (left) {};
        
        \node (reality) [right of = left, xshift = 2cm]{\includegraphics[width=.15\textwidth]{Fundamentação/Realidade Extendida/real.png}}
        node(t_reality)[below of = reality,yshift=-0.75cm] {Real}
        node(t_reality2)[below of = t_reality,yshift=0.5cm] {Environment};
        
        \node (midLeft) [right of = reality, xshift = 1cm] {};
        
        \node (ar) [right of=reality, xshift=3cm] {\includegraphics[width=.15\textwidth]{Fundamentação/Realidade Extendida/ar.png}}
        node(t_ar)[below of = ar,yshift=-0.75cm] {Augmented}
        node(t_ar2)[below of = t_ar,yshift=0.5cm] {Reality};
        
        \node (av) [right of=ar, xshift=3cm] {\includegraphics[width=.15\textwidth]{Fundamentação/Realidade Extendida/av.png}}
        node(t_av)[below of = av,yshift=-0.75cm] {Augmented}
        node(t_av2)[below of = t_av,yshift=0.5cm] {Virtuality};
        
        \node (vr) [right of=av, xshift=3cm] {\includegraphics[width=.15\textwidth]{Fundamentação/Realidade Extendida/vr.png}}
        node(t_vr)[below of = vr,yshift = -0.75cm] {Virtual} 
        node(t_vr2)[below of = t_vr,yshift = 0.5cm] {Reality};
        
        \node (midRight) [left of = vr, xshift = -1cm] {};
        
        \node (right) [right of = vr, xshift = 2cm] {};
        
        \node (mr) [above of=ar, right of=ar, xshift=1cm, yshift=3cm] {\includegraphics[width=.15\textwidth]{Fundamentação/Realidade Extendida/mr.png}} 
        node(t_mr) [below of = mr, yshift = -0.50cm]{Mixed}
        node(t_mr2) [below of = t_mr, yshift = 0.5cm]{Reality};
        
        \draw [arrow] (reality) to (left);
        \draw [--blue] (reality) to (ar);
        \draw [--blue] (ar) to (av);
        \draw [--blue] (av) to (vr);
        \draw [arrow] (vr) to (right);
        \draw [--black] (mr.west) -- +(-2.6,0) to (midLeft);
        \draw [--black] (mr.east) -- +(2.6,0) to (midRight);
         
        
        \node (left) [left of = reality, xshift = -2cm] {};
        \node (right) [right of = vr, xshift = 2cm] {};
    
        
    \end{tikzpicture}
    }
    \centering
    \caption{The Virtuality Continuum concept. Adapted from \cite{milgram1994taxonomy}}
    \label{fig:virtuality_continuum}
\end{figure}

In the augmented reality system, the user sees some digital elements that are laid over the real environment. without making the user lose his sense of presence in the real world. These elements can be text, images, video, etc. Augmented reality can be used to assist workers in manufacturing and assembly tasks, as well as training \cite{doolani2020review, farrell2018learning, ma2007virtuality}.
    
While the augmented reality brings digital elements to the real environment, the augmented virtuality creates an environment that could only exist digitally, such as a fantasy world from games or movies. This scenario is the background of some other activity that is being done by the user in the real environment. An example is to train a pilot in a virtual environment but with an accurate mock-up of the cockpit, which provides physical buttons and inceptors for the pilot to touch and hold \cite{farshid2018go}. Another example is to play sports, such as tennis, golf or baseball, in a complete digital arena but using the actual equipment with a tracker. \citeonline{kirner2012using} also add that augmented reality has three characteristics: it combines real and virtual elements; it has real time interaction; and three-dimensional.

The mixed reality stays in between the real and virtual environments. Unlike augmented reality and augmented virtuality, in a mixed reality system the user can manipulate digital elements as if they were inside the real world \cite{doolani2020review}. One example is when a client from a furniture store uses mixed reality not only to see how the furniture fits inside his room, but he can also move it and change its color, size and shape before buying or even going to the shop.
    
On the far right of the virtuality continuum, the virtual reality is when the user is the only non-digital element, everything else is digital, immersing the user in a virtual environment, but, of course, inside the physical limits of the real environment \cite{ma2007virtuality}. If the feeling of presence inside that environment is well tailored, the user can momentarily forget about the real environment and act and react accordingly to the virtual environment \cite{farrell2018learning}. 
    
Virtual reality is a powerful tool that allows a user to be transported to a tridimensional environment that could be out of reach or that does not exist but is needed for testing or training reasons \cite{mujber2004virtual}. Inside this virtual environment, the user can walk, look around and feel as if the environment was real \cite{salah2019virtual}.

Figure \ref{fig:ar_av_mr} shows the representations of each of these Extended Reality classes.

\begin{figure}[!htb]
    \tikzstyle{arrow} = [ccmblue, rounded corners, line width = 2mm, ->]
    \tikzstyle{--blue} = [ccmblue, rounded corners, line width = 2mm]
    \tikzstyle{--black} = [rounded corners, line width = 1mm]
    
    \resizebox{0.85\width}{!}{
    \begin{tikzpicture}[node distance=1cm]
        \centering
    
        \node (ar) {\includegraphics[width=0.3\textwidth]{Fundamentação/Realidade Extendida/ar.png}}
        node(t_ar)[below of = ar,yshift=-2.0cm] {Augmented}
        node(t_ar2)[below of = t_ar,yshift=0.5cm] {Reality};
        
        \node (av) [right of=ar, xshift=5cm, yshift = 0.1cm] {\includegraphics[width=.34\textwidth]{Fundamentação/Realidade Extendida/av.png}}
        node(t_av)[below of = av,xshift = 1.0cm, yshift=-2.0cm] {Augmented}
        node(t_av2)[below of = t_av,yshift=0.5cm] {Virtuality};
        
        \node (mr) [below of=ar, yshift=-5.5cm] {\includegraphics[width=.31\textwidth]{Fundamentação/Realidade Extendida/mr.png}}
        node(t_mr)[below of = mr,yshift=-2.0cm] {Mixed}
        node(t_mr2)[below of = t_mr,yshift=0.5cm] {Reality};
        
        \node (vr) [right of=mr, xshift=5.0cm, yshift = -0.5cm] {\includegraphics[width=.25\textwidth]{Fundamentação/Realidade Extendida/vr.png}}
        node(t_vr)[below of = vr,yshift = -1.5cm] {Virtual} 
        node(t_vr2)[below of = t_vr,yshift = 0.5cm] {Reality};
        
        \node (n) [right of = ar, above of = ar , xshift = 2cm, yshift = 3.0cm] {};
        \node (l) [right of = av, below of = av, xshift = 3cm, yshift = -3.1cm] {};
        \node (s) [left of = vr, below of = vr, xshift = -2cm, yshift = -3.0cm] {};
        \node (o) [left of = mr, above of = mr, xshift = -3cm, yshift = 1.6cm] {};
        
        
        %\draw [arrow] (reality) to (left);
        %\draw [--blue] (reality) to (ar);
        %\draw [--blue] (ar) to (av);
        %\draw [--blue] (av) to (vr);
        %\draw [arrow] (vr) to (right);
        \draw [--black] (n) to (s);
        \draw [--black] (l) to (o);
         
        
        %\node (left) [left of = reality, xshift = -2cm] {};
        %\node (right) [right of = vr, xshift = 2cm] {};
    
        
    \end{tikzpicture}
    }
    \centering
    \caption{A represantion of the differences between AR, AV, MR and VR. Made by the author}
    \label{fig:ar_av_mr}
\end{figure}

Although some researchers still use this classification, it is considered outdated by others. Nowadays the shift between augmented reality and augmented virtuality is not gradual because they group different styles of interaction, have different equipments necessary to create the interface and has other technologies that at the time the virtuality continuum was conceived it was not considered \cite{cerqueira2019tangible}.

For \citeonline{kirner2012using} consider an additional class: cross-reality. It is the combination of having a virtual environment and a real environment connected with sensors and actuators creating a bidirection information feed between both environments.    
    

\section{Co-Design}
\label{sec:co_design}

    Co-design, or collaborative design, refers to a design process in which individuals of the design team have different backgrounds or bring different experiences, which can be essencial for the product under design. It is based on good communication and information sharing among the team \cite{chiu2002organizational}.

\citeonline{kleinsmann2006understanding} provides the following definition:.

\begin{quote}
    \textit{"Collaborative design is the process in which actors from different disciplines share their knowledge about both the design process and the design content. They do that to create a shared understanding of both aspects, to be able to integrate and explore their knowledge and to achieve the larger common objective: the new product to be designed."} \cite{kleinsmann2006understanding}.
\end{quote}

This definition emphasizes two critical aspects of co-design: knowledge sharing and integration. According to \citeonline{kleinsmann2006understanding} knowledge is the data after the receiver's understanding or translating process, in a state that is possible to record or register, so that the person can remember and use it later. During the collaborative design, ideas, facts or concepts are exchanged between the actors. This exchange is a fundamental part of the co-design process since it is responsible for the growth of each individual's knowledge. Once the knowledge is shared among the actors, they can use it when performing their tasks, resulting in knowledge integration \cite{kleinsmann2006understanding}.

\chapter{Literature review}
\label{ch:revisao}
This chapter discusses a set of selected works from the literature that are related to different aspects of this work. Their selection was performed in the Scopus and Web of Science databases, using keywords such as “human factors”, “virtual reality” and “blindness”. From an initial set of 344 papers, a set of seven were selected as more relevant to this work and are detailed in the next sections.

\section{Virtual reality for BVI users}
\label{sec:vr_without_vision}
Motivated by the popularization of virtual reality technology, \citeonline{siu2020virtual} developed a white cane to be used by BVI users in a virtual environment. Their purpose was to make virtual reality applications available for BVI users. 

The traditional white cane transmits three sources of information to the user: detection of obstacles, surface topography and foot placement preview. In their work, these sources of information were transmitted through sounds or haptics \cite{siu2020virtual}, which would be defined based on the cane position in the virtual environment. For obstacle detection, the cane was built with a three-degree-of-freedom brake mechanism that would stop the movement when the cane hit an obstacle. A coil actuator was used to simulate surface properties. Lastly, a wave-based acoustic simulation was used to render geometry-aware sound effects in other to give the user a sense of the surroundings (echo localization).

In order to evaluate their proposal, the authors performed an experiment where the participants had to play a “scavenger hunt” using an HTC Vive system. During the experiment, each participant had two tasks: collect targets along the way (primary task) and avoid virtual obstacles and walls (secondary task). The targets appeared, one at a time, once the previous target was collected, and they emitted a sound that acted like an audio beacon for the participant. The obstacles did not emit any sound as a beacon, but the participant could detect it by the shape and the noise it emitted when in contact with the cane. The experiment was performed with 8 blind users (4 female, 4 male) from 25 to 70 years old. All of them did a training section where the virtual environment was presented. 

Among the relevant findings of \citeonline{siu2020virtual} is that not all the participants reacted the same to a particular stimulus. The vibration of the cane was considered confusing by some participants, while others were familiar with it. This difference affected the performance of the participants. The ones that had already used vibrating devices performed better. It shows that user's previous experiences can impact their performance in the virtual environment.

Another interesting observation was that, similar to what happens in the real world, it was easier for the participants to navigate in larger areas than in tight spaces. Moreover, the authors observed that the participants focused their attention on the primary task, without freely exploring the environment, which might have impacted the low time to achieve the goal and the low number of obstacle hits. 

Among the limitations pointed out by the authors is the lack of feedback possibilities for situations such as when the obstacle contacts a point along with the cane, not the tip of it, and the fact that the brake system did not stop the participant when he/she walked forwards toward a wall.

Comparing the work of \citeonline{siu2020virtual} to this work, \citeonline{siu2020virtual} were focused on providing mechanisms for a BVI user to navigate inside virtual environments. In this work, the purpose is to use the virtual environment to collect data about how the BVI user would navigate in a real environment. Another difference is in the functioning of the virtual cane, which in this work is limited to vibration, with no brake system, as the BVI user does not need to touch the environment with it. One common observation of both works is the sound importance for the BVI guidance and the need to use high-quality spatialized audio to increase the realism of the virtual environment.

\section{Feeling of presence in virtual reality}
\label{sec:emotion_presence_vr}
One of the many feelings that flourish during the use of a VR is the feeling of presence. This feeling, inside the virtuality context, is when someone feels draw into a VE and starts to occupy the VE instead of the real one \cite{cummings2016immersive}.

\citeonline{jicol2021effects} explores this feeling in its work. The authors aim to correlate the feeling of presence with one's agency (which is the self perception that the user is in control of a situation or some actions \cite{farrer2002experiencing}) and emotion, both of these in a VE context. Besides assessing this correlation, the author also did a structural equation model (SEM) based on their findings. The author did this by creating two different VE, one that would trigger happy emotions, and another that would trigger fear. For each VE there was two different variations of it, one that the user could interact with it's elements and another that it could not. So at the end, four different VE were designed as the Figures \ref{fig:happy_without} to \ref{fig:fear_with} show.

\begin{figure}[!h] 
    \begin{subfigure}{0.45\textwidth}
    \centering
    \includegraphics[width=\textwidth]{Revisao/Emotion Presence/Scene Happy No.png} 
    \caption{Without agency.} 
    \label{fig:happy_without}
  \end{subfigure}
  \hfill
  \begin{subfigure}{0.45\textwidth}
    \centering
    \includegraphics[width=\textwidth]{Revisao/Emotion Presence/Scene Happy Yes.png}
    \caption{With agency.} 
    \label{fig:happy_with}
  \end{subfigure}
\caption{Happy environment \cite{jicol2021effects}.} 
\label{fig:happy_environment}
\end{figure}

\begin{figure}[!h] 
    \begin{subfigure}{0.45\textwidth}
    \centering
    \includegraphics[width=\textwidth]{Revisao/Emotion Presence/Scene Fear No.png}
    \caption{Without agency.} 
    \label{fig:fear_without}
  \end{subfigure}
  \hfill
  \begin{subfigure}{0.45\textwidth}
    \centering
    \includegraphics[width=\textwidth]{Revisao/Emotion Presence/Scene Fear Yes.png}
    \caption{With agency.} 
    \label{fig:fear_with}
  \end{subfigure}
\caption{Fear environment \cite{jicol2021effects}.} 
\label{fig:fear_environment}
\end{figure}

This experiment had 121 participants and they were randomly assigned to one of the four VE. Participants with a neurological disease, fear of dogs, psychological or emotional issues, epilepsy or use of medical device were excluded.

The authors had three hypothesis about their experiment:
\begin{enumerate}
    \item The intensity of the dominant emotion in each VE will correlate positively with the presence
    \item Presence will be significantly higher in environments where participants have agency
    \item Agency will moderate the effect of the emotion on the presence
\end{enumerate}

The first hypothesis was confirmed. No matter if the feeling is positive (happiness) or negative (fear), the users did felt a stronger presence when the positive or negative feeling were more intense.

The second hypothesis was partially correct. In the VE that provoked fear, agency did make a difference and induced a higher feeling of presence, whilst in the VE that provoked happiness, agency did not affected the presence. The same could be said about the third hypothesis.

This is an important work for its findings about the user's presence feeling. Inside a VE, users that have a direct interaction inside it do find a bigger feeling of presence. This is important for this master's thesis experiment. It is possible that, if the participant did not feel "present" inside the VE, the gathered data could be less sensitive to the experiment's goals.

This experiment did not assess directly the feeling of presence, but the feeling of presence inside a VE with BVI users could be a suggestion for future works or even a base study.



\section{Information for BVI navigation}
\label{sec:bradley_dunlop}
Bradley and Dunlop published two works (\citeyear{bradley2002investigating,bradley2005experimental}) about how BVI navigates and how much it is similar or different to how a sighted person navigates. 

The first work of Bradley and Dunlop was published in \citeyear{bradley2002investigating} and discussed which type of information BVI uses to navigate in an environment and how it compares to sighted people. The data were collected during structured interviews where the participant had to explain how to arrive at two different locations as if they were talking to someone with the same vision condition \cite{bradley2002investigating}.

Based on the answers, the authors defined 11 categories of information: 1) directional (e.g. left/right, north/south); 2) structural (e.g. road, monument, church); 3) environmental (e.g. hill, river, tree); 4) textual-structural (e.g. name of shops, places, restaurants); 5) textual-area/street-based (e.g. name of street, neighbourhoods, squares); 6) numerical (e.g. first, second, 100m); 7) descriptive (e.g. steep, tall); 8) temporal/distance based (e.g. ”walk until you reach...” or ”before you get to”); 9) sensory (e.g. the sound of engines, the smell of bread from a bakery); 10) motion (e.g. cars passing by, doors opening); 11) social contact (e.g. asking people or using a guide dog for help) \cite{bradley2002investigating}.

As an output from the interviews, the authors provided the average number which each category was used by each group and is reproduced in Figure \ref{fig:bradley_2002}. From the results, the researchers observed that BVI participants used less text-based information than the sighted participants. However BVI participants used more words to describe a path than the sighted participants. Another essential result was that visually impaired people used, on average, 9 to 10 categories to describe a route, while sighted people used around 6 categories.

\begin{figure}[!htb]
    \centering
    \tikzstyle{barraVI} = [fill = cor1]
    \tikzstyle{barraS} = [fill = cor2]
    \tikzstyle{legenda} = [fill = white, line width = 0.25mm]
    \tikzstyle{--} = [line width = 0.25mm]
    
    \resizebox{0.8\linewidth}{!}{
    \begin{tikzpicture}[node distance=0cm]
        % Fundo do gráfico
    
        \renewcommand{\tamX}{13.875cm}
        \renewcommand{\tamY}{5.4cm}
        
        \node (origin) {};
        \node (endX) [xshift = \tamX] {};
        \node (endY) [yshift = \tamY] {};
        \node (endXY) [above of = endX, yshift = \tamY] {};
        
        %Título
        \node (titulo) [xshift = \tamX*0.5, yshift = \tamY + 1.25cm] {\textbf{Average nº of utterances used within each contextual category}}
        node [below of = titulo, yshift = -0.5cm] {\textbf{between sighted and visually impaired participants}};
        
        \draw[--] (origin.west) node[anchor = east]{\footnotesize 0} to (endX.center) node[anchor = north, xshift = -7.5cm, yshift = -2.0cm]{\textbf{Type of contextual categories}};
        \draw[--] (origin.south) to (endY.center) node[anchor = east, xshift = -1.0cm,rotate = 90]{\textbf{Average nº of utterance}};
        \draw[--] (endX.center) to (endXY.center);
        
        \foreach \r/\n in        {1/5,2/10,3/15,4/20,5/25,6/30,7/35,8/40,9/45,10/50,11/55,12/60,13/65,14/70,15/75,16/80,17/85,18/90}
        {
            \draw [--] (-0.15,0.30cm*\r) node[anchor = east]{\footnotesize \n} to (\tamX,0.30cm*\r);
        }
        
        \renewcommand{\largX}{0.25}
        \renewcommand{\altY}{0}
        \renewcommand{\distX}{0.5}
        %Direct
        \draw[barraVI] (\distX,0) node[yshift = -1.0cm, rotate = 45]{Direct} rectangle ++(\largX,5.3);
        \draw[barraS] (\distX+\largX,0) rectangle ++(\largX,1.6);
        \draw[--] (\distX+3.5*\largX,0) to ++(0,-0.2);
        
        \renewcommand{\distX}{1.75}
        %Struct
        \draw[barraVI] (\distX,0) node[yshift = -1.0cm, rotate = 45]{Struct} rectangle ++(\largX,3.6);
        \draw[barraS] (\distX+\largX,0) rectangle ++(\largX,0.5);
        \draw[--] (\distX+3.5*\largX,0) to ++(0,-0.2);
        
        \renewcommand{\distX}{3.0}
        %Struct
        \draw[barraVI] (\distX,0) node[yshift = -1.0cm, rotate = 45]{Environ} rectangle ++(\largX,0.5);
        \draw[barraS] (\distX+\largX,0) rectangle ++(\largX,0.2);
        \draw[--] (\distX+3.5*\largX,0) to ++(0,-0.2);
        
        \renewcommand{\distX}{4.25}
        %Struct
        \draw[barraVI] (\distX,0) node[yshift = -1.0cm, rotate = 45]{Text-struct} rectangle ++(\largX,0.2);
        \draw[barraS] (\distX+\largX,0) rectangle ++(\largX,0.4);
        \draw[--] (\distX+3.5*\largX,0) to ++(0,-0.2);
        
        \renewcommand{\distX}{5.5}
        %Struct
        \draw[barraVI] (\distX,0) node[yshift = -1.0cm, rotate = 45]{Text-area/st} rectangle ++(\largX,0.5);
        \draw[barraS] (\distX+\largX,0) rectangle ++(\largX,0.7);
        \draw[--] (\distX+3.5*\largX,0) to ++(0,-0.2);
        
        \renewcommand{\distX}{6.75}
        %Struct
        \draw[barraVI] (\distX,0) node[yshift = -1.0cm, rotate = 45]{Numer} rectangle ++(\largX,1.3);
        \draw[barraS] (\distX+\largX,0) rectangle ++(\largX,0.2);
        \draw[--] (\distX+3.5*\largX,0) to ++(0,-0.2);
        
        \renewcommand{\distX}{8.0}
        %Struct
        \draw[barraVI] (\distX,0) node[yshift = -1.0cm, rotate = 45]{Desc} rectangle ++(\largX, 4.2);
        \draw[barraS] (\distX+\largX,0) rectangle ++(\largX,0.45);
        \draw[--] (\distX+3.5*\largX,0) to ++(0,-0.2);
        
        \renewcommand{\distX}{9.25}
        %Struct
        \draw[barraVI] (\distX,0) node[yshift = -1.0cm, rotate = 45]{Sensory} rectangle ++(\largX,0.8);
        \draw[barraS] (\distX+\largX,0) rectangle ++(\largX,0.0);
        \draw[--] (\distX+3.5*\largX,0) to ++(0,-0.2);
        
        \renewcommand{\distX}{10.5}
        %Struct
        \draw[barraVI] (\distX,0) node[yshift = -1.0cm, rotate = 45]{Tem/Dist} rectangle ++(\largX,0.95);
        \draw[barraS] (\distX+\largX,0) rectangle ++(\largX,0.35);
        \draw[--] (\distX+3.5*\largX,0) to ++(0,-0.2);
        
        \renewcommand{\distX}{11.75}
        %Struct
        \draw[barraVI] (\distX,0) node[yshift = -1.0cm, rotate = 45]{Motion} rectangle ++(\largX,0.1);
        \draw[barraS] (\distX+\largX,0) rectangle ++(\largX,0.0);
        \draw[--] (\distX+3.5*\largX,0) to ++(0,-0.2);
        
        \renewcommand{\distX}{13.0}
        %Struct
        \draw[barraVI] (\distX,0) node[yshift = -1.0cm, rotate = 45]{Social} rectangle ++(\largX,0.2);
        \draw[barraS] (\distX+\largX,0) rectangle ++(\largX,0.0);
        \draw[--] (\distX+3.5*\largX,0) to ++(0,-0.2);
        
        %Legenda
        \draw[legenda] (\tamX-3.0cm,\tamY-1.5cm) rectangle (\tamX+1.5cm, \tamY+0.25cm);
        \draw[barraVI] (\tamX-2.5cm,\tamY-0.25cm) rectangle ++(0.25cm,0.25cm) node[anchor = west, xshift = 0.15cm, yshift = -0.15cm]{Visualy Impaired};
        \draw[barraS] (\tamX-2.5cm,\tamY-1.0cm) rectangle ++(0.25cm,0.25cm) node[anchor = west, xshift = 0.15cm, yshift = -0.15cm]{Sighted};
        
    \end{tikzpicture}
    }
    \centering
    \caption{Comparison between sighted participants with BVI participants (Adapted from \citeonline{bradley2002investigating}).}
    \label{fig:bradley_2002}
\end{figure}

Among the comments provided by BVI participants, a common one was about the limitations of available navigation options, such as white canes and guide dogs. They also emphasize that, when navigating, using different senses is essential for confirming one piece of information. 

To extend the findings of their previous work, Bradley and Dunlop designed an experiment to investigate if there is a difference between the perceived workload of BVI participants and sighted participants when they navigate using user-tailored information created with the results of the previous experiments \cite{bradley2005experimental}.

The experiment was performed with 16 participants, 8 sighted and 8 BVI, who were recruited to walk to four pre-determined landmarks in the centre of Glasgow. They followed the orientations recorded during the interviews from their previous work. For each participant, orientations for 2 of the 4 landmarks were made using sighted users' interviews, while the other 2 used data from BVI interviews. The results showed that BVI users reached landmarks significantly quicker when given the information made for that group, but still longer than sighted users. 

Another issue analysed during the experiment was the perceived workload. After each landmark, the participant was asked to complete the NASA-TLX questionnaire. The average score for each dimension of the NASA-TLX is reproduced in Figure \ref{fig:bradley_2005_participants}. As expected, it shows that BVI participants systematically have a higher workload than sighted participants. It also confirms that BVI did have a higher workload when guided by orientations provided by sighted people, as well as the sighted participants did with orientations from BVI. Another essential piece of information that stands out is the high frustration score given by the BVI users when they were guided by the orientations of sighted people.

\begin{figure}[htbp]
    \centering
    \begin{subfigure}{.49\textwidth}
        \centering
        \resizebox{\linewidth}{!}{
        \input{Revisao/Bradley Dunlop/Grafico_2005_Nasa_7}
        }
        \caption{Condition 1.}
        \label{fig:bradley_2005_nasa_participants_1}
    \end{subfigure}
    \hfill
    \begin{subfigure}{.49\textwidth}
        \centering
        \resizebox{\linewidth}{!}{
        \tikzstyle{barraVI} = [fill = cor1]
\tikzstyle{barraS} = [fill = cor2]
\tikzstyle{legenda} = [fill = white, line width = 0.25mm]
\tikzstyle{--} = [line width = 0.25mm]

%\resizebox{0.8\linewidth}{!}{
\begin{tikzpicture}[node distance=0cm]
    % Fundo do gráfico

    \renewcommand{\tamX}{12.0cm}
    \renewcommand{\tamY}{6.0cm}
    
    \node (origin) {};
    \node (endX) [xshift = \tamX] {};
    \node (endY) [yshift = \tamY] {};
    \node (endXY) [above of = endX, yshift = \tamY] {};
    
    %Título
    \node (titulo) [xshift = \tamX*0.5, yshift = \tamY + 1cm] {\textbf{\LARGE Orientation from BVI people}};
    \draw[--] (origin.west) node[anchor = east]{\Large 0} to (endX.center) node[anchor = north, xshift = -\tamX*0.5, yshift = -5.0cm]{\textbf{\LARGE Workload dimensions}};
    \draw[--] (origin.south) to (endY.center) 
    node(eixoY)[anchor = east, xshift = -2.5cm, yshift = 0.5cm, rotate = 90]{\textbf{\LARGE Average weighted score}};
    \draw[--] (endX.center) to (endXY.center);
    
    \foreach \r/\n in {1/50,2/100,3/150,4/200,5/250,6/300}
    {
        \draw [--] (-0.15,1cm*\r) node[anchor = east]{\Large \n} to (\tamX,1cm*\r);
    }
    
    \renewcommand{\largX}{0.5}
    \renewcommand{\altY}{0}
    \renewcommand{\distX}{0.5}
    
    %Mental Demand
    %\draw[barraVI] (\distX,0) node[xshift = \largX*1cm, yshift = -0.5cm]{\textbf{\LARGE MD}} rectangle ++(\largX,5.6);
    \draw[barraVI] (\distX,0) node[xshift = \largX*-1.65cm, yshift = -2.25cm]{\rotatebox{50}{\textbf{\LARGE Mental Demand}}} rectangle ++(\largX,5.6);
    \draw[barraS] (\distX+\largX,0) rectangle ++(\largX,3.1);
    \draw[--] (\distX+3*\largX,0) to ++(0,-0.2);
    
    \renewcommand{\distX}{2.5}
    %Physical Demand
    %\draw[barraVI] (\distX,0) node[xshift = \largX*1cm, yshift = -0.5cm]{\textbf{\LARGE PD}} rectangle ++(\largX,0.7);
    \draw[barraVI] (\distX,0) node[xshift = \largX*-1.75cm, yshift = -2.25cm]{\rotatebox{50}{\textbf{\LARGE Physical Demand}}} rectangle ++(\largX,0.7);
    \draw[barraS] (\distX+\largX,0) rectangle ++(\largX,0.5);
    \draw[--] (\distX+3*\largX,0) to ++(0,-0.2);
    
    \renewcommand{\distX}{4.5}
    %Temporal demand
    %\draw[barraVI] (\distX,0) node[xshift = \largX*1cm, yshift = -0.5cm]{\textbf{\LARGE TD}} rectangle ++(\largX,0.9);
    \draw[barraVI] (\distX,0) node[xshift = \largX*-2cm, yshift = -2.5cm]{\rotatebox{50}{\textbf{\LARGE Temporal Demand}}} rectangle ++(\largX,0.9);
    \draw[barraS] (\distX+\largX,0) rectangle ++(\largX,0.7);
    \draw[--] (\distX+3*\largX,0) to ++(0,-0.2);
    
    \renewcommand{\distX}{6.5}
    %Performance
    %\draw[barraVI] (\distX,0) node[xshift = \largX*1cm, yshift = -0.5cm]{\textbf{\LARGE OP}} rectangle ++(\largX,2.1);
    \draw[barraVI] (\distX,0) node[xshift = \largX*-1.25cm, yshift = -1.65cm]{\rotatebox{50}{\textbf{\LARGE Performance}}} rectangle ++(\largX,2.1);
    \draw[barraS] (\distX+\largX,0) rectangle ++(\largX,0.5);
    \draw[--] (\distX+3*\largX,0) to ++(0,-0.2);
    
    \renewcommand{\distX}{8.5}
    %Effort
    %\draw[barraVI] (\distX,0) node[xshift = \largX*1cm, yshift = -0.5cm]{\textbf{\LARGE EF}} rectangle ++(\largX,3.8);
    \draw[barraVI] (\distX,0) node[xshift = \largX*-0.0cm, yshift = -0.850cm]{\rotatebox{50}{\textbf{\LARGE Effort}}} rectangle ++(\largX,3.8);
    \draw[barraS] (\distX+\largX,0) rectangle ++(\largX,2.5);
    \draw[--] (\distX+3*\largX,0) to ++(0,-0.2);
    
    \renewcommand{\distX}{10.5}
    %Frustation
    %\draw[barraVI] (\distX,0) node[xshift = \largX*1cm, yshift = -0.5cm]{\textbf{\LARGE FR}} rectangle ++(\largX,1.5);
    \draw[barraVI] (\distX,0) node[xshift = \largX*-1.0cm, yshift = -1.35cm]{\rotatebox{50}{\textbf{\LARGE Frustation}}} rectangle ++(\largX,1.5);
    \draw[barraS] (\distX+\largX,0) rectangle ++(\largX,1.1);
    \draw[--] (\distX+3*\largX,0) to ++(0,-0.2);
    
    
    %Legenda
    \draw[legenda] (\tamX-3.0cm,\tamY-1.5cm) rectangle ++(6.5cm,2cm);
    \draw[barraVI] (\tamX-2.75cm,\tamY-0.25cm) rectangle ++(0.25cm,0.25cm) 
    node[anchor = west, xshift = 0.15cm, yshift = -0.15cm]{\LARGE BVI};
    \draw[barraS] (\tamX-2.75cm,\tamY-1.0cm) rectangle ++(0.25cm,0.25cm) 
    node[anchor = west, xshift = 0.15cm, yshift = -0.15cm]{\LARGE Sighted};
    
\end{tikzpicture}
%}
        }
        \caption{Condition 2.}
        \label{fig:bradley_2005_nasa_participants_2}
    \end{subfigure}
\caption{Comparison of the NASA-TLX between the participants (Adapted from \citeonline{bradley2005experimental}).}
\label{fig:bradley_2005_participants}
\end{figure}

The work of \citeonline{bradley2002investigating} brings some relevant information for developing this work. Firstly, it shows the differences between the way sighted and BVI people navigate, highlighting the importance of including BVI in the design process of assistive technologies. It confirms the limitations of the current solutions. It brings essential insights on what type of information to include in developing audio systems, which is one of the assistive devices evaluated in this work. Finally, it shows the importance of using different workload assessment techniques when evaluating assistive technologies.

\section{Audio navigation for BVI}
\label{sec:auditory_navigation}
Despite the several existing navigation systems for BVI users, their limitations have been pointed out in many works. \citeonline{yang2014design} explored the effect of two factors when BVIs use a standard GPS navigation system. The first factor is the amount of detail of the provided information (information completeness). The second one is the distance between the information reproduction and the object referred by that same information (broadcasting timing).

In order to evaluate the impact of these two factors, \citeonline{yang2014design} experimented with BVI users where each factor had two levels. The completeness of the information could be “complete” and “simple” and the broadcast timing could be 5m and 7m. As outputs the authors evaluated the participants' performance by their precision and time in finding a goal, and evaluated their perceived workload with NASA-TLX. 

The independent variables were analysed by a two-way ANOVA hypothesis test. They found that the precision in finding the goal was only influenced by the broadcasting timing. The time in finding the goal was influenced by both variables. The task's workload was influenced by the broadcasting timing and the interaction between it and the information completeness.

The work of \citeonline{yang2014design} shows the importance of synchronizing the information provided by the audio system with the current position of the BVI user – a point to be taken into account when developing the audio solution used in this work. However, concerning the lack of influence of information completeness, it is relevant to observe that, in a certain way, this result contradicts the conclusions of Bradley and Dunlop of \citeyear{bradley2002investigating,bradley2005experimental} and, therefore, should be considered with caution. It may be due to the difference between the two levels (complete and simple) adopted in the experiment. Finally, \citeonline{yang2014design} confirms the NASA-TLX as a feasible tool to evaluate workload in experiments with BVI participants.


\section{Comparison of assistive devices}
\label{sec:evaluation_spatial_display}
In the work of \cite{marston2006evaluation}, the author wanted to test a prototype developed in previous research on the street and in a park with a blind user. This experiment would also compare two different guidance displays, one based on haptics transmission and another based on sounds.

8 BVI participants attended the experiment, which was divided into one training set and two test sites. The first was in a busy block that had a variety of street furniture, parked bicycles and people and the participant needed to pass through 4 waypoints for a total of 244m. The second site was inside a park, with paths made of concrete, crushed gravel and paver blocks, with 7 waypoints for a total of 187m. Each participant did each route with both guidance displays.

The researchers collected the time to collect all waypoints, the errors made, the distance traveled and the percentage of the total time that the users accessed the guidance device. All participants were able to complete all routes and collect all waypoints with both devices. This shows that they were able to be guided by new sound or haptic devices. The mean time to collect all the waypoints using the sound device was lower than with the haptic device, as shown in the Figure \ref{fig:evaluation_mean_time}. This Figure shows a standardized time made based on the time that two researchers took to complete the route, both of them blindfolded and with a cane, but already had made the same route many times before and during the experiment.

%\begin{figure}[htbp]
%    \centering
%    \includegraphics[width = \linewidth]{Revisao/Evaluation Spatial Display/Evaluation mean time.png}
%    \caption{Standardize mean completion time for each subject with each device in each route \cite{marston2006evaluation}.}
%    \label{fig:evaluation_mean_time}
%\end{figure}

\begin{figure}[htbp]
    \centering    
    \tikzstyle{barraHPIRua} = [fill = cor1]
    \tikzstyle{barraSomRua} = [fill = cor2]
    \tikzstyle{barraHPIParque} = [fill = cor3]
    \tikzstyle{barraSomParque} = [fill = cor4]
    \tikzstyle{legenda} = [fill = white, line width = 0.25mm]
    \tikzstyle{--} = [line width = 0.25mm]
    
    \resizebox{\linewidth}{!}{
    \begin{tikzpicture}[node distance=0cm]
        \centering    
        % Fundo do gráfico
        \renewcommand{\tamX}{16.0cm}
        \renewcommand{\tamY}{6.0cm}
        
        \node (origin) {};
        \node (endX) [xshift = \tamX] {};
        \node (endY) [yshift = \tamY] {};
        \node (endXY) [above of = endX, yshift = \tamY] {};
        
        %Título
        \node (titulo) [xshift = \tamX*0.5, yshift = \tamY + 1cm] {};
        \draw[--] (origin.west) node[anchor = east]{ 1} to (endX.center) node[anchor = north, xshift = -\tamX*0.5, yshift = -1.0cm]{\textbf{\Large Subjects}};
        \draw[--] (origin.south) to (endY.center) 
        node(eixoY)[anchor = east, xshift = -2.5cm, yshift = 0.5cm, rotate = 90]{\textbf{\Large Relative Access Measure}}
        node[right of = eixoY, anchor = west, xshift = 1.0cm, yshift = -0.75cm, rotate = 90]{\textbf{\Large (RAM)}};
        \draw[--] (endX.center) to (endXY.center);
        
       \foreach \r/\n in {1/1.5, 2/2.0, 3/2.5, 4/3.0, 5/3.5, 6/4.0}
        {
            \draw [--] (-0.15,1cm*\r) node[anchor = east]{\n} to (\tamX,1cm*\r);
        }
        
        \renewcommand{\largX}{0.25}
        \renewcommand{\altY}{0}
        
        \renewcommand{\distX}{0.5}
        %S1
        \draw[barraHPIRua] (\distX,0) rectangle ++(\largX,2.25);
        \draw[barraSomRua] (\distX+\largX,0) node[xshift = \largX*1cm, yshift = -0.5cm]{\textbf{\Large S1}} rectangle ++(\largX,2.1);
        \draw[barraHPIParque] (\distX+2*\largX,0) rectangle ++(\largX,3.9);
        \draw[barraSomParque] (\distX+3*\largX,0) rectangle ++(\largX,0);
        \draw[--] (\distX+6*\largX,0) to ++(0,-0.2);
        
        \renewcommand{\distX}{2.5}
        %S2
        \draw[barraHPIRua] (\distX,0) rectangle ++(\largX,0.6);
        \draw[barraSomRua] (\distX+\largX,0) node[xshift = \largX*1cm, yshift = -0.5cm]{\textbf{\Large S2}} rectangle ++(\largX,0.5);
        \draw[barraHPIParque] (\distX+2*\largX,0) rectangle ++(\largX,0.4);
        \draw[barraSomParque] (\distX+3*\largX,0) rectangle ++(\largX,0.3);
        \draw[--] (\distX+6*\largX,0) to ++(0,-0.2);
        
        \renewcommand{\distX}{4.5}
        %S3
        \draw[barraHPIRua] (\distX,0) rectangle ++(\largX,2.1);
        \draw[barraSomRua] (\distX+\largX,0) node[xshift = \largX*1cm, yshift = -0.5cm]{\textbf{\Large S3}} rectangle ++(\largX,1.8);
        \draw[barraHPIParque] (\distX+2*\largX,0) rectangle ++(\largX,1.9);
        \draw[barraSomParque] (\distX+3*\largX,0) rectangle ++(\largX,1.7);
        \draw[--] (\distX+6*\largX,0) to ++(0,-0.2);
        
        \renewcommand{\distX}{6.5}
        %S4
        \draw[barraHPIRua] (\distX,0) rectangle ++(\largX,1.6);
        \draw[barraSomRua] (\distX+\largX,0) node[xshift = \largX*1cm, yshift = -0.5cm]{\textbf{\Large S4}} rectangle ++(\largX,1.5);
        \draw[barraHPIParque] (\distX+2*\largX,0) rectangle ++(\largX,2.0);
        \draw[barraSomParque] (\distX+3*\largX,0) rectangle ++(\largX,0.7);
        \draw[--] (\distX+6*\largX,0) to ++(0,-0.2);
        
        \renewcommand{\distX}{8.5}
        %S5
        \draw[barraHPIRua] (\distX,0) rectangle ++(\largX,1.1);
        \draw[barraSomRua] (\distX+\largX,0) node[xshift = \largX*1cm, yshift = -0.5cm]{\textbf{\Large S5}} rectangle ++(\largX,1.7);
        \draw[barraHPIParque] (\distX+2*\largX,0) rectangle ++(\largX,1.6);
        \draw[barraSomParque] (\distX+3*\largX,0) rectangle ++(\largX,0);
        \draw[--] (\distX+6*\largX,0) to ++(0,-0.2);
        
        \renewcommand{\distX}{10.5}
        %S6
        \draw[barraHPIRua] (\distX,0) rectangle ++(\largX,3.1);
        \draw[barraSomRua] (\distX+\largX,0) node[xshift = \largX*1cm, yshift = -0.5cm]{\textbf{\Large S6}} rectangle ++(\largX,4.9);
        \draw[barraHPIParque] (\distX+2*\largX,0) rectangle ++(\largX,4.8);
        \draw[barraSomParque] (\distX+3*\largX,0) rectangle ++(\largX,2.4);
        \draw[--] (\distX+6*\largX,0) to ++(0,-0.2);
        
        \renewcommand{\distX}{12.5}
        %S7
        \draw[barraHPIRua] (\distX,0) rectangle ++(\largX,3.5);
        \draw[barraSomRua] (\distX+\largX,0) node[xshift = \largX*1cm, yshift = -0.5cm]{\textbf{\Large S7}} rectangle ++(\largX,3.6);
        \draw[barraHPIParque] (\distX+2*\largX,0) rectangle ++(\largX,3.0);
        \draw[barraSomParque] (\distX+3*\largX,0) rectangle ++(\largX,2.8);
        \draw[--] (\distX+6*\largX,0) to ++(0,-0.2);
        
        \renewcommand{\distX}{14.5}
        %S8
        \draw[barraHPIRua] (\distX,0) rectangle ++(\largX,2.9);
        \draw[barraSomRua] (\distX+\largX,0) node[xshift = \largX*1cm, yshift = -0.5cm]{\textbf{\Large S8}} rectangle ++(\largX,1.8);
        \draw[barraHPIParque] (\distX+2*\largX,0) rectangle ++(\largX,4.5);
        \draw[barraSomParque] (\distX+3*\largX,0) rectangle ++(\largX,3.2);
        \draw[--] (\distX+6*\largX,0) to ++(0,-0.2);
        
        %Legenda
        \draw[legenda] (\tamX-6.0cm,\tamY-0.75cm) rectangle ++(6.5cm,3.25cm);
        \draw[barraHPIRua] (\tamX-5.75cm,\tamY+2.0cm) rectangle ++(0.25cm,0.25cm) 
        node[anchor = west, xshift = 0.15cm, yshift = -0.15cm]{\Large Street HPI};
        \draw[barraSomRua] (\tamX-5.75cm,\tamY+1.25cm) rectangle ++(0.25cm,0.25cm) 
        node[anchor = west, xshift = 0.15cm, yshift = -0.15cm]{\Large Street Virtual Sound};
        \draw[barraHPIParque] (\tamX-5.75cm,\tamY+0.5cm) rectangle ++(0.25cm,0.25cm) 
        node[anchor = west, xshift = 0.15cm, yshift = -0.15cm]{\Large Park HPI};
        \draw[barraSomParque] (\tamX-5.75cm,\tamY-0.25cm) rectangle ++(0.25cm,0.25cm) 
        node[anchor = west, xshift = 0.15cm, yshift = -0.15cm]{\Large Park Virtual Sound};
        
    \end{tikzpicture}
    }
    \caption{Standardize mean completion time for each subject with each device in each route (Adapted from \citeonline{marston2006evaluation}).}
    \label{fig:evaluation_mean_time}
\end{figure}

Another finding from this work is about the use of the haptic device caused some strain on the arm and was less acceptable as compared to the sound device, which required no use of the arms.

This study was relevant for current work because it also compares the same types of guidance devices. The participants were asked to score both devices in three questions from 1 = very unacceptable to 5 = very acceptable. These scores are presented in the Table \ref{tab:evaluation_table}. As said above, the participants were able to perform the full experiment with both devices, but there seems to be a preference for the sound-based device.

\begin{table}[htbp]
    \centering
    \caption{Scores of the device}
    \label{tab:evaluation_table}
    \begin{tabular}{|l|l|l|l|l|}
        \hline
        \multirow{2}{*}{\textbf{Statement}} &
        \multirow{2}{*}{\textbf{\begin{tabular}[c]{@{}l@{}}Haptic device's\\ mean score \end{tabular}}} & \multirow{2}{*}{\textbf{SD}} &
        \multirow{2}{*}{\textbf{\begin{tabular}[c]{@{}l@{}}Sound device's \\ mean score\end{tabular}}} & 
        \multirow{2}{*}{\textbf{SD}} \\
        &&&& \\ \hline
        Precision of the directiona information & 4.0 & 0    & 4.1 & 0.83 \\ \hline
        Personal safety while using the device  & 4.1 & 0.35 & 4.0 & 0.76 \\ \hline
        Ease of use                             & 3.5 & 0.53 & 4.6 & 0.52 \\ \hline
    \end{tabular}
\end{table}

But what about being able to use both devices? That's one of the questions that the experiment of this master's thesis aims to answer.

\section{Virtual reality in the design process}
\label{sec:vr_cabin}
% \lipsum[2-4]

VR is also being studied by the aeronautics and aerospace industries. \citeauthor{moerland2021application} proposed the use of the VR during the aircraft cabin's design procedure. The idea is to create easy communication between the development actors and its clients.

The cabin design procedure is often said to be a complex product because it involves a lot of users and stakeholder and each of them have their own set of preferences and requirements \citeauthor{moerland2021application}. The time needed to attend to all of these demands tends to be long and expensive. To understand better the design process of an aircraft cabin, \citeauthor{moerland2021application} interviewed a cabin designer and this interview concluded that the cabin design needed in general 2 years to be concluded. As an example, the interviewees cited a design, in which multiple mock-ups and more than ten meetings with the stakeholder were needed while designing the cabin. The Figure \ref{fig:simplified_cabin_process} illustrates a simplified cabin design process and shows that the traditional process has a high chance to return to initials phases even in the final phases.

%\begin{figure}[h]
%\centering
%\begin{minipage}{\textwidth}
%    \centering
%    \includegraphics[width = 0.5\textwidth]{Revisao/VR Cabin/Process ilustration.png}
%    \caption{Simplified cabin design process \cite{moerland2021application}.}
%    \label{fig:simplified_cabin_process}
%\end{minipage}
%\hfil
%\begin{minipage}{\textwidth}
%    \centering
%    \includegraphics[width = 0.5\textwidth]{Revisao/VR Cabin/Design User involment.png}
%    \caption{Best moments for user involvement \cite{moerland2021application}.}
%    \label{fig:user_involvement}
%\end{minipage}
%\end{figure}

\begin{figure}[!htbp]
    \centering    
    \tikzstyle{hexag} = [regular polygon, regular polygon sides=6, minimum size = 3cm, inner sep = 0cm, xshift = 1.5cm]
    \tikzstyle{hexagText} = [text = white, xshift = 1.5cm]
    \tikzstyle{legenda} = [fill = white, line width = 0.25mm]
    \tikzstyle{--} = [line width = 0.25mm]
    
    \tikzstyle{arrow} = [rounded corners, line width = 1mm, -to]
    \tikzstyle{arrow_blue} = [cor5, arrow]
    
    \resizebox{\linewidth}{!}{
    \begin{tikzpicture}[node distance=1.75cm]
        \centering    

        \node [hexag, draw = cor1, fill = cor1] (emphatize) {};
        \node [hexagText] (emphatizeText) {};
        
        \node [hexag, draw = cor2, fill = cor2, right of = emphatize] (define) {};
        \node [hexagText, right of = emphatize] (defineText) {DEFINE};
        
        \node(inicioFlecha1) [xshift = 3.75cm, yshift = 2.25cm] {};
        \node(fimFlecha1) [right of = emphatize, xshift = 1cm, yshift = 1.2cm] {};
        \node(inicioFlecha2) [xshift = 2.75cm, yshift = 1.5cm] {};
        \node(fimFlecha2) [right of = emphatize, xshift = 0.5cm, yshift = 0.8cm] {};
        \node(inicioFlecha3) [xshift = 1.7cm, yshift = 0.8cm] {};
        \node(fimFlecha3) [right of = emphatize, xshift = 0.25cm, yshift = 0.4cm] {};
        \node(inicioFlecha4) [xshift = 2cm, yshift = -0.5cm] {};
        \node(fimFlecha4) [right of = emphatize, xshift = 0.15cm, yshift = 0cm] {};
        \node(inicioFlecha5) [xshift = 2.77cm, yshift = -1.6cm] {};
        \node(fimFlecha5) [right of = emphatize, xshift = 0.25cm, yshift = -0.5cm] {};
        \node(inicioFlecha6) [xshift = 3.4cm, yshift = -2.2cm] {};
        \node(fimFlecha6) [right of = emphatize, xshift = 0.8cm, yshift = -1.2cm] {};
        
        \draw[arrow] (inicioFlecha1) .. controls ++(0.3,-0.5) .. (fimFlecha1);
        \draw[arrow] (inicioFlecha2) .. controls ++(0.5,-0.15) .. (fimFlecha2);
        \draw[arrow] (inicioFlecha3) .. controls ++(0.75,0) .. (fimFlecha3);
        \draw[arrow] (inicioFlecha4) .. controls ++(0.5,0.25) .. (fimFlecha4);
        \draw[arrow] (inicioFlecha5) .. controls ++(0.25,0.5) .. (fimFlecha5);
        \draw[arrow] (inicioFlecha6) .. controls ++(0.25,0.5) .. (fimFlecha6);
        
        \node [hexag, draw = cor3, fill = cor3, right of = define] (ideate) {};
        \node [hexagText, right of = define] (ideateText) {IDEATE};
        
        \node(inicioDFlecha1) [above of = define, xshift = 0.25cm, yshift = -0.1cm] {};
        \node(fimDFlecha1) [above of = ideate, xshift = -0.25cm, yshift = -0.1cm] {};
        \node(inicioIFlecha1) [below of = ideate, xshift = -0.25cm, yshift = 0.1cm] {};
        \node(fimIFlecha1) [below of = define, xshift = 0.25cm, yshift = 0.1cm] {};
        
        \draw[arrow, draw = cor3] (inicioDFlecha1.south) .. controls ++(0.25,0.5) and ++(-0.75,0.5).. (fimDFlecha1.south);
        \draw[arrow, draw = cor3] (inicioIFlecha1.north) .. controls ++(-0.25,-0.5) and ++(0.75,-0.5).. (fimIFlecha1.north);
        
        \node [hexag, draw = cor4, fill = cor4, right of = ideate] (prototype) {};
        \node [hexagText, right of = ideate] (prototypeText) {PROTOTYPE};
        
        \node(inicioIFlecha2) [above of = ideate, xshift = 0.25cm, yshift = -0.2cm] {};
        \node(fimIFlecha2) [above of = prototype, xshift = -0.25cm, yshift = -0.2cm] {};
        \node(inicioPFlecha1) [below of = prototype, xshift = -0.25cm, yshift = 0.2cm] {};
        \node(fimPFlecha1) [below of = ideate, xshift = 0.25cm, yshift = 0.2cm] {};
        \node(inicioPFlecha2) [above of = prototype, xshift = 0.25cm, yshift = -0.2cm] {};
        \node(fimPFlecha2) [above of = define, xshift = -0.25cm, yshift = -0.2cm] {};
        
        \draw[arrow, draw = cor4] (inicioIFlecha2.center) .. controls ++(0.25,0.5) and ++(-0.75,0.5).. (fimIFlecha2.east);
        \draw[arrow, draw = cor4] (inicioPFlecha1.east) .. controls ++(-0.25,-0.5) and ++(0.75,-0.5).. (fimPFlecha1.center);
        \draw[arrow, draw = cor1] (inicioPFlecha2.west) .. controls ++(-1,1.75) and ++(1.2,1.5).. (fimPFlecha2);
        
        
        \node [hexag, draw = cor5, fill = cor5, right of = prototype] (assess) {};
        
        \node(inicioPFlecha3) [above of = prototype, xshift = 0.25cm, yshift = -0.2cm] {};
        \node(fimPFlecha3) [above of = assess, xshift = -0.25cm, yshift = -0.2cm] {};
        \node(inicioAFlecha1) [below of = assess, xshift = -0.25cm, yshift = 0.2cm] {};
        \node(fimAFlecha1) [below of = prototype, xshift = 0.25cm, yshift = 0.2cm] {};
        \node(inicioAFlecha2) [below of = assess, xshift = 0cm, yshift = 0.2cm] {};
        \node(fimAFlecha2) [below of = ideate, xshift = -0.25cm, yshift = 0.1cm] {};
        \node(inicioAFlecha3) [below of = assess, xshift = 0.25cm, yshift = 0.2cm] {};
        \node(fimAFlecha3) [below of = define, xshift = -0.25cm, yshift = 0.1cm] {};
        
        \draw[arrow, draw = cor5] (inicioPFlecha3.center) .. controls ++(0.5,1) and ++(-0.75,1).. (fimPFlecha3.center);
        \draw[arrow, draw = cor5] (inicioAFlecha1.center) .. controls ++(-0.25,-0.5) and ++(0.75,-1).. (fimAFlecha1.west);
        \draw[arrow, draw = cor2] (inicioAFlecha2.center) .. controls ++(-0.25,-1.25) and ++(0.75,-1).. (fimAFlecha2.south east);
        \draw[arrow, draw = cor1] (inicioAFlecha3.center) .. controls ++(-0.25,-1.75) and ++(1.2,-1.5).. (fimAFlecha3.north);
        
        \node [hexagText, right of = prototype] (assessText) {ASSESS};
        

        
        
    \end{tikzpicture}
    }
    \caption{Simplified cabin design process (Adapted from \citeonline{moerland2021application}).}
    \label{fig:simplified_cabin_process}
\end{figure}
\begin{figure}[!htbp]
    \centering    
    \tikzstyle{hexag} = [regular polygon, regular polygon sides=6, minimum size = 3cm, inner sep = 0cm, xshift = 1.5cm]
    \tikzstyle{hexagText} = [text = white, xshift = 1.5cm]
    \tikzstyle{legenda} = [fill = white, line width = 0.25mm]
    \tikzstyle{--} = [line width = 0.25mm]
    
    \resizebox{\linewidth}{!}{
    \begin{tikzpicture}[node distance=1.75cm]
        \centering    

        \node [hexag, draw = cor1, fill = cor1] {};
        \node [hexagText] (emphatize) {EMPHATIZE};
        %\node [regular polygon, regular polygon sides=6, text = white, draw = cor1, minimum size = 5cm, fill = cor1] (emphatize) {EMPHATIZE};
        \node [hexag, draw = cor2, fill = cor2, right of = emphatize] (define) {};
        \node [hexagText, right of = emphatize] (defineText) {DEFINE};
        %\node [regular polygon, regular polygon sides=6, text = white, draw = cor2, minimum size = 5cm, fill = cor2, right of = emphatize] (define) {DEFINE};
        \node [hexag, draw = cor3, fill = cor3, right of = define] (ideate) {};
        \node [hexagText, right of = define] (ideateText) {IDEATE};
        %\node [regular polygon, regular polygon sides=6, text = white, draw = cor3, minimum size = 5cm, fill = cor3, right of = define] (ideate) {IDEATE};
        \node [hexag, draw = cor4, fill = cor4, right of = ideate] (prototype) {};
        \node [hexagText, right of = ideate] (prototypeText) {PROTOTYPE};
        %\node [regular polygon, regular polygon sides=6, text = white, draw = cor4, minimum size = 5cm, fill = cor4, right of = ideate] (prototype) {PROTOTYPE};
        \node [hexag, draw = cor5, fill = cor5, right of = prototype] (assess) {};
        \node [hexagText, right of = prototype] (assessText) {ASSESS};
        %\node [regular polygon, regular polygon sides=6, text = white, draw = cor5, minimum size = 5cm, fill = cor5, right of = prototype] (assess) {ASSESS};

        \node(insertNorth1) [right of = emphatize, xshift = -0.125cm, yshift = 0.25cm] {};
        \node(insertNorthWest1) [above of = insertNorth1, left of = insertNorth1, xshift = 0.75cm] {};
        \node(insertNorthEast1) [above of = insertNorth1, right of = insertNorth1, xshift = -0.75cm] {};
        \draw[--] (insertNorthWest1.center) to (insertNorth1.center) to (insertNorthEast1.center);
        
        \node(insertSouth1) [right of = emphatize, xshift = -0.125cm, yshift = -0.25cm] {};
        \node(insertSouthWest1) [below of = insertSouth1, left of = insertSouth1, xshift = 0.75cm] {};
        \node(insertSouthEast1) [below of = insertSouth1, right of = insertSouth1, xshift = -0.75cm] {};
        \draw[--] (insertSouthWest1.center) to (insertSouth1.center) to (insertSouthEast1.center);
        
        \node(insertNorth2) [right of = define, xshift = -0.125cm, yshift = 0.25cm] {};
        \node(insertNorthWest2) [above of = insertNorth2, left of = insertNorth2, xshift = 0.75cm] {};
        \node(insertNorthEast2) [above of = insertNorth2, right of = insertNorth2, xshift = -0.75cm] {};
        \draw[--] (insertNorthWest2.center) to (insertNorth2.center) to (insertNorthEast2.center);
        
        \node(insertSouth2) [right of = define, xshift = -0.125cm, yshift = -0.25cm] {};
        \node(insertSouthWest2) [below of = insertSouth2, left of = insertSouth2, xshift = 0.75cm] {};
        \node(insertSouthEast2) [below of = insertSouth2, right of = insertSouth2, xshift = -0.75cm] {};
        \draw[--] (insertSouthWest2.center) to (insertSouth2.center) to (insertSouthEast2.center);


    \end{tikzpicture}
    }
    \caption{Best moments for user involvement (Adapted from \citeonline{moerland2021application}).}
    \label{fig:user_involvement}
\end{figure}


\citeauthor{moerland2021application} are inside the German Aerospace Center (DLR, \textit{Deutsch Zentrum für Luft- Raumfahrt}) and decided to study a new procedure that could bring the involvement of the final users in the design process. This procedure is based on co-design, where the users can influence the product's development from the beginning until the end. The Figure \ref{fig:user_involvement} shows the best moments to bring the users to the process. But for the involvement to happen, a communication channel needed to be established. The authors choose to use \textit{Reality Works} and test on a DLR's inside project. This project's goal was to design a new cabin that would be incorporated into a large workflow, but the design process was to be completely made in a digital environment. This was the perfect test case for the VR use in the cabin's design procedure.

A pilot use case was made with the members of this project. Three different designers (two with around 5 years of experience and another with more than 35 years of experience) initiated a cabin design. 
The Figures \ref{fig:cabin_sketch} and \ref{fig:cabin_3d_model} show the results using the traditional method. The sketch can only present a glance of what the cabin will be. The 3D model has more details, but any change to this representation needs a new rendering session and this can take hours, or even days, to be made.

\begin{figure}[h]
\centering
\begin{minipage}{.45\textwidth}
    \centering
    \includegraphics[width = \linewidth]{Revisao/VR Cabin/Sketch.png}
    \vspace{0.4cm}
    \caption{Cabin sketch made with Adobe Photoshop \cite{moerland2021application}.}
    \label{fig:cabin_sketch}
\end{minipage}
\hfil
\begin{minipage}{.45\textwidth}
    \centering
    \includegraphics[width = \linewidth]{Revisao/VR Cabin/3D Model.png}
    \caption{Cabin 3D model made with Rhyno \cite{moerland2021application}.}
    \label{fig:cabin_3d_model}
\end{minipage}
\end{figure}

The Figures \ref{fig:vr_sketch} and \ref{fig:vr_3d_model} show the same representation but made in a VR environment. The sketch was made inside the aircraft cabin and this could have been done with a client or a stakeholder and they could also draw and give their opinions from the beginning. The 3D models can be imported to increase the sketch's level of detail.

\begin{figure}[h]
\centering
\begin{minipage}{.45\textwidth}
    \centering
    \includegraphics[width = \linewidth]{Revisao/VR Cabin/VR sketch.png}
    \caption{VR navigation with sketching \cite{moerland2021application}.}
    \label{fig:vr_sketch}
\end{minipage}
\hfil
\begin{minipage}{.45\textwidth}
    \centering
    \includegraphics[width = \linewidth]{Revisao/VR Cabin/VR 3D Model.png}
    \caption{VR navigation with imported 3D models \cite{moerland2021application}.}
    \label{fig:vr_3d_model}
\end{minipage}
\end{figure}

This case was well received by the design team and they have chosen to continue to use the VR tool. The benefits disadvantages pointed by \citeauthor{moerland2021application} are listed in the Table \ref{tab:benefits_disvantages_vr_cabin}. The VR helps to bring the clients closer to the design team, allows them to draw quick sketches in brainstorming gatherings and has a steep learning curve for the designers. On the other hand, is its a high-cost tool, the use for a long time can cause nausea and maybe other health implications, even though the learning curve is steep, there is still a learning curve and the user needs to get used to the exposure to others that can see the user from outside the virtual environment (some find this situation uncomfortable).

\begin{table}[h]
    \centering
    \caption{The benefits and disadvantages noted by the authors \citeauthor{moerland2021application}.}
    \label{tab:benefits_disvantages_vr_cabin}
    \begin{tabular}{|l|l|}
        \hline
        \textbf{Benefits}                         & \textbf{Disadvantages}                                  \\ \hline
        Bottleneck at early concept design stages & High cost                                               \\ \hline
        Quick sketchs during brainstorming        & Nausea and other health implacations                    \\ \hline
        Steep learning curve                      & \begin{tabular}[c]{@{}l@{}}There is a learning and personal\\ adaptation to exposure\end{tabular} \\ \hline
    \end{tabular}
\end{table}

The current master's thesis isn't about designing or aircraft cabins, but this research shows that VR is being studied to be implemented inside industries. The current research could be done by any product industry that wanted to create a test environment for their clients to increase the user's approval or to bring other teams close to reducing the full design time.

\section{Final Remarks}
\label{sec:final_remarks3}

In this chapter, 7 papers were reviewed with the current thesis in mind. These 7 papers are related to one of the three keywords, or a combination of them: "human factors", "virtual reality" and "blindness".

These papers pointed that BVI users are sensible to sensorial inputs. \ref{sec:vr_without_vision} showed that they were sensible with the vibration of the cane. This sensibility was based on their past experience with canes made from different materials and with vibration devices, but this paper was more about designing a navigation tool, not a framework for testing human factors.

When designing an virtual environment, it is important to consider the feeling of presence, and in the Section \ref{sec:emotion_presence_vr} presented a paper that concluded that to increase this feeling one should allow the user agency inside the environment. That means that the user will feel more present (will forget that he/she is inside the virtual environment) when he/she is induced to interact with the elements from the virtual environment. The current experiment used real furniture and real actors mixed with virtual sounds to increase the sensation that the user was not inside a laboratory, but inside a medical clinic reception.

The Section \ref{sec:bradley_dunlop} was about to papers from the same authors that studied the differences between the information used by BVI and sighted users during the navigation. They found out that the last group used more text-based information, while the first group used information with more words and from a wider variety of categories. Another conclusion an information that was made for BVI users was more mentally demandful for sighted users, and vice-verse. This conclusion was importante for two of the used methods in the thesis, which used audio information for the navigation.

But when is the perfect moment to announce a audio information for BVI users? The Section \ref{sec:auditory_navigation} showed the the moment that an information is transmitted has affect on the accuracy they reach a goal. This conclusion was made by analysing their navigation task with a NASA-TLX and the results confirmed that this tool can be used for their tasks.

One of the evaluations that this work does is comparing different BVI devices. This evalutation was presented in \ref{sec:evaluation_spatial_display} and the author made a similar comparison. They compared an haptic and sound information in different environments. Their conclusion was that the haptic were less acceptable. This work employs two models of haptic devices and a audio method. Is likely that the conclusion will be comparable.

Finally, the last section of this chapter, Section \ref{sec:vr_cabin}, showed the benefits and disavantages of implementing virtual reality with co-design on a design team. Similarly the ideia of this work is to used virtual reality to allow the users to test new concepts of products and give the designer their feedbacks and impressions.

All these papers contributed to the designing, improvement, and decision making of the current experiment. The next chapter will describe all these steps with more details.



\chapter{Proposal description}
\label{ch:metodologia}
% CAPÍTULO 3: DESCRIÇÃO DA SUA PROPOSTA OU CONTRIBUIÇÃO
%   ESSE É O CAPÍTULO MAIS IMPORTANTE DE SEU TRABALHO


%1. Elabore um parágrafo que introduz o capítulo: Este capítulo apresenta(descreva o objetivo do  capítulo...). É constituído de N seções a saber...
%2. Caso você tenha dúvida do nível de detalhamento da descrição, coloque-se no lugar do leitor!
%3. Elabore um parágrafo que conclui o capítulo e introduz o capítulo seguinte.


This chapter is about the proposed methodology of this master thesis experiments. The Figure \ref{fig:diag_metodologia} shows the phases and the tasks inside each phase. This chapter will explain each phase and task presented in the Figure \ref{fig:diag_metodologia}.

\begin{figure}[!htb]
    \centering
    \tikzstyle{every node}=[font=\large]
    
    \tikzstyle{task} = [rectangle, rounded corners, minimum width=4cm, minimum height=1.5cm,text centered, draw=black, fill=white!30, text width=3.5cm, font=\large]
    \tikzstyle{phase} = [rectangle, minimum width=4cm, minimum height=1.5cm,text centered, draw=black, fill=white!30, text width=5.0cm, font=\Large]
    \tikzstyle{--gray} = [ccmLGray, dashed, dash pattern=on 1cm off 1cm , rounded corners, line width = 2mm]
    \tikzstyle{--red} = [ccmRed, rounded corners, line width = 2mm]
    \tikzstyle{--blue} = [ccmDBlue, rounded corners, line width = 2mm]
    
    \tikzstyle{arrow_blue} = [ccmDBlue, rounded corners, line width = 2mm, ->]
    \tikzstyle{d--blue} = [ccmDBlue, dashed, dash pattern=on 1.0cm off 0.65cm, rounded corners, line width = 2mm]
    \tikzstyle{arrow_--_blue} = [ccmDBlue, dashed, dash pattern=on 1.0cm off 0.65cm, rounded corners, line width = 2mm, ->]
    \tikzstyle{arrow_red} = [ccmRed, rounded corners, line width = 2mm, ->]
    
    
    \resizebox{\linewidth}{!}{
    \begin{tikzpicture}[node distance=1cm]
        
        \node (interview) [phase] {Phase 1 – Context definition:};
        \node (interview_hospital) [task, below of = interview, yshift = -2cm] {1.1 Interview with environment specialists;};
        \node (interview_bvi) [task, below of = interview_hospital, yshift = -3.5cm] {1.2 Interview with BVI consultants.};
        
        \node (a1) [right of = interview, above of = interview, xshift = 2.0cm] {};
        \node (a2) [below of = a1, yshift = -11cm] {};
        \draw [--gray] (a1) to (a2);
        
        \node (scope) [phase] [phase, right of = interview, xshift = 5cm] {Phase 2 – Specification:};
        \node (virtual_environment) [task, below of = scope, yshift = -2cm] {2.1 Virtual environment specification;};
        \node (human_factors) [task, below of = virtual_environment, yshift = -2cm] {2.2 Assessment specification;};
        \node (guidance_methods) [task, below of = human_factors, yshift = -2cm] {2.3 Guidance specification.};
        
        \node (b1) [right of = scope, above of = scope, xshift = 2.0cm] {};
        \node (b2) [below of = b1, yshift = -11cm] {};
        \draw [--gray] (b1) to (b2);
        
        \node (development) [phase] [phase, right of = scope, xshift = 5cm] {Phase 3 – Development:};
        \node (ve_creation) [task, below of = development, yshift = -2cm] {3.1 Implementation of virtual environment;};
        \node (tools_definition) [task, below of = ve_creation, yshift = -2cm] {3.2 Proposal of assessment techniques and tools;};
        \node (guidance_development) [task, below of = tools_definition, yshift = -2cm] {3.3 Development of guidance devices.};
        
        \node (c1) [right of = development, above of = development, xshift = 2.0cm] {};
        \node (c2) [below of = c1, yshift = -11cm] {};
        \draw [--gray] (c1) to (c2);
        
        \node (tryouts) [phase] [phase, right of = development, xshift = 5cm] {Phase 4 – Preliminary evaluation:};
        \node (tryouts_task) [task, below of = tryouts, yshift = -5.0cm] {4.1 Try-outs.};
        
        \node (d1) [right of = tryouts, above of = tryouts, xshift = 2.0cm] {};
        \node (d2) [below of = d1, yshift = -11cm] {};
        \draw [--gray] (d1) to (d2);
        
        \node (experiment) [phase] [phase, right of = tryouts, xshift = 5cm] {Phase 5 – Systematic evaluation:};
        \node (experiment_task) [task, below of = experiment, yshift = -5.0cm] {5.1 Controlled experiments.};
    
        \draw[arrow_blue] (interview_hospital.east) to (virtual_environment.west);
        \draw[arrow_blue] (interview_bvi.east) to ++(1.0,0) to ++(0,4.5) to (virtual_environment.west);
        \draw[arrow_blue] (interview_bvi.east) to ++(1.0,0) to ++(0,1.5) to (human_factors.west);
        \draw[arrow_blue] (interview_bvi.east) to ++(1.0,0) to ++(0,-1.5) to (guidance_methods.west);  
        
        \draw[arrow_blue] (virtual_environment.east) to (ve_creation.west);
        \draw[arrow_blue] (human_factors.east) to (tools_definition.west);
        \draw[arrow_blue] (guidance_methods.east) to (guidance_development.west);
        
        \draw[arrow_blue] (ve_creation.east) to ++(1.0,0) to ++(0,-3.0) to (tryouts_task.west);
        \draw[arrow_blue] (tools_definition.east) to ++(1.0,0) to  (tryouts_task.west);
        \draw[arrow_blue] (guidance_development.east) to ++(1.0,0) to ++(0,3.0) to (tryouts_task.west);
        
        \draw[arrow_blue] (tryouts_task.east) to (experiment_task.west);
        \draw[--red] (tryouts_task.north) to (tryouts.south);
        \draw[--red] [opacity=0.2] (tryouts.south) to (tryouts.center) to (tryouts.west);
        \draw[arrow_red] (tryouts.west) to (development.east);

        \draw[d--blue] (experiment_task.north) to (experiment.south);
        \draw[d--blue] [opacity=0.2] (experiment.south) to (experiment.center) to (experiment.north);
        \draw[arrow_--_blue] (experiment.north) to ++(0,1.0) to ++(-18.0,0.0) to (scope.north);
        
    \end{tikzpicture}
    }
    \caption{Method's diagram}
    \label{fig:diag_metodologia}
\end{figure}


\section{Interviews' phase}
\label{sec:interviews_phase}
    The first phase of this project was the Interviews' phase. In these phase, the researchers' main goal was to gather information, especially those related to the COVID-19 pandemic, about the main procedures that happen inside a hospital and about the daily life of BVI people.

    \subsection{Interview with the hospital}
    
        To understand the procedures that hospital and medical clinics followed during their day to day activities and during the COVID-19 pandemic, two hospitals were interviewed. The interview was aimed to find out how a new patient does a check-in and the following steps until he/she get in the proper medic's office.
        
        At the project start, the scenario was supposed to be a reception inside a hospital, but, because of the physical space needed to simulate that virtual environment, the scenario changed along the project several times.

    \subsection{Interview with the BVI consultants}
    
        One of the motivations of this master thesis is that the current BVI guidance products are not effective enough and one of the likely reasons is that the BVI users were not consulted during the products development process.
    
        With that though in mind, BVI users were consulted in order to design a virtual environment that would be familiar to their reality. Two users with different visual impairments were interviewed, one person that became blind with 13 years old, and other that with Usher's disease. These were critical to understand how they perceive a medical clinic as they walk in and how they interact with the environment and these notes were used in the next phase.

\section{Experiment idealization's phase}
\label{sec:idealization_phase}
    At this phase, the proceedings and the interview notes are used to take key decisions about the virtual environment used in the experiment, about which human factors are going to be assessed and which guidance methods are going to be used
    

    \subsection{Experiment's virtual world definition}
        As said before, the original idea was to use a hospital reception as model to the virtual environment for the experiment, however the physical space needed to fit the hospital was to big. So instead of a whole hospital reception, it was decided to use it a medical clinic reception but still with the same proceedings.

    \subsection{Assessed human factors definition}
        In order to reach the experiment's objective, a set of human factors had to be choose. The objectives \ref{itm:obj_second} and \ref{itm:obj_third} could be reach if the assessed human factor represented the user's workload and the developed mental map. Both these could be evaluated using:

        \begin{itemize}
            \item Mental workload
            \item Situation awareness
        \end{itemize}
        
        The details about each method are explained in the sections \ref{sec:mental_workload} and \ref{sec:situation_awareness}.

    \subsection{Guidance methods definition}
        The variety of BVI users is wide, as is the variety of assistive products. All of these products must communicate with the users and they use sound, vibration or both to transmit these information. With this though in mind, it was decided that that would be used at least two methods: one that rely only in audio and another that transmits only vibration. Of course, the interaction between those two methods would be also evaluated. This interaction became the third method.
        
        Another interesting property that could be evaluated in those products is the effect of a information being transmitted with and without the user's command. This evaluation split the vibration method in two: one that worked \textit{without} the user's command and another the worked \textit{with} the user's command.
        
        At the end, the following methods were chosen to be analysed:
        
        \begin{itemize}
            \item A usual guidance method;
            \item Audio guidance;
            \item Vibration guidance without command;
            \item Vibration guidance with command;
            \item Mixture of audio and vibration.
        \end{itemize}
    
\section{Development and creation's phase}
\label{sec:creation_phase}
    
    With the decision of the previous phase it is possible to start the development of the virtual environment, the guidance methods and the tools for the assessment of the human factors.

    \subsection{Virtual world creation}
    \label{subsec:virtual_world_creation}
        The virtual reality application was made using the software Unity3D, which is a famous tool for virtual reality applications and game development. It has some built-in tools but is also possible to customize functions for more specific use \cite{wang2010new}. The virtual environments, or scenes (as it is called inside the Unity3D), were made with the dimensions to fit in the CCM entry hall, that has a flat area of 8x4m. Inside the environment there was some typical furniture or devices found in hospital reception, a reception desk and a waiting area, composed of 2-3 chairs. The participant had \ref{itm:n_tasks} tasks at the scene and they are displayed at Figure \ref{fig:task_diagram}. More details ahead at Chapter \ref{ch:cenario}.
        
        \begin{enumerate}
            \item Clean the hands at the sanitizer totem (COVID-19 procedures);
            \item Go to the reception desk to receive a queue number;
            \item Go to the waiting area and wait for the number calling;
            \item Leave the room when called \label{itm:n_tasks}
        \end{enumerate}
        
        \begin{figure}[!htb]
    \centering
    \tikzstyle{every node}=[font=\large]
    
    \tikzstyle{task} = [rectangle, rounded corners, minimum width=4cm, minimum height=1.5cm,text centered, draw=black, fill=white!30, text width=3.5cm, font=\large]
    \tikzstyle{--gray} = [ccmGray, dashed, dash pattern=on 1cm off 1cm , rounded corners, line width = 2mm]
    \tikzstyle{--red} = [ccmRed, rounded corners, line width = 2mm]
    \tikzstyle{entryLine} = [ccmDBlue, rounded corners, line width = 2mm]
    \tikzstyle{arrow} = [ccmRed, rounded corners, line width = 1mm, ->]
    \tikzstyle{arrowX} = [line width = 2mm]
    
    \begin{tikzpicture}[node distance=1cm]
        
        \node (entryA) {};
        \node (exit) [below of = entryA, xshift = 0.5cm] {Exit};
        \node (entryB) [right of = entryA, xshift = 1.5cm] {};
        \node (entry) [below of = entryB, xshift = -0.5cm] {Entry};
        \draw[entryLine] (entryA) to (entryB);
        
        \node (totem) [right of = entryB, above of = entryB, draw, rectangle, minimum width=1cm, minimum height=1cm, draw=black, fill=white!30] {};
        \node (totemTexto) [below of = totem, yshift = 0.3cm, font = \small] {Sanitizer totem};
        
        \node (desk) [above of = entryB, yshift = 5.0cm, draw, rectangle, minimum width=3cm, minimum height=1.5cm, draw=black, fill=white!30, font = \small] {Reception};
        \node (nurse) [above of = desk, yshift = 0.6cm] {\includegraphics[scale = 0.09]{Metodologia/enfermeira.png}};
        \node (visitor) [below of = desk, yshift = -0.6cm] {\includegraphics[angle=180, origin=c, scale = 0.09]{Metodologia/visita.png}};
        
        \node (waiting1) [left of = desk, xshift = -5.0cm] {\includegraphics[angle=90, origin=c, scale = 0.09]{Metodologia/visita.png}};
        \node (waiting2) [below of = waiting1, yshift = -0.3cm] {\includegraphics[angle=90, origin=c, scale = 0.09]{Metodologia/visita(x).png}};
        \node (waiting3) [below of = waiting2, yshift = -0.3cm] {\includegraphics[angle=90, origin=c, scale = 0.09]{Metodologia/visita.png}};
        \node (waitingTexto) [left of = waiting2, rotate = 90, font = \small] {Waiting area};
        
        
        \draw (-5,0) rectangle(5cm,8.5cm);
        
        \draw[arrow] (entry) to ++(0,2.0) to (totem.west);
        \draw[arrow] (totem.north) to ++(0,0.5) to ++(-1.0,0) to (visitor.south);
        \draw[arrow] (visitor.west) to ++ (-3.0,0) to ++(0,-1.0cm) to (waiting3.east);
        \draw[arrow] (waiting3.south) to ++(0,-1.0) to ++(4.0,0) to (exit);  
        
    \end{tikzpicture}
    \caption{Scheduled task of the experiment and their order.}
    \label{fig:task_diagram}
\end{figure}

        
        The goal of these tasks is to engage the user to navigate through the room and see if it is able to draw a mental map of the scene as well as use the information of the obstacles in order to avoid them when needed. Beside theses mains components, there are also some minor distractions that is common to hear at a clinical, such as telephone ringing, keyboard typing, people taking and others. These were put to increase the immersion and to be a distraction as well, otherwise it wouldn't simulate the reality of these scenarios.
    
    \subsection{Tools and methods definition}
        There were three types of human factors' assessment tools that were applied at the experiment:
        \begin{itemize}
            \item Task performance; \\  Measured using the time and the number contacts between the user and the furniture throughout the experiment.
            \item Physiological measures; \\ Measured using and ECG sensor, a GSR sensor and a temperature sensor.
            \item Subjective measures. \\ Measured using a NASA-TLX, a SAGAT Adapted questionnaire and a guidance method evaluation questionnaire.
        \end{itemize}
        The details about each method are explained in the sections \ref{subsec:task_performance}, \ref{subsec:physiological_measures} and \ref{subsec:subjective_measures}.
        
    \subsection{Guidance methods development}
        As said in the last section, three different guidance methods were established to be used in the experiment besides the White Cane:, a haptic belt and a virtual cane.
        \begin{itemize}
            \item A audio guidance method; \\ The audio only guidance method will be straight simple. In the course of the experiment the participant could give two different voice commands:
            \begin{itemize}
                \item  "What is around me?"; \\ The answer of this command was a quickly description of the closest furniture around the user.
                \item  "Where is (something)?". \\ The answer of this command was the direction and distance of something asked by the user.
            \end{itemize}
            Each command was answered by a member of the experiment team accordingly.
            
            \item The haptic belt; \\
            That is a belt that had appended 8 vibration devices that vibrate accordingly to the direction and distance of the closest object around the user. More information on the Haptic Belt ahead at Chapter \ref{ch:cinto}.
            
            \item The virtual cane. \\ 
            This was based on the white cane mechanics, that the user "points" the cane to check near obstacles in the direction of the cane. The virtual cane has a similar function, but instead of connecting the user to the object through the cane, it vibrates when it detects an obstacle in the direction pointed by the user. A VR hand-control was used as canes and the user point's it to where he/she wanted. The algorithm used on the Virtual Cane is in the Appendix \ref{ap:virtual_apend}
        \end{itemize}
    

\section{Tryouts and tests' phase}
\label{sec:tests_phase}
        
        At this phase a few tests were performed to evaluate if the experiment was going as planned and to avoid any unfortunate events or errors during the real experiment. It was expected that changes could be needed to be made before the real experiment and there were a few. It was at this phase that the final dimension of the virtual environment and the physical space were defined.

\section{Experiment}
\label{sec:experiment}
        
        As the proper section name says, this phase is were the proper experiment was made.

After all these phases were completed, the next step was to analyze all the data and elaborate their conclusions. Instead of going to the results and discussions, the next Chapters \ref{ch:cenario} and \ref{ch:cinto} will deepen in the virtual environment development and in the haptic belt development in these order. The Figure \ref{fig:ve_re} show a comparisson between the virtual environment created in Unity3D and the real environment assembled in the CCM's entry hall.

%\begin{figure}[!htb]
    \centering
    \tikzstyle{every node}=[font=\large]
    
    \tikzstyle{task} = [rectangle, rounded corners, minimum width=4cm, minimum height=1.5cm,text centered, draw=black, fill=white!30, text width=3.5cm, font=\large]
    \tikzstyle{phase} = [rectangle, minimum width=4cm, minimum height=1.5cm,text centered, draw=black, fill=white!30, text width=3.5cm, font=\Large]
    \tikzstyle{--} = [black, line width = 2mm]
    \tikzstyle{--red} = [ccmRed, rounded corners, line width = 2mm]
    \tikzstyle{arrow_blue} = [ccmDBlue, rounded corners, line width = 2mm, ->]
    \tikzstyle{arrow_red} = [ccmRed, rounded corners, line width = 2mm, ->]
    
    
    \setlength{\fboxsep}{0pt}%
    \setlength{\fboxrule}{1mm}

    \resizebox{\linewidth}{!}{
    \begin{tikzpicture}[node distance=1cm]
        
        \node (image1) [] {\fbox{\includegraphics[width = 0.5\linewidth]{Metodologia/VE.png}}};
        \node (image2) [right of = image1, xshift = 7.15cm] {\fbox{\includegraphics[width = 0.5\linewidth]{Metodologia/RE.jpg}}};

        \node (image3) [below of = image1, yshift = -10cm] {\includegraphics[width = 0.5\linewidth]{Metodologia/VE.png}};
        \node (image4) [right of = image3, xshift = 7.15cm] {\includegraphics[width = 0.5\linewidth]{Metodologia/RE.jpg}};

        %\node (ponto1) [above of = image1, yshift = 2.5cm, left of = image1, xshift = -3cm] {};
        %\node (ponto2) [above of = image1, yshift = 2.5cm, right of = image1, xshift = 3cm]{};
    
        %\draw[--] (ponto1.west) -- (ponto2.east);
        
    \end{tikzpicture}
    }
    \caption{Virtual and Real environment comparisson}
    \label{fig:ve_re}
\end{figure}

\begin{figure}[!htb]
    \centering
    \begin{subfigure}[b]{0.49\textwidth}
        \centering
        \includegraphics[width=\textwidth]{Metodologia/VE.png}
        \caption{Virtual environment screenshot}
        \label{fig:ve_photo}
    \end{subfigure}
    \hfill
    \begin{subfigure}[b]{0.49\textwidth}
        \centering
        \includegraphics[width=\textwidth]{Metodologia/RE.jpg}
        \caption{Real environment photo}
        \label{fig:re_photo}
    \end{subfigure}
       \caption{Environment comparisson}
       \label{fig:ve_re}
\end{figure}

\chapter{Virtual environment development}
\label{ch:cenario}

The main background and the source of the sensorial input was the virtual environment. Its development can be divided in \ref{itm:final_clinic} steps. The whole procedure is represented in the Figure \ref{fig:ve_process}.

\begin{enumerate}
    \item Procedures
    \item City Hospital
    \item Medical Clinic
    \item Adjustments
    \item Final clinic \label{itm:final_clinic}
\end{enumerate}

\begin{figure}[!htb]
    \centering
    %\tikzstyle{every node}=[font=\large]
    
    \tikzstyle{start} = [rectangle, rounded corners, minimum width=3cm, minimum height=1.0cm,text centered, draw=black, fill=white!30, text width=3cm]
    \tikzstyle{scene} = [rectangle, minimum width=3cm, minimum height=1.0cm, text centered, draw=black, fill=white!30, text width=3cm]
    \tikzstyle{perks} = [diamond, minimum width=1cm, minimum height=1.0cm,  text centered, text width=1.75cm, draw=black, fill=white!30]
    
    \tikzstyle{arrow} = [rounded corners, line width = 1mm, ->]
    \tikzstyle{arrow_blue} = [ccmDBlue, rounded corners, line width = 1mm, ->]
    \tikzstyle{arrow_red} = [ccmRed, rounded corners, line width = 1mm, ->]
    
    \resizebox{\linewidth}{!}{
    \begin{tikzpicture}[node distance=2.5cm]
        \centering
        \node (start) [start] {Hospital procedures};
        \node (city) [scene, below of=start] {City hospital};
        
        \draw [arrow] (start) to node[midway,right]{1st idea} (city);
        
        \node (clinic1) [scene, below of=city] {Medical Clinic V1};
        
        \draw [arrow] (city) to node[midway,right]{Too big} (clinic1);
        
        \node (furniture) [perks, aspect=2.5, below of=clinic1, xshift = -0.5cm, yshift = -0.5cm] {Chairs and desk};
        \node (telephone) [perks, aspect=2.5, right of=furniture, xshift = 2.5cm] {Telephone ringing};
        \node (keyboard) [perks, aspect=2.5, left of=furniture, xshift = -2.5cm] {Keyboard typing};
        
        \draw [arrow_blue] (-0.5cm,-5.5cm) to (furniture);
        \draw [arrow_blue] (-0.5cm,-5.5cm) to ++(0,-1.0cm) to +(-5cm,0) to (keyboard);
        \draw [arrow_blue] (-0.5cm,-5.5cm) to ++(0,-1.0cm) to +(5cm,0) to (telephone);
        
        \node (conv1) [below of = furniture, yshift = 0.5cm] {};
        \node (conv1Texto) [right of = conv1, xshift = -1.0cm] {1st presentation};
        
        \node (tv) [perks, aspect=2.5, below of=conv1, yshift = 0.5cm] {TV playing};
        \node (people) [perks, aspect=2.5, left of=tv, xshift = -2.5cm] {People chatting};
        \node (ticket) [perks, aspect=2.5, right of=tv, xshift = 2.5cm] {Queue machine};
        
        \draw [arrow_blue] (furniture) to (tv);
        \draw [arrow_blue] (keyboard) to ++(0,-1.5cm) to ++(5cm,0) to (conv1) to ++(0,-0.5cm) to ++(-5cm,0) to (people);
        \draw [arrow_blue] (telephone) to ++(0,-1.5cm) to ++(-5cm,0) to (conv1) to ++(0,-0.5cm) to ++(-5cm,0) to (people);
        \draw [arrow_blue] (conv1) to ++(0,-0.5cm) to ++(5cm,0) to (ticket);
        
        \node (clinic2) [scene, right of=ticket, above of = ticket, xshift = 0.5cm, yshift = -0.5cm] {Medical Clinic V2};
        
        \draw [arrow_red] (0.5cm,-5.5cm) to ++(0,-0.5cm) to ++(7.0cm,0) to (clinic2);
        
        \node (clinic3) [scene, right of=ticket, below of = ticket, xshift = 0.5cm] {Medical Clinic V3};
        
        \draw [arrow_red] (clinic2) to (clinic3);
        
        \node (conv2) [below of = tv, yshift = 0.75cm] {};
        
        \node (clinic4) [scene, below of = conv2, xshift = 0.5cm] {Medical Clinic V4};
        
        \draw [arrow_blue] (tv) to ++(0,-3.7cm);
        \draw [arrow_blue] (people) to ++(0,-1.25cm) to ++(5cm,0) to (conv2) to ++(0,-1.95cm);
        \draw [arrow_blue] (ticket) to ++(0,-1.25cm) to ++(-5cm,0) to (conv2) to ++(0,-1.95cm);
        \draw [arrow_red] (clinic3) to node[black,midway,above]{Auditorium complexity} ++(-7.0cm,0) to ++(0,-1.2cm);
        
        \node (exterior) [perks, aspect=2.5, below of=clinic4] {Exterior sounds};
        
        \draw [arrow] (clinic4) to node[midway, right]{1st test} (exterior);
        
        \node (clinic5) [start, below of = exterior] {Medical Clinic V5};
        
        \draw [arrow] (exterior) to (clinic5);
        
    \end{tikzpicture}
    }
        \centering
        \caption{Virtual environment development process}
        \label{fig:ve_process}
\end{figure}

\section{Procedures}

    The first step of the research was to learn how hospitals operate, especially throughout the COVID-19 pandemic. Two hospitals from the city of São José dos Campos - São Paulo were interviewed on how does the reception procedure worked and both of them had a similar operation:
    
    \begin{enumerate}
        \item Patient enters the hospital
        \item Uses the sanitizer to clean their hands
        \item Take a queue number and wait for the call of the receptionist
        \item Go to the receptionist and does his/her check-in
        \item Sits on the waiting area and wait until it's name is called \label{itm:name_call}
    \end{enumerate}
    
    The tasks in the experiment were to be similar to these procedures. The only exception was the name-calling, step \ref{itm:name_call}, because of the complexity of creating a routine inside the virtual environment that could call the participant's name. One possible solution was to use an actor, but because of the COVID-19 procedures that limit the number of people inside a room, this solution was discarded.
    
    Since the procedures were from hospitals, the first idea of a virtual environment was to build a virtual hospital reception.

\section{City Hospital}

    If the virtual environment was a hospital reception, it would be possible to include a lot of artifacts that could increase the participant's sense of presence, such as people walking and the sound of elevators, and that was very appealing.
    
    One problem with that idea was the physical space needed to simulate that. It would be needed a closed-quarters space with enough area to allow the participant to walk through the whole reception. The original space was approximately 15x20m and the laboratory, or the university, didn't have somewhere like that.
    
    So the solution for that was to shrink down the area to fit inside the laboratory, so it was decided not to simulate a hospital reception, but a medical clinic reception

\section{Medical Clinic}

    The laboratory didn't have a room that could fit a hospital reception, but it did have plenty of space that could fit a medical clinic reception, especially in the laboratory's auditorium. The laboratory has 7x10m and that was the dimension of the first version of the virtual medical clinic. At the first moment, it was decided that this would be the setting for the experiment and its development went towards the definition of the interior details (blue path on the Figure \ref{fig:ve_process}), but other problems appeared along with the development that the room dimensions needed to be redefined (red path on the Figure \ref{fig:ve_process}). Both of these modification are going to be detailed in the following \nameref{subsec:interior} and \nameref{subsec:exterior} subsections.
    
    \subsection{Interior}
    \label{subsec:interior}
    
        The goal of the interior was to increase the presence and feeling of the participant inside the virtual environment. The inspiration was from the typical objects and furniture that a patient notices when waiting in reception. The first objects positioned inside the reception were the desk, chairs (both normal and some with "X" in the seats to represent a COVID-19 procedure), a telephone and a laptop. The last two also emitted sounds to increase the feeling of presence and to point to the BVI participant where the reception desk was located. The telephone and the laptop had a C\# script to play their sounds randomly.
        
        This virtual environment was presented to two BVI members of the research team and they pointed out that it needed to have more noise, to increase even more the feeling of presence. They felt the lack of people chattering and the noise that came from a TV show, both were included in the virtual environment. To simulate the people chattering, dialogues from video or series between two people were used. The TV noise was made similarly, but with audio from famous Brazilian tv programs. Another missing artifact noticed by the team was the queue machine that was also included. All these added objects also had a script that played a specific dialogue/program/queue order for each created scene, never repeating once, to increase the sensation of a different day \footnote{During one of the experiments, a BVI participant commented that he/she felt different day times for each time he/she did the scene}. After all of these objects were included, the interior was ready for a trial.
        
    \subsection{Exterior}
    \label{subsec:exterior}
        
        The first version of the clinic had 7x10m, which was the exact dimension of the \textit{Auditório Romi}, the room that was selected to be the physical space for the experiment. Since it was the exact dimension, it became the first change, since an extra space is needed to place the two VR Stations, that in the experiment were assembled on a tripod basis. That modification became the second version of the clinic, with 5x7m.
        
        The second modification came from the maximum distance between the VR Stations. According to "SteamVR" (the software that was the interface between the computer and the VR) the maximum distance was 5m, besides that it could not guarantee the correct operation of the device. Besides that, the auditorium was filled with chairs and without a computer. Every time an experiment was going to be realized, it would be needed to rearrange the entire room, costing almost half an hour to clear the space and another half an hour to return to its place. The solution was to reduce, once more, the virtual environment dimensions to 4x4 and that was the third version of the medical clinic.
        
        The fourth version was reached because, even though smaller, the rearrangement of the auditorium was still a nuisance. The answer to that was to experiment in the entry hall. This was a space, just in front of the room where the computer with all the files was stationed. The only problem was that people passed by until 17h, but since the chosen auditorium was the physical space, it was scheduled that the experiment was going to be performed only during non-working hours.
        
    With these \nameref{subsec:exterior} and \nameref{subsec:interior} modification, the environment was ready to receive its first BVI participant.
        
\section{Adjustments}

    The first BVI participant was the blind member of the research team and he enjoyed the final result, but still found a thing that could help to increase the feeling of presence. He pointed out that BVI people normally find the exit of a room by searching the following sounds in sequence and repeatedly:
    \begin{enumerate}
        \item Sound of a door opening;
        \item Noise from an exterior space (like people walking, cars passing by, horns, etc.);
        \item Sound of a door closing.
    \end{enumerate}
    After that note, a sound-emitting point was added to each environment. This point played this sequence of sounds, but in a random interval.

\section{Final Clinic}

    After that last addition, the clinic reached its final version.




\chapter{Haptic belt development}
\label{ch:cinto}
Since haptic is one of the types of information that a BVI user can rely on, it was a good idea to test haptic devices in the experiment. These haptic devices would not detect the real object per se, but would receive the information from Unity3D based on the position of the user inside the virtual environment.
 
 The virtual cane was a simple development, since the controller already had a vibration motor inside of it. Knowing that, was only a matter to find the right commands and write an algorithm that worked. A pseudo-code is presented at Appendix \ref{ap:virtual_apend}. The two differences between the virtual cane and the haptic belt are the command to check the distance and the fact that with the cane the user must point to the direction where he/she wants to investigate if there is an obstacle whilst the belt indicates to the user the direction of the closest object.
 
 The idea to design a haptic belt came as a suggestion from one of the research members. It was possible to buy one directly from the internet but the cost was too high, so it was decided to assemble one from scratch. The project was based on a haptic compass \cite{kylecorry31_instructables_2020}, but instead of having the input being made by a magnetometer, it was made by the Unity3D.
 
 The first prototype was made using a Arduino Mega 2560, LEDs and a protoboard. If Unity3D could send a command to turn the LED on, then the software would be able to do the same with a coin vibrator. After checking the communication between Unity3D and Arduino, it was time to build the proper belt.
 
 The materials used were:
 \begin{itemize}
     \item DOIT ESP32 DevKit v1. (Datasheet in the Annex \ref{an:esp32_annex});
     \item A printed circuit board (PCB)
     \item A leather belt;
     \item 8 Coin Vibrator 1027;
     \item 16 female P2 jacks or PJ-320B;
     \item 16 P2 male or PJ cable connectors;
     \item 8 straps;
     \item Duct tape;
     \item A 3D printed case.
 \end{itemize}
 
 The first step was to correct and adapt the algorithm used on the Arduino to be used on ESP32 also using the LEDs. After it was made sure that it would also work with an ESP32, the system was designed on the EasyEDA website \cite{easyeda} them a PCB was ordered with the schematic presented in the Appendix \ref{ap:haptic_apend}. While the PCB didn't arrive the coin vibrator and the cables were being soldered. When the PCB arrived, it was time to solder the board P2 jacks and design a case for it, represented in the Figure \ref{fig:case_cinto}. After everything was soldered, printed, and connected it was ready, as is represented in the Figure \ref{fig:cinto_haptico}.
 

 
 Until this moment the belt was working cabled, but since the participant could walk great distances it was decided that the correct way to connect Unity3D with the ESP32 would be by wireless and it was decided to use a Bluetooth connection. The pseudo-code used in the development are in the Appendix \ref{ap:haptic_apend}.


\chapter{Results' analysis and discussion}
\label{ch:resultados}
%CAPÍTULO 4: ANÁLISE DOS RESULTADOS E DISCUSSÃO

%1. Elabore um parágrafo que introduz o capítulo: Este capítulo apresenta (descreva o objetivo do capítulo ...). É constituído de N seções a saber...
%2. Caso vc tenha aplicado a sua contribuição (modelo, produto, processo etc.) em um caso (empresa, laboratório, simulação etc.), apresente a descrição e análise dos resultados. Na seção de discussão cabem as análises de cenários What-If ou de sensibilidade. Exemplo: se o parâmetro X aumentar de N para N+1, o resultado poderia mudar de Y para Z?
%3. Elabore um parágrafo que conclui o capítulo e introduz o capítulo seguinte.

The purpose of the experiment discussed in this chapter is to investigate the two research questions proposed for this work:

\begin{itemize}%[leftmargin = 3.5em, label = $H_\arabic*$:]
    \item Is it possible to evaluate and compare concepts of assistive device from a human factors’ perspective in a virtual environment? What are the main limitations of the use of a virtual reality environment?
    \item Do non-BVI users, when deprived from their vision, evaluate assistive devices in a similar way as BVI users?
\end{itemize}

For this purpose, the experiment described in Section \ref{ch:metodologia} was performed with the following groups:

\begin{itemize}
    \item Blind group: composed of 4 participants with age varying from 26 to 56, all male, three of them graduated and one with graduation in course. 

    \item Sighted group: composed of 4 participants with age varying from 22 to 31, three males and one woman, all of them graduated.
\end{itemize}

In order to answer the two research questions, this chapter is organized in the following way. Section \ref{sec:results_obj_1} is dedicated to the first question and brings an analysis performed only with data from blind participants. Then, Section \ref{sec:results_obj_2} repeats the same analysis now with data from sighted participants and compare the results with those obtained from blind participants, in order to answer the second research question.

In both sections, the data analysis follows the following sequence:

\begin{itemize}
    \item Analysis of subjective questionnaires:
    \begin{itemize}
        \item NASA-TLX: it aims at assessing the workload perceived by the user in six dimensions, including ‘mental demand’. It is expected that the mental workload would decrease from the ‘first’ to the ‘return’ round. It is also expected that some guidance methods would differ regarding the required mental workload.
        \item Adapted SAGAT: it aims at assessing the situation awareness and the mental map of the user. It is expected that the SAGAT score would increase from the ‘first’ to the ‘return’ round. It is also expected that some guidance methods would differ regarding the required situation awareness provided to the user.
        \item Guidance method’s questionnaire: it aims at assessing the user experience with each method. It is also expected that some guidance methods would differ regarding the score received in this questionnaire.
    \end{itemize}
    \item Analysis of physiological sensors:
    \begin{itemize}
        \item ECG: it aims at assessing the user workload. Two features are extracted from the ECG signal, heartrate (BPM) and heartrate variance (SDNN). It is expected that the heartrate slight decreases from the ‘first’ to the ‘return’ round, while the heartrate variance is expected to slight increase.
        \item GSR: it aims at assessing the user workload and stress. It is expected that the GSR average would increase at every ‘first’ round and then a slight decrease in the ‘return’ round.
    \end{itemize}
\end{itemize}

Particularly in the case of this work, an additional round of tests is added to the experiment in order to investigate the differences between the evaluation performed by BVI users and sighted (non-BVI) users. For this purpose, the experiment is repeated with a set of non-BVI users and the same data is collected. The purpose is to investigate whether or not performing the analysis with non-BVI users could lead to different conclusions.


\section{Evaluation of assistive device from a human factors’ perspective in a virtual environment}
\label{sec:results_obj_1}

\subsection{Subjective data}
\subsubsection{NASA-TLX}
\label{subsubsec:results_nasa_tlx_1}

The NASA-TLX provides two information that are relevant to the workload analysis. The first one is the score attributed to the ‘mental demand’ dimension. The second one is the average obtained from the scores of the six dimensions of NASA-TLX. The two analyses are presented in the next subsections.

\paragraph{Analysis of the mental demand scale}\mbox{}\\

Table \ref{tab:md_table_blind} presents the ‘mental demand’ score attributed by each blind participant to each guidance method. The ‘base’ method refers to the guidance method that the person uses in his/her daily life (e.g., white cane). 


\begin{table}[!htb]
\centering
\caption{Score of NASA-TLX mental demand for the blind participants.}
\label{tab:md_table_blind}
\begin{tabular}{llrrrrr}
\toprule
     &        & Base & Audio & \begin{tabular}[c]{@{}l@{}}Haptic\\ Belt\end{tabular} & \begin{tabular}[c]{@{}l@{}}Virtual\\ Cane\end{tabular} & Mixture \\
Participant & Round &      &       &                                                       &                                                        &         \\
\midrule
001C & First &    3 &     1 &                                                    14 &                                                      3 &       6 \\
     & Return &    1 &     1 &                                                    10 &                                                      2 &       6 \\
002C & First &    5 &     1 &                                                     1 &                                                     10 &      12 \\
     & Return &    1 &     1 &                                                     1 &                                                     10 &       3 \\
003C & First &    5 &     5 &                                                     5 &                                                      8 &       1 \\
     & Return &    3 &     1 &                                                     1 &                                                      2 &       1 \\
004C & First &    9 &    10 &                                                    15 &                                                     10 &      10 \\
     & Return &    7 &    10 &                                                    14 &                                                      8 &      10 \\
\bottomrule
\end{tabular}
\end{table}



The mean value obtained for each guidance method is illustrated in Figure \ref{fig:barplot_md_avg_5_scene_blind}. It shows a systematic reduction on the perceived mental workload between the rounds for all methods, confirming that the participants get familiar with the devices after the first use. It also shows that although the haptic belt obtained the largest mean, it also had the largest variation, showing that the effort required from the user may vary significantly.

\begin{figure}[!htb]
    \centering
    \includegraphics[width = 0.8\linewidth]{Resultados/Nasa/Figuras/png/barplot_md_avg_5_scene_blind.png}
    \caption{Mean and standard deviation of mental demand of blind participants for each method.}
    \label{fig:barplot_md_avg_5_scene_blind}
\end{figure}

Figure \ref{fig:boxplot_md_blind_scene}  presents a box plot of the mental demand score grouped by method. This Figure shows that there may be two different groups: one associated with lower demand, composed of ‘base’ and ‘audio’, and another with higher demand, composed of ‘haptic belt’, ‘virtual cane’ and ‘mixture’. It indicates that maybe a guidance method that uses vibration as input is not intuitive. Following, Figure \ref{fig:boxplot_md_blind_rounds} presents a box plot of the mental demand grouped by the rounds, confirming the general tendency to reduce the required ‘mental demand’. 

\begin{figure}[!htb]
    \centering
    \begin{minipage}{0.45\textwidth}
        \centering
        \includegraphics[width = 0.8\linewidth]{Resultados/Nasa/Figuras/png/boxplot_md_blind_scene.png}
        \caption{Boxplot of the mental demand of the blind participants grouped by method.}
        \label{fig:boxplot_md_blind_scene}
    \end{minipage}
    \begin{minipage}{0.45\textwidth}
        \centering
        \includegraphics[width = 0.8\linewidth]{Resultados/Nasa/Figuras/png/boxplot_md_blind_rounds.png}
        \caption{Boxplot of the mental demand of the blind participants grouped by round.}
        \label{fig:boxplot_md_blind_rounds}
    \end{minipage}
\end{figure}

In order to support the statistical analysis, Figures \ref{fig:qqplot_md_avg_two_way_blind} and \ref{fig:residplot_md_avg_two_way_blind} presents the QQ-plot and the residual plot of the ‘mental demand’ data, confirming that the data follow a normal distribution and the residues are homogenous.

Following, Figures \ref{fig:qqplot_md_avg_two_way_blind} and \ref{fig:residplot_md_avg_two_way_blind} shows the distribution and variance of the Table \ref{tab:md_table_blind}. These Figures shows that the data are normally distributed and that the methods have a similar variance. The Table \ref{tab:blocanova_md_avg_two_way_blind} shows the Anova test p-values of the mental demand of the ‘blind” sample between the guidance methods. The method’s and the round’s p-values indicates that there is no influence from them in the mental demand. The interaction between the methods and the round also does not influences the mental demand.

\begin{figure}[!htb]
    \centering
    \begin{minipage}{0.45\textwidth}
        \centering
        \includegraphics[width = 0.8\linewidth]{Resultados/Nasa/Figuras/png/qqplot_md_avg_two_way_blind.png}
        \caption{QQ plot of the mental demand of the blind participants on each method.}
        \label{fig:qqplot_md_avg_two_way_blind}
    \end{minipage}
    \begin{minipage}{0.45\textwidth}
        \centering
        \includegraphics[width = 0.8\linewidth]{Resultados/Nasa/Figuras/png/residplot_md_avg_two_way_blind.png}
        \caption{Residual plot of the mental demand score the blind participants on each method.}
        \label{fig:residplot_md_avg_two_way_blind}
    \end{minipage}
\end{figure}

Following, the statistical model of Equation 5.1 is used for the analysis of variance (ANOVA): 

\begin{equation}
    \label{eq:statistical_model}
    y_{ijk} = \mu + \tau_i + \beta_j + \msout{\omega_k} + (\tau\beta_{ij}) + e
\end{equation}

where:

\begin{itemize}
    \item $y_{ij}$ - output variable for method $i$, round $j$ and participant $k$;
    \item $\mu$ - mean of all the observations;
    \item $\tau_i$ - variance from method $i$;
    \item $\beta_j$ - variance from round $j$;
    \item \sout{$\omega_k$} - variance from participant k, which treated as a block;
    \item $\tau\beta_{ij}$ - combined variance from the interaction between method i and round j;
    \item $e$ - residual error.
\end{itemize}

The results of ANOVA are presented in Table \ref{tab:blocanova_md_avg_two_way_blind}. ANOVA tests the hypothesis that the means of independent groups of data are equal or not. In the literature, a p-value of 0.05 is commonly adopted as a threshold to confirm the hypothesis. A p-value < 0.05 indicates that the means of the groups are statistically different with 95\% of confidence. According to this criterion neither method or round have a significant influence on the mental demand.

However, due to the low number of participants, the threshold of 0.1 could also be considered. In this case, it indicates, with 90\% confidence, that the mean of the ‘first’ and ‘return’ rounds are different. For guidance method, the p-value of 0.170 is close to the threshold but slightly higher, suggesting that the means may be different but this hypothesis is not statistically confirmed with the current data. 


\begin{table}[!htb]
\centering
\caption{Anova p-value for the mental demand average on each method for blinded users.}
\label{tab:blocanova_md_avg_two_way_blind}
\begin{tabular}{lrrrrl}
\toprule
               Source &  Squared sum &  DOF & Squared average &     F & \begin{tabular}[c]{@{}l@{}}P-Value \\ $(F_{0} > F)$\end{tabular} \\
\midrule
Participants (Blocks) &      298.475 &    3 &          99.492 & 8.133 &                                                                  \\
         \    Methods &       85.150 &    4 &          21.288 & 1.740 &                                                            0.170 \\
          \    Rounds &       42.025 &    1 &          42.025 & 3.436 &                                                            0.075 \\
     \    Interaction &        2.850 &    4 &           0.712 & 0.058 &                                                            0.993 \\
   Experimental Error &      330.275 &   27 &          12.232 &       &                                                                  \\
                Total &      758.775 &   39 &                 &       &                                                                  \\
\bottomrule
\end{tabular}
\end{table}



In order to conclude the analysis of the NASA-TLX mental demand, Table \ref{tab:md_var_average_group_blind} brings the average of difference between the mental demand of the ‘first’ and ‘return’ rounds. Unexpectedly, it shows that the largest variation is obtained to the ‘base’, i.e., the guidance method the participant is used to and, therefore, should not present a significant variation. The method with the lower variation was ‘audio’, probably because it already had a very low score in the first round. 


\begin{table}[!htb]
\centering
\caption{Mental demand variation grouped by participant and visual condition}
\label{tab:md_var_average_group_blind}
\begin{tabular}{lrrrrrr}
\toprule
{} &  Base & Audio & \begin{tabular}[c]{@{}l@{}}Haptic\\ Belt\end{tabular} & \begin{tabular}[c]{@{}l@{}}Virtual\\ Cane\end{tabular} & Mixture \\
Visual Condition &       &       &                                                       &                                                        &         \\
\midrule
Blind            &  -2.5 &  -1.0 &                                                  -2.2 &                                                   -2.2 &    -2.2 \\
\bottomrule
\end{tabular}
\end{table}



\FloatBarrier

%%%%%%%%%%%%%%%%%%%%%%%%%%%%%%%%%%%%%%%%%%%%%%%%%%%%%%%%%%%%%%%%%%%%%%%%%%%
%%%%%%%%%%%%%%%%%%%%%%%%%%%%%%%%%%%%%%%%%%%%%%%%%%%%%%%%%%%%%%%%%%%%%%%%%%%
%%%%%%%%%%%%%%%%%%%%%%%%%%%%%%%%%%%%%%%%%%%%%%%%%%%%%%%%%%%%%%%%%%%%%%%%%%%
%%%%%%%%%%%%%%%%%%%%%%%%%%%%%%%%%%%%%%%%%%%%%%%%%%%%%%%%%%%%%%%%%%%%%%%%%%%


\paragraph{Analysis of the NASA-TLX score}\mbox{}\\

This section repeats the analysis steps of the previous section but now considering the mean value of all dimension of NASA-TLX, referred in this text as global score. Table \ref{tab:nasa_table_blind} presents the global score of each blind participant. 


\begin{table}[!htb]
\centering
\caption{NASA-TLX score felled by the blinded participants.}
\label{tab:nasa_table_blind}
\begin{tabular}{llrrrrr}
\toprule
     &        &  Base &  Audio & \begin{tabular}[c]{@{}l@{}}Haptic\\ Belt\end{tabular} & \begin{tabular}[c]{@{}l@{}}Virtual\\ Cane\end{tabular} & Mixture \\
Participant & Round &       &        &                                                       &                                                        &         \\
\midrule
001C & First & 4.833 &  4.000 &                                                 8.833 &                                                  5.167 &   6.333 \\
     & Return & 4.167 &  4.000 &                                                 6.667 &                                                  4.500 &   6.167 \\
002C & First & 6.333 &  4.833 &                                                 4.833 &                                                  9.000 &   7.000 \\
     & Return & 4.500 &  4.833 &                                                 4.833 &                                                  7.000 &   5.167 \\
003C & First & 4.000 &  4.000 &                                                 5.333 &                                                  6.667 &   3.500 \\
     & Return & 4.000 &  3.833 &                                                 3.667 &                                                  3.500 &   3.500 \\
004C & First & 9.833 & 10.000 &                                                12.667 &                                                  9.667 &  11.000 \\
     & Return & 8.667 &  9.167 &                                                11.667 &                                                  9.333 &  10.833 \\
\bottomrule
\end{tabular}
\end{table}



Figure \ref{fig:barplot_nasa_avg_5_scene_blind} brings the corresponding barplot with the mean value and standard deviation for each guidance method and each round. In a qualitative comparison with Figure \ref{fig:barplot_md_avg_5_scene_blind}, the differences between the methods are confirmed but softened. It is possible to notice that the mean score of ‘audio’ and ‘base’ are still lower than that of the other methods. The differences between ‘first’ and ‘return’ rounds are also reduced. However, the standard deviation among are also considerably reduced for all methods, and especially for the haptic belt.

\begin{figure}[!htb]
    \centering
    \includegraphics[width = 0.8\linewidth]{Resultados/Nasa/Figuras/png/barplot_nasa_avg_5_scene_blind.png}
    \caption{Barplot of the average NASA-TLX score of the blind participants on each method.}
    \label{fig:barplot_nasa_avg_5_scene_blind}
\end{figure}

Figure \ref{fig:boxplot_nasa_blind_scene} presents the box plot with the NASA-TLX global score grouped by method. Similar to what happened for the ‘mental demand’, it shows that it possible to split the methods in two different groups: ’base’ and ‘audio’, which requires a lower level of workload, and another group, which requires a higher level. Figure \ref{fig:boxplot_nasa_blind_rounds} presents a box plot with the NASA-TLX global score grouped by the rounds, apparently showing that the two groups are still different. 

\begin{figure}[!htb]
    \centering
    \begin{minipage}{0.45\textwidth}
        \centering
        \includegraphics[width = 0.8\linewidth]{Resultados/Nasa/Figuras/png/boxplot_nasa_blind_scene.png}
        \caption{QQ plot of the NASA-TLX score of the blind participants on each method.}
        \label{fig:boxplot_nasa_blind_scene}
    \end{minipage}
    \begin{minipage}{0.45\textwidth}
        \centering
        \includegraphics[width = 0.8\linewidth]{Resultados/Nasa/Figuras/png/boxplot_nasa_blind_rounds.png}
        \caption{Residual plot of the NASA-TLX score the blind participants on each method.}
        \label{fig:boxplot_nasa_blind_rounds}
    \end{minipage}
\end{figure}

%The Table \ref{tab:nasa_average_group_blind} shows the average NASA-TLX score in the “blind” sample and is possible to notice how the average score by the “blind” sample was lower during the “Audio” and the “Base” methods.
%
%
\begin{table}[!htb]
\centering
\caption{Average NASA-TLX score of the blind participants}
\label{tab:nasa_average_group_blind}
\begin{tabular}{lrrrrrr}
\toprule
{} &  Base & Audio & \begin{tabular}[c]{@{}l@{}}Haptic\\ Belt\end{tabular} & \begin{tabular}[c]{@{}l@{}}Virtual\\ Cane\end{tabular} &  Mixture \\
Visual Condition &       &       &                                                       &                                                        &          \\
\midrule
Blind            &  5.79 &  5.58 &                                                  7.31 &                                                   6.85 &    6.688 \\
\bottomrule
\end{tabular}
\end{table}



Figures \ref{fig:qqplot_nasa_avg_two_way} and \ref{fig:residplot_nasa_avg_two_way} presents the QQ plot and residual distribution of the NASA-TLX global score, showing that apparently the data are normally distributed, but the residuals are not so homogeneous as in the previous case, showing that the participants have different variability among them.

\begin{figure}[!htb]
    \centering
    \begin{minipage}{0.45\textwidth}
        \centering
        \includegraphics[width = 0.8\linewidth]{Resultados/Nasa/Figuras/png/qqplot_nasa_avg_two_way.png}
        \caption{QQ plot of the NASA-TLX score variation of the blind participants on each method.}
        \label{fig:qqplot_nasa_avg_two_way}
    \end{minipage}
    \begin{minipage}{0.45\textwidth}
        \centering
        \includegraphics[width = 0.8\linewidth]{Resultados/Nasa/Figuras/png/residplot_nasa_avg_two_way.png}
        \caption{Residual plot of the NASA-TLX score variation the blind participants on each method.}
        \label{fig:residplot_nasa_avg_two_way}
    \end{minipage}
\end{figure}

Table \ref{tab:blocanova_nasa_avg_two_way} brings the p-value resulting from ANOVA. In this case, both the method and round where appointed as significant variables that influence the mean value of the NASA-TLX global score. 


\begin{table}[!htb]
\centering
\caption{Anova p-value for the NASA-TLX score on each method for blinded users.}
\label{tab:blocanova_nasa_avg_two_way}
\begin{tabular}{lrrrrl}
\toprule
               Source &  Squared sum &  DOF & Squared average &      F & \begin{tabular}[c]{@{}l@{}}P-Value \\ $(F_{0} > F)$\end{tabular} \\
\midrule
Participants (Blocks) &      211.041 &    3 &          70.347 & 51.869 &                                                                  \\
         \    Methods &       17.185 &    4 &           4.296 &  3.168 &                                                          0.029** \\
          \    Rounds &        7.951 &    1 &           7.951 &  5.862 &                                                          0.022** \\
     \    Interaction &        2.115 &    4 &           0.529 &  0.390 &                                                            0.814 \\
   Experimental Error &       36.619 &   27 &           1.356 &        &                                                                  \\
                Total &      274.910 &   39 &                 &        &                                                                  \\
\bottomrule
\end{tabular}
\end{table}



Finally, Table \ref{tab:lsd_nasa_avg_two_way} presents the results of a pairwise Fisher LSD test comparing each pair of guidance method. The results show that only ‘audio’ is similar ‘base’, all the other methods are different among each other

\input{Resultados/Nasa/Tabelas/lsd_nasa_avg_two_way}

Table \ref{tab:nasa_var_group_blind} shows the difference in the NASA-TLX global score between the first and return rounds. It shows that the ‘audio’ difference is the lowest among all methods, while the highest difference is for the ‘virtual cane’.

%The Table \ref{tab:nasa_var_group_blind} shows the average of the NASA-TLX score variation between the rounds. This table shows that the variation from the “Audio” was the lowest variation and the highest variation was the "Virtual Cane".

\input{Resultados/Nasa/Tabelas/nasa_var_group_blind}

%The Figures \ref{fig:qqplot_nasa_var} and \ref{fig:residplot_nasa_var} shows the distribution and variance of the NASA-TLX score variation of the Table \ref{tab:nasa_table_blind}. These Figures shows that the data are normally distributed and that the methods have a similar variance.
%The Table \ref{tab:blocanova_nasa_var} shows the Anova test p-value of the NASA-TLX score of the "blind" sample between the guidance methods. The p-value indicates that there are no difference between the variation of any method. 
%
%
\begin{table}[!htb]
\centering
\caption{Anova p-value for the NASA-TLX score variation on each method for blinded users.}
\label{tab:blocanova_nasa_var}
\begin{tabular}{lrrrrr}
\toprule
Source & P-Value \\
\midrule
Method &   0.402 \\
\bottomrule
\end{tabular}
\end{table}


%
%\begin{figure}[!htb]
%    \centering
%    \begin{minipage}{0.45\textwidth}
%        \centering
%        \includegraphics[width = 0.8\linewidth]{Resultados/Nasa/Figuras/png/qqplot_nasa_var.png}
%        \caption{Bar plot of the average NASA-TLX score of the blind participants on each method.}
%        \label{fig:qqplot_nasa_var}
%    \end{minipage}
%    \begin{minipage}{0.45\textwidth}
%        \centering
%        \includegraphics[width = 0.8\linewidth]{Resultados/Nasa/Figuras/png/residplot_nasa_var.png}
%        \caption{Bar plot of the average NASA-TLX score of the sighted participants on each method.}
%        \label{fig:residplot_nasa_var}
%    \end{minipage}
%\end{figure}
%
%%The Table \ref{tab:lsdbloc_nasa_var} presents the conclusion of a pairwise Fisher LSD test of the blind NASA-TLX score between all the guidance methods. The results show that all methods have similar variations.
%
%%\input{Resultados/Nasa/Tabelas/lsdbloc_nasa_var}
%
%To close up, according to the LSD test at Table \ref{tab:lsd_nasa_avg_two_way} only the "Audio" method has a NASA-TLX score that could be said to be similar to the "Base" method, which indicates that the existance of an haptic device increased the NASA-TLX score and that the round has some impact on the score, which means that there was a learning effect from the "First" to the "Return" round. Probably this effect was reflected in the other dimensions of the NASA-TLX.
%
%The \ref{tab:blocanova_nasa_avg_two_way} concludes that the rounds and the interaction between the rounds and the methods have no influence on the variation of the NASA-TLX score.
%
\FloatBarrier
\subsubsection{Adapted SAGAT}
\label{subsubsec:results_adapted_sagat_1}

This section discusses the results of the adapted SAGAT questionnaire, which aims at assessing the participant situation awareness and mental map of the environment. 

For each question of the SAGAT questionnaire, the participant could score 1 point or a fraction of it. The total score achieved by each blind participant is presented in the Table \ref{tab:sagat_table_blind}. Figure  \ref{fig:barplot_sagat_avg_5_scene_blind} illustrates the corresponding bar plot, indicating the mean and standard deviation for each guidance method and each round. This figure shows clearly that the participants improved its situation awareness in the return round, when they already had some information about the environment. Also, it is possible to observe that the worst situation awareness is obtained in the ‘first’ round for the ‘virtual cane’. However, on the ‘return’ round, the SAGAT mean score becomes equivalent to that of the ‘audio’ method.


\begin{table}[!htb]
\centering
\caption{SAGAT global score felled by the blinded participants.}
\label{tab:sagat_table_blind}
\begin{tabular}{llrrrrr}
\toprule
     &        &   Base &  Audio & \begin{tabular}[c]{@{}l@{}}Haptic\\ Belt\end{tabular} & \begin{tabular}[c]{@{}l@{}}Virtual\\ Cane\end{tabular} & Mixture \\
Participant & Round &        &        &                                                       &                                                        &         \\
\midrule
001C & First &   6.25 &   5.50 &                                                  5.33 &                                                   5.83 &   3.500 \\
     & Return &   6.25 &   6.50 &                                                  8.50 &                                                   5.50 &   5.500 \\
002C & First &   6.75 &   4.50 &                                                  3.99 &                                                   4.50 &   6.250 \\
     & Return &   5.25 &   5.00 &                                                  4.00 &                                                   6.50 &   8.500 \\
003C & First &   7.25 &   7.50 &                                                  7.49 &                                                   4.66 &   9.000 \\
     & Return &  10.00 &  10.00 &                                                  8.50 &                                                   9.00 &   9.000 \\
004C & First &   7.50 &   6.00 &                                                  7.66 &                                                   4.99 &   6.500 \\
     & Return &   9.00 &   6.00 &                                                  9.25 &                                                   7.25 &   9.000 \\
\bottomrule
\end{tabular}
\end{table}



\begin{figure}[!htb]
    \centering
    \includegraphics[width = 0.8\linewidth]{Resultados/Sagat/Figuras/png/barplot_sagat_avg_5_scene_blind.png}
    \caption{Barplot of the average SAGAT score of the blind participants on each method.}
    \label{fig:barplot_sagat_avg_5_scene_blind}
\end{figure}

Figure \ref{fig:boxplot_sagat_blind_scene} brings the boxplot of the SAGAT score grouped by guidance method. It shows that the methods can be divided in two groups. The first one is composed of ‘base’, ‘haptic belt’ and the ‘mixture’. This group received scores higher than the second group, composed of ‘audio’ and ‘virtual cane’. Following, Figure \ref{fig:boxplot_sagat_blind_rounds} shows the boxplot of the data grouped by round and confirms the general improvement of situation awareness from the ‘first’ to the ‘return’ round. 

\begin{figure}[!htb]
    \centering
    \begin{minipage}{0.45\textwidth}
        \centering
        \includegraphics[width = 0.8\linewidth]{Resultados/Sagat/Figuras/png/boxplot_sagat_blind_scene.png}
        \caption{Boxplot of the SAGAT score of the blind participants grouped by method.}
        \label{fig:boxplot_sagat_blind_scene}
    \end{minipage}
    \begin{minipage}{0.45\textwidth}
        \centering
        \includegraphics[width = 0.8\linewidth]{Resultados/Sagat/Figuras/png/boxplot_sagat_blind_rounds.png}
        \caption{Boxplot of the SAGAT score of the blind participants grouped by round.}
        \label{fig:boxplot_sagat_blind_rounds}
    \end{minipage}
\end{figure}

%The Table \ref{tab:sagat_average_group_blind} shows the average SAGAT score in the “blind” sample and is possible to notice how the average score by the “blind” sample was lower during the “Audio” and the “Base” methods.
%
%
\begin{table}[!htb]
\centering
\caption{SAGAT score average grouped by participant and visual condition}
\label{tab:sagat_average_group_blind}
\begin{tabular}{lrrrrrr}
\toprule
{} &  Base & Audio & \begin{tabular}[c]{@{}l@{}}Haptic\\ Belt\end{tabular} & \begin{tabular}[c]{@{}l@{}}Virtual\\ Cane\end{tabular} &  Mixture \\
Visual Condition &       &       &                                                       &                                                        &          \\
\midrule
Blind            &  7.28 &  6.38 &                                                  6.84 &                                                   6.03 &    7.156 \\
\bottomrule
\end{tabular}
\end{table}



Proceeding to the statistical analysis of the data, Figures \ref{fig:qqplot_sagat_avg_two_way_blind} and \ref{fig:residplot_sagat_avg_two_way_blind} present the QQ plot and the residual distribution, which confirms the normal distribution assumption and the homogeneity of variances

\begin{figure}[!htb]
    \centering
    \begin{minipage}{0.45\textwidth}
        \centering
        \includegraphics[width = 0.8\linewidth]{Resultados/Sagat/Figuras/png/qqplot_sagat_avg_two_way_blind.png}
        \caption{QQ plot of the SAGAT score of the blind participants on each method.}
        \label{fig:qqplot_sagat_avg_two_way_blind}
    \end{minipage}
    \begin{minipage}{0.45\textwidth}
        \centering
        \includegraphics[width = 0.8\linewidth]{Resultados/Sagat/Figuras/png/residplot_sagat_avg_two_way_blind.png}
        \caption{Residual plot of the SAGAT score the blind participants on each method.}
        \label{fig:residplot_sagat_avg_two_way_blind}
    \end{minipage}
\end{figure}

Finally, Table \ref{tab:blocanova_sagat_avg_two_way_blind} shows the Anova test p-value of the SAGAT score. It indicates that the round is a significant variable that influences the value of the SAGAT score. The same cannot be said for the method, which, apparently, has no significant influence.


\begin{table}[!htb]
\centering
\caption{Anova p-value for the SAGAT score on each method for blinded users.}
\label{tab:blocanova_sagat_avg_two_way_blind}
\begin{tabular}{lrrrrr}
\toprule
          Source & P-Value \\
\midrule
    \    Methods &   0.277 \\
     \    Rounds & 0.002** \\
\    Interaction &   0.834 \\
\bottomrule
\end{tabular}
\end{table}



%The Table \ref{tab:lsd_sagat_avg_two_way} presents the conclusion of a pairwise Fisher LSD test of the blind NASA-TLX score between all the guidance methods. The results show that only the "Audio" has a similar NASA-TLX score as the "Base" method, as it was also posible to notice at Figure \ref{fig:boxplot_sagat_blind_scene}.

%\input{Resultados/Sagat/Tabelas/lsd_sagat_avg_two_way}

Finally, Table \ref{tab:sagat_var_group_blind} brings the mean difference in the SAGAT score between the first and return round for each guidance method. It shows that the ‘base’ and ‘audio’ methods have the lowest difference, while the highest one was obtained for the ‘virtual cane’.


\begin{table}[!htb]
\centering
\caption{Adapted Sagat global score variation grouped by participant and visual Condition}
\label{tab:sagat_var_group_blind}
\begin{tabular}{lrrrrrr}
\toprule
{} &  Base &  Audio & \begin{tabular}[c]{@{}l@{}}Haptic\\ Belt\end{tabular} & \begin{tabular}[c]{@{}l@{}}Virtual\\ Cane\end{tabular} & Mixture \\
Visual Condition &       &        &                                                       &                                                        &         \\
\midrule
Blind            &  8.93 &  15.66 &                                                 23.49 &                                                  44.30 &   32.90 \\
\bottomrule
\end{tabular}
\end{table}



%The Figures \ref{fig:qqplot_sagat_var_blind} and \ref{fig:residplot_sagat_var_blind} shows the distribution and variance of the SAGAT score variation of the Table \ref{tab:sagat_table_blind}. These Figures shows that the data are normally distributed and that the methods have a similar variance.
%The Table \ref{tab:blocanova_sagat_var_blind} shows the Anova test p-value of the SAGAT score of the "blind" sample between the guidance methods. The p-value indicates that there are no difference between the variation in any method. 
%
%
\begin{table}[!htb]
\centering
\caption{Anova p-value for the SAGAT score variation on each method for blinded users.}
\label{tab:blocanova_sagat_var_blind}
\begin{tabular}{lrrrrr}
\toprule
               Source &  Squared sum &  DOF & Squared average &     F & \begin{tabular}[c]{@{}l@{}}P-Value \\ $(F_{0} > F)$\end{tabular} \\
\midrule
Participants (blocks) &     1176.902 &    3 &         782.885 & 0.473 &                                                                  \\
               Method &     3131.542 &    4 &         392.301 & 0.944 &                                                            0.472 \\
   Experimental error &     9956.458 &   12 &         829.705 &       &                                                                  \\
                Total &    14264.902 &   19 &                 &       &                                                                  \\
\bottomrule
\end{tabular}
\end{table}


%
%\begin{figure}[!htb]
%    \centering
%    \begin{minipage}{0.45\textwidth}
%        \centering
%        \includegraphics[width = 0.8\linewidth]{Resultados/Sagat/Figuras/png/qqplot_sagat_var_sight.png}
%        \caption{QQ plot of the SAGAT score variation of the blind participants on each method.}
%        \label{fig:qqplot_sagat_var_blind}
%    \end{minipage}
%    \begin{minipage}{0.45\textwidth}
%        \centering
%        \includegraphics[width = 0.8\linewidth]{Resultados/Sagat/Figuras/png/residplot_sagat_var_sight.png}
%        \caption{Residual plot of the SAGAT score variation of the blind participants on each method.}
%        \label{fig:residplot_sagat_var_blind}
%    \end{minipage}
%\end{figure}
%
%%The Table \ref{tab:lsdbloc_nasa_var} presents the conclusion of a pairwise Fisher LSD test of the blind NASA-TLX score between all the guidance methods. The results show that all methods have similar variations.
%
%%\input{Resultados/Nasa/Tabelas/lsdbloc_nasa_var}
%
%To close up, according to the ANOVA test at Table \ref{tab:blocanova_sagat_avg_two_way_blind} the methods caused no reaction on the SAGAT score, but the rounds did. That means that the participants were able in all methods to learn a little about their environment and that learning impacted their environmental perception in the next round. The fact that the test has not found any influence of the methods on the SAGAT score may be because of the small sample size, since it is posible to notice a difference between the methods at Figure \ref{fig:boxplot_sagat_blind_scene}. Also the interaction between method and round caused no influence in the Sagat score. According to the ANOVA test at Table \ref{tab:blocanova_sagat_var_blind}, the methods did not influenced the SAGAT score.
%
\FloatBarrier
\subsubsection{Guidance method's questionnaire.}
\label{subsubsec:results_questionnaires}

The data from the questionnaire for evaluating the user experience with each guidance method is also analysed. The higher the score, the more satisfied the user is with the method. It is essential to observe that this analysis does not include the base method as the questions are specific about each method and the base may vary among the participants. Also, there is no distinction between first and return rounds. Each questionnaire is answered only once for each method.

Table \ref{tab:questionnaire_average_blind} presents the score attributed to each method by each participant. The mean values are plotted in Figure \ref{fig:barplot_questionnaire_scene_blind} and show a dissatisfaction with the methods that only use vibration for communicating with the participant, i.e., the haptic belt and the virtual cane. 


\begin{table}[!htb]
\centering
\caption{ Guidance method questionnaire score felled by the blinded participants.}
\label{tab:questionnaire_average_blind}
\begin{tabular}{llrrrrr}
\toprule
{} &  Audio &  \begin{tabular}[c]{@{}l@{}}Haptic\\ Belt\end{tabular} &  \begin{tabular}[c]{@{}l@{}}Virtual\\ Cane\end{tabular} &  Mixture \\
Participant &        &                                                        &                                                         &          \\
\midrule
001C        &  0.774 &                                                  0.543 &                                                   0.629 &    0.865 \\
002C        &  0.857 &                                                  0.743 &                                                   0.543 &    0.935 \\
003C        &  0.929 &                                                  0.571 &                                                   0.543 &    0.745 \\
004C        &  0.881 &                                                  0.486 &                                                   0.400 &    0.730 \\
\bottomrule
\end{tabular}
\end{table}



\begin{figure}[!htb]
    \centering
    \includegraphics[width = \textwidth]{Resultados/Questionario/Figuras/pdf/barplot_questionnaire_scene_blind.pdf}
    \caption{Barplot of the average questionaire score of the blind participants on each method.}
    \label{fig:barplot_questionnaire_scene_blind}
\end{figure}

Figure \ref{fig:boxplot_quest_blind_scene} brings the questionnaire boxplot, which clearly shows the difference between two groups: haptic belt and virtual cane, and audio and mixture. 

\begin{figure}[!htb]
    \centering
    \includegraphics[width = 0.45\textwidth]{Resultados/Questionario/Figuras/pdf/boxplot_questionnaire_scene_blind.pdf}
    \caption{Boxplot of the questionaire score of the blind participants grouped by the methods.}
    \label{fig:boxplot_quest_blind_scene}
\end{figure}

%The Table \ref{tab:questionnaire_average_group_blind} show the the average questionnaire score on each method. It also shows a disatisfaction with the haptic devices alone.
%
%
\begin{table}[!htb]
\centering
\caption{Guidance method questionnaire average score for the blind participants.}
\label{tab:questionnaire_average_group_blind}
\begin{tabular}{lrrrrr}
\toprule
{} & Audio & Haptic Belt & Virtual Cane & Mixture \\
Visual Condition &       &             &              &         \\
\midrule
Blind            &  0.86 &        0.59 &         0.53 &    0.82 \\
\bottomrule
\end{tabular}
\end{table}



Figures \ref{fig:qqplot_questionnaire_blind} and \ref{fig:residplot_questionnaire_blind} show that the data follows a normal distribution. However, the residual variance is not strictly homogenous among the participants. 

\begin{figure}[!thb]
    \centering
    \begin{minipage}{0.45\textwidth}
        \centering
        \includegraphics[width = \textwidth]{Resultados/Questionario/Figuras/pdf/qqplot_questionnaire_blind.pdf}
        \caption{QQ plot of the questionnaire score of the blind participants on each method.}
        \label{fig:qqplot_questionnaire_blind}
    \end{minipage}
    \begin{minipage}{0.075\textwidth}
        \hfill
    \end{minipage}
    \begin{minipage}{0.45\textwidth}
        \centering
        \includegraphics[width = \textwidth]{Resultados/Questionario/Figuras/pdf/residplot_questionnaire_blind.pdf}
        \caption{Residual plot of the questionnaire score the blind participants on each method.}
        \label{fig:residplot_questionnaire_blind}
    \end{minipage}
\end{figure}

The results of ANOVA are presented in Table  \ref{tab:blocanova_questionnaire_blind} and it shows that the method, with a p-value of 0.001, is indeed a significant variable that affects the user's satisfaction.


\begin{table}[!htb]
\centering
\caption{Anova p-value for the questionnaire score on each method for blinded users.}
\label{tab:blocanova_questionnaire_blind}
\begin{tabular}{lrrrrr}
\toprule
Source & P-Value \\
\midrule
Method & 0.001** \\
\bottomrule
\end{tabular}
\end{table}



In order to complement the ANOVA analysis, the pairwise comparison of the methods obtained from the Fisher LSD test is presented in Table \ref{tab:lsd_questionnaire_blind}. The results show that audio and mixture are equivalent from the perspective of user satisfaction. All the other comparisons indicate there is a difference between the methods.

\input{Resultados/Questionario/Tabelas/lsd_questionnaire_blind.tex}

Additional to the scores, the participants also expressed their dissatisfaction with the answers to the open questions of the questionnaire, where they commented that the haptic belt and the virtual cane are confusing, are not precise enough, and are very different from what they are used to.

\FloatBarrier

\subsection{Physiological data}

During the experiment, data from two physiological sensors were captured: ECG and GSR. As commonly found in the literature, these data are used to assess mental workload. The corresponding analysis is presented in this section.

\begin{itemize}
    \item \nameref{subsubsec:results_ecg_1};
    
        Two features are extracted from the ECG, heartrate (BPM) and heartrate variance (SDNN).
    
        Is expected that the heartrate slight decrease from the "First" to the "Return" round. The heartrate variance is expected to slight increase from the "First" to the "Return" round.
    

    \item \nameref{subsubsec:results_gsr_temp_1};
    
        Is expected that the GSR average to increase at every “First” round and then a slight decrease in the next round.

\end{itemize}
\subsubsection{Electrocardiogram (ECG) data}
\label{subsubsec:results_ecg_1}

The ECG analysis is divided into two different types

\begin{itemize}
    \item Heart rate;
    
        This analysis checks the heartbeat frequency;

    \item Heart rate variance.
    
        This analysis checks the heartbeat frequency variance and it is done by analyzing the variation of the interval between beats.

\end{itemize}

At the beginning of each experience, a baseline data was gathered to establish a comparison between the normal state of the user and the scenes’ induced state. After the data gathering, an algorithm in Python was used to read the data and separate it accordingly to each participant, method and round. The algorithm followed the steps above:

\begin{itemize}
    \item Outliers remotion;
        Since the participants moved during the whole experience a lot of noise was collected by the sensors
    \item Normalization between -1 and 1;
    \item Peak detection;
        If the results were appropriate:
        \begin{itemize}
            \item Heartbeat interval calculation;
            \item File save to be used in Kubius HRV Standard.
        \end{itemize} 
        If the results were not appropriate:
        \begin{itemize}
            \item Tune peak detection method’s parameters;
            \item Heartbeat interval calculation;
            \item File save to be used in the next software.
        \end{itemize}    
\end{itemize}

This judgment was made by analyzing the plotted ECG signal and the detected peaks. Kubios HRV Standard is a heart rate variability (HRV) analysis software for personal non-commercial use. The Kubios HRV Standard makes it possible to use your HR monitor to examine the health of the cardiovascular system or to evaluate stress and recovery \cite{kubios}. At Kubius, the file with the intervals was analyzed and the results were saved in a report file to be read in python again. Back in python the results were plotted, tabled and statistically tested as the other data. In Appendix D there is a diagram with a pseudo-algorithm of this process.

This analysis was made by comparing the baseline values with the values of each round individually and between the round values themselves.

\paragraph{Analysis of the heartbeat frequency (BPM)}\mbox{}\\

\input{Resultados/ECG/bpm1.tex}

%%%%%%%%%%%%%%%%%%%%%%%%%%%%%%%%%%%%%%%%%%%%%%%%%%%%%%%%%%%%%%%%%%%%%%%%%%%%
%%%%%%%%%%%%%%%%%%%%%%%%%%%%%%%%%%%%%%%%%%%%%%%%%%%%%%%%%%%%%%%%%%%%%%%%%%%%
%%%%%%%%%%%%%%%%%%%%%%%%%%%%%%%%%%%%%%%%%%%%%%%%%%%%%%%%%%%%%%%%%%%%%%%%%%%%
%%%%%%%%%%%%%%%%%%%%%%%%%%%%%%%%%%%%%%%%%%%%%%%%%%%%%%%%%%%%%%%%%%%%%%%%%%%%
%
%
\paragraph{Analysis of the heartbeat variance (SDNN)}\mbox{}\\
%
\input{Resultados/ECG/sdnn1}

\subsubsection{Galvanic skin reaction and temperature data;}
\label{subsubsec:results_gsr_temp_1}

The GSR analysis is based in the average level of the signal during each run of the experiment and its comparison to the participant baseline, collected before each round. For the experiment, the GSR sensor was worn on the left hand for right-handed participant and on the right hand for left-handed participants. One of the blind participants had the GSR sensor removed during the experiment because it was not appropriately fixed.

Table \ref{tab:gsr_table_blind} presents the GSR mean values for each of the three remaining participants. For all the participants, the baseline was smaller than the values obtained during the experiment, as expected. Moreover, in most of the cases, the skin conductance has risen from the ‘first’ to the ‘return’, indicating an increase in the mental workload.

\input{Resultados/GSR/Tabelas/gsr_table_blind.tex}

Table \ref{tab:gsr_var_blind} brings the percentual increase in the GSR mean when compared to the baseline value. Figure \ref{fig:barplot_gsr_avg_5_scene_blind} shows the corresponding barplot. Apparently, the presence of a haptic device causes an increase in the skin conductance, hence its mental workload. Also, it is possible to observe the increase in GSR mean between the two rounds, excepted for the haptic belt.

\input{Resultados/GSR/Tabelas/gsr_var_blind.tex}

\begin{figure}[!htb]
    \centering
    \includegraphics[width = 0.8\linewidth]{Resultados/GSR/Figuras/png/barplot_gsr_avg_5_scene_blind.png}
    \caption{Barplot of the average SDNN of the blind participants on each method.}
    \label{fig:barplot_gsr_avg_5_scene_blind}
\end{figure}

Figure \ref{fig:boxplot_gsr_avg_blind_scene} presents the boxplot of the percentual variation in the skin conductance for each method. The ‘base’ method has the lowest variation among all methods. Also, the introduction of vibration increases the method variance. Figure \ref{fig:boxplot_gsr_avg_blind_rounds} presents the GSR grouped by the rounds. In this case there is no apparent different between the rounds.

\begin{figure}[!htb]
    \centering
    \begin{minipage}{0.45\textwidth}
        \centering
        \includegraphics[width = 0.8\linewidth]{Resultados/GSR/Figuras/png/boxplot_gsr_avg_blind_scene.png}
        \caption{Boxplot of the GSR of the blind participants grouped by method.}
        \label{fig:boxplot_gsr_avg_blind_scene}
    \end{minipage}
    \begin{minipage}{0.45\textwidth}
        \centering
        \includegraphics[width = 0.8\linewidth]{Resultados/GSR/Figuras/png/boxplot_gsr_avg_blind_rounds.png}
        \caption{Boxplot of the GSR of the blind participants grouped by round.}
        \label{fig:boxplot_gsr_avg_blind_rounds}
    \end{minipage}
\end{figure}

Figures \ref{fig:qqplot_gsr_two_way_blind} and \ref{fig:residplot_gsr_two_way_blind} shows the QQ plot and the residual distribution. The Table \ref{tab:blocanova_gsr_two_way_blind} shows the ANOVA test p-value for the GSR percentual variance. Although the p-value for the method is not below the threshold of 0.05, it close to it, indicating that probably the GSR is affected by it. 

The Figures  shows the distribution and variance of the Table \ref{tab:gsr_var_blind}. These Figures shows that the data are normally distributed but the participants had different  that the methods have a similar variance.

\begin{figure}[!htb]
    \centering
    \begin{minipage}{0.45\textwidth}
        \centering
        \includegraphics[width = 0.8\linewidth]{Resultados/GSR/Figuras/png/qqplot_gsr_two_way_blind.png}
        \caption{QQ plot of the SDNN of the blind participants on each method.}
        \label{fig:qqplot_gsr_two_way_blind}
    \end{minipage}
    \begin{minipage}{0.45\textwidth}
        \centering
        \includegraphics[width = 0.8\linewidth]{Resultados/GSR/Figuras/png/residplot_gsr_two_way_blind.png}
        \caption{Residual plot of the SDNN of the blind participants on each method.}
        \label{fig:residplot_gsr_two_way_blind}
    \end{minipage}
\end{figure}

\input{Resultados/GSR/Tabelas/blocanova_gsr_two_way_blind.tex}
%
%%\input{Resultados/GSR/Tabelas/lsd_gsr_two_way.tex}
%
%The Table \ref{tab:blocanova_gsr_two_way_blind} does not prove that any method or round has some influence in the skin conductance variation, thus in the Mental Workload. Although, in the Figure \ref{fig:boxplot_gsr_avg_blind_scene} it is posible to notice that the "Base" and the "Audio" method have a different distribution. As it has already commented before, maybe the result of the anova test is a conseguence of a small sample size.
%
\FloatBarrier

 

\subsection{Final Remarks}

% Mental demand
% • ANOVA não deu resultado
% • Grafico - Presença de haptico aumenta a demanda mental

% NASA TLX
% • ANOVA - Efeito do método e to round.
% •• Metodo - 3 grupos
% ••• Base e Audio menores Carga mental
% ••• Haptic alta Carga mental
% 
% SAGAT
% ANOVA - Round tem efeito no SAGAT
%
% Satisfação
% ANOVA mostrou que o método fez diferença na satisfação do usuario
% Audio e Mix os preferidos
% Participantes comentando que o haptic e o virtual eram imprecisos e confusos

% BPM
% ANOVA - não deu resultado
% Gráfico mostra uma pequena diferença entre os métodos

% SDNN
% ANOVA - não deu resultado
% Gráfico - "Base" tem uma variância menor

% GSR
% ANOVA - não deu resultado
% Grafico - Base e Audio

The “Audio” method showed a higher performance among the other methods and the use of a haptic device decreased the performance of all methods. This probably happened because the participants are already used to use sound to guide themselves, especially environmental sounds. The environment sounds used inside the scenes that gave hints about locations where always the same (telephone ringing, laptop keyboard sounds, exterior noise, door opening and closing). It is likely that the participants felt more secure when it only had to focus on the sounds around him/her. This is reinforced by the fact that, during the “Audio” only guidance, half of the participants did not called for any command, or used only a few times the audio command option.

The fact that the haptic devices caused a higher average and a higher variation is probably due to the fact that the users had to learn and get used with them. Besides, for being just conceptual, their precision was not as big as they were expecting. That explains why their results were not as good as the “Base” or “Audio” methods and these results are correctly related to the satisfaction questionnaires, which scored them as the unsatisfied devices.

The ANOVA test from the mental demand did not prove any influence of the method nor the rounds, but the same test for the NASA-TLX score proved a influence of these both factors. That means that the methods influenced the score in other NASA-TLX dimension.

The SAGAT's ANOVA proved that the round did influenced the participants average score, meaning that they did learn from one round to another and that effect was a natural thing, not an effect caused by the devices.

The other ANOVA tests were inconclusive, but most of their boxplots most showed a difference between the methods. In these cases the reason is the small sample size and the sample’s variety. The blind participants age range was from 26 to 56, with an average of 43.5 years, and the education range was from High School to 2 graduations. That may impact the user experience as well in the questionnaires' answer.

But all the participants showed a great enthusiasm before, during and after the research. They also recommend some modifications that would bring more realism for they. And of course, they made some complaints, such as:

\begin{itemize}
    \item The speakers inside the HMD were not could enough for some to give them the precise location of its origin;
    \item The HMD was big enough to cover have of the participant’s face and that gave them a strange sensation, since some of them use the air or the wind feeling on the face to give them hints about the location of walls or other high obstacles;
    \item As said before, the precision of the vibration was not good for them to use the devices. That is mainly because of how the HMD position the user inside the virtual environment. \\
    The user is represented as a vertical capsule, and the HMD is positioned on the top end of that capsule. If the user tilts his/her head down, as if they were facing the ground, the capsule rotates in relation to the HMD point making the virtual body of the user occupy a total different space from the reality. The Figure \ref{fig:user_envelope} represents that situation. 

    %\begin{figure}[!htb]
    %    \centering
    %    \begin{minipage}{0.45\textwidth}
    %        \centering
    %        \includegraphics[width = 0.8\linewidth]{Resultados/envelope1.png}
    %        \subcaption{The user's capsule while the participant is straight and looking forward.}
    %        \label{fig:user_straight}
    %    \end{minipage}
    %    \begin{minipage}{0.45\textwidth}
    %        \centering
    %        \includegraphics[width = 0.8\linewidth]{Resultados/envelope2.png}
    %        \subcaption{The user's capsule while the participant is straight but looking down.}
    %        \label{fig:user_looking_down}
    %    \end{minipage}
    %    \caption{Two different capsule positions based on the user's head position.}
    %    \label{fig:user_envelope}
    %\end{figure}
    
    \item The vibration from the haptic belt was not intense enough sometimes.
\end{itemize}

\section{Comparison between BVI users and sighted users}
\label{sec:results_obj_2}

In this section, the second goal of this experiment, “do non-BVI users, when deprived from their vision, evaluate assistive devices in a similar way as BVI users?”, will be linked with the gathered data and then compared with the results of the first goal's section. It is expected that both results would be different. As was the last section, this section will also be divided in the same subsections.

\subsection{Subjective data}

Only two of the questionnaires will be analyzed, the NASA-TLX and the Adapted SAGAT, and it is expected that for:

\begin{itemize}
    \item \nameref{subsubsec:results_nasa_tlx_2};
    
        There will be a noticeable difference between the sight sample mental workload and the blind sample mental workload.

    \item \nameref{subsubsec:results_adapted_sagat_2};
    
        Is expected to notice a difference between the “blind” sample and the “sight” sample.

    \item \nameref{subsubsec:results_questionnaires}.

        Meant to assess the user experience with each method.

\end{itemize}

\subsubsection{NASA-TLX}
\label{subsubsec:results_nasa_tlx_2}

\paragraph{Analysis of the mental demand scale}\mbox{}\\

The Table \ref{tab:md_table_noBase} presents the ‘mental demand’ score of all participants, while the corresponding barplot is presented in Figure \ref{fig:barplot_md_avg_4_scene_blind_sight}. It is interesting to observe that sighted people gave a higher score to audio, as they are not so familiar to use sounds as source of guidance.

\input{Resultados/Nasa/Tabelas/md_table_noBase}

 %The Figures \ref{fig:barplot_md_avg_4_scene_blind} and \ref{fig:barplot_md_avg_4_scene_sight} show a systematic reduction on the perceived mental demand in all methods between the rounds for both groups. But the Figure \ref{fig:barplot_md_avg_4_scene} shows that the average of each method was very different between the two groups.

\begin{figure}[!htb]
    \centering
    \begin{minipage}{\textwidth}
        \centering
        \includegraphics[width = 0.8\linewidth]{Resultados/Nasa/Figuras/png/barplot_md_avg_4_scene_blind.png}
        \subcaption{Blind participants}
        \label{fig:barplot_md_avg_4_scene_blind}
    \end{minipage}
    \begin{minipage}{\textwidth}
        \centering
        \includegraphics[width = 0.8\linewidth]{Resultados/Nasa/Figuras/png/barplot_md_avg_4_scene_sight.png}
        \subcaption{Sight participants}
        \label{fig:barplot_md_avg_4_scene_sight}
    \end{minipage}
    \caption{Barplot of the average mental demand on each method and round.}
    \label{fig:barplot_md_avg_4_scene_blind_sight}
\end{figure}
%\begin{figure}[!htb]
%    \centering
%    \includegraphics[width = 0.8\linewidth]{Resultados/Nasa/Figuras/png/barplot_md_avg_4_scene.png}
%    \caption{Barplot of the average mental demand of both participants on each method.}
%    \label{fig:barplot_md_avg_4_scene}
%\end{figure}

Figures \ref{fig:boxplot_noBase_md_4_scene} and \ref{fig:boxplot_noBase_md_4_rounds} presents the box plot for both groups, organized by method and round. It is clear that the mental demand is systematically higher for sighted people, which is something expected. But while blind participants considered the ‘audio’ method less demanding, sighted participants gave preference to the virtual cane. For both groups, we observe a decrease in the mental demand.

\begin{figure}[!htb]
    \centering
    \begin{minipage}{0.45\textwidth}
        \centering
        \includegraphics[width = 0.8\linewidth]{Resultados/Nasa/Figuras/png/boxplot_noBase_md_4_scene.png}
        \caption{Boxplot of the mental demand of the participants grouped by method.}
        \label{fig:boxplot_noBase_md_4_scene}
    \end{minipage}
    \begin{minipage}{0.45\textwidth}
        \centering
        \includegraphics[width = 0.8\linewidth]{Resultados/Nasa/Figuras/png/boxplot_noBase_md_4_rounds.png}
        \caption{Boxplot of the mental demand of the participants grouped by round.}
        \label{fig:boxplot_noBase_md_4_rounds}
    \end{minipage}
\end{figure}

%The Table \ref{tab:md_average_group_noBase} shows the average mental demand of both samples and is possible to notice how the average perceived mental demand by the sight sample was higher in every method.
%
%\input{Resultados/Nasa/Tabelas/md_average_group_noBase.tex}

Figures \ref{fig:qqplot_md_avg_two_way_sight} and \ref{fig:residplot_md_avg_two_way_sight} show the QQ plot and residual distribution for the sighted data, confirming that the data is normally distributed and participants have similar variance. Table \ref{tab:blocanova_md_avg_two_way_blind_sight} brings the results of ANOVA. Different from the blind participants, in the case of sighted ones, the p-value for ‘method’ is below the threshold of 0.05, confirming it as a significant variable for the mental demand. In the case of ‘round’, data from both sighted and blind participants resulted in the same p-value of 0.075, which is close to the traditional threshold of 0.05, but slightly higher. 

\begin{table}
    \caption{Anova p-value for the mental demand average on each method'}
    \label{tab:blocanova_md_avg_two_way_blind_sight}
\begin{minipage}{0.45\textwidth}
    \subcaption{Blind participants}
    \input{Resultados/Nasa/Tabelas/blocanova_md_avg_two_way_blindSemBegin.tex}
\end{minipage}
\begin{minipage}{0.45\textwidth}
    \subcaption{Sight participants}
    \input{Resultados/Nasa/Tabelas/blocanova_md_avg_two_way_sightSemBegin.tex}    
\end{minipage}
\end{table}

\begin{figure}[!htb]
    \centering
    \begin{minipage}{0.45\textwidth}
        \centering
        \includegraphics[width = 0.8\linewidth]{Resultados/Nasa/Figuras/png/qqplot_md_avg_two_way_sight.png}
        \caption{QQ plot of the mental demand of the sight participants on each method.}
        \label{fig:qqplot_md_avg_two_way_sight}
    \end{minipage}
    \begin{minipage}{0.45\textwidth}
        \centering
        \includegraphics[width = 0.8\linewidth]{Resultados/Nasa/Figuras/png/residplot_md_avg_two_way_sight.png}
        \caption{Residual plot of the mental demand score the sighted participants on each method.}
        \label{fig:residplot_md_avg_two_way_sight}
    \end{minipage}
\end{figure}

%The Table \ref{tab:lsd_md_avg_two_way_sight} presents the conclusion of a pairwise Fisher LSD test of the previous ANOVA test. The results show that only the "Audio" has a similar mental demand as the "Mixture" method.
%
%\input{Resultados/Nasa/Tabelas/lsd_md_avg_two_way_sight.tex}
%
%The Table \ref{tab:md_var_average_group} shows the average of the mental demand variation between the rounds. This table shows that the mental demand variation from the “Audio” has the lower variation, and the rest are similar variations.
%
%\input{Resultados/Nasa/Tabelas/md_var_average_group}
%
%The Figures \ref{fig:qqplot_md_var_sight} and \ref{fig:residplot_md_var_sight} shows the distribution and variance of the mental demand variation of the Table \ref{tab:md_table_blind}. These Figures shows that the data are normally distributed and that the methods have a similar variance.
%The Table \ref{tab:blocanova_md_var_sight} shows the Anova test p-value of the mental demand of the "sight" sample between the guidance methods. The p-value indicates that there is no influence of the methods in the variation of mental demand between the rounds. 
%
%\input{Resultados/Nasa/Tabelas/blocanova_md_var_sight.tex}
%
%\begin{figure}[!htb]
%    \centering
%    \begin{minipage}{0.45\textwidth}
%        \centering
%        \includegraphics[width = 0.8\linewidth]{Resultados/Nasa/Figuras/png/qqplot_md_var_sight.png}
%        \caption{Residual plot of the mental demand variation of the blind participants on each method.}
%        \label{fig:qqplot_md_var_sight}
%    \end{minipage}
%    \begin{minipage}{0.45\textwidth}
%        \centering
%        \includegraphics[width = 0.8\linewidth]{Resultados/Nasa/Figuras/png/residplot_md_var_sight.png}
%        \caption{Residual plot of the mental demand variation of the sighted participants on each method.}
%        \label{fig:residplot_md_var_sight}
%    \end{minipage}
%\end{figure}
%
%%The Table \ref{tab:lsdbloc_mental_demand_var} presents the conclusion of a pairwise Fisher LSD test of the blind mental demand between all the guidance methods. The results show that all methods have similar variations.
%
%%\input{Resultados/Nasa/Tabelas/lsdbloc_mental_demand_var.tex}
%
%To close up, according to the ANOVA test at Table \ref{tab:lsd_md_avg_two_way_sight} the method do have influence on the mental demand of the sighted participant and that the "Audio" and the "Mixture" method have the same mental demand for them. This differs from the result of the previous section that was the ANOVA did not prove any effect and that the "Audio" and "Mixture" methods could not be said to be similar. Although for the "blind" users, they were also the methods that caused the lowest mental demand.
%
%There is no influence in the tested methods in the participants mental demand variation, as shown in the Table \ref{tab:blocanova_md_var_sight}.

\FloatBarrier

%%%%%%%%%%%%%%%%%%%%%%%%%%%%%%%%%%%%%%%%%%%%%%%%%%%%%%%%%%%%%%%%%%%%%%%%%%%%
%%%%%%%%%%%%%%%%%%%%%%%%%%%%%%%%%%%%%%%%%%%%%%%%%%%%%%%%%%%%%%%%%%%%%%%%%%%%
%%%%%%%%%%%%%%%%%%%%%%%%%%%%%%%%%%%%%%%%%%%%%%%%%%%%%%%%%%%%%%%%%%%%%%%%%%%%
%%%%%%%%%%%%%%%%%%%%%%%%%%%%%%%%%%%%%%%%%%%%%%%%%%%%%%%%%%%%%%%%%%%%%%%%%%%%


\paragraph{Analysis of the NASA-TLX score}\mbox{}\\

Table \ref{tab:nasa_table_noBase} brings the NASA-TLX global score of all participants, while the corresponding barplot is presented in Figure \ref{fig:barplot_nasa_avg_4_scene}.

\input{Resultados/Nasa/Tabelas/nasa_table_noBase}

From Figure \ref{fig:barplot_nasa_avg_4_scene} it is possible to see that, similar to blind participants, sighted participants also consider that the workload of the return round was lower than that of the first round. However, similar to what happened for the mental demand, sighted participants considered ‘virtual cane’ as the method with the lowest workload, while, for  blind participants, it was the ‘audio’.

\begin{figure}[!htb]
    \centering
    \begin{minipage}{\textwidth}
        \centering
        \includegraphics[width = 0.8\linewidth]{Resultados/Nasa/Figuras/png/barplot_nasa_avg_4_scene_blind.png}
        \subcaption{Blind participants.}
        \label{fig:barplot_nasa_avg_4_scene_blind}
    \end{minipage}
    \begin{minipage}{\textwidth}
        \centering
        \includegraphics[width = 0.8\linewidth]{Resultados/Nasa/Figuras/png/barplot_nasa_avg_4_scene_sight.png}
        \subcaption{Sight participants.}
        \label{fig:barplot_nasa_avg_4_scene_sight}
    \end{minipage}
    \caption{Barplot of the NASA-TLX score on each method and round.}
    \label{fig:barplot_nasa_avg_4_scene}
\end{figure}
%\begin{figure}[!htb]
%    \centering
%    \includegraphics[width = 0.8\linewidth]{Resultados/Nasa/Figuras/png/barplot_nasa_avg_4_scene.png}
%    \caption{Barplot of the NASA-TLX score of both participants on each method.}
%    \label{fig:barplot_nasa_avg_4_scene}
%\end{figure}

Figures \ref{fig:boxplot_noBase_nasa_4_scene} and \ref{fig:boxplot_noBase_nasa_4_rounds} present the boxplots of NASA-TLX global score. Again, it is possible to see that sighted people usually give higher workload scores than blind ones. The influence of the round is approximately the same. But the order of preference of the methods are different.

\begin{figure}[!htb]
    \centering
    \begin{minipage}{0.45\textwidth}
        \centering
        \includegraphics[width = 0.8\linewidth]{Resultados/Nasa/Figuras/png/boxplot_noBase_nasa_4_scene.png}
        \caption{Boxplot of the NASA-TLX score of the participants grouped by method.}
        \label{fig:boxplot_noBase_nasa_4_scene}
    \end{minipage}
    \begin{minipage}{0.45\textwidth}
        \centering
        \includegraphics[width = 0.8\linewidth]{Resultados/Nasa/Figuras/png/boxplot_noBase_nasa_4_rounds.png}
        \caption{Boxplot of the NASA-TLX score of the participants grouped by round.}
        \label{fig:boxplot_noBase_nasa_4_rounds}
    \end{minipage}
\end{figure}

%The Table \ref{tab:nasa_average_group_noBase} shows the average NASA-TLX score of both samples and is possible to notice how the average perceived NASA-TXL average by the sight sample was also higher in every method.
%
%\input{Resultados/Nasa/Tabelas/nasa_average_group_noBase.tex}
Figures \ref{fig:qqplot_nasa_avg_two_way_sight} and \ref{fig:residplot_nasa_avg_two_way_sight} bring the QQ plot and residual distribution of the data from sighted participants, showing that ANOVA can be used. The p-values for both groups are presented in Table \ref{tab:blocanova_nasa_avg_two_way_blind_sight}. It confirms the influence of the round for both sighted and blind people. In the case of the method, the p-value of ‘blind’ is lower than the threshold of 0.5, while that of ‘sighted’ is slightly higher.

\begin{table}
    \caption{Anova p-value for the mental demand average on each method'}
    \label{tab:blocanova_nasa_avg_two_way_blind_sight}
    \begin{minipage}{0.45\textwidth}
        \subcaption{Blind participants}
        \input{Resultados/Nasa/Tabelas/blocanova_nasa_avg_two_way_blindSemBegin.tex}
    \end{minipage}
    \begin{minipage}{0.45\textwidth}
        \subcaption{Sight participants}
        \input{Resultados/Nasa/Tabelas/blocanova_nasa_avg_two_way_sightSemBegin.tex}    
    \end{minipage}
\end{table}


\begin{figure}[!htb]
    \centering
    \begin{minipage}{0.45\textwidth}
        \centering
        \includegraphics[width = 0.8\linewidth]{Resultados/Nasa/Figuras/png/qqplot_nasa_avg_two_way_sight.png}
        \caption{QQ plot of the NASA-TLX score of the sight participants on each method.}
        \label{fig:qqplot_nasa_avg_two_way_sight}
    \end{minipage}
    \begin{minipage}{0.45\textwidth}
        \centering
        \includegraphics[width = 0.8\linewidth]{Resultados/Nasa/Figuras/png/residplot_nasa_avg_two_way_sight.png}
        \caption{Residual plot of the NASA-TLX score the sight participants on each method.}
        \label{fig:residplot_nasa_avg_two_way_sight}
    \end{minipage}
\end{figure}

%The Table \ref{tab:lsd_nasa_avg_two_way_sight} presents the conclusion of a pairwise Fisher LSD test of the previous ANOVA test and it shows that all the method had an effect in the NASA-TLX score.
%
%\input{Resultados/Nasa/Tabelas/lsd_nasa_avg_two_way_sight.tex}

%The Table \ref{tab:nasa_var_group} shows the average of the NASA-TLX score variation between the rounds. This table shows that the score variation from the “Audio” has the lower variation, and the rest are similar variations.

%\input{Resultados/Nasa/Tabelas/nasa_var_group.tex}

%The Figures \ref{fig:qqplot_nasa_var_sight} and \ref{fig:residplot_nasa_var_sight} shows the distribution and variance of the NASA-TLX score variation of the Table \ref{tab:md_table_blind}. These Figures shows that the data are normally distributed and that the methods have a similar variance.

%The Table \ref{tab:blocanova_nasa_var_sight} shows the Anova test p-value of the NASA-TLX score of the "sight" sample between the guidance methods and it proves that there is no influence of the methods in the variation of score between the rounds. 

%\input{Resultados/Nasa/Tabelas/blocanova_nasa_var_sight.tex}

%\begin{figure}[!htb]
%    \centering
%    \begin{minipage}{0.45\textwidth}
%        \centering
%        \includegraphics[width = 0.8\linewidth]{Resultados/Nasa/Figuras/png/qqplot_nasa_var_sight.png}
%        \caption{Residual plot of the variation NASA-TLX score of the blind participants on each method.}
%        \label{fig:qqplot_nasa_var_sight}
%    \end{minipage}
%    \begin{minipage}{0.45\textwidth}
%        \centering
%        \includegraphics[width = 0.8\linewidth]{Resultados/Nasa/Figuras/png/residplot_nasa_var_sight.png}
%        \caption{Residual plot of the variation NASA-TLX score of the sighted participants on each method.}
%        \label{fig:residplot_nasa_var_sight}
%    \end{minipage}
%\end{figure}

%The Table \ref{tab:lsdbloc_mental_demand_var} presents the conclusion of a pairwise Fisher LSD test of the blind mental demand between all the guidance methods. The results show that all methods have similar variations.

%\input{Resultados/Nasa/Tabelas/lsdbloc_mental_demand_var.tex}

%To close up, according to the ANOVA test at Table \ref{tab:blocanova_nasa_avg_two_way_sight} the methods do not an effect on the score, but rounds do. The blind users felt an impact on both the method and the round.
%
%There is no influence in the tested methods in the participants NASA-TLX score variation, as shown in the Table \ref{tab:blocanova_nasa_var_sight}.

\FloatBarrier

\subsubsection{Adapted SAGAT}
\label{subsubsec:results_adapted_sagat_2}

Table\ref{tab:sagat_table_noBase} presents the SAGAT score of all participants. The corresponding barplot is presented in Figure \ref{fig:barplot_sagat_avg_4_scene_blind_sight}.

\input{Resultados/sagat/Tabelas/sagat_table_noBase}

Figure \ref{fig:barplot_sagat_avg_4_scene_blind_sight}. shows that the SAGAT score for sighted participants is on average lower than that of blind participants, which is expected as they are not used to navigate without vision. Also, the increase in situation awareness from the first to the return round is lower. In the case of the mixture method, the SAGAT score did not improve at all. For both groups, the ‘virtual cane’ was the method with lowest score in the first round.

\begin{figure}[!htb]
    \centering
    \begin{minipage}{\textwidth}
        \centering
        \includegraphics[width = 0.8\linewidth]{Resultados/Sagat/Figuras/png/barplot_sagat_avg_4_scene_blind.png}
        \subcaption{Blind participants.}
        \label{fig:barplot_sagat_avg_4_scene_blind}
    \end{minipage}
    \begin{minipage}{\textwidth}
        \centering
        \includegraphics[width = 0.8\linewidth]{Resultados/Sagat/Figuras/png/barplot_sagat_avg_4_scene_sight.png}
        \subcaption{Sight participants.}
        \label{fig:barplot_sagat_avg_4_scene_sight}
    \end{minipage}
    \caption{Barplot of the SAGAT score on each method and round.}
    \label{fig:barplot_sagat_avg_4_scene_blind_sight}
\end{figure}
%\begin{figure}[!htb]
%    \centering
%    \includegraphics[width = 0.8\linewidth]{Resultados/Sagat/Figuras/png/barplot_sagat_avg_4_scene.png}
%    \caption{Barplot of the SAGAT score of both participants on each method.}
%    \label{fig:barplot_sagat_avg_4_scene}
%\end{figure}

Figures \ref{fig:boxplot_sagat_4_scene} and \ref{fig:boxplot_sagat_4_rounds} bring the boxplots. According to Figure \ref{fig:boxplot_sagat_4_scene}, both groups presented a higher situation awareness with ‘mixture’ and ‘haptic’. On the other hand, Figure \ref{fig:boxplot_sagat_4_rounds} confirms that the difference between the rounds is greater for blind participants. 

\begin{figure}[!htb]
    \centering
    \begin{minipage}{0.45\textwidth}
        \centering
        \includegraphics[width = 0.8\linewidth]{Resultados/Sagat/Figuras/png/boxplot_sagat_4_scene.png}
        \caption{Boxplot of the Sagat score of the participants grouped by method.}
        \label{fig:boxplot_sagat_4_scene}
    \end{minipage}
    \begin{minipage}{0.45\textwidth}
        \centering
        \includegraphics[width = 0.8\linewidth]{Resultados/Sagat/Figuras/png/boxplot_sagat_4_rounds.png}
        \caption{Boxplot of the Sagat score of the participants grouped by round.}
        \label{fig:boxplot_sagat_4_rounds}
    \end{minipage}
\end{figure}

%The Table \ref{tab:sagat_average_group_noBase} shows the average SAGAT score of both samples and is possible to notice how the average score by the blind users was higher in every method.
%
%\input{Resultados/Sagat/Tabelas/sagat_average_group_noBase.tex}

Figures \ref{fig:qqplot_sagat_avg_two_way_sight} and \ref{fig:residplot_sagat_avg_two_way_sight} brings the QQ plot and residual distribution. It is clear that the residuals variance is not equal among the participants. Table \ref{tab:blocanova_sagat_avg_two_way_blind_sight} brings the p-value from ANOVA. While for blind participants the round is a significant factor and the method is not, for sighted participants the result is the opposite, showing a significant influence of the method and not of the round.

\begin{table}
    \caption{Anova p-value for the SAGAT score on each method}
    \label{tab:blocanova_sagat_avg_two_way_blind_sight}
\begin{minipage}{0.45\textwidth}
    \subcaption{Blind participants}
    \input{Resultados/Sagat/Tabelas/blocanova_sagat_avg_two_way_blindSemBegin.tex}
\end{minipage}
\begin{minipage}{0.45\textwidth}
    \subcaption{Sight participants}
    \input{Resultados/Sagat/Tabelas/blocanova_sagat_avg_two_way_sightSemBegin.tex}    
\end{minipage}
\end{table}

\begin{figure}[!htb]
    \centering
    \begin{minipage}{0.45\textwidth}
        \centering
        \includegraphics[width = 0.8\linewidth]{Resultados/Sagat/Figuras/png/qqplot_sagat_avg_two_way_sight.png}
        \caption{QQ plot of the mental demand of the sight participants on each method.}
        \label{fig:qqplot_sagat_avg_two_way_sight}
    \end{minipage}
    \begin{minipage}{0.45\textwidth}
        \centering
        \includegraphics[width = 0.8\linewidth]{Resultados/Sagat/Figuras/png/residplot_sagat_avg_two_way_sight.png}
        \caption{Residual plot of the mental demand score the sight participants on each method.}
        \label{fig:residplot_sagat_avg_two_way_sight}
    \end{minipage}
\end{figure}

%%%%%%%%%%%%%%%%%%%%%%%%%%%%%%%%%%%%%%%%%%%%%%%%%%%%%

%The Table \ref{tab:lsd_sagat_avg_two_way_sight} presents the conclusion of a pairwise Fisher LSD test of the previous ANOVA test. The results show that only the "Audio" and the "Haptic Belt" had a similar SAGAT score. This is different than the result from the ANOVA of the blind users, which indicated for them that the rounds had an effect.

%\input{Resultados/Sagat/Tabelas/lsd_sagat_avg_two_way_sight.tex}

%The Table \ref{tab:sagat_var_group_noBase} shows the average of the SAGAT score variation between the rounds. This table and the Figure \ref{fig:boxplot_sagat_4_rounds} show that, besides the higher average score, the blind users also had a higher variation between the rounds.

%\input{Resultados/Sagat/Tabelas/sagat_var_group_noBase.tex}



%\begin{figure}[!htb]
%    \centering
%    \begin{minipage}{0.45\textwidth}
%        \centering
%        \includegraphics[width = 0.8\linewidth]{Resultados/Sagat/Figuras/png/qqplot_sagat_var_sight.png}
%        \caption{Residual plot of the variation SAGAT score of the blind participants on each method.}
%        \label{fig:qqplot_sagat_var_sight}
%    \end{minipage}
%    \begin{minipage}{0.45\textwidth}
%        \centering
%        \includegraphics[width = 0.8\linewidth]{Resultados/Sagat/Figuras/png/residplot_sagat_var_sight.png}
%        \caption{Residual plot of the variation SAGAT score of the sighted participants on each method.}
%        \label{fig:residplot_sagat_var_sight}
%    \end{minipage}
%\end{figure}

%The Table \ref{tab:lsdbloc_mental_demand_var} presents the conclusion of a pairwise Fisher LSD test of the blind mental demand between all the guidance methods. The results show that all methods have similar variations.

%\input{Resultados/Nasa/Tabelas/lsdbloc_mental_demand_var.tex}

%Figures \ref{tab:sagat_average_group_noBase} and \ref{tab:sagat_var_group_noBase} brings the QQ plot and residual distribution. It is clear that the residuals variance is not equal among the participants. Table \ref{tab:blocanova_sagat_avg_two_way} brings the p-value from ANOVA. While for blind participants the round is a significant factor and the method is not, for sighted participants the result is the opposite, showing a significant influence of the method and not of the round.
%
%To close up, according to the Tables  with the Figures \ref{fig:boxplot_sagat_4_scene} and \ref{fig:boxplot_sagat_4_rounds} the blind user scored a higher SAGAT score than the sight user with the same conditions and devices. Besides that, the ANOVA and the LSD Fisher Test at Tables \ref{tab:blocanova_sagat_avg_two_way_sight} and \ref{tab:lsd_sagat_avg_two_way_sight} show that for the sight user the methods impact more their score, whilst the blind user were affected more with the rounds.
%
%There is no influence in the tested methods in the participants mental demand variation, as shown in the Table \ref{tab:blocanova_sagat_var_sight}.

\FloatBarrier

\subsubsection{Guidance method's questionnaire.}
\label{subsec:results_questionnaires}

Finally, the Questionnaire is analyzed to give an idea about the impressions of the users with each device. This is an important evaluation to seek their impressions of each method. The higher the score, the more the user was satisfaction with that method. The Table \ref{tab:questionnaire_average_blind} shows the score of each method and they are plotted in the Figure \ref{fig:barplot_questionnaire_scene_blind}. The Figure show a disatisfaction with the haptic devices alone.

\input{Resultados/Questionario/Tabelas/questionnaire_average_blind.tex}

\begin{figure}[!htb]
    \centering
    \includegraphics[width = 0.8\linewidth]{Resultados/Questionario/Figuras/png/barplot_questionnaire_scene_blind.png}
    \caption{Barplot of the average questionaire score of the blind participants on each method.}
    \label{fig:barplot_questionnaire_scene_blind}
\end{figure}

\begin{figure}[!htb]
    \centering
    \includegraphics[width = 0.6\linewidth]{Resultados/Questionario/Figuras/png/boxplot_questionnaire_scene_blind.png}
    \caption{Boxplot of the questionaire score of the blind participants grouped by method.}
    \label{fig:boxplot_quest_blind_scene}
\end{figure}

The Table \ref{tab:questionnaire_average_group_blind} show the the average questionnaire score on each method. It also shows a disatisfaction with the haptic devices alone.

\input{Resultados/Questionario/Tabelas/questionnaire_average_group_blind}

The Figures \ref{fig:qqplot_sagat_avg_two_way} and \ref{fig:residplot_sagat_avg_two_way} shows the distribution and variance of the Table \ref{tab:sagat_table_blind}. These Figures shows that the data are normally distributed and that the methods have a similar variance.
The Table \ref{tab:blocanova_sagat_avg_two_way} shows the Anova test p-value of the SAGAT score of the "blind" sample. The p-values indicates that the method have influence on the questionnaire score. Meaning that the participants had differents level os satisfaction about each method.

\input{Resultados/Questionario/Tabelas/blocanova_questionnaire.tex}

\begin{figure}[!htb]
    \centering
    \begin{minipage}{0.45\textwidth}
        \centering
        \includegraphics[width = 0.8\linewidth]{Resultados/Questionario/Figuras/png/qqplot_questionnaires.png}
        \caption{QQ plot of the questionnaire score of the blind participants on each method.}
        \label{fig:qqplot_sagat_avg_two_way}
    \end{minipage}
    \begin{minipage}{0.45\textwidth}
        \centering
        \includegraphics[width = 0.8\linewidth]{Resultados/Questionario/Figuras/png/residplot_questionnaires.png}
        \caption{Residual plot of the questionnaire score the blind participants on each method.}
        \label{fig:residplot_questionnaires}
    \end{minipage}
\end{figure}


The Table \ref{tab:lsd_questionnaire} presents the conclusion of a pairwise Fisher LSD test of the blind NASA-TLX score between all the guidance methods. The results show that only the "Audio" and "Mixture" have the same statistically result and that there is a difference between the both "Haptic Belt" and "Virtual Cane".

\input{Resultados/Questionario/Tabelas/lsd_questionnaire.tex}

The LSD Table \ref{tab:lsd_questionnaire} confirms the information of the Figure \ref{fig:boxplot_quest_blind_scene} that the “Audio” and the ”Mixture” methods were the most favorite by the blind participants, whilst the “Haptic Belt” and “Virtual Cane” were the most unfavorite devices. The participants did comment about those two last devices, saying that they were not precise enough, confusing and very different from what they are used to use.

\FloatBarrier
\subsection{Physiological data}

The same sensors used for the first objective are used for the second objective. The expectations for all of the results is a difference between the “blind” sample and the “sight” sample. This subsection was divided in the same way as before:

\begin{itemize}
    \item \nameref{subsubsec:results_ecg_2};
    
        Two features are extracted from the ECG, heartrate (BPM) and heartrate variance (SDNN).
    
        Is expected that the heartrate increases at every “First” round and then a slight decrease in the next round. The heartrate variance is expected to decrease in the “First” round and a slight increase in the next round.    

    \item \nameref{subsubsec:results_gsr_temp_2};
    
        Is expected that the GSR average to increase at every “First” round and then a slight decrease in the next round.

\end{itemize}
\subsubsection{Electrocardiogram (ECG) data}
\label{subsubsec:results_ecg_2}

\paragraph{Analysis of the heartbeat frequency (BPM)}\mbox{}\\

\input{Resultados/ECG/bpm2.tex}

%%%%%%%%%%%%%%%%%%%%%%%%%%%%%%%%%%%%%%%%%%%%%%%%%%%%%%%%%%%%%%%%%%%%%%%%%%%%
%%%%%%%%%%%%%%%%%%%%%%%%%%%%%%%%%%%%%%%%%%%%%%%%%%%%%%%%%%%%%%%%%%%%%%%%%%%%
%%%%%%%%%%%%%%%%%%%%%%%%%%%%%%%%%%%%%%%%%%%%%%%%%%%%%%%%%%%%%%%%%%%%%%%%%%%%
%%%%%%%%%%%%%%%%%%%%%%%%%%%%%%%%%%%%%%%%%%%%%%%%%%%%%%%%%%%%%%%%%%%%%%%%%%%%
%
%
\paragraph{Analysis of the heartbeat variance (SDNN)}\mbox{}\\
%
\input{Resultados/ECG/sdnn2}

\subsubsection{Galvanic skin response and temperature data;}
\label{subsubsec:results_gsr_temp_2}

Table \ref{tab:gsr_table_noBase} presents the average skin conductance for both groups, while the percentual variation related to the baseline is presented in Table \ref{tab:gsr_var_blind}.

\input{Resultados/GSR/Tabelas/gsr_table_noBase.tex}

\input{Resultados/GSR/Tabelas/gsr_var_noBase.tex}

The barplots of the two groups are presented in Figure \ref{fig:barplot_gsr_avg_4_scene_blind_sight}. If the variation between the round and the Baseline is positive, it means that the user had an increase on his/her Mental Workload or stress. While the GSR varied for the blind participants, increasing for methods with vibration, the same does not happen for sighted participants. Also, the variance of GSR data for blind participants is significantly higher than that of sighted ones. The same conclusion can be drawn from the boxplots in Figures \ref{fig:boxplot_ecg_sdnn_4_scene} and \ref{fig:boxplot_ecg_sdnn_4_rounds}. 

\begin{figure}[!htb]
    \centering
    \begin{minipage}{\textwidth}
        \centering
        \includegraphics[width = \textwidth]{Resultados/GSR/Figuras/pdf/barplot_gsr_avg_4_scene_blind.pdf}
        \subcaption{Blind participants.}
        \label{fig:barplot_gsr_avg_4_scene_blind}
    \end{minipage}
    \begin{minipage}{\textwidth}
        \centering
        \includegraphics[width = \textwidth]{Resultados/GSR/Figuras/pdf/barplot_gsr_avg_4_scene_sight.pdf}
        \subcaption{Sight participants.}
        \label{fig:barplot_gsr_avg_4_scene_sight}
    \end{minipage}
    \caption{Barplot of the average GSR on each method and round.}
    \label{fig:barplot_gsr_avg_4_scene_blind_sight}
\end{figure}

\begin{figure}[!htb]
    \centering
    \begin{minipage}{0.45\textwidth}
        \centering
        \includegraphics[width = \textwidth]{Resultados/GSR/Figuras/pdf/boxplot_gsr_avg_4_scene.pdf}
        \caption{Boxplot of the average GSR of the participants grouped by method.}
        \label{fig:boxplot_gsr_avg_4_scene}
    \end{minipage}
    \begin{minipage}{0.075\textwidth}
        \hfill
    \end{minipage}
    \begin{minipage}{0.45\textwidth}
        \centering
        \includegraphics[width = \textwidth]{Resultados/GSR/Figuras/pdf/boxplot_gsr_avg_4_rounds.pdf}
        \caption{Boxplot of the average GSR of the participants grouped by round.}
        \label{fig:boxplot_gsr_avg_4_rounds}
    \end{minipage}
\end{figure}

Figures \ref{fig:qqplot_gsr_two_way_sight} and \ref{fig:residplot_gsr_two_way_sight} bring the QQ Plot and residual distribution. The results from ANOVA are presented in Table \ref{tab:blocanova_gsr_two_way_blind_sight}. In the case of blind participants, the p-value for the method is just slightly over the threshold, indicating a possible influence of the method. The same does not happen with sighted participants, where the p-value of the method factor is the highest and well above the 0.05 threshold.
 
%The Table \ref{tab:gsr_average_group_noBase} shows the average skin conductance variation of both samples. It also shows that the presence of a haptic device increases the GSR, whilst the sight user had a basically constant GSR.
%
%\input{Resultados/GSR/Tabelas/gsr_average_group_noBase.tex}

\begin{table}[!htb]
    \caption{Anova p-value for the skin conductance average on each method}
    \label{tab:blocanova_gsr_two_way_blind_sight}
\begin{minipage}{0.45\textwidth}
    \subcaption{Blind participants}
    \input{Resultados/GSR/Tabelas/blocanova_gsr_two_way_blindsemBegin.tex}
\end{minipage}
\begin{minipage}{0.45\textwidth}
    \subcaption{Sight participants}
    \input{Resultados/GSR/Tabelas/blocanova_gsr_two_way_sightsemBegin.tex}
\end{minipage}
\end{table}

\begin{figure}[!htb]
    \centering
    %\vspace{-15.0cm}
    \begin{minipage}{0.45\textwidth}
        \centering
        \includegraphics[width = \textwidth]{Resultados/GSR/Figuras/pdf/qqplot_gsr_two_way_sight.pdf}
        \caption{QQ plot of the average skin conductance of the sight participants on each method.}
        \label{fig:qqplot_gsr_two_way_sight}
    \end{minipage}
    \begin{minipage}{0.075\textwidth}
        \hfill
    \end{minipage}
    \begin{minipage}{0.45\textwidth}
        \centering
        \includegraphics[width = \textwidth]{Resultados/GSR/Figuras/pdf/residplot_gsr_two_way_sight.pdf}
        \caption{Residual plot of the average skin conductance score the sight participants on each method.}
        \label{fig:residplot_gsr_two_way_sight}
    \end{minipage}
\end{figure}

\FloatBarrier 

\subsection{Final Remarks}

% Mental demand
% • ANOVA - Audio e Mix     VS   não deu resultado
% • Grafico - Não tem uma relação específica VS  Presença de haptico aumenta a demanda mental

% NASA TLX
% • ANOVA - Round  VS  - Efeito do método e do round.
% •• Difere do ANOVA MD     Metodo - 3 grupos
% •••                       Base e Audio menores Carga mental
% ••• Sem padrão no tipo de dispositivo  Haptic alta Carga mental
% 
% SAGAT
% ANOVA - Método tem efeito     VS   Round tem efeito no SAGAT
% Nota menor nas mesmas condições
%

% BPM
% ANOVA - não deu resultado VS IDEM
% Gráfico mostra uma diferença entre os métodos
% Maior BPM que os cegos

% SDNN
% ANOVA - não deu resultado VS IDEM
% Gráfico - "Base" tem uma variância menor
% Valor de SDNN maior que o dos cegos

% GSR
% ANOVA - não deu resultado VS IDEM
% Grafico - métodos muito parecidos VS metodos com impactos diferentes

Differently than the blind users, the results from the mental demand discipline of the NASA-TLX proved that the sight users felt a higher mental demand than the blind users. 

The overall NASA-TLX score also proved a different conclusion than the one in the Section \ref{subsubsec:results_nasa_tlx_1}. For the sighted users, the round impacted more the overall score than the methods, whilst for the blind user were the opposite. This may be because the overall score is composed of 6 dimensions. Probably for the sighted user the mental demand score was higher and for the blind user it was not. But even so, the average score of the sight user was higher than the score of the blind user.

The Adapted SAGAT questionnaire for the sight users proved that the method impacts their situation awareness. The conclusion was different than the one proved by the blind users in the Section \ref{subsubsec:results_adapted_sagat_1}, who felt a bigger impact between the rounds than between the methods. The sight performance was also poorer than the blind user.

The guidance questionnaire of both groups had a similar distribution, both groups enjoy the same methods, "Audio" and "Mixture". The difference was in the last two. The sighted users rather use the "Virtual Cane" and the blind users rather use the "Haptic Belt".

These conclusion show that the sighted users were more sensible to the methods than the blind users, although the effects were different. The blind users were more impacted by the methods than by the rounds, and when impacted by the methods, it was not possible to detect a pattern about the presence or not of a haptic device, as happened with the blind users.

The ECG sensors shown a difference in the heartrate between the methods, but the ANOVA test was not able to prove that difference, the same conclusion of the blind users in the Section \ref{subsubsec:results_ecg_1}. Another observation is that the heartrate frequency of the sighte user was higher than the blind users, meaning that their mental workload was probably higher.

According to the ANOVA test, The heartbeat variance also was not impact by the method or by the rounds, the same conclusion for the blind users in the Section \ref{subsubsec:results_ecg_1}. Graphically there was a small difference in the methods. Despite the results of the heartrate, the variance of the sight user was higher than the results from the blind user, meaning that the mental workload of the "sight" sample was higher than the one of "blind" sample.

The GSR ANOVA test also did not detect any impact from the methods or from the rounds, as it happened with the blind users in the Section \ref{subsubsec:results_gsr_temp_1}. Graphically the sight user variations were very similar in all methods. This is a different effect than the one observed in the blind users, which graphically showed different GSR distributions on different methods. The sight GSR also show a small variation between the rounds and methods, which means that the sight user were not stressed or had a low mental workload during the experiment.

Despite the proved and not proved tests, there is a consideration to be made. The sight sample group profiling. As already explained before, the profile of the “blind” sample group was very wide and that can impact negatively in their performance. But the opposite effect may had happened with the “sight” sample group. This group was composed basically by researchers and engineer students, people that are typically involved with computers and technological devices, aging from 22 to 31 with an average of 27.5 years. This may biased the results with better performance when using the HMD and being able to feel present inside a virtual environment.

Besides these results, the “sighted” sample also commented the experiment. They all felt a lot more insecurity when walking, exploring and even when hand guided by the researcher before the start of the round. The “blind” sample group was already used to bumping their body when exploring new closed quarters. The “sighted” group did not want that to happen and approached the furniture with a lot more caution. They also noticed the lack of precision of the haptic devices, but they did rely more on then to navigate.

%The processing of each data collected is rather similar and follows these steps:
%\begin{enumerate}
%    \item Separate the Blind sample and the Sight sample;
%    \item Check if the samples are normally distributed; \label{itm:results_shapiro} \\
%        If the data is normally distributed then it is possible to use other statistical analyses and verify the results statistically.
%    \item Check if the "blind" sample is statistically different then the "sight" sample; \label{itm:results_t_test} \\ 
%        This is one of the goals. To verify that the workload and the situation awareness of the blind participants are different from the sighted participants.
%    \item For the physiological data:
%    \begin{enumerate}
%        \item Calculate the variation between the scene and the baseline  in each method;
%        \item Calculate the variation between the scenes in each method;
%        \item Calculate the average variation in the group in each method.
%    \end{enumerate}
%    \item For the other data:
%    \begin{enumerate}
%        \item Calculate the variation between the scenes in each method; \label{itm:results_average_method_particpant}
%        \item Calculate the average variation in the group in each method.
%        \label{itm:results_average_method}
%    \end{enumerate}
%        The variation mentioned above is calculated as Equation \ref{eq:variation}:
%    \begin{equation}
%        \label{eq:variation}
%        Var_{ij} = \frac{Obs_j - Obs_i}{Obs_i}*100 [\%]
%    \end{equation}
%        Where:\\
%        $Var_{ij}$ = Variation between the two sequential observations \\
%        $Obs_i$ = First observation \\
%        $Obs_j$ = Second observation 
%    
%    \item Check if there are methods that are statistically different from the rest. \\
%        This is done by doing an analyses of variance (ANOVA) in the data. For this analyses is required that the observerd residues are normally distributed and the variance is constant. The residues are all plotted in the figures of the Appendix \ref{ap:figures}. The variance constancy can be verified by the box plot presented in the next sections.
%
%\end{enumerate}

\chapter{Conclusion}
\label{ch:conclusao}
% CAPÍTULO 5 – CONCLUSÕES
%   NÃO PODE TER APENAS UMA PÁGINA!
%   O assentamento do último tijolo é tão importante quanto o do primeiro

%1. Elabore um parágrafo que introduz o capítulo: Este capítulo apresenta (descreva o objetivo do capítulo...). É constituído de N seções a saber...
%2. O Capítulo Conclusões não “gosta” de novidades
%3. Responda a pergunta da pesquisa com os elementos que você pesquisou e desenvolveu
%4. Análise do atendimento dos objetivos específicos:
%   • Descreva SE e COMO os objetivos específicos foram atendidos. Utilize as informações e resultados já apresentados nos Capítulos 3 e 4.
%5. Principais resultados obtidos:
%   • reafirme os resultados mais importantes;
%   • retome o posicionamento do seu trabalho em relação à literatura.
%6. Limitações do trabalho
%   • descreva as dificuldades encontradas;
%   • analise a delimitação do trabalho (Cap. 1) e as limitações de sua contribuição.


%7. Propostas de desenvolvimentos futuros – aquilo que você faria se tivesse tempo.
%   • Esta seção não é uma lista com marcadores!
%   • Descreva 3 propostas baseadas nas limitações de seu trabalho e as detalhe a um ponto que um outro pesquisador possa retomar e desenvolver essa proposta.

In this final chapter, the goals will be revised along with the results collected. It will be divided into four sections, one for each goal and a final one for future works and suggestions, and each section will have four more subsections, one for each data source gathered and one for a conclusion and commentaries for that goal.

%    \item Do BVI users feel present in the VE as if they were in the real world? \label{itm:obj_first}
%    \item BVI users rely on audio cues and haptic feedback to guide. But does it rely more on your type of information than the other? \label{itm:obj_second}
%    \item Do non-BVI users have the same demands and skills as BVI users when designing assistive products? \label{itm:obj_third}

%%%%%%%%%%%%%%%%%%%%%%%%%%%%%%%%%%%%%%%%%%%%%%%%%%%%%%%%%%%%%%%%%%%%%%%

\subsection*{Is it possible to evaluate and compare concepts of assistive device from a human factors’ perspective in a virtual environment? What are the main limitations of the use of a virtual reality environment?
}

As for the experiment used for to study this goal, the blind users were more affected by the rounds them by the methods, in both mental workload and situation awareness. And when impacted by the methods, the presence of a haptic device provoked negative conseguences on their perception or mental workload.

For the sighted users it were affected by both method and round in some cases, meaning that they were more sensible to the experiment than the blind users. There was no pattern in which method they performed better or not but in overall their performance was inferior than the perfomance from blind users.
    
Based on the gathered data, there was a variation in the mental workload and in the situation awareness during the experiment. This variation show that the users were impacted by the experiment in the virtual reality, but since no experiment outside the virtual reality was made, it is not possible to compare this data and verify that they are similar to one provided by a real scenario.

Some problems of the method was that the users walked approximately half of the experiment duration, and problably added some noise to the sensor data, leaving to unrelatable results. The heartbeat and the interbeat interval standard deviation did not show the same results as the NASA-TLX indicated. 

This could also be caused because the experiment was made using a virtual reality, a technology still unvisited by most of the participants. That could have made the participant anxious and risen their heartbeat at the beginning of the experiment.

As for the limitations, the participants complained about the sound. The integrate headphone of the VIVE HMD did not provide sounds with a quality good enought for them to locate. A common commentarie was "I feel like the sound origin is inside my head", which was not true. But this may be solved by placing a real sound source in the real environment and use the HMD only for geolocalizing the participant inside de virtual environment.

Another limitation is the real time position of the furniture. More than once, after a "First round" a furniture was not well aligned with the its virtual model. That caused some frustation on the participant as well in the researchers that had to stop the experiment to fix their position. A solution for this it would be to install real time locator on each piece of furniture.

\subsection*{Do non-BVI users, when deprived from their vision, evaluate assistive devices in a similar way as BVI users?}

Comparing the results from the analyzes of the "blind" sample and the "sight" sample one realizes that the groups felt different reaction. Most of the blind users felt a bigger impact between the rounds than by the methods, whilst the sight users felt a bigger impact by the method.

This may be biased, since most of the conception of the devices was made by a sighted researcher, even thought there was recommendations from BVI researchers. But this may only reinterate that sighted users have a difficulty imagine how a blind users perceives the world and how to develop ways to assist them.


%%%%%%%%%%%%%%%%%%%%%%%%%%%%%%%%%%%%%%%%%%%%%%%%%%%%%%%%%%%%%%%%%%%%%%%

\section{Future works and suggestions}

For future works related to this one it could be suggested:

\begin{itemize}
    \item Repeat the experiment in a real situation and compare it with this one to verify the first goal;
    
    This experiment was made exclusivily using virtual reality, hence it is not posibly to verify the efficiency or the quality of a experiment made using Virtual Reality.

    \item Repeat the experiment not using the sound from the HMD;
     
    The BVI users complained about the VIVE HMD sound system. They got confused sometimes and could figure it out if a sound source was coming from forwards or backwards. One alternative for this problem is to add a physical sound source at each point in the real environment where it was supose to be in the virtual environment. It still related to the virtual reality but it is more realistic. 

    \item Repeat the experiment with bigger sample size and a more diverse sample to verify if the results of the hypothesis test do remain the same;
    
    As commented before, most ANOVA tests showed one result and the figures showed a different conclusion. This happend because of the small sample size. If the sample size was bigger, maybe both conclusions would be the same.

\end{itemize}

% REFERENCIAS BIBLIOGRAFICAS
\renewcommand\bibname{\itareferencesnamebabel} %renomear título do capítulo referências
%\bibliographystyle{abnt-num}
\bibliography{Bibliography/bibliography}

% Apendices
\appendix
\chapter{Questionnaires}
\label{ap:questionnaires}
%APÊNDICE

%É um conteúdo que você elaborou (você ainda tem o seu apêndice intestinal?)

\section{BVI condition}
\label{sec:bvi_condition}
\vspace{0.25in}
{\color{gray}

Código de identificação do voluntário: \rule{1in}{.2mm}

\textit{Os dados deste questionário são anônimos e somente serão utilizados para fins acadêmicos e de pesquisa, ficando proibido seu manuseio ou uso sem consentimento do coordenador da pesquisa.}
}

\begin{center}
\textbf{Parte 1 - Questionário sobre dados pessoais.}
\end{center}


\textbf{Idade:} \rule{1in}{.2mm} \hfill \textbf{Sexo:}   ( ) Masculino \hfill ( ) Feminino

\vspace{2cm}

Como deficiente visual você se identifica como:

\noindent
( ) Cego \hfill \break
( ) Surdo-Cego \hfill \break
( ) Baixa Visão \hfill \break
( ) Monocular \hfill \break
( ) Não sou deficiente visual \hfill \break

USO DO PESQUISADOR

\vspace{1cm}

Ordem dos cenários:

\vspace{1cm}

\rule{10cm}{0.1mm}

\pagebreak


\section{NASA-TLX}
\label{apsec:nasa_tlx}
{\color{gray}

Código de identificação do voluntário: \rule{1in}{.2mm}

\textit{Os dados deste questionário são anônimos e somente serão utilizados para fins acadêmicos e de pesquisa, ficando proibido seu manuseio ou uso sem consentimento do coordenador da pesquisa.}}

\begin{center}
\textbf{Parte 2 - Questionário sobre carga mental (NASA-TLX).}
\end{center}

\noindent
\textbf{TESTE} ( ) BASE \hfill ( ) ÁUDIO \hfill ( ) CINTO HÁPTICO \hfill ( ) BENGALA \hfill ( ) MISTO

Avalie cada um dos itens abaixo em uma escala de 1 a 20.

\hspace{1cm}

%1) Demanda Mental
%
%%\resizebox{\linewidth}{\height}{
\resizebox{\linewidth}{!}{
    \begin{tabular}{llllllllllllllllllllll}
        &   &   &   &   &   &   &   &   &   &  &    &    &    &    &    &    &    &    &    &  & \\
        \multicolumn{1}{lV{3.56}}{}&   &   &   &   &   &   &   &   &   &\multicolumn{1}{lV{3.56}}{} &    &    &    &    &    &    &    &    &    & \multicolumn{1}{lV{3.56}}{}   & \\
        \multicolumn{1}{lV{3.56}}{}& \multicolumn{1}{l|}{1} & \multicolumn{1}{l|}{2} & \multicolumn{1}{l|}{3} & \multicolumn{1}{l|}{4} & \multicolumn{1}{l|}{5} & \multicolumn{1}{l|}{6} & \multicolumn{1}{l|}{7} & \multicolumn{1}{l|}{8} & \multicolumn{1}{l|}{9} & \multicolumn{1}{lV{3.56}}{10} & \multicolumn{1}{l|}{11} & \multicolumn{1}{l|}{12} & \multicolumn{1}{l|}{13} & \multicolumn{1}{l|}{14} & \multicolumn{1}{l|}{15} & \multicolumn{1}{l|}{16} & \multicolumn{1}{l|}{17} & \multicolumn{1}{l|}{18} & \multicolumn{1}{l|}{19} & \multicolumn{1}{lV{3.56}}{20} & \\
        \clineB{2-21}{3.56}
        \multicolumn{3}{l}{Baixo} &   &   &   &   &   &   &   &    &    &    &    &    &    &    &    &    &  \multicolumn{3}{r}{Alto} \\
        &   &   &   &   &   &   &   &   &   &  &    &    &    &    &    &    &    &    &    &  & \\
    \end{tabular}
}
%
%2) Demanda Física
%
%%\resizebox{\linewidth}{\height}{
\resizebox{\linewidth}{!}{
    \begin{tabular}{llllllllllllllllllllll}
        &   &   &   &   &   &   &   &   &   &  &    &    &    &    &    &    &    &    &    &  & \\
        \multicolumn{1}{lV{3.56}}{}&   &   &   &   &   &   &   &   &   &\multicolumn{1}{lV{3.56}}{} &    &    &    &    &    &    &    &    &    & \multicolumn{1}{lV{3.56}}{}   & \\
        \multicolumn{1}{lV{3.56}}{}& \multicolumn{1}{l|}{1} & \multicolumn{1}{l|}{2} & \multicolumn{1}{l|}{3} & \multicolumn{1}{l|}{4} & \multicolumn{1}{l|}{5} & \multicolumn{1}{l|}{6} & \multicolumn{1}{l|}{7} & \multicolumn{1}{l|}{8} & \multicolumn{1}{l|}{9} & \multicolumn{1}{lV{3.56}}{10} & \multicolumn{1}{l|}{11} & \multicolumn{1}{l|}{12} & \multicolumn{1}{l|}{13} & \multicolumn{1}{l|}{14} & \multicolumn{1}{l|}{15} & \multicolumn{1}{l|}{16} & \multicolumn{1}{l|}{17} & \multicolumn{1}{l|}{18} & \multicolumn{1}{l|}{19} & \multicolumn{1}{lV{3.56}}{20} & \\
        \clineB{2-21}{3.56}
        \multicolumn{3}{l}{Baixo} &   &   &   &   &   &   &   &    &    &    &    &    &    &    &    &    &  \multicolumn{3}{r}{Alto} \\
        &   &   &   &   &   &   &   &   &   &  &    &    &    &    &    &    &    &    &    &  & \\
    \end{tabular}
}
%
%3) Demanda Temporal
%
%%\resizebox{\linewidth}{\height}{
\resizebox{\linewidth}{!}{
    \begin{tabular}{llllllllllllllllllllll}
        &   &   &   &   &   &   &   &   &   &  &    &    &    &    &    &    &    &    &    &  & \\
        \multicolumn{1}{lV{3.56}}{}&   &   &   &   &   &   &   &   &   &\multicolumn{1}{lV{3.56}}{} &    &    &    &    &    &    &    &    &    & \multicolumn{1}{lV{3.56}}{}   & \\
        \multicolumn{1}{lV{3.56}}{}& \multicolumn{1}{l|}{1} & \multicolumn{1}{l|}{2} & \multicolumn{1}{l|}{3} & \multicolumn{1}{l|}{4} & \multicolumn{1}{l|}{5} & \multicolumn{1}{l|}{6} & \multicolumn{1}{l|}{7} & \multicolumn{1}{l|}{8} & \multicolumn{1}{l|}{9} & \multicolumn{1}{lV{3.56}}{10} & \multicolumn{1}{l|}{11} & \multicolumn{1}{l|}{12} & \multicolumn{1}{l|}{13} & \multicolumn{1}{l|}{14} & \multicolumn{1}{l|}{15} & \multicolumn{1}{l|}{16} & \multicolumn{1}{l|}{17} & \multicolumn{1}{l|}{18} & \multicolumn{1}{l|}{19} & \multicolumn{1}{lV{3.56}}{20} & \\
        \clineB{2-21}{3.56}
        \multicolumn{3}{l}{Baixo} &   &   &   &   &   &   &   &    &    &    &    &    &    &    &    &    &  \multicolumn{3}{r}{Alto} \\
        &   &   &   &   &   &   &   &   &   &  &    &    &    &    &    &    &    &    &    &  & \\
    \end{tabular}
}
%
%4) Desempenho
%
%%\resizebox{\linewidth}{\height}{
\resizebox{\linewidth}{!}{
    \begin{tabular}{llllllllllllllllllllll}
        &   &   &   &   &   &   &   &   &   &  &    &    &    &    &    &    &    &    &    &  & \\
        \multicolumn{1}{lV{3.56}}{}&   &   &   &   &   &   &   &   &   &\multicolumn{1}{lV{3.56}}{} &    &    &    &    &    &    &    &    &    & \multicolumn{1}{lV{3.56}}{}   & \\
        \multicolumn{1}{lV{3.56}}{}& \multicolumn{1}{l|}{1} & \multicolumn{1}{l|}{2} & \multicolumn{1}{l|}{3} & \multicolumn{1}{l|}{4} & \multicolumn{1}{l|}{5} & \multicolumn{1}{l|}{6} & \multicolumn{1}{l|}{7} & \multicolumn{1}{l|}{8} & \multicolumn{1}{l|}{9} & \multicolumn{1}{lV{3.56}}{10} & \multicolumn{1}{l|}{11} & \multicolumn{1}{l|}{12} & \multicolumn{1}{l|}{13} & \multicolumn{1}{l|}{14} & \multicolumn{1}{l|}{15} & \multicolumn{1}{l|}{16} & \multicolumn{1}{l|}{17} & \multicolumn{1}{l|}{18} & \multicolumn{1}{l|}{19} & \multicolumn{1}{lV{3.56}}{20} & \\
        \clineB{2-21}{3.56}
        \multicolumn{3}{l}{Baixo} &   &   &   &   &   &   &   &    &    &    &    &    &    &    &    &    &  \multicolumn{3}{r}{Alto} \\
        &   &   &   &   &   &   &   &   &   &  &    &    &    &    &    &    &    &    &    &  & \\
    \end{tabular}
}
%
%5) Esforço
%
%%\resizebox{\linewidth}{\height}{
\resizebox{\linewidth}{!}{
    \begin{tabular}{llllllllllllllllllllll}
        &   &   &   &   &   &   &   &   &   &  &    &    &    &    &    &    &    &    &    &  & \\
        \multicolumn{1}{lV{3.56}}{}&   &   &   &   &   &   &   &   &   &\multicolumn{1}{lV{3.56}}{} &    &    &    &    &    &    &    &    &    & \multicolumn{1}{lV{3.56}}{}   & \\
        \multicolumn{1}{lV{3.56}}{}& \multicolumn{1}{l|}{1} & \multicolumn{1}{l|}{2} & \multicolumn{1}{l|}{3} & \multicolumn{1}{l|}{4} & \multicolumn{1}{l|}{5} & \multicolumn{1}{l|}{6} & \multicolumn{1}{l|}{7} & \multicolumn{1}{l|}{8} & \multicolumn{1}{l|}{9} & \multicolumn{1}{lV{3.56}}{10} & \multicolumn{1}{l|}{11} & \multicolumn{1}{l|}{12} & \multicolumn{1}{l|}{13} & \multicolumn{1}{l|}{14} & \multicolumn{1}{l|}{15} & \multicolumn{1}{l|}{16} & \multicolumn{1}{l|}{17} & \multicolumn{1}{l|}{18} & \multicolumn{1}{l|}{19} & \multicolumn{1}{lV{3.56}}{20} & \\
        \clineB{2-21}{3.56}
        \multicolumn{3}{l}{Baixo} &   &   &   &   &   &   &   &    &    &    &    &    &    &    &    &    &  \multicolumn{3}{r}{Alto} \\
        &   &   &   &   &   &   &   &   &   &  &    &    &    &    &    &    &    &    &    &  & \\
    \end{tabular}
}
%
%6) Frustração
%
%%\resizebox{\linewidth}{\height}{
\resizebox{\linewidth}{!}{
    \begin{tabular}{llllllllllllllllllllll}
        &   &   &   &   &   &   &   &   &   &  &    &    &    &    &    &    &    &    &    &  & \\
        \multicolumn{1}{lV{3.56}}{}&   &   &   &   &   &   &   &   &   &\multicolumn{1}{lV{3.56}}{} &    &    &    &    &    &    &    &    &    & \multicolumn{1}{lV{3.56}}{}   & \\
        \multicolumn{1}{lV{3.56}}{}& \multicolumn{1}{l|}{1} & \multicolumn{1}{l|}{2} & \multicolumn{1}{l|}{3} & \multicolumn{1}{l|}{4} & \multicolumn{1}{l|}{5} & \multicolumn{1}{l|}{6} & \multicolumn{1}{l|}{7} & \multicolumn{1}{l|}{8} & \multicolumn{1}{l|}{9} & \multicolumn{1}{lV{3.56}}{10} & \multicolumn{1}{l|}{11} & \multicolumn{1}{l|}{12} & \multicolumn{1}{l|}{13} & \multicolumn{1}{l|}{14} & \multicolumn{1}{l|}{15} & \multicolumn{1}{l|}{16} & \multicolumn{1}{l|}{17} & \multicolumn{1}{l|}{18} & \multicolumn{1}{l|}{19} & \multicolumn{1}{lV{3.56}}{20} & \\
        \clineB{2-21}{3.56}
        \multicolumn{3}{l}{Baixo} &   &   &   &   &   &   &   &    &    &    &    &    &    &    &    &    &  \multicolumn{3}{r}{Alto} \\
        &   &   &   &   &   &   &   &   &   &  &    &    &    &    &    &    &    &    &    &  & \\
    \end{tabular}
}
\textbf{Primeira visita}

\begin{table}[!h]
\centering
    \def\arraystretch{0.5}
    \begin{tabular}{m{0.25\textwidth} m{0.70\textwidth}}
        1) Demanda Mental & \begin{center}\includegraphics[width = 0.7\textwidth]{ApendC_(Questionarios)/Escala_NASA.png}\end{center}\\
        2) Demanda Física & \begin{center}\includegraphics[width = 0.7\textwidth]{ApendC_(Questionarios)/Escala_NASA.png}\end{center}\\
        3) Demanda Temporal & \begin{center}\includegraphics[width = 0.7\textwidth]{ApendC_(Questionarios)/Escala_NASA.png}\end{center}\\
        4) Desempenho     & \begin{center}\includegraphics[width = 0.7\textwidth]{ApendC_(Questionarios)/Escala_NASA.png}\end{center}\\
        5) Esforço        & \begin{center}\includegraphics[width = 0.7\textwidth]{ApendC_(Questionarios)/Escala_NASA.png}\end{center}\\
        6) Frustração     & \begin{center}\includegraphics[width = 0.7\textwidth]{ApendC_(Questionarios)/Escala_NASA.png}\end{center}\\
    \end{tabular}
\end{table}
\pagebreak

\textbf{Retorno}

\begin{table}[!h]
\centering
    \def\arraystretch{0.5}
    \begin{tabular}{m{0.25\textwidth} m{0.70\textwidth}}
        1) Demanda Mental & \begin{center}\includegraphics[width = 0.7\textwidth]{ApendC_(Questionarios)/Escala_NASA.png}\end{center}\\
        2) Demanda Física & \begin{center}\includegraphics[width = 0.7\textwidth]{ApendC_(Questionarios)/Escala_NASA.png}\end{center}\\
        3) Demanda Temporal & \begin{center}\includegraphics[width = 0.7\textwidth]{ApendC_(Questionarios)/Escala_NASA.png}\end{center}\\
        4) Desempenho     & \begin{center}\includegraphics[width = 0.7\textwidth]{ApendC_(Questionarios)/Escala_NASA.png}\end{center}\\
        5) Esforço        & \begin{center}\includegraphics[width = 0.7\textwidth]{ApendC_(Questionarios)/Escala_NASA.png}\end{center}\\
        6) Frustração     & \begin{center}\includegraphics[width = 0.7\textwidth]{ApendC_(Questionarios)/Escala_NASA.png}\end{center}\\
    \end{tabular}
\end{table}

\pagebreak

\pagebreak

\section{Adapted SAGAT}
\label{apsec:sagat}
%{\color{gray}
%
%Código de identificação do voluntário: \rule{1in}{.2mm}
%
%\textit{Os dados deste questionário são anônimos e somente serão utilizados para fins acadêmicos e de pesquisa, ficando proibido seu manuseio ou uso sem consentimento do coordenador da pesquisa.}
%}
%
%\begin{center}
%\textbf{Parte 3 - Questionário sobre consciência situacional (SAGAT).}
%\end{center}
%
%\noindent
%\textbf{TESTE} ( ) BASE \hfill ( ) ÁUDIO \hfill ( ) CINTO HÁPTICO \hfill ( ) BENGALA \hfill ( ) MISTO
%
%\textbf{Primeira visita}

\begin{table}[!htb]
    \centering
    \begin{tabular}{m{1\linewidth}}

        {\color{gray}
        
        Código de identificação do voluntário: \rule{1in}{.2mm}
        
        \textit{Os dados deste questionário são anônimos e somente serão utilizados para fins acadêmicos e de pesquisa, ficando proibido seu manuseio ou uso sem consentimento do coordenador da pesquisa.}
        }
        
        \begin{center}
        \textbf{Parte 3 - Questionário sobre consciência situacional (SAGAT).}
        \end{center}
        
        \noindent
        \textbf{TESTE} ( ) BASE \hfill ( ) ÁUDIO \hfill ( ) CINTO HÁPTICO \hfill ( ) BENGALA \hfill ( ) MISTO
        
    \end{tabular}
%\end{table}

%\begin{table}[!htb]
    \centering
    \begin{tabular}{m{1\linewidth}}
        \textbf{Primeira visita}
    \end{tabular}
%\end{table}

\hspace{0.5cm}

%\begin{table}[!htb]
    \centering
    \begin{tabular}{m{0.5\linewidth} m{0.5\linewidth}}
        \large{Nível 1 – Percepção}  &\\
        & \\
        %---------------------------------
        1) Existe algum objeto próximo de você? & 2) Sinalize onde o objeto está: \\
        & \\
        \rule{\linewidth}{.2mm} & \begin{center}\multirow{5}{*}{\includegraphics[width = 0.45\linewidth]{ApendC_(Questionarios)/diagrama_sagat.png}} \end{center}\\
        \rule{\linewidth}{.2mm} & \\
        & \\
        \rule{\linewidth}{.2mm} & \\
        & \\
        \rule{\linewidth}{.2mm} & \\
        & \\
\end{tabular}
\end{table}
\begin{table}[!htb]
    \begin{tabular}{m{0.5\linewidth} m{0.5\linewidth}}
         %---------------------------------
         3)	Existe alguém perto de você? & 4) Sinalize onde a pessoa está: \\
        & \\
        \rule{\linewidth}{.2mm} & \begin{center}\multirow{5}{*}{\includegraphics[width = 0.45\linewidth]{ApendC_(Questionarios)/diagrama_sagat.png}} \end{center}\\
        \rule{\linewidth}{.2mm} & \\
        & \\
        \rule{\linewidth}{.2mm} & \\
        & \\
        \rule{\linewidth}{.2mm} & \\
        & \\
\end{tabular}
\end{table}
\begin{table}[!htb]
    \begin{tabular}{m{0.5\linewidth} m{0.5\linewidth}}
        %---------------------------------
         5)	Você percebeu alguma fonte de som característica do lugar onde você se encontra? & 6)	Sinalize de onde vem o som: \\
        & \\
        \rule{\linewidth}{.2mm} & \begin{center}\multirow{5}{*}{\includegraphics[width = 0.45\linewidth]{ApendC_(Questionarios)/diagrama_sagat.png}} \end{center}\\
        \rule{\linewidth}{.2mm} & \\
        & \\
        \rule{\linewidth}{.2mm} & \\
        & \\
        \rule{\linewidth}{.2mm} & \\
        & \\
        %---------------------------------
    \end{tabular}
\end{table}
\begin{table}[!htb]
    \begin{tabular}{m{0.5\linewidth} m{0.5\linewidth}}
        %---------------------------------
        \large{Nível 2 – Compreensão}  &\\
        & \\
        & \\
        %---------------------------------
        7)	Sinalize em que direção está a recepcionista: & 8)	Sinalize em que direção está a saída: \\
        & \\
        \begin{center}\multirow{5}{*}{\includegraphics[width = 0.45\linewidth]{ApendC_(Questionarios)/diagrama_sagat.png}} \end{center} & \begin{center}\multirow{5}{*}{\includegraphics[width = 0.45\linewidth]{ApendC_(Questionarios)/diagrama_sagat.png}} \end{center}\\
        & \\
        & \\
        & \\
        & \\
        & \\
        %---------------------------------
    \end{tabular}
\end{table}
\begin{table}[!htb]
    \begin{tabular}{m{0.5\linewidth} m{0.5\linewidth}}
        %---------------------------------
        \large{Nível 3 – Projeção}  &\\
        & \\
        & \\
        %---------------------------------
        9)	O que você espera encontrar no caminho ao retornar à mesa de recepção? & 10)	Qual será a sua trajetória para sair? Descreva seus próximos movimentos. \\
        & \\
        \rule{\linewidth}{.2mm} & \rule{\linewidth}{.2mm}\\
        & \\
        \rule{\linewidth}{.2mm} & \rule{\linewidth}{.2mm}\\
        & \\
        \rule{\linewidth}{.2mm} & \rule{\linewidth}{.2mm}\\
        & \\
        \rule{\linewidth}{.2mm} & \\
        & \\
\end{tabular}
\end{table}
\begin{table}[!htb]
    \begin{tabular}{m{0.5\linewidth} m{0.5\linewidth}}
        %---------------------------------
        11)	Qual a distância você imagina que existe entre você e a mesa da recepção? & 12)	Qual a distância você imagina que existe entre você e a saída? \\
        & \\
        \rule{\linewidth}{.2mm} & \rule{\linewidth}{.2mm}\\
        & \\
        \rule{\linewidth}{.2mm} & \rule{\linewidth}{.2mm}\\
        & \\
        \rule{\linewidth}{.2mm} & \rule{\linewidth}{.2mm}\\
        & \\
        \rule{\linewidth}{.2mm} & \\
        & \\
        %---------------------------------
    \end{tabular}
\end{table}
\vfill


%\FloatBarrier

%\begin{table}[!htb]
    \centering
    \begin{tabular}{p{0.5\linewidth} p{0.5\linewidth}}
        \large{Nível 1 – Percepção}  &\\
        & \\
        %---------------------------------
        1) Existe algum objeto próximo de você? & 2) Sinalize onde o objeto está: \\
        & \\
        \rule{\linewidth}{.2mm} & \begin{center}\multirow{5}{*}{\includegraphics[width = 0.45\linewidth]{ApendC_(Questionarios)/diagrama_sagat.png}} \end{center}\\
        \rule{\linewidth}{.2mm} & \\
        & \\
        \rule{\linewidth}{.2mm} & \\
        & \\
        \rule{\linewidth}{.2mm} & \\
        & \\
\end{tabular}
%\end{table}
%\begin{table}[!htb]
    \begin{tabular}{p{0.5\linewidth} p{0.5\linewidth}}
         %---------------------------------
         3)	Existe alguém perto de você? & 4) Sinalize onde a pessoa está: \\
        & \\
        \rule{\linewidth}{.2mm} & \begin{center}\multirow{5}{*}{\includegraphics[width = 0.45\linewidth]{ApendC_(Questionarios)/diagrama_sagat.png}} \end{center}\\
        \rule{\linewidth}{.2mm} & \\
        & \\
        \rule{\linewidth}{.2mm} & \\
        & \\
        \rule{\linewidth}{.2mm} & \\
        & \\
\end{tabular}
%\end{table}
%\begin{table}[!htb]
    \begin{tabular}{p{0.5\linewidth} p{0.5\linewidth}}
        %---------------------------------
         5)	Você percebeu alguma fonte de som característica do lugar onde você se encontra? & 6)	Sinalize de onde vem o som: \\
        & \\
        \rule{\linewidth}{.2mm} & \begin{center}\multirow{5}{*}{\includegraphics[width = 0.45\linewidth]{ApendC_(Questionarios)/diagrama_sagat.png}} \end{center}\\
        \rule{\linewidth}{.2mm} & \\
        & \\
        \rule{\linewidth}{.2mm} & \\
        & \\
        \rule{\linewidth}{.2mm} & \\
        & \\
        %---------------------------------
    \end{tabular}
\end{table}

\begin{table}[!htb]
    \centering
    \begin{tabular}{m{1\linewidth}}

        {\color{gray}
        
        Código de identificação do voluntário: \rule{1in}{.2mm}
        
        \textit{Os dados deste questionário são anônimos e somente serão utilizados para fins acadêmicos e de pesquisa, ficando proibido seu manuseio ou uso sem consentimento do coordenador da pesquisa.}
        }
        
        \begin{center}
        \textbf{Parte 3 - Questionário sobre consciência situacional (SAGAT).}
        \end{center}
        
        \noindent
        \textbf{TESTE} ( ) BASE \hfill ( ) ÁUDIO \hfill ( ) CINTO HÁPTICO \hfill ( ) BENGALA \hfill ( ) MISTO
        
    \end{tabular}
%end{table}
%\begin{table}[!htb]
    \begin{tabular}{p{0.5\linewidth} p{0.5\linewidth}}
        %---------------------------------
        \large{Nível 2 – Compreensão}  &\\
        & \\
        & \\
        %---------------------------------
        7)	Sinalize em que direção está a recepcionista: & 8)	Sinalize em que direção está a saída: \\
        & \\
        \begin{center}\multirow{5}{*}{\includegraphics[width = 0.45\linewidth]{ApendC_(Questionarios)/diagrama_sagat.png}} \end{center} & \begin{center}\multirow{5}{*}{\includegraphics[width = 0.45\linewidth]{ApendC_(Questionarios)/diagrama_sagat.png}} \end{center}\\
        & \\
        & \\
        & \\
        & \\
        & \\
        %---------------------------------
    \end{tabular}
%\end{table}
%\begin{table}[!htb]
    \begin{tabular}{p{0.5\linewidth} p{0.5\linewidth}}
        %---------------------------------
        \large{Nível 3 – Projeção}  &\\
        & \\
        & \\
        %---------------------------------
        9)	O que você espera encontrar no caminho ao retornar à mesa de recepção? & 10)	Qual será a sua trajetória para sair? Descreva seus próximos movimentos. \\
        & \\
        \rule{\linewidth}{.2mm} & \rule{\linewidth}{.2mm}\\
        & \\
        \rule{\linewidth}{.2mm} & \rule{\linewidth}{.2mm}\\
        & \\
        \rule{\linewidth}{.2mm} & \rule{\linewidth}{.2mm}\\
        & \\
        \rule{\linewidth}{.2mm} & \rule{\linewidth}{.2mm}\\
        & \\
\end{tabular}
%\end{table}
%\begin{table}[!htb]
    \begin{tabular}{p{0.5\linewidth} p{0.5\linewidth}}
        %---------------------------------
        11)	Qual a distância você imagina que existe entre você e a mesa da recepção? & 12)	Qual a distância você imagina que existe entre você e a saída? \\
        & \\
        \rule{\linewidth}{.2mm} & \rule{\linewidth}{.2mm}\\
        & \\
        \rule{\linewidth}{.2mm} & \rule{\linewidth}{.2mm}\\
        & \\
        \rule{\linewidth}{.2mm} & \rule{\linewidth}{.2mm}\\
        & \\
        \rule{\linewidth}{.2mm} & \rule{\linewidth}{.2mm}\\
        & \\
        %---------------------------------
    \end{tabular}
\end{table}
%
\begin{table}[!htb]
    \centering
    \begin{tabular}{m{1\linewidth}}

        {\color{gray}
        
        Código de identificação do voluntário: \rule{1in}{.2mm}
        
        \textit{Os dados deste questionário são anônimos e somente serão utilizados para fins acadêmicos e de pesquisa, ficando proibido seu manuseio ou uso sem consentimento do coordenador da pesquisa.}
        }
        
        \begin{center}
        \textbf{Parte 3 - Questionário sobre consciência situacional (SAGAT).}
        \end{center}
        
        \noindent
        \textbf{TESTE} ( ) BASE \hfill ( ) ÁUDIO \hfill ( ) CINTO HÁPTICO \hfill ( ) BENGALA \hfill ( ) MISTO
        
    \end{tabular}
%\end{table}

%\begin{table}[!htb]
    \centering
    \begin{tabular}{m{1\linewidth}}
        \textbf{Retorno}
    \end{tabular}
%\end{table}

\hspace{0.5cm}

%\begin{table}[!htb]
    \centering
    \begin{tabular}{m{0.5\linewidth} m{0.5\linewidth}}
        \large{Nível 1 – Percepção}  &\\
        & \\
        %---------------------------------
        1) Existe algum objeto próximo de você? & 2) Sinalize onde o objeto está: \\
        & \\
        \rule{\linewidth}{.2mm} & \begin{center}\multirow{5}{*}{\includegraphics[width = 0.45\linewidth]{ApendC_(Questionarios)/diagrama_sagat.png}} \end{center}\\
        \rule{\linewidth}{.2mm} & \\
        & \\
        \rule{\linewidth}{.2mm} & \\
        & \\
        \rule{\linewidth}{.2mm} & \\
        & \\
\end{tabular}
\end{table}
\begin{table}[!htb]
    \begin{tabular}{m{0.5\linewidth} m{0.5\linewidth}}
         %---------------------------------
         3)	Existe alguém perto de você? & 4) Sinalize onde a pessoa está: \\
        & \\
        \rule{\linewidth}{.2mm} & \begin{center}\multirow{5}{*}{\includegraphics[width = 0.45\linewidth]{ApendC_(Questionarios)/diagrama_sagat.png}} \end{center}\\
        \rule{\linewidth}{.2mm} & \\
        & \\
        \rule{\linewidth}{.2mm} & \\
        & \\
        \rule{\linewidth}{.2mm} & \\
        & \\
\end{tabular}
\end{table}
\begin{table}[!htb]
    \begin{tabular}{m{0.5\linewidth} m{0.5\linewidth}}
        %---------------------------------
         5)	Você percebeu alguma fonte de som característica do lugar onde você se encontra? & 6)	Sinalize de onde vem o som: \\
        & \\
        \rule{\linewidth}{.2mm} & \begin{center}\multirow{5}{*}{\includegraphics[width = 0.45\linewidth]{ApendC_(Questionarios)/diagrama_sagat.png}} \end{center}\\
        \rule{\linewidth}{.2mm} & \\
        & \\
        \rule{\linewidth}{.2mm} & \\
        & \\
        \rule{\linewidth}{.2mm} & \\
        & \\
        %---------------------------------
    \end{tabular}
\end{table}
\begin{table}[!htb]
    \begin{tabular}{m{0.5\linewidth} m{0.5\linewidth}}
        %---------------------------------
        \large{Nível 2 – Compreensão}  &\\
        & \\
        & \\
        %---------------------------------
        7)	Sinalize em que direção está a recepcionista: & 8)	Sinalize em que direção está a saída: \\
        & \\
        \begin{center}\multirow{5}{*}{\includegraphics[width = 0.45\linewidth]{ApendC_(Questionarios)/diagrama_sagat.png}} \end{center} & \begin{center}\multirow{5}{*}{\includegraphics[width = 0.45\linewidth]{ApendC_(Questionarios)/diagrama_sagat.png}} \end{center}\\
        & \\
        & \\
        & \\
        & \\
        & \\
        %---------------------------------
    \end{tabular}
\end{table}
\begin{table}[!htb]
    \begin{tabular}{m{0.5\linewidth} m{0.5\linewidth}}
        %---------------------------------
        \large{Nível 3 – Projeção}  &\\
        & \\
        & \\
        %---------------------------------
        9)	O que você espera encontrar no caminho ao retornar à mesa de recepção? & 10)	Qual será a sua trajetória para sair? Descreva seus próximos movimentos. \\
        & \\
        \rule{\linewidth}{.2mm} & \rule{\linewidth}{.2mm}\\
        & \\
        \rule{\linewidth}{.2mm} & \rule{\linewidth}{.2mm}\\
        & \\
        \rule{\linewidth}{.2mm} & \rule{\linewidth}{.2mm}\\
        & \\
        \rule{\linewidth}{.2mm} & \\
        & \\
\end{tabular}
\end{table}
\begin{table}[!htb]
    \begin{tabular}{m{0.5\linewidth} m{0.5\linewidth}}
        %---------------------------------
        11)	Qual a distância você imagina que existe entre você e a mesa da recepção? & 12)	Qual a distância você imagina que existe entre você e a saída? \\
        & \\
        \rule{\linewidth}{.2mm} & \rule{\linewidth}{.2mm}\\
        & \\
        \rule{\linewidth}{.2mm} & \rule{\linewidth}{.2mm}\\
        & \\
        \rule{\linewidth}{.2mm} & \rule{\linewidth}{.2mm}\\
        & \\
        \rule{\linewidth}{.2mm} & \\
        & \\
        %---------------------------------
    \end{tabular}
\end{table}
\vfill


%\FloatBarrier

%\begin{table}[!htb]
    \centering
    \begin{tabular}{p{0.5\linewidth} p{0.5\linewidth}}
        \large{Nível 1 – Percepção}  &\\
        & \\
        %---------------------------------
        1) Existe algum objeto próximo de você? & 2) Sinalize onde o objeto está: \\
        & \\
        \rule{\linewidth}{.2mm} & \begin{center}\multirow{5}{*}{\includegraphics[width = 0.45\linewidth]{ApendC_(Questionarios)/diagrama_sagat.png}} \end{center}\\
        \rule{\linewidth}{.2mm} & \\
        & \\
        \rule{\linewidth}{.2mm} & \\
        & \\
        \rule{\linewidth}{.2mm} & \\
        & \\
\end{tabular}
%\end{table}
%\begin{table}[!htb]
    \begin{tabular}{p{0.5\linewidth} p{0.5\linewidth}}
         %---------------------------------
         3)	Existe alguém perto de você? & 4) Sinalize onde a pessoa está: \\
        & \\
        \rule{\linewidth}{.2mm} & \begin{center}\multirow{5}{*}{\includegraphics[width = 0.45\linewidth]{ApendC_(Questionarios)/diagrama_sagat.png}} \end{center}\\
        \rule{\linewidth}{.2mm} & \\
        & \\
        \rule{\linewidth}{.2mm} & \\
        & \\
        \rule{\linewidth}{.2mm} & \\
        & \\
\end{tabular}
%\end{table}
%\begin{table}[!htb]
    \begin{tabular}{p{0.5\linewidth} p{0.5\linewidth}}
        %---------------------------------
         5)	Você percebeu alguma fonte de som característica do lugar onde você se encontra? & 6)	Sinalize de onde vem o som: \\
        & \\
        \rule{\linewidth}{.2mm} & \begin{center}\multirow{5}{*}{\includegraphics[width = 0.45\linewidth]{ApendC_(Questionarios)/diagrama_sagat.png}} \end{center}\\
        \rule{\linewidth}{.2mm} & \\
        & \\
        \rule{\linewidth}{.2mm} & \\
        & \\
        \rule{\linewidth}{.2mm} & \\
        & \\
        %---------------------------------
    \end{tabular}
\end{table}

\begin{table}[!htb]
    \centering
    \begin{tabular}{m{1\linewidth}}

        {\color{gray}
        
        Código de identificação do voluntário: \rule{1in}{.2mm}
        
        \textit{Os dados deste questionário são anônimos e somente serão utilizados para fins acadêmicos e de pesquisa, ficando proibido seu manuseio ou uso sem consentimento do coordenador da pesquisa.}
        }
        
        \begin{center}
        \textbf{Parte 3 - Questionário sobre consciência situacional (SAGAT).}
        \end{center}
        
        \noindent
        \textbf{TESTE} ( ) BASE \hfill ( ) ÁUDIO \hfill ( ) CINTO HÁPTICO \hfill ( ) BENGALA \hfill ( ) MISTO
        
    \end{tabular}
%end{table}
%\begin{table}[!htb]
    \begin{tabular}{p{0.5\linewidth} p{0.5\linewidth}}
        %---------------------------------
        \large{Nível 2 – Compreensão}  &\\
        & \\
        & \\
        %---------------------------------
        7)	Sinalize em que direção está a recepcionista: & 8)	Sinalize em que direção está a saída: \\
        & \\
        \begin{center}\multirow{5}{*}{\includegraphics[width = 0.45\linewidth]{ApendC_(Questionarios)/diagrama_sagat.png}} \end{center} & \begin{center}\multirow{5}{*}{\includegraphics[width = 0.45\linewidth]{ApendC_(Questionarios)/diagrama_sagat.png}} \end{center}\\
        & \\
        & \\
        & \\
        & \\
        & \\
        %---------------------------------
    \end{tabular}
%\end{table}
%\begin{table}[!htb]
    \begin{tabular}{p{0.5\linewidth} p{0.5\linewidth}}
        %---------------------------------
        \large{Nível 3 – Projeção}  &\\
        & \\
        & \\
        %---------------------------------
        9)	O que você espera encontrar no caminho ao retornar à mesa de recepção? & 10)	Qual será a sua trajetória para sair? Descreva seus próximos movimentos. \\
        & \\
        \rule{\linewidth}{.2mm} & \rule{\linewidth}{.2mm}\\
        & \\
        \rule{\linewidth}{.2mm} & \rule{\linewidth}{.2mm}\\
        & \\
        \rule{\linewidth}{.2mm} & \rule{\linewidth}{.2mm}\\
        & \\
        \rule{\linewidth}{.2mm} & \rule{\linewidth}{.2mm}\\
        & \\
\end{tabular}
%\end{table}
%\begin{table}[!htb]
    \begin{tabular}{p{0.5\linewidth} p{0.5\linewidth}}
        %---------------------------------
        11)	Qual a distância você imagina que existe entre você e a mesa da recepção? & 12)	Qual a distância você imagina que existe entre você e a saída? \\
        & \\
        \rule{\linewidth}{.2mm} & \rule{\linewidth}{.2mm}\\
        & \\
        \rule{\linewidth}{.2mm} & \rule{\linewidth}{.2mm}\\
        & \\
        \rule{\linewidth}{.2mm} & \rule{\linewidth}{.2mm}\\
        & \\
        \rule{\linewidth}{.2mm} & \rule{\linewidth}{.2mm}\\
        & \\
        %---------------------------------
    \end{tabular}
\end{table}

\FloatBarrier
\pagebreak


\pagebreak

\section{Guidance method evaluation}
\label{apsec:guidace_evaluation}
%{\color{gray}
%
%Código de identificação do voluntário: \rule{1in}{.2mm}
%
%\textit{Os dados deste questionário são anônimos e somente serão utilizados para fins acadêmicos e de pesquisa, ficando proibido seu manuseio ou uso sem consentimento do coordenador da pesquisa.}}
%
%\begin{center}
%\textbf{Parte 4 - Questionário sobre o método de navegação.}
%\end{center}
\begin{table}[!thb]
    \begin{tabular}{m{1\linewidth}}

        {\color{gray}
        
        Código de identificação do voluntário: \rule{1in}{.2mm}
        
        \textit{Os dados deste questionário são anônimos e somente serão utilizados para fins acadêmicos e de pesquisa, ficando proibido seu manuseio ou uso sem consentimento do coordenador da pesquisa.}
        }
        
        \begin{center}
        \textbf{Parte 4 - Questionário sobre o método de navegação}
        \end{center}
        
    \end{tabular}
%\end{table}
%
%\begin{table}[!thb]   
    \begin{tabular}{>{\centering\arraybackslash}m{1\linewidth}}
        {\large TESTE 1 - ORIENTAÇÃO VIA ÁUDIO}
    \end{tabular}
    \vfill
    \begin{tabular}{m{1\linewidth}}
        \vspace{1ex}
        1)	Os sons foram de fácil interpretação?        
    \end{tabular}
    %\begin{tabular}{|>{\centering\arraybackslash}m{0.11\linewidth}|>{\centering\arraybackslash}m{0.11\linewidth}|>{\centering\arraybackslash}m{0.11\linewidth}|>{\centering\arraybackslash}m{0.11\linewidth}|>{\centering\arraybackslash}m{0.11\linewidth}|>{\centering\arraybackslash}m{0.11\linewidth}|>{\centering\arraybackslash}m{0.11\linewidth}|}
        %\hline
        % 1 & 2 & 3 & 4 & 5 & 6 & 7 \\ \hline
        %\begin{tabular}[c]{@{}c@{}}Muito\\fácil\end{tabular} &&& 
        %Médio &&&
        %\begin{tabular}[c]{@{}c@{}}Muito\\ díficil\end{tabular} \\ \hline
    %\end{tabular}
        \begin{tabular}{|>{\centering\arraybackslash}m{0.11\linewidth}|>{\centering\arraybackslash}m{0.11\linewidth}|>{\centering\arraybackslash}m{0.11\linewidth}|>{\centering\arraybackslash}m{0.11\linewidth}|>{\centering\arraybackslash}m{0.11\linewidth}|>{\centering\arraybackslash}m{0.11\linewidth}|>{\centering\arraybackslash}m{0.11\linewidth}|}
    \hline
        \begin{table}[!h]
\centering
\begin{tabularx}{\textwidth}{|Y|Y|Y|Y|Y|Y|Y|}
\hline
 1 & 2 & 3 & 4 & 5 & 6 & 7 \\ \hline
\begin{tabular}[c]{@{}c@{}}Muito\\fácil\end{tabular} &&& 
Médio &&&
\begin{tabular}[c]{@{}c@{}}Muito\\ díficil\end{tabular} \\ \hline
\end{tabularx}
\end{table}

\FloatBarrier

    \begin{tabular}{m{1\linewidth}}
        \vspace{1ex}
        2)	Os sons causaram algum incômodo durante o uso?
    \end{tabular}

        \begin{tabular}{|>{\centering\arraybackslash}m{0.11\linewidth}|>{\centering\arraybackslash}m{0.11\linewidth}|>{\centering\arraybackslash}m{0.11\linewidth}|>{\centering\arraybackslash}m{0.11\linewidth}|>{\centering\arraybackslash}m{0.11\linewidth}|>{\centering\arraybackslash}m{0.11\linewidth}|>{\centering\arraybackslash}m{0.11\linewidth}|}
    \hline
        \begin{table}[!h]
\centering
\begin{tabularx}{\textwidth}{|Y|Y|Y|Y|Y|Y|Y|}
\hline
 1 & 2 & 3 & 4 & 5 & 6 & 7 \\ \hline
\begin{tabular}[c]{@{}c@{}}Nenhum\\incômodo\end{tabular} &&& 
Médio &&&
\begin{tabular}[c]{@{}c@{}}Muito\\ incômodo\end{tabular} \\ \hline
\end{tabularx}
\end{table}

\FloatBarrier
    
    \begin{tabular}{m{1\linewidth}}
        \vspace{1ex}
        3)	Os comandos foram claros?
    \end{tabular}

        \begin{tabular}{|>{\centering\arraybackslash}m{0.11\linewidth}|>{\centering\arraybackslash}m{0.11\linewidth}|>{\centering\arraybackslash}m{0.11\linewidth}|>{\centering\arraybackslash}m{0.11\linewidth}|>{\centering\arraybackslash}m{0.11\linewidth}|>{\centering\arraybackslash}m{0.11\linewidth}|>{\centering\arraybackslash}m{0.11\linewidth}|}
    \hline
        \input{ApendC_(Questionarios)/escalas_guidance/claro}

    \begin{tabular}{m{1\linewidth}}
        \vspace{1ex}
        4)	Como você avalia a quantidade de comandos?
    \end{tabular}

    \begin{tabular}{|>{\centering\arraybackslash}m{0.11\linewidth}|>{\centering\arraybackslash}m{0.11\linewidth}|>{\centering\arraybackslash}m{0.11\linewidth}|>{\centering\arraybackslash}m{0.11\linewidth}|>{\centering\arraybackslash}m{0.11\linewidth}|>{\centering\arraybackslash}m{0.11\linewidth}|>{\centering\arraybackslash}m{0.11\linewidth}|}
    \hline
    \input{ApendC_(Questionarios)/escalas_guidance/pouca_excessiva}
    
    \begin{tabular}{m{1\linewidth}}
        \vspace{1ex}
        5)	Os momentos em que os comandos foram reproduzidos foram inadequados?
    \end{tabular}

        \begin{tabular}{|>{\centering\arraybackslash}m{0.11\linewidth}|>{\centering\arraybackslash}m{0.11\linewidth}|>{\centering\arraybackslash}m{0.11\linewidth}|>{\centering\arraybackslash}m{0.11\linewidth}|>{\centering\arraybackslash}m{0.11\linewidth}|>{\centering\arraybackslash}m{0.11\linewidth}|>{\centering\arraybackslash}m{0.11\linewidth}|}
    \hline
        \input{ApendC_(Questionarios)/escalas_guidance/adequados}
    
    \begin{tabular}{m{1\linewidth}}
        \vspace{1ex}
        6)	O som ambiente atrapalhou a reprodução de algum comando
    \end{tabular}

        \begin{tabular}{|>{\centering\arraybackslash}m{0.11\linewidth}|>{\centering\arraybackslash}m{0.11\linewidth}|>{\centering\arraybackslash}m{0.11\linewidth}|>{\centering\arraybackslash}m{0.11\linewidth}|>{\centering\arraybackslash}m{0.11\linewidth}|>{\centering\arraybackslash}m{0.11\linewidth}|>{\centering\arraybackslash}m{0.11\linewidth}|}
    \hline
        \begin{table}[!h]
\centering
\begin{tabularx}{\textwidth}{|Y|Y|Y|Y|Y|Y|Y|}
\hline
 1 & 2 & 3 & 4 & 5 & 6 & 7 \\ \hline
\begin{tabular}[c]{@{}c@{}}Não\\atrapalhou\end{tabular} &&& 
Médio &&&
\begin{tabular}[c]{@{}c@{}}Atrapalhou\\muito\end{tabular} \\ \hline
\end{tabularx}
\end{table}

\FloatBarrier
    
    \begin{tabular}{m{1\linewidth}}
        \vspace{1ex}
        7)	Você sentiu falta de alguma informação durante a orientação via áudio? \\

        \noindent
        \rule{6in}{.2mm} \\
        \rule{6in}{.2mm} \\
        \rule{6in}{.2mm}

    \end{tabular}
\end{table}

\FloatBarrier
%\vspace{2cm}

\begin{table}[!thb]
    \begin{tabular}{m{1\linewidth}}

        {\color{gray}
        
        Código de identificação do voluntário: \rule{1in}{.2mm}
        
        \textit{Os dados deste questionário são anônimos e somente serão utilizados para fins acadêmicos e de pesquisa, ficando proibido seu manuseio ou uso sem consentimento do coordenador da pesquisa.}
        }
        
        \begin{center}
        \textbf{Parte 4 - Questionário sobre o método de navegação}
        \end{center}
        
    \end{tabular}
%\end{table}
%
%\begin{table}[!thb]   
    \begin{tabular}{>{\centering\arraybackslash}m{1\linewidth}}
        {\large TESTE 2 – CINTO HÁPTICO}
    \end{tabular}

    \begin{tabular}{m{1\linewidth}}
        \vspace{2ex}
        8)	A informação da vibração do cinto foi precisa?
    \end{tabular}

        \begin{tabular}{|>{\centering\arraybackslash}m{0.11\linewidth}|>{\centering\arraybackslash}m{0.11\linewidth}|>{\centering\arraybackslash}m{0.11\linewidth}|>{\centering\arraybackslash}m{0.11\linewidth}|>{\centering\arraybackslash}m{0.11\linewidth}|>{\centering\arraybackslash}m{0.11\linewidth}|>{\centering\arraybackslash}m{0.11\linewidth}|}
    \hline
        \input{ApendC_(Questionarios)/escalas_guidance/precisa}

    \begin{tabular}{m{1\linewidth}}
        \vspace{2ex}
        9)	A vibração do cinto causou algum incômodo durante o uso?
    \end{tabular}

        \begin{tabular}{|>{\centering\arraybackslash}m{0.11\linewidth}|>{\centering\arraybackslash}m{0.11\linewidth}|>{\centering\arraybackslash}m{0.11\linewidth}|>{\centering\arraybackslash}m{0.11\linewidth}|>{\centering\arraybackslash}m{0.11\linewidth}|>{\centering\arraybackslash}m{0.11\linewidth}|>{\centering\arraybackslash}m{0.11\linewidth}|}
    \hline
        \begin{table}[!h]
\centering
\begin{tabularx}{\textwidth}{|Y|Y|Y|Y|Y|Y|Y|}
\hline
 1 & 2 & 3 & 4 & 5 & 6 & 7 \\ \hline
\begin{tabular}[c]{@{}c@{}}Nenhum\\incômodo\end{tabular} &&& 
Médio &&&
\begin{tabular}[c]{@{}c@{}}Muito\\ incômodo\end{tabular} \\ \hline
\end{tabularx}
\end{table}

\FloatBarrier

    \begin{tabular}{m{1\linewidth}}
        \vspace{2ex}
        10)	A vibração do cinto causou alguma confusão durante a navegação?
    \end{tabular}

        \begin{tabular}{|>{\centering\arraybackslash}m{0.11\linewidth}|>{\centering\arraybackslash}m{0.11\linewidth}|>{\centering\arraybackslash}m{0.11\linewidth}|>{\centering\arraybackslash}m{0.11\linewidth}|>{\centering\arraybackslash}m{0.11\linewidth}|>{\centering\arraybackslash}m{0.11\linewidth}|>{\centering\arraybackslash}m{0.11\linewidth}|}
    \hline
        \begin{table}[!h]
\centering
\begin{tabularx}{\textwidth}{|Y|Y|Y|Y|Y|Y|Y|}
\hline
 1 & 2 & 3 & 4 & 5 & 6 & 7 \\ \hline
\begin{tabular}[c]{@{}c@{}}Nenhuma\\confusão\end{tabular} &&& 
Médio &&&
\begin{tabular}[c]{@{}c@{}}Muita\\confusão\end{tabular} \\ \hline
\end{tabularx}
\end{table}

\FloatBarrier

    \begin{tabular}{m{1\linewidth}}
        \vspace{2ex}
        11)	A vibração do cinto trouxe alguma segurança na navegação?
    \end{tabular}

        \begin{tabular}{|>{\centering\arraybackslash}m{0.11\linewidth}|>{\centering\arraybackslash}m{0.11\linewidth}|>{\centering\arraybackslash}m{0.11\linewidth}|>{\centering\arraybackslash}m{0.11\linewidth}|>{\centering\arraybackslash}m{0.11\linewidth}|>{\centering\arraybackslash}m{0.11\linewidth}|>{\centering\arraybackslash}m{0.11\linewidth}|}
    \hline
        \input{ApendC_(Questionarios)/escalas_guidance/segurança}

    \begin{tabular}{m{1\linewidth}}
        \vspace{2ex}
        12)	O cinto causou algum incômodo durante o uso?
    \end{tabular}

        \begin{tabular}{|>{\centering\arraybackslash}m{0.11\linewidth}|>{\centering\arraybackslash}m{0.11\linewidth}|>{\centering\arraybackslash}m{0.11\linewidth}|>{\centering\arraybackslash}m{0.11\linewidth}|>{\centering\arraybackslash}m{0.11\linewidth}|>{\centering\arraybackslash}m{0.11\linewidth}|>{\centering\arraybackslash}m{0.11\linewidth}|}
    \hline
        \begin{table}[!h]
\centering
\begin{tabularx}{\textwidth}{|Y|Y|Y|Y|Y|Y|Y|}
\hline
 1 & 2 & 3 & 4 & 5 & 6 & 7 \\ \hline
\begin{tabular}[c]{@{}c@{}}Nenhum\\incômodo\end{tabular} &&& 
Médio &&&
\begin{tabular}[c]{@{}c@{}}Muito\\ incômodo\end{tabular} \\ \hline
\end{tabularx}
\end{table}

\FloatBarrier

    \begin{tabular}{m{1\linewidth}}
        \vspace{2ex}
        13)	Você sentiu falta de alguma informação durante a orientação usando o cinto?

        \noindent
        \rule{6in}{.2mm} \\
        \rule{6in}{.2mm} \\
        \rule{6in}{.2mm}

    \end{tabular}
\end{table}

\FloatBarrier
%\vspace{2cm}

\begin{table}[!thb]
    \begin{tabular}{m{1\linewidth}}

        {\color{gray}
        
        Código de identificação do voluntário: \rule{1in}{.2mm}
        
        \textit{Os dados deste questionário são anônimos e somente serão utilizados para fins acadêmicos e de pesquisa, ficando proibido seu manuseio ou uso sem consentimento do coordenador da pesquisa.}
        }
        
        \begin{center}
        \textbf{Parte 4 - Questionário sobre o método de navegação}
        \end{center}
        
    \end{tabular}
%\end{table}
%
%\begin{table}[!thb]   
    \begin{tabular}{>{\centering\arraybackslash}m{1\linewidth}}
        {\large TESTE 3 - BENGALA VIRTUAL}
    \end{tabular}


    \begin{tabular}{m{1\linewidth}}
        \vspace{2ex}
        14)	A informação da vibração do cinto foi precisa?
    \end{tabular}

    \begin{tabular}{|>{\centering\arraybackslash}m{0.11\linewidth}|>{\centering\arraybackslash}m{0.11\linewidth}|>{\centering\arraybackslash}m{0.11\linewidth}|>{\centering\arraybackslash}m{0.11\linewidth}|>{\centering\arraybackslash}m{0.11\linewidth}|>{\centering\arraybackslash}m{0.11\linewidth}|>{\centering\arraybackslash}m{0.11\linewidth}|}
    \hline
    \input{ApendC_(Questionarios)/escalas_guidance/precisa}

    \begin{tabular}{m{1\linewidth}}
        \vspace{2ex}
        15)	A bengala virtual funcionou semelhantemente à bengala tradicional?
    \end{tabular}

    \begin{tabular}{|>{\centering\arraybackslash}m{0.11\linewidth}|>{\centering\arraybackslash}m{0.11\linewidth}|>{\centering\arraybackslash}m{0.11\linewidth}|>{\centering\arraybackslash}m{0.11\linewidth}|>{\centering\arraybackslash}m{0.11\linewidth}|>{\centering\arraybackslash}m{0.11\linewidth}|>{\centering\arraybackslash}m{0.11\linewidth}|}
    \hline
    %\begin{table}[!h]
%\centering
%\begin{tabularx}{\textwidth}{|Y|Y|Y|Y|Y|Y|Y|}
%\hline
 1 & 2 & 3 & 4 & 5 & 6 & 7 \\ \hline
\begin{tabular}[c]{@{}c@{}}Pouco\\semelhante\end{tabular} &&& 
Médio &&&
\begin{tabular}[c]{@{}c@{}}Muito\\semelhante\end{tabular} \\ \hline
\end{tabular}
%\end{tabularx}
%\end{table}
%
%\FloatBarrier

    \begin{tabular}{m{1\linewidth}}
        \vspace{2ex}
        16)	O uso da bengala foi intuitivo?
    \end{tabular}

    \begin{tabular}{|>{\centering\arraybackslash}m{0.11\linewidth}|>{\centering\arraybackslash}m{0.11\linewidth}|>{\centering\arraybackslash}m{0.11\linewidth}|>{\centering\arraybackslash}m{0.11\linewidth}|>{\centering\arraybackslash}m{0.11\linewidth}|>{\centering\arraybackslash}m{0.11\linewidth}|>{\centering\arraybackslash}m{0.11\linewidth}|}
    \hline
    \input{ApendC_(Questionarios)/escalas_guidance/intuitivo}

    \begin{tabular}{m{1\linewidth}}
        \vspace{2ex}
        17)	A bengala causou algum incômodo durante o uso?
    \end{tabular}

    \begin{tabular}{|>{\centering\arraybackslash}m{0.11\linewidth}|>{\centering\arraybackslash}m{0.11\linewidth}|>{\centering\arraybackslash}m{0.11\linewidth}|>{\centering\arraybackslash}m{0.11\linewidth}|>{\centering\arraybackslash}m{0.11\linewidth}|>{\centering\arraybackslash}m{0.11\linewidth}|>{\centering\arraybackslash}m{0.11\linewidth}|}
    \hline
    \begin{table}[!h]
\centering
\begin{tabularx}{\textwidth}{|Y|Y|Y|Y|Y|Y|Y|}
\hline
 1 & 2 & 3 & 4 & 5 & 6 & 7 \\ \hline
\begin{tabular}[c]{@{}c@{}}Nenhum\\incômodo\end{tabular} &&& 
Médio &&&
\begin{tabular}[c]{@{}c@{}}Muito\\ incômodo\end{tabular} \\ \hline
\end{tabularx}
\end{table}

\FloatBarrier

    \begin{tabular}{m{1\linewidth}}
        \vspace{2ex}
        18)	A bengala causou alguma confusão durante a navegação?
    \end{tabular}

    \begin{tabular}{|>{\centering\arraybackslash}m{0.11\linewidth}|>{\centering\arraybackslash}m{0.11\linewidth}|>{\centering\arraybackslash}m{0.11\linewidth}|>{\centering\arraybackslash}m{0.11\linewidth}|>{\centering\arraybackslash}m{0.11\linewidth}|>{\centering\arraybackslash}m{0.11\linewidth}|>{\centering\arraybackslash}m{0.11\linewidth}|}
    \hline
    \begin{table}[!h]
\centering
\begin{tabularx}{\textwidth}{|Y|Y|Y|Y|Y|Y|Y|}
\hline
 1 & 2 & 3 & 4 & 5 & 6 & 7 \\ \hline
\begin{tabular}[c]{@{}c@{}}Nenhuma\\confusão\end{tabular} &&& 
Médio &&&
\begin{tabular}[c]{@{}c@{}}Muita\\confusão\end{tabular} \\ \hline
\end{tabularx}
\end{table}

\FloatBarrier

    \begin{tabular}{m{1\linewidth}}
        \vspace{2ex}
        19)	Você sentiu falta de alguma informação durante a orientação? \\


        \noindent
        \rule{6in}{.2mm} \\
        \rule{6in}{.2mm} \\
        \rule{6in}{.2mm}

    \end{tabular}
    \begin{tabular}{m{1\linewidth}}
        \vspace{2ex}
        20)	Você prefere mais a bengala virtual ou o cinto? Por quê? \\


        \noindent
        \rule{6in}{.2mm} \\
        \rule{6in}{.2mm} \\
        \rule{6in}{.2mm}


    \end{tabular}
\end{table}

%\vspace{2cm}
\FloatBarrier

\begin{table}[!thb]
    \begin{tabular}{m{1\linewidth}}

        {\color{gray}
        
        Código de identificação do voluntário: \rule{1in}{.2mm}
        
        \textit{Os dados deste questionário são anônimos e somente serão utilizados para fins acadêmicos e de pesquisa, ficando proibido seu manuseio ou uso sem consentimento do coordenador da pesquisa.}
        }
        
        \begin{center}
        \textbf{Parte 4 - Questionário sobre o método de navegação}
        \end{center}
        
    \end{tabular}
%\end{table}
%
%\begin{table}[!thb]   
    \begin{tabular}{>{\centering\arraybackslash}m{1\linewidth}}
        {\large TESTE 4 - MISTURADO}
    \end{tabular}


    \begin{tabular}{m{1\linewidth}}
        \vspace{1ex}
        21)	Os sons foram de fácil interpretação?
    \end{tabular}

    \begin{tabular}{|>{\centering\arraybackslash}m{0.11\linewidth}|>{\centering\arraybackslash}m{0.11\linewidth}|>{\centering\arraybackslash}m{0.11\linewidth}|>{\centering\arraybackslash}m{0.11\linewidth}|>{\centering\arraybackslash}m{0.11\linewidth}|>{\centering\arraybackslash}m{0.11\linewidth}|>{\centering\arraybackslash}m{0.11\linewidth}|}
    \hline
    \begin{table}[!h]
\centering
\begin{tabularx}{\textwidth}{|Y|Y|Y|Y|Y|Y|Y|}
\hline
 1 & 2 & 3 & 4 & 5 & 6 & 7 \\ \hline
\begin{tabular}[c]{@{}c@{}}Muito\\fácil\end{tabular} &&& 
Médio &&&
\begin{tabular}[c]{@{}c@{}}Muito\\ díficil\end{tabular} \\ \hline
\end{tabularx}
\end{table}

\FloatBarrier

    \begin{tabular}{m{1\linewidth}}
        \vspace{1ex}
        22)	Os sons causaram algum incômodo durante o uso?
    \end{tabular}

    \begin{tabular}{|>{\centering\arraybackslash}m{0.11\linewidth}|>{\centering\arraybackslash}m{0.11\linewidth}|>{\centering\arraybackslash}m{0.11\linewidth}|>{\centering\arraybackslash}m{0.11\linewidth}|>{\centering\arraybackslash}m{0.11\linewidth}|>{\centering\arraybackslash}m{0.11\linewidth}|>{\centering\arraybackslash}m{0.11\linewidth}|}
    \hline
    \begin{table}[!h]
\centering
\begin{tabularx}{\textwidth}{|Y|Y|Y|Y|Y|Y|Y|}
\hline
 1 & 2 & 3 & 4 & 5 & 6 & 7 \\ \hline
\begin{tabular}[c]{@{}c@{}}Nenhum\\incômodo\end{tabular} &&& 
Médio &&&
\begin{tabular}[c]{@{}c@{}}Muito\\ incômodo\end{tabular} \\ \hline
\end{tabularx}
\end{table}

\FloatBarrier

    \begin{tabular}{m{1\linewidth}}
        \vspace{1ex}
        23)	Os comandos foram claros?
    \end{tabular}

    \begin{tabular}{|>{\centering\arraybackslash}m{0.11\linewidth}|>{\centering\arraybackslash}m{0.11\linewidth}|>{\centering\arraybackslash}m{0.11\linewidth}|>{\centering\arraybackslash}m{0.11\linewidth}|>{\centering\arraybackslash}m{0.11\linewidth}|>{\centering\arraybackslash}m{0.11\linewidth}|>{\centering\arraybackslash}m{0.11\linewidth}|}
    \hline
    \input{ApendC_(Questionarios)/escalas_guidance/claro}

    \begin{tabular}{m{1\linewidth}}
        \vspace{1ex}
        24)	Como você avalia a quantidade de comandos?
    \end{tabular}

    \begin{tabular}{|>{\centering\arraybackslash}m{0.11\linewidth}|>{\centering\arraybackslash}m{0.11\linewidth}|>{\centering\arraybackslash}m{0.11\linewidth}|>{\centering\arraybackslash}m{0.11\linewidth}|>{\centering\arraybackslash}m{0.11\linewidth}|>{\centering\arraybackslash}m{0.11\linewidth}|>{\centering\arraybackslash}m{0.11\linewidth}|}
    \hline
    \input{ApendC_(Questionarios)/escalas_guidance/pouca_excessiva}

    \begin{tabular}{m{1\linewidth}}
        \vspace{1ex}
        25)	Os momentos em que os comandos foram reproduzidos foram inadequados?
    \end{tabular}

    \begin{tabular}{|>{\centering\arraybackslash}m{0.11\linewidth}|>{\centering\arraybackslash}m{0.11\linewidth}|>{\centering\arraybackslash}m{0.11\linewidth}|>{\centering\arraybackslash}m{0.11\linewidth}|>{\centering\arraybackslash}m{0.11\linewidth}|>{\centering\arraybackslash}m{0.11\linewidth}|>{\centering\arraybackslash}m{0.11\linewidth}|}
    \hline
    \input{ApendC_(Questionarios)/escalas_guidance/adequados}

    \begin{tabular}{m{1\linewidth}}
        \vspace{1ex}
        26)	O som ambiente atrapalhou a reprodução de algum comando
    \end{tabular}

    \begin{tabular}{|>{\centering\arraybackslash}m{0.11\linewidth}|>{\centering\arraybackslash}m{0.11\linewidth}|>{\centering\arraybackslash}m{0.11\linewidth}|>{\centering\arraybackslash}m{0.11\linewidth}|>{\centering\arraybackslash}m{0.11\linewidth}|>{\centering\arraybackslash}m{0.11\linewidth}|>{\centering\arraybackslash}m{0.11\linewidth}|}
    \hline
    \begin{table}[!h]
\centering
\begin{tabularx}{\textwidth}{|Y|Y|Y|Y|Y|Y|Y|}
\hline
 1 & 2 & 3 & 4 & 5 & 6 & 7 \\ \hline
\begin{tabular}[c]{@{}c@{}}Não\\atrapalhou\end{tabular} &&& 
Médio &&&
\begin{tabular}[c]{@{}c@{}}Atrapalhou\\muito\end{tabular} \\ \hline
\end{tabularx}
\end{table}

\FloatBarrier

    \begin{tabular}{m{1\linewidth}}
        \vspace{1ex}
        27)	A informação da vibração do cinto foi precisa?
    \end{tabular}

    \begin{tabular}{|>{\centering\arraybackslash}m{0.11\linewidth}|>{\centering\arraybackslash}m{0.11\linewidth}|>{\centering\arraybackslash}m{0.11\linewidth}|>{\centering\arraybackslash}m{0.11\linewidth}|>{\centering\arraybackslash}m{0.11\linewidth}|>{\centering\arraybackslash}m{0.11\linewidth}|>{\centering\arraybackslash}m{0.11\linewidth}|}
    \hline
    \input{ApendC_(Questionarios)/escalas_guidance/precisa}

    \begin{tabular}{m{1\linewidth}}
        \vspace{1ex}
        28)	A vibração do cinto causou algum incômodo durante o uso?
    \end{tabular}

    \begin{tabular}{|>{\centering\arraybackslash}m{0.11\linewidth}|>{\centering\arraybackslash}m{0.11\linewidth}|>{\centering\arraybackslash}m{0.11\linewidth}|>{\centering\arraybackslash}m{0.11\linewidth}|>{\centering\arraybackslash}m{0.11\linewidth}|>{\centering\arraybackslash}m{0.11\linewidth}|>{\centering\arraybackslash}m{0.11\linewidth}|}
    \hline
    \begin{table}[!h]
\centering
\begin{tabularx}{\textwidth}{|Y|Y|Y|Y|Y|Y|Y|}
\hline
 1 & 2 & 3 & 4 & 5 & 6 & 7 \\ \hline
\begin{tabular}[c]{@{}c@{}}Nenhum\\incômodo\end{tabular} &&& 
Médio &&&
\begin{tabular}[c]{@{}c@{}}Muito\\ incômodo\end{tabular} \\ \hline
\end{tabularx}
\end{table}

\FloatBarrier

\end{table}

\FloatBarrier

\begin{table}[!thb]
    \begin{tabular}{m{1\linewidth}}

        {\color{gray}
        
        Código de identificação do voluntário: \rule{1in}{.2mm}
        
        \textit{Os dados deste questionário são anônimos e somente serão utilizados para fins acadêmicos e de pesquisa, ficando proibido seu manuseio ou uso sem consentimento do coordenador da pesquisa.}
        }
        
        \begin{center}
        \textbf{Parte 4 - Questionário sobre o método de navegação}
        \end{center}
        
    \end{tabular}
%\end{table}
%
%\begin{table}[!thb]

    \begin{tabular}{m{1\linewidth}}
        \vspace{1ex}
        29)	A vibração do cinto causou alguma confusão durante a navegação?
    \end{tabular}

    \begin{tabular}{|>{\centering\arraybackslash}m{0.11\linewidth}|>{\centering\arraybackslash}m{0.11\linewidth}|>{\centering\arraybackslash}m{0.11\linewidth}|>{\centering\arraybackslash}m{0.11\linewidth}|>{\centering\arraybackslash}m{0.11\linewidth}|>{\centering\arraybackslash}m{0.11\linewidth}|>{\centering\arraybackslash}m{0.11\linewidth}|}
    \hline
    \begin{table}[!h]
\centering
\begin{tabularx}{\textwidth}{|Y|Y|Y|Y|Y|Y|Y|}
\hline
 1 & 2 & 3 & 4 & 5 & 6 & 7 \\ \hline
\begin{tabular}[c]{@{}c@{}}Nenhuma\\confusão\end{tabular} &&& 
Médio &&&
\begin{tabular}[c]{@{}c@{}}Muita\\confusão\end{tabular} \\ \hline
\end{tabularx}
\end{table}

\FloatBarrier

    \begin{tabular}{m{1\linewidth}}
        \vspace{1ex}
        30)	A vibração do cinto trouxe alguma segurança na navegação
    \end{tabular}

    \begin{tabular}{|>{\centering\arraybackslash}m{0.11\linewidth}|>{\centering\arraybackslash}m{0.11\linewidth}|>{\centering\arraybackslash}m{0.11\linewidth}|>{\centering\arraybackslash}m{0.11\linewidth}|>{\centering\arraybackslash}m{0.11\linewidth}|>{\centering\arraybackslash}m{0.11\linewidth}|>{\centering\arraybackslash}m{0.11\linewidth}|}
    \hline
    \input{ApendC_(Questionarios)/escalas_guidance/segurança}

    \begin{tabular}{m{1\linewidth}}
        \vspace{1ex}
        31)	O cinto causou algum incômodo durante o uso?
    \end{tabular}

    \begin{tabular}{|>{\centering\arraybackslash}m{0.11\linewidth}|>{\centering\arraybackslash}m{0.11\linewidth}|>{\centering\arraybackslash}m{0.11\linewidth}|>{\centering\arraybackslash}m{0.11\linewidth}|>{\centering\arraybackslash}m{0.11\linewidth}|>{\centering\arraybackslash}m{0.11\linewidth}|>{\centering\arraybackslash}m{0.11\linewidth}|}
    \hline
    \begin{table}[!h]
\centering
\begin{tabularx}{\textwidth}{|Y|Y|Y|Y|Y|Y|Y|}
\hline
 1 & 2 & 3 & 4 & 5 & 6 & 7 \\ \hline
\begin{tabular}[c]{@{}c@{}}Nenhum\\incômodo\end{tabular} &&& 
Médio &&&
\begin{tabular}[c]{@{}c@{}}Muito\\ incômodo\end{tabular} \\ \hline
\end{tabularx}
\end{table}

\FloatBarrier

    \begin{tabular}{m{1\linewidth}}
        \vspace{1ex}
        32)	A mistura de comandos de áudio com vibração te trouxe mais segurança?
    \end{tabular}

    \begin{tabular}{|>{\centering\arraybackslash}m{0.11\linewidth}|>{\centering\arraybackslash}m{0.11\linewidth}|>{\centering\arraybackslash}m{0.11\linewidth}|>{\centering\arraybackslash}m{0.11\linewidth}|>{\centering\arraybackslash}m{0.11\linewidth}|>{\centering\arraybackslash}m{0.11\linewidth}|>{\centering\arraybackslash}m{0.11\linewidth}|}
    \hline
    \input{ApendC_(Questionarios)/escalas_guidance/segurança}

    \begin{tabular}{m{1\linewidth}}
        \vspace{1ex}
        33)	De forma geral, como você avalia a quantidade de informação recebida?
    \end{tabular}

    \begin{tabular}{|>{\centering\arraybackslash}m{0.11\linewidth}|>{\centering\arraybackslash}m{0.11\linewidth}|>{\centering\arraybackslash}m{0.11\linewidth}|>{\centering\arraybackslash}m{0.11\linewidth}|>{\centering\arraybackslash}m{0.11\linewidth}|>{\centering\arraybackslash}m{0.11\linewidth}|>{\centering\arraybackslash}m{0.11\linewidth}|}
    \hline
    \input{ApendC_(Questionarios)/escalas_guidance/pouca_excessiva}

    \begin{tabular}{m{1\linewidth}}
        \vspace{1ex}
        34)	A informação da vibração da bengala foi precisa?
    \end{tabular}

    \begin{tabular}{|>{\centering\arraybackslash}m{0.11\linewidth}|>{\centering\arraybackslash}m{0.11\linewidth}|>{\centering\arraybackslash}m{0.11\linewidth}|>{\centering\arraybackslash}m{0.11\linewidth}|>{\centering\arraybackslash}m{0.11\linewidth}|>{\centering\arraybackslash}m{0.11\linewidth}|>{\centering\arraybackslash}m{0.11\linewidth}|}
    \hline
    \input{ApendC_(Questionarios)/escalas_guidance/precisa.tex}

    \begin{tabular}{m{1\linewidth}}
        \vspace{1ex}
        35)	A bengala virtual funcionou semelhantemente à bengala tradicional?
    \end{tabular}

    \begin{tabular}{|>{\centering\arraybackslash}m{0.11\linewidth}|>{\centering\arraybackslash}m{0.11\linewidth}|>{\centering\arraybackslash}m{0.11\linewidth}|>{\centering\arraybackslash}m{0.11\linewidth}|>{\centering\arraybackslash}m{0.11\linewidth}|>{\centering\arraybackslash}m{0.11\linewidth}|>{\centering\arraybackslash}m{0.11\linewidth}|}
    \hline
    %\begin{table}[!h]
%\centering
%\begin{tabularx}{\textwidth}{|Y|Y|Y|Y|Y|Y|Y|}
%\hline
 1 & 2 & 3 & 4 & 5 & 6 & 7 \\ \hline
\begin{tabular}[c]{@{}c@{}}Pouco\\semelhante\end{tabular} &&& 
Médio &&&
\begin{tabular}[c]{@{}c@{}}Muito\\semelhante\end{tabular} \\ \hline
\end{tabular}
%\end{tabularx}
%\end{table}
%
%\FloatBarrier

    \begin{tabular}{m{1\linewidth}}
        \vspace{1ex}
        36)	O uso da bengala foi intuitivo?
    \end{tabular}

    \begin{tabular}{|>{\centering\arraybackslash}m{0.11\linewidth}|>{\centering\arraybackslash}m{0.11\linewidth}|>{\centering\arraybackslash}m{0.11\linewidth}|>{\centering\arraybackslash}m{0.11\linewidth}|>{\centering\arraybackslash}m{0.11\linewidth}|>{\centering\arraybackslash}m{0.11\linewidth}|>{\centering\arraybackslash}m{0.11\linewidth}|}
    \hline
    \input{ApendC_(Questionarios)/escalas_guidance/intuitivo.tex}

\end{table}

\FloatBarrier
\pagebreak

\begin{table}[!thb]
    \begin{tabular}{m{1\linewidth}}

        {\color{gray}
        
        Código de identificação do voluntário: \rule{1in}{.2mm}
        
        \textit{Os dados deste questionário são anônimos e somente serão utilizados para fins acadêmicos e de pesquisa, ficando proibido seu manuseio ou uso sem consentimento do coordenador da pesquisa.}
        }
        
        \begin{center}
        \textbf{Parte 4 - Questionário sobre o método de navegação}
        \end{center}
        
    \end{tabular}
%\end{table}
%
%\begin{table}[!thb]

    \begin{tabular}{m{1\linewidth}}
        \vspace{1ex}
        37)	A bengala causou algum incômodo durante o uso?
    \end{tabular}

    \begin{tabular}{|>{\centering\arraybackslash}m{0.11\linewidth}|>{\centering\arraybackslash}m{0.11\linewidth}|>{\centering\arraybackslash}m{0.11\linewidth}|>{\centering\arraybackslash}m{0.11\linewidth}|>{\centering\arraybackslash}m{0.11\linewidth}|>{\centering\arraybackslash}m{0.11\linewidth}|>{\centering\arraybackslash}m{0.11\linewidth}|}
    \hline
    \begin{table}[!h]
\centering
\begin{tabularx}{\textwidth}{|Y|Y|Y|Y|Y|Y|Y|}
\hline
 1 & 2 & 3 & 4 & 5 & 6 & 7 \\ \hline
\begin{tabular}[c]{@{}c@{}}Nenhum\\incômodo\end{tabular} &&& 
Médio &&&
\begin{tabular}[c]{@{}c@{}}Muito\\ incômodo\end{tabular} \\ \hline
\end{tabularx}
\end{table}

\FloatBarrier

    \begin{tabular}{m{1\linewidth}}
        \vspace{1ex}
        38)	A bengala causou alguma confusão durante a navegação?
    \end{tabular}

    \begin{tabular}{|>{\centering\arraybackslash}m{0.11\linewidth}|>{\centering\arraybackslash}m{0.11\linewidth}|>{\centering\arraybackslash}m{0.11\linewidth}|>{\centering\arraybackslash}m{0.11\linewidth}|>{\centering\arraybackslash}m{0.11\linewidth}|>{\centering\arraybackslash}m{0.11\linewidth}|>{\centering\arraybackslash}m{0.11\linewidth}|}
    \hline
    \begin{table}[!h]
\centering
\begin{tabularx}{\textwidth}{|Y|Y|Y|Y|Y|Y|Y|}
\hline
 1 & 2 & 3 & 4 & 5 & 6 & 7 \\ \hline
\begin{tabular}[c]{@{}c@{}}Nenhuma\\confusão\end{tabular} &&& 
Médio &&&
\begin{tabular}[c]{@{}c@{}}Muita\\confusão\end{tabular} \\ \hline
\end{tabularx}
\end{table}

\FloatBarrier

    \begin{tabular}{m{1\linewidth}}
        \vspace{1ex}
        39)	Você sentiu falta de alguma informação durante a orientação?

        \noindent
        \rule{6in}{.2mm} \\
        \rule{6in}{.2mm} \\
        \rule{6in}{.2mm}

    \end{tabular}

    \begin{tabular}{m{1\linewidth}}
        40)	Considerando a emissão de comandos via áudio e via cinto háptico, você considera que algum deles é desnecessário?

        \noindent
        \rule{6in}{.2mm} \\
        \rule{6in}{.2mm} \\
        \rule{6in}{.2mm}

    \end{tabular}
\end{table}



\chapter{Virtual Cane}
\label{ap:virtual_apend}
%APÊNDICE

%É um conteúdo que você elaborou (você ainda tem o seu apêndice intestinal?)

\section{Virtual Cane algorithm}
%\pagebreak
\input{Apend Bengala/virtual_cane}
    


\chapter{Haptic Belt}
\label{ap:haptic_apend}
%APÊNDICE

%É um conteúdo que você elaborou (você ainda tem o seu apêndice intestinal?)

\section{Printed circuit board used on the haptic belt}


    
\pagebreak

\section{Vibration unit}

    \begin{figure}[htbp]
        \centering
        \hspace{1cm}
        \includegraphics[width=.45\textwidth]{Apend Cinto/Unidade Vibracao.pdf}
        \caption{Vibration Unit}
        \label{vibration_unit}
    \end{figure}
    
\pagebreak
    
\section{Algorithm running in the ESP32}
\tikzstyle{start} = [rectangle, rounded corners, minimum width=4cm, minimum height=1.0cm,text centered, draw=black, fill=white!30, text width=3cm]
\tikzstyle{process} = [rectangle, minimum width=4cm, minimum height=1.0cm, text centered, draw=black, fill=white!30, text width=3cm]
\tikzstyle{decision} = [diamond, minimum width=4cm, minimum height=1.0cm,  text centered, text width=1.5cm, draw=black, fill=white!30, text width=3cm]
\tikzstyle{arrow_flow} = [ccmDBlue, rounded corners, line width = 2mm, ->]
\tikzstyle{arrow_return} = [ccmRed, rounded corners, line width = 2mm, ->]

\begin{tikzpicture}[node distance=2cm]
    \centering
    \node (start) [start] {Connected to Unity};
    \node (read) [process, below of=start,yshift=-1cm] {Read Unity's command};
    \node (dec1) [decision, aspect=2.5, below of=read,yshift=-1cm] {Activate or deactivate?};
    
    \node (id_activate) [process, right of=dec1, xshift=2cm, yshift=-3cm] {Identify Vibrator(s)};
    \node (intensity) [process, below of=id_activate, yshift=-1cm] {Set intensity};
    
    \node (id_deactivate) [process, left of=dec1, xshift=-2cm, yshift=-3cm] {Identify Vibrator(s)};
    \node (activate) [process, below of=intensity, yshift=-1cm] {Activate Vibrator(s)};
    \node (deactivate) [process, below of=id_deactivate, yshift=-1cm] {Deactivate Vibrator(s)};
    
    \draw [arrow_flow,line width=2mm] (start) -- (read);
    \draw [arrow_flow] (read) -- (dec1);
    
    \draw [arrow_flow] (dec1) -- node[anchor=south] {activate} +(4,0) -- (id_activate);
    \draw [arrow_flow] (id_activate) -- (intensity);
    \draw [arrow_flow] (intensity) -- (activate);
    \draw [arrow_return] (activate) -- ++(0,-2) -- ++(3,0) -- (7,-3) -- (read);
    
    \draw [arrow_flow] (dec1) -- node[anchor=south] {deactivate} +(-4,0) -- (id_deactivate);
    \draw [arrow_flow] (id_deactivate) -- (deactivate);
    \draw [arrow_return] (deactivate) -- ++(0,-2) -- ++(-3,0) -- (-7,-3) --(read);
\end{tikzpicture}

\begin{figure}[!h]
    \centering
    \caption{Unity's message flowchart inside ESP32}
    \label{fig:esp32_algorithim}
\end{figure}

\pagebreak

\section{Algorithm running in the Unity}
\tikzstyle{start} = [rectangle, rounded corners, minimum width=4cm, minimum height=1.0cm,text centered, draw=black, fill=white!30, text width=3cm]
\tikzstyle{process} = [rectangle, minimum width=4cm, minimum height=1.0cm, text centered, draw=black, fill=white!30, text width=3cm]
\tikzstyle{decision} = [diamond, minimum width=4cm, minimum height=1.0cm,  text centered, text width=1.5cm, draw=black, fill=white!30, text width=3cm]
\tikzstyle{arrow_flow} = [ccmDBlue, rounded corners, line width = 2mm, ->]
\tikzstyle{arrow_return} = [ccmRed, rounded corners, line width = 2mm, ->]

\begin{tikzpicture}[node distance=2cm]
    \centering
    \node (start) [start] {Connect to esp32};
    \node (get) [process, below of=start,yshift=-0.5cm] {Get near obstacles};
    \node (check) [process, aspect=2.5, below of=get, yshift=-0.5cm, text width=4cm] {Check distance (using closest point)};
    \node (dec1) [decision, aspect=2.5, below of=check, yshift=-0.5cm] {Impact?};
    
    \node (yes_impact) [process, left of=dec1, xshift=-2cm, yshift=-2cm] {Add 1 impact};
    \node (yes_message) [process, below of=yes_impact, yshift=-0.7cm] {Send impact message};
    
    \node (id_obstacle) [process, right of=dec1, xshift=2cm, yshift=-2cm] {Identify nearest obstacle};
    \node (dir_obstacle) [process, below of=id_obstacle, yshift=-0.5cm, text width=4cm] {Identify direction of the obstacle};
    \node (int_obstacle) [process, below of=dir_obstacle, yshift=-0.5cm, text width=4cm] {Identify intensity based on distance};
    \node (no_message) [process, below of=int_obstacle, yshift=-0.5cm] {Send message to vibrator};
    
    \draw [arrow_flow] (start) -- (get);
    \draw [arrow_flow] (get) -- (check);
    \draw [arrow_flow] (check) -- (dec1);
    
    \draw [arrow_flow] (dec1) -- node[anchor=south] {no} +(4,0) -- (id_obstacle);
    \draw [arrow_flow] (id_obstacle) -- (dir_obstacle);
    \draw [arrow_flow] (dir_obstacle) -- (int_obstacle);
    \draw [arrow_flow] (int_obstacle) -- (no_message);
    \draw [arrow_return] (no_message) -- ++(0,-1.5) -- ++(3,0) -- (7,-3) -- (read);
    
    \draw [arrow_flow] (dec1) -- node[anchor=south] {yes} +(-4,0) -- (yes_impact);
    \draw [arrow_flow] (yes_impact) -- (yes_message);
    \draw [arrow_return] (yes_message) -- ++(0,-1.5) -- ++(-3,0) -- (-7,-3) -- (read);
    
\end{tikzpicture}

\begin{figure}[!h]
    \centering
    \caption{Unity's process of identifying obstacle and setting the vibration intensity.}
    \label{fig:unity_algorithim}
\end{figure}


\chapter{ECG processing algorithm}
\label{ap:ecg_processing_apend}
%APÊNDICE

%É um conteúdo que você elaborou (você ainda tem o seu apêndice intestinal?)

%\section{Printed circuit board used on the haptic belt}


\begin{figure}[!h]
    \centering
    \tikzstyle{start} = [rectangle, rounded corners, minimum width=4cm, minimum height=1.0cm,text centered, draw=black, fill=white!30, text width=3cm]
\tikzstyle{process} = [rectangle, minimum width=4cm, minimum height=1.0cm, text centered, draw=black, fill=white!30, text width=3.5cm]
\tikzstyle{decision} = [diamond, minimum width=3.5cm, minimum height=1.0cm,  text centered, text width=3.5cm, draw=black, fill=white!30]
\tikzstyle{arrow_flow} = [ccmDBlue, rounded corners, line width = 2mm, ->]
\tikzstyle{arrow_return} = [ccmRed, rounded corners, line width = 2mm, ->]

\begin{tikzpicture}[node distance=2cm]
    \centering
    \node (start) [start] {Collect the ECG Data};
    \node (read) [process, below of=start,yshift=-0.5cm] {Python - Read the ECG file};
    \node (outlier) [process, aspect=2.5, below of=read, yshift=-0.5cm, text width=4cm] {Python - Remove the outlier noise};
    \node (normalize) [process, aspect=2.5, below of=outlier, yshift=-0.5cm] {Python - Normalize (-1 and 1)};
    \node (findPeaks) [process, aspect=2.5, below of=normalize, yshift=-0.5cm] {Python - Run peak finder};
    \node (graphical) [decision, aspect=2.5, below of=findPeaks, yshift=-1cm] {Graphical analysis};

    \node (no_graphical) [process, below of=graphical, yshift=-0.75cm] {Python - Tune the peak finder};

    \node (yes_graphical) [process, right of=graphical, xshift=2.5cm, yshift=2.5cm]{Python - Calculate the time difference between peaks};
    \node (savePeak) [process, above of=yes_graphical, yshift=1.0cm, text width=4cm]{Python - Save the peak file};
    \node (readPeak) [process, above of=savePeak, yshift=0.5cm, text width=4cm] {Kubius - Read the peak file};
    \node (analysis) [process, above of=readPeak, yshift=0.5cm, text width=4cm] {Kubius - Run Analysis};
    \node (saveAnalysis) [process, right of=analysis, xshift=2.5cm, yshift=-2.5cm] {Kubius - Save a report file};
    \node (readAnalysis) [process, below of=saveAnalysis, yshift=-1.0cm] {Python - Read report file};
    
    \draw [arrow_flow] (start.south) -- (read.north);
    \draw [arrow_flow] (read.south) -- (outlier.north);
    \draw [arrow_flow] (outlier.south) -- (normalize.north);
    \draw [arrow_flow] (normalize.south) -- (findPeaks.north);
    \draw [arrow_flow] (findPeaks.south) -- (graphical.north);
    
    \draw [arrow_flow] (graphical.east) -- node[anchor=north] {Peaks fit} +(1.8,0) -- (yes_graphical.south);
    \draw [arrow_flow] (yes_graphical.north) -- (savePeak.south);
    \draw [arrow_flow] (savePeak.north) -- (readPeak.south);
    \draw [arrow_flow] (readPeak.north) -- (analysis.south);
    \draw [arrow_flow] (analysis.east) -- ++(2.4,0) -- (saveAnalysis.north);
    \draw [arrow_flow] (saveAnalysis.south) -- (readAnalysis.north);
    \draw [arrow_return] (readAnalysis) -- ++(0,-1.5) -- ++(2.25,0) -- (11.25,1) -- (-3,1) -- (-3,-2.5) -- (read.west);
    
    \draw [arrow_flow] (graphical.west) -- ++(-0.3,0) -- node[anchor=south west] {Peaks} node[anchor=west] {fit not} ++(0,-2.75) -- (no_graphical.west);

    \draw [arrow_return] (no_graphical.north) -- (graphical.south);
    
\end{tikzpicture}
    \caption{ECG's data treatment algorithim.}
    \label{fig:ecg_algorithim}
\end{figure}

\pagebreak

\chapter{GSR processing algorithm}
\label{ap:gsr_processing_apend}
%APÊNDICE

%É um conteúdo que você elaborou (você ainda tem o seu apêndice intestinal?)

\begin{figure}[!h]
    \centering
    \tikzstyle{start} = [rectangle, rounded corners, minimum width=4cm, minimum height=1.0cm,text centered, draw=black, fill=white!30, text width=3cm]
\tikzstyle{process} = [rectangle, minimum width=4cm, minimum height=1.0cm, text centered, draw=black, fill=white!30, text width=3.5cm]
\tikzstyle{decision} = [diamond, minimum width=3.5cm, minimum height=1.0cm,  text centered, text width=3.5cm, draw=black, fill=white!30]
\tikzstyle{arrow_flow} = [ccmDBlue, rounded corners, line width = 2mm, ->]
\tikzstyle{arrow_return} = [ccmRed, rounded corners, line width = 2mm, ->]

\begin{tikzpicture}[node distance=2cm]
    \centering
    \node (start) [start] {Collect the ECG Data};
    \node (read) [process, below of=start,yshift=-0.5cm] {Read the GSR file};
    \node (average) [process, aspect=2.5, below of=read, yshift=-0.5cm] {Calculate the average};
    %\node (std) [process, aspect=2.5, below of=average, yshift=-0.5cm] {Calculate the standard deviaton};
    
    \draw [arrow_flow] (start.south) -- (read.north);
    \draw [arrow_flow] (read.south) -- (average.north);
    %\draw [arrow_flow] (average.south) -- (std.north);
    %\draw [arrow_flow] (std.south) -- (findPeaks.north);
    
\end{tikzpicture}
    \caption{GSR's data treatment algorithim.}
    \label{fig:gsr_algorithim}
\end{figure}

\pagebreak


% Anexos
\annex

\chapter{DOIT Esp32 DevKit v1 datasheet}
\label{an:esp32_annex}
%ANEXO

%É um conteúdo elaborado por outros (ex. datasheet de equipamento)

\section{Especifications}

    \begin{table}[h]
        \begin{tabular}{ll}
            CPU: & Xtensa® Dual-Core 32-bit LX6 \\
            ROM: & 448 KBytes \\
            RAM: & 520 Kbytes \\
            Flash & 4 MB \\
            Max Clock: & 240MHz \\
            \multicolumn{2}{l}{Wireless 802.11 b/g/n} \\
            \multicolumn{2}{l}{Built-in Antena} \\
            Conector: & Micro-usb \\
            \multicolumn{2}{l}{Wi-Fi Direct (P2P), P2P Discovery, P2P Group Owner mode e P2P Power Management} \\
            Modos de operação: & STA/AP/STA+AP \\
            Bluetooth BLE 4.2 \\
            Portas GPIO: & 11 \\ 
            GPIO com funções de PWM, I2C, SPI, etc \\
            Tensão de operação: & 4,5 ~ 9V \\
            Taxa de transferência: & 110-460800bps \\
            Suports Firmware remote upgrade \\
            Conversor analógico digital (ADC) \\
            Distância entre pinos: & 2,54mm \\
            Dimensões: & 49 x 25,5 x 7 mm
        \end{tabular}
    \end{table}
\pagebreak


\section{Pinnout}

    \begin{figure}[h]
        \centering
        \includegraphics[width=\textwidth]{AnexC/esp32-devkitC-v4-pinout.png}
        \caption{ESP32 pinnout diagramm \cite{espressif}.}
        \label{esp32_pins}
    \end{figure}

\chapter{Kubius}
\label{an:kubius_annex}
%ANEXO

%É um conteúdo elaborado por outros (ex. datasheet de equipamento)

\section{Kubius software}

Kubios HRV Standard is a heart rate variability (HRV) analysis software for personal
non-commercial use. The Kubios HRV Standard makes it possible to use your HR monitor to
examine the health of the cardiovascular system or to evaluate stress and recovery.




% Glossario
%\itaglossary
%\printglossary
%GLOSSÁRIO

%Conceitos e definições usados no trabalho.


% Folha de Registro do Documento
% Valores dos campos do formulario
\FRDitadata{}
\FRDitadocnro{DCTA/ITA/DM-018/2015} %(o número de registro você solicita a biblioteca)
\FRDitaorgaointerno{Instituto Tecnológico de Aeronáutica -- ITA}
%Exemplo no caso de pós-graduação: Instituto Tecnol{\'o}gico de Aeron{\'a}utica -- ITA
\FRDitapalavrasautor{Realidade Virtual; Fatores Humanos; Simulação}
\FRDitapalavrasresult{Cupim; Dilema; Construção}
%Exemplo no caso de graduação (TG):
%\FRDitapalavraapresentacao{Trabalho de Graduação, ITA, São José dos Campos, 2015. \NumPenultimaPagina\ páginas.}
%Exemplo no caso de pós-graduação (msc, dsc):
\FRDitapalavraapresentacao{ITA, São José dos Campos. Curso de Mestrado. Programa de Pós-Graduação em Engenharia Aeronáutica e Mecânica. Área de Sistemas Aeroespaciais e Mecatrônica. Orientador: Prof.~Dr. Adalberto Santos Dupont. Coorientadora: Prof$^\textnormal{a}$.~Dr$^\textnormal{a}$. Doralice Serra. Defesa em 05/03/2015. Publicada em 25/03/2015.}
\FRDitaresumo{%% Resumo

%A sociedade alcançou tecnologia para criar veículos autônomos e conectar diferentes aparelhos e máquinas umas às outras a fim de trocar informações e otimizar a eficiência de produção. Com essa tecnologia, logo será possível obter melhores métodos para orientar usuários cegos e deficientes visuais (CDV) nas suas atividades diárias. Os produtos que estão disponíveis no mercado hoje em dia possuem um número de limitações e não agradam os usuários CDV. Acredita-se que uma das razões desse problema é a ausência do envolvimento de indivíduos CDV no desenvolvimento desses produtos. A falta de uma solução eficiente para a navegação desse público tornou-se mais grave com a pandemia da SARS-CoV 2, quando pessoas eram instruídas a praticar isolamento social e evitar contato em superfícies que possam estar contaminadas. O objetivo desse trabalho é propor um método para avaliação de opções de design para produtos assistivos para CDV baseados em Realidade Virtual (RV). A ideia é usar o RV como um campo de teste, onde o usuário pode experimentar diferentes soluções em diferentes cenários. Com isso, ele se torna integrante do design e da avaliação, resultando em um produto melhor e com uma interface mais simples. O método proposto inclui, além da montagem do ambiente virtual, o uso de sensores fisiológicos e testes subjetivos que aferem a carga mental e a consciência situacional nas diferentes situações e produtos que estão em desenvolvimento. Para ilustrar o método proposto, é estudado a navegação de indivíduos CDV em um hospital que usa protocolos COVID-19. Esse estudo de caso foi escolhido devido a ocorrente pandemia e a situação crítica que ela causa à população CDV. O cenário virtual foi feito usando Unity3D, uma plataforma de desenvolvimento de aplicações para realidade virtual largamente utilizada. O aparelho RV é o Tobbi Eye Tracking VR. São óculos que foram desenvolvidos usando o HTC VIVE. Esses óculos são utilizados para definir a posição e orientação do usuário no ambiente virtual do Unity. Para inferir a carga mental, foram utilizados os sensores fisiológicos da TEA Capitv T-Sens. Eles são o eletrocardiograma (ECG), usado para coletar a frequência cardíaca e a variância cardíaca, e o GSR (Galvanic skin reaction, reação galvânica da pele), para captar a condutância da pele. Além desses sensores, os voluntários também responderam os testes NASA-TLX, também para verificar a carga mental, e uma versão adaptada do SAGAT, para determinar a consciência situacional. Entre os benefícios esperados pelo método é a flexibilidade e a agilidade para se criar diferentes cenários e também a possibilidade de testar eles no mesmo espaço físico. Isso pode acelerar o design de novas soluções e melhorar a qualidade dos produtos. Outro resultado esperado da pesquisa é a identificação de características chaves dos produtos que causam o aumento ou diminuição da carga mental ou da consciência situacional nos usuários CDV.

Society has developed technology to create autonomous vehicles and to connect different devices and machinery to exchange data and optimize production efficiency.  With this technology, soon, it will be possible to achieve better methods to guide blind and visually impaired (BVI) users in their daily activities. The available products in the market have several limitations and do not satisfy BVI users. We believe that one of the reasons behind this problem is that they are not members of the development team or are not consulted by these. 
The lack of an efficient solution for BVI users' navigation became even more significant with the SARS-CoV2 pandemic, in which people had to avoid contact with one another and not touch another surface.
The purpose of this paper is to use virtual reality (VR) to test and evaluate different designs of BVI products. Also to verify if BVI and non-BVI users have the same mental demand and situation awareness when using assistive products. The idea is to use VR as a testing ground where a BVI user can try different assistive solutions in different scenarios. By doing so, the user becomes part of the product design and evaluation, resulting in better and more user-friendly products. The proposed method includes not only the setup of the virtual environment but also the use of physiological sensors and subjective tests to assess the mental workload and situational awareness in different situations.
To illustrate the proposed method, a case study is proposed, in which the navigation of BVI users inside a medical clinic is studied. This case study is chosen due to the current undergoing SARS-CoV-2 pandemic and the impact on BVI people, so the simulated clinic is also applying COVID health protocols.
The scenes were made using Unity3D, a widely used development platform for virtual reality applications. The VR device was the Tobii Eye Tracking VR, a head-mounted display for virtual reality developed using the HTC VIVE. This VR device is used for defining the user position and orientation inside the virtual environment. Based on the current situation in the virtual environment, inputs are provided to the user using aural commands and haptics devices. To assess the mental workload, physiological sensors, from TEA Captiv T-Sens, are used. Among them, are an electrocardiogram sensor (ECG), to gather heart-rate and heart-rate variance data, and a galvanic skin response sensor (GSR), to collect skin conductance. Besides these sensors, the users are also expected to answer mental workload assessment tests and situation awareness questionnaires.
Among the proposed method's expected benefits are the flexibility and agility to create different scenarios, and also the possibility to test all of them in the same physical room. The method could not only speed the design of new solutions but also improve the overall quality of the products and verify the need of a BVI user in the development team of an assistive product.}
%  Primeiro Parametro: Nacional ou Internacional -- N/I
%  Segundo parametro: Ostensivo, Reservado, Confidencial ou Secreto -- O/R/C/S
\FRDitaOpcoes{N}{O}
% Cria o formulario
\itaFRD

\end{document}
% Fim do Documento. O massacre acabou!!! :-)
