Extended reality refers to the interaction of a Human-Machine system with a real and virtual interface together. It has four differents forms:
\begin{itemize}
    \item Augmented Reality;
    \item Augmented Virtuality;
    \item Mixed Reality;
    \item Virtual Reality.
\end{itemize}

These forms differs from one another based on a leverage of reality and virtuality involved on the system. To help to visualise these differences, \citeonline{milgram1994taxonomy} created the concept of "virtuality continuum" and is presented on Figure \ref{fig:virtuality_continuum}. 

\begin{figure}[!htb]
    \tikzstyle{arrow} = [ccmDBlue, rounded corners, line width = 2mm, ->]
    \tikzstyle{--blue} = [ccmDBlue, rounded corners, line width = 2mm]
    \tikzstyle{--black} = [rounded corners, line width = 1mm]
    
    %\tikzstyle{arrow_flow} = [ccmblue, rounded corners, line width = 2mm, ->]
    %\tikzstyle{arrow_return} = [ccmred, rounded corners, line width = 2mm, ->]
    
    \resizebox{0.80\width}{!}{
    \begin{tikzpicture}[node distance=1cm]
        \centering
    
        \node (left) {};
        
        \node (reality) [right of = left, xshift = 2cm]{\includegraphics[width=.15\textwidth]{Fundamentação/Realidade Extendida/real.png}}
        node(t_reality)[below of = reality,yshift=-0.75cm] {Real}
        node(t_reality2)[below of = t_reality,yshift=0.5cm] {Environment};
        
        \node (midLeft) [right of = reality, xshift = 1cm] {};
        
        \node (ar) [right of=reality, xshift=3cm] {\includegraphics[width=.15\textwidth]{Fundamentação/Realidade Extendida/ar.png}}
        node(t_ar)[below of = ar,yshift=-0.75cm] {Augmented}
        node(t_ar2)[below of = t_ar,yshift=0.5cm] {Reality};
        
        \node (av) [right of=ar, xshift=3cm] {\includegraphics[width=.15\textwidth]{Fundamentação/Realidade Extendida/av.png}}
        node(t_av)[below of = av,yshift=-0.75cm] {Augmented}
        node(t_av2)[below of = t_av,yshift=0.5cm] {Virtuality};
        
        \node (vr) [right of=av, xshift=3cm] {\includegraphics[width=.15\textwidth]{Fundamentação/Realidade Extendida/vr.png}}
        node(t_vr)[below of = vr,yshift = -0.75cm] {Virtual} 
        node(t_vr2)[below of = t_vr,yshift = 0.5cm] {Reality};
        
        \node (midRight) [left of = vr, xshift = -1cm] {};
        
        \node (right) [right of = vr, xshift = 2cm] {};
        
        \node (mr) [above of=ar, right of=ar, xshift=1cm, yshift=3cm] {\includegraphics[width=.15\textwidth]{Fundamentação/Realidade Extendida/mr.png}} 
        node(t_mr) [below of = mr, yshift = -0.50cm]{Mixed}
        node(t_mr2) [below of = t_mr, yshift = 0.5cm]{Reality};
        
        \draw [arrow] (reality) to (left);
        \draw [--blue] (reality) to (ar);
        \draw [--blue] (ar) to (av);
        \draw [--blue] (av) to (vr);
        \draw [arrow] (vr) to (right);
        \draw [--black] (mr.west) -- +(-2.6,0) to (midLeft);
        \draw [--black] (mr.east) -- +(2.6,0) to (midRight);
         
        
        \node (left) [left of = reality, xshift = -2cm] {};
        \node (right) [right of = vr, xshift = 2cm] {};
    
        
    \end{tikzpicture}
    }
    \centering
    \caption{The Virtuality Continuum concept. Adapted from \cite{milgram1994taxonomy}}
    \label{fig:virtuality_continuum}
\end{figure}

The extreme left means full reality, where the stimuli does not come, or is not produced, from any computer or any other digital system. Along the path to the right, the environment starts to have some digital elements until it reaches the far right, where all the elements in the environment have a digital origin \cite{nijholt2005virtuality, doolani2020review}. The first step from the Real Environment towards Virtual Reality is the Augmented Reality.

\subsection{Augmented Reality (AR)}
\label{subsec:augmented_reality}

    In augmented reality, the user can see some digital elements, that could be text, images, video, etc, that are laid in a real environment without the user losing the sense of presence in the real world. Some uses for AR are to assist workers in the manufacturing, assembly tasks and in training \cite{doolani2020review, farrell2018learning, ma2007virtuality}.

\subsection{Augmented Virtuality (AV)}
\label{subsec:augmented_virtuality}
    
    While AR brings digital elements inside a real environment, Augmented Virtuality creates an environment that could only exist with a digital origin, like a fantasy world from games or movies. This scenario is the background of some other activity that is being done in a real environment. An example could be using a virtual environment during the a pilot or driver training or an engineer visualizing a real-time model of an aircraft in flight \cite{farshid2018go}. Other example could be playing sports the use an equipment to play it, like tennis, golf or baseball but the arena is completely digital. The user can use the real equipment with a tracker, but, besides that, the rest would be all digital.

\subsection{Mixed Reality (MR)}
\label{subsec:mixed_reality}

    Mixed Reality stay in between Real and Virtual Environment. But what is the difference between MR and AR or AV? In MR the user can manipulated the digital element, as if it where inside the real world \cite{doolani2020review}. For example, a client from a furniture store could use MR to see what product fit inside his/her room. He/she can move the furniture inside the room and see if the colors, size and shape fit before buying or even going to the shop.
    
\subsection{Virtual Reality (VR)}
\label{subsec:virtual_reality}

    Resting in the far right of the virtuality continuum, the Virtual Reality has its user as the only element that hasn't a digital origin, making he/she totally immersed in a virtual environment, but, of course, inside the physical limits of a real environment \cite{ma2007virtuality}. If the feeling of presence of that environment is well done, the user can momentarily forget about the real environment and act and react accordingly to the virtual environment \cite{farrell2018learning}. 
    
    VR is a powerful tool that allow an user to be transported to a tridimensional environment that could be out of reach or that doesn't exist but is perfect to test or train some situation. Inside this virtual environment the user can walk and look around and interact with the many elements as if they were real \cite{mujber2004virtual} and this technique becomes more effective and valuable when one can simulate a real situation and use it for training \cite{salah2019virtual}.
    
The Figure \ref{fig:ar_av_mr} shows the representations of each of these Extended Reality subsections.
    
\begin{figure}[!htb]
    \tikzstyle{arrow} = [ccmblue, rounded corners, line width = 2mm, ->]
    \tikzstyle{--blue} = [ccmblue, rounded corners, line width = 2mm]
    \tikzstyle{--black} = [rounded corners, line width = 1mm]
    
    \resizebox{0.85\width}{!}{
    \begin{tikzpicture}[node distance=1cm]
        \centering
    
        \node (ar) {\includegraphics[width=0.3\textwidth]{Fundamentação/Realidade Extendida/ar.png}}
        node(t_ar)[below of = ar,yshift=-2.0cm] {Augmented}
        node(t_ar2)[below of = t_ar,yshift=0.5cm] {Reality};
        
        \node (av) [right of=ar, xshift=5cm, yshift = 0.1cm] {\includegraphics[width=.34\textwidth]{Fundamentação/Realidade Extendida/av.png}}
        node(t_av)[below of = av,xshift = 1.0cm, yshift=-2.0cm] {Augmented}
        node(t_av2)[below of = t_av,yshift=0.5cm] {Virtuality};
        
        \node (mr) [below of=ar, yshift=-5.5cm] {\includegraphics[width=.31\textwidth]{Fundamentação/Realidade Extendida/mr.png}}
        node(t_mr)[below of = mr,yshift=-2.0cm] {Mixed}
        node(t_mr2)[below of = t_mr,yshift=0.5cm] {Reality};
        
        \node (vr) [right of=mr, xshift=5.0cm, yshift = -0.5cm] {\includegraphics[width=.25\textwidth]{Fundamentação/Realidade Extendida/vr.png}}
        node(t_vr)[below of = vr,yshift = -1.5cm] {Virtual} 
        node(t_vr2)[below of = t_vr,yshift = 0.5cm] {Reality};
        
        \node (n) [right of = ar, above of = ar , xshift = 2cm, yshift = 3.0cm] {};
        \node (l) [right of = av, below of = av, xshift = 3cm, yshift = -3.1cm] {};
        \node (s) [left of = vr, below of = vr, xshift = -2cm, yshift = -3.0cm] {};
        \node (o) [left of = mr, above of = mr, xshift = -3cm, yshift = 1.6cm] {};
        
        
        %\draw [arrow] (reality) to (left);
        %\draw [--blue] (reality) to (ar);
        %\draw [--blue] (ar) to (av);
        %\draw [--blue] (av) to (vr);
        %\draw [arrow] (vr) to (right);
        \draw [--black] (n) to (s);
        \draw [--black] (l) to (o);
         
        
        %\node (left) [left of = reality, xshift = -2cm] {};
        %\node (right) [right of = vr, xshift = 2cm] {};
    
        
    \end{tikzpicture}
    }
    \centering
    \caption{A represantion of the differences between AR, AV, MR and VR. Made by the author}
    \label{fig:ar_av_mr}
\end{figure}
    