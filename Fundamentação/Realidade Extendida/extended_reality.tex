Extended reality is a broad term that refers to all different ways of combining virtual and real entities in a human-machine interface system. It is usually decomposed into four classes (augmented reality, augmented virtuality, mixed reality, and virtual reality) that differ on the level of reality and virtuality involved in the interface system. 

\citeonline{milgram1994taxonomy} organized these classes and created the concept of the “virtuality continuum”, as illustrated in Figure, as illustrated in Figure \ref{fig:virtuality_continuum}. On the extreme left, the real environment represents the cases where the user operates physical elements inside that environment. Along the path to the right, the environment starts to incorporate digital elements until it reaches the far right, where all the elements in the environment are virtual and have a digital origin \cite{nijholt2005virtuality, doolani2020review}. The first step from the "Real Environment" to "Virtual Reality" is the augmented reality.

\begin{figure}[!htb]
    \tikzstyle{arrow} = [ccmDBlue, rounded corners, line width = 2mm, ->]
    \tikzstyle{--blue} = [ccmDBlue, rounded corners, line width = 2mm]
    \tikzstyle{--black} = [rounded corners, line width = 1mm]
    
    %\tikzstyle{arrow_flow} = [ccmblue, rounded corners, line width = 2mm, ->]
    %\tikzstyle{arrow_return} = [ccmred, rounded corners, line width = 2mm, ->]
    
    \resizebox{0.80\width}{!}{
    \begin{tikzpicture}[node distance=1cm]
        \centering
    
        \node (left) {};
        
        \node (reality) [right of = left, xshift = 2cm]{\includegraphics[width=.15\textwidth]{Fundamentação/Realidade Extendida/real.png}}
        node(t_reality)[below of = reality,yshift=-0.75cm] {Real}
        node(t_reality2)[below of = t_reality,yshift=0.5cm] {Environment};
        
        \node (midLeft) [right of = reality, xshift = 1cm] {};
        
        \node (ar) [right of=reality, xshift=3cm] {\includegraphics[width=.15\textwidth]{Fundamentação/Realidade Extendida/ar.png}}
        node(t_ar)[below of = ar,yshift=-0.75cm] {Augmented}
        node(t_ar2)[below of = t_ar,yshift=0.5cm] {Reality};
        
        \node (av) [right of=ar, xshift=3cm] {\includegraphics[width=.15\textwidth]{Fundamentação/Realidade Extendida/av.png}}
        node(t_av)[below of = av,yshift=-0.75cm] {Augmented}
        node(t_av2)[below of = t_av,yshift=0.5cm] {Virtuality};
        
        \node (vr) [right of=av, xshift=3cm] {\includegraphics[width=.15\textwidth]{Fundamentação/Realidade Extendida/vr.png}}
        node(t_vr)[below of = vr,yshift = -0.75cm] {Virtual} 
        node(t_vr2)[below of = t_vr,yshift = 0.5cm] {Reality};
        
        \node (midRight) [left of = vr, xshift = -1cm] {};
        
        \node (right) [right of = vr, xshift = 2cm] {};
        
        \node (mr) [above of=ar, right of=ar, xshift=1cm, yshift=3cm] {\includegraphics[width=.15\textwidth]{Fundamentação/Realidade Extendida/mr.png}} 
        node(t_mr) [below of = mr, yshift = -0.50cm]{Mixed}
        node(t_mr2) [below of = t_mr, yshift = 0.5cm]{Reality};
        
        \draw [arrow] (reality) to (left);
        \draw [--blue] (reality) to (ar);
        \draw [--blue] (ar) to (av);
        \draw [--blue] (av) to (vr);
        \draw [arrow] (vr) to (right);
        \draw [--black] (mr.west) -- +(-2.6,0) to (midLeft);
        \draw [--black] (mr.east) -- +(2.6,0) to (midRight);
         
        
        \node (left) [left of = reality, xshift = -2cm] {};
        \node (right) [right of = vr, xshift = 2cm] {};
    
        
    \end{tikzpicture}
    }
    \centering
    \caption{The Virtuality Continuum concept. Adapted from \cite{milgram1994taxonomy}}
    \label{fig:virtuality_continuum}
\end{figure}

In the augmented reality system, the user sees some digital elements that are laid over the real environment. without making the user lose his sense of presence in the real world. These elements can be text, images, video, etc. Augmented reality can be used to assist workers in manufacturing and assembly tasks, as well as training \cite{doolani2020review, farrell2018learning, ma2007virtuality}.
    
While the augmented reality brings digital elements to the real environment, the augmented virtuality creates an environment that could only exist digitally, such as a fantasy world from games or movies. This scenario is the background of some other activity that is being done by the user in the real environment. An example is to train a pilot in a virtual environment but with an accurate mock-up of the cockpit, which provides physical buttons and inceptors for the pilot to touch and hold \cite{farshid2018go}. Another example is to play sports, such as tennis, golf or baseball, in a complete digital arena but using the actual equipment with a tracker. \citeonline{kirner2012using} also add that augmented reality has three characteristics: it combines real and virtual elements; it has real time interaction; and three-dimensional.

The mixed reality stays in between the real and virtual environments. Unlike augmented reality and augmented virtuality, in a mixed reality system the user can manipulate digital elements as if they were inside the real world \cite{doolani2020review}. One example is when a client from a furniture store uses mixed reality not only to see how the furniture fits inside his room, but he can also move it and change its color, size and shape before buying or even going to the shop.
    
On the far right of the virtuality continuum, the virtual reality is when the user is the only non-digital element, everything else is digital, immersing the user in a virtual environment, but, of course, inside the physical limits of the real environment \cite{ma2007virtuality}. If the feeling of presence inside that environment is well tailored, the user can momentarily forget about the real environment and act and react accordingly to the virtual environment \cite{farrell2018learning}. 
    
Virtual reality is a powerful tool that allows a user to be transported to a tridimensional environment that could be out of reach or that does not exist but is needed for testing or training reasons \cite{mujber2004virtual}. Inside this virtual environment, the user can walk, look around and feel as if the environment was real \cite{salah2019virtual}.

Figure \ref{fig:ar_av_mr} shows the representations of each of these Extended Reality classes.

\begin{figure}[!htb]
    \tikzstyle{arrow} = [ccmblue, rounded corners, line width = 2mm, ->]
    \tikzstyle{--blue} = [ccmblue, rounded corners, line width = 2mm]
    \tikzstyle{--black} = [rounded corners, line width = 1mm]
    
    \resizebox{0.85\width}{!}{
    \begin{tikzpicture}[node distance=1cm]
        \centering
    
        \node (ar) {\includegraphics[width=0.3\textwidth]{Fundamentação/Realidade Extendida/ar.png}}
        node(t_ar)[below of = ar,yshift=-2.0cm] {Augmented}
        node(t_ar2)[below of = t_ar,yshift=0.5cm] {Reality};
        
        \node (av) [right of=ar, xshift=5cm, yshift = 0.1cm] {\includegraphics[width=.34\textwidth]{Fundamentação/Realidade Extendida/av.png}}
        node(t_av)[below of = av,xshift = 1.0cm, yshift=-2.0cm] {Augmented}
        node(t_av2)[below of = t_av,yshift=0.5cm] {Virtuality};
        
        \node (mr) [below of=ar, yshift=-5.5cm] {\includegraphics[width=.31\textwidth]{Fundamentação/Realidade Extendida/mr.png}}
        node(t_mr)[below of = mr,yshift=-2.0cm] {Mixed}
        node(t_mr2)[below of = t_mr,yshift=0.5cm] {Reality};
        
        \node (vr) [right of=mr, xshift=5.0cm, yshift = -0.5cm] {\includegraphics[width=.25\textwidth]{Fundamentação/Realidade Extendida/vr.png}}
        node(t_vr)[below of = vr,yshift = -1.5cm] {Virtual} 
        node(t_vr2)[below of = t_vr,yshift = 0.5cm] {Reality};
        
        \node (n) [right of = ar, above of = ar , xshift = 2cm, yshift = 3.0cm] {};
        \node (l) [right of = av, below of = av, xshift = 3cm, yshift = -3.1cm] {};
        \node (s) [left of = vr, below of = vr, xshift = -2cm, yshift = -3.0cm] {};
        \node (o) [left of = mr, above of = mr, xshift = -3cm, yshift = 1.6cm] {};
        
        
        %\draw [arrow] (reality) to (left);
        %\draw [--blue] (reality) to (ar);
        %\draw [--blue] (ar) to (av);
        %\draw [--blue] (av) to (vr);
        %\draw [arrow] (vr) to (right);
        \draw [--black] (n) to (s);
        \draw [--black] (l) to (o);
         
        
        %\node (left) [left of = reality, xshift = -2cm] {};
        %\node (right) [right of = vr, xshift = 2cm] {};
    
        
    \end{tikzpicture}
    }
    \centering
    \caption{A represantion of the differences between AR, AV, MR and VR. Made by the author}
    \label{fig:ar_av_mr}
\end{figure}

Although some researchers still use this classification, it is considered outdated by others. Nowadays the shift between augmented reality and augmented virtuality is not gradual because they group different styles of interaction, have different equipments necessary to create the interface and has other technologies that at the time the virtuality continuum was conceived it was not considered \cite{cerqueira2019tangible}.

For \citeonline{kirner2012using} consider an additional class: cross-reality. It is the combination of having a virtual environment and a real environment connected with sensors and actuators creating a bidirection information feed between both environments.