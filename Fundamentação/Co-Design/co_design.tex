Collaborative design is a way to design in which each element inside a development team has a different experience, resources, ideas or formation which can be important for the product effectiveness. It is based on good communication and information sharing \cite{chiu2002organizational}. Still, it is common to say "collaboration" interchangeably with "interaction" and "cooperation". Some authors define those words differently. "Interaction" is a more formal relationship between the elements \cite{kahn1996interdepartmental} and "cooperation" focus more on the coordination and the mutual gain or benefit between the elements \cite{smith1996top}.

For collaborations, the shared vision and the process to be followed are important \cite{kleinsmann2006understanding}.

\begin{quote}
    Collaborative design is the process in which actors from different disciplines share their knowledge about both the design process and the design content. They do that to create a shared understanding of both aspects, to be able to integrate and explore their knowledge and to achieve the larger common objective: the new product to be designed \cite{kleinsmann2006understanding}.
\end{quote}

According to \citeonline{kleinsmann2006understanding} two aspects are important for Collaborative Design:

Information is a data after the receiver's understanding, or translating, process. Knowledge is the data in a state that is possible to record or register to remember later inside the individual's memory. These can be ideas, facts or concepts. During the collaborative design, these ideas, facts or concepts are exchanged between the actors. This exchange is a fundamental part of this method since it is responsible for the growth of each individual's knowledge and this is used to perform their tasks. This brings us to the second aspect, the knowledge integration \cite{kleinsmann2006understanding}.

With both of these aspects in mind, \citeonline{kleinsmann2006understanding} defines Collaborative design as "the process in which actors from different disciplines share their knowledge about both the design process and the design content". This happens to increase the team's understanding to help them to design a new product based on all of their knowledge and experience.