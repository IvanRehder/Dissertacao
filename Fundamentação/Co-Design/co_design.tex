Co-design, or collaborative design, refers to a design process in which individuals of the design team have different backgrounds or bring different experiences, which can be essencial for the product under design. It is based on good communication and information sharing among the team \cite{chiu2002organizational}.

\citeonline{kleinsmann2006understanding} provides the following definition:.

\begin{quote}
    \textit{"Collaborative design is the process in which actors from different disciplines share their knowledge about both the design process and the design content. They do that to create a shared understanding of both aspects, to be able to integrate and explore their knowledge and to achieve the larger common objective: the new product to be designed."} %\cite{kleinsmann2006understanding}.
\end{quote}

This definition emphasizes two critical aspects of co-design: knowledge sharing and integration. According to \citeonline{kleinsmann2006understanding} knowledge is the data after the receiver's understanding or translating process, in a state that is possible to record or register, so that the person can remember and use it later. During the collaborative design, ideas, facts or concepts are exchanged between the actors. This exchange is a fundamental part of the co-design process since it is responsible for the growth of each individual's knowledge. Once the knowledge is shared among the actors, they can use it when performing their tasks, resulting in knowledge integration \cite{kleinsmann2006understanding}.