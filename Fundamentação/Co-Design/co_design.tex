%-------------------------------------------------------
%-------------------------------------------------------
%
%COLAB DESIGN
%
%-------------------------------------------------------
%-------------------------------------------------------

Collaborative design is a way to design that each element inside the design team has a different experience, resourcers, ideas or formation that is important for the product effectiveness. It is based on a good communication and in the information sharing \cite{chiu2002organizational}. Still, it is common to say "collaboration" interchangibly with "interaction" and "cooperation". Some authors define those words differently. "Interaction" is a more formal relationship between the elements \cite{kahn1996interdepartmental} and "cooperation" focus more on the coordination and the mutual gain or benefit between the elements \cite{smith1996top}.

For collaborations, the shared vision and the process to be followed are imporant \cite{kleinsmann2006understanding}.

\begin{quote}
    Collaborative design is the process in which actors from different disciplines share their knowledge about both the design process and the design content. They do that in order to create shared understanding on both aspects, to be able to integrate and explore their knowledge and to achieve the larger common objective: the new product to be designed \cite{kleinsmann2006understanding}.
\end{quote}

%\begin{quote}
%    Colaborative design is an activity that requires participation of  individuals for sharing information and organizing design tasks and resources. Particularly in a complex and large project, design often involves multiple persons or groups collaborating in the design process. The purpose of design collaboration is to share expertise, ideas, resources, or responsibilities. Design communication is central to design development in the process. The effectiveness of design communication becomes critical for designers in sharing design information, in decision-making and coordinating design tasks. During the last decade, design practice has changed due to globalization and computerization. The use of computer technology in design practice \cite{chiu2002organizational}.
%\end{quote}

%Chiu's definition contains two important aspects of collaboration: sharing and organizing both tasks and resources. Chiu named the process of sharing expertise, ideas, resources and responsibilities design communication. Design communication is about the content of the design process. Valkenburg (2000) calls this communication about the design content. Valkenburg, Chiu and other researchers claim that the effectiveness of design communication is critical for the quality of product design. This is valid from an efficiency point of view, as well as for the quality of the product (Bucciarelli, 1996), (Valkenburg, 2000), (Song et al., 2003). 

%Although the definitions of Chiu (2002) and Kahn (1996) contain important aspects of collaboration and collaborative design, we would like to elaborate on two aspects. 

According to \citeonline{kleinsmann2006understanding} there are two aspects that are important for Collaborative Design:

Information is a data after the receiver understanding, or translating, process. Knowledge is the data in a state that is possible record, register to remember later inside the individual's memory. This can be ideas, facts or concepts. During the collaborative design, these ideas, facts or concepts are exchanged between the actors. This exchange is a fundamental part of this method since it is responsible for the growth of each individuals's knowledge and this is used to perform their individual tasks. This brings us to the second aspect, the knowledge integration \cite{kleinsmann2006understanding}.

With both of these aspects in mind, \citeonline{kleinsmann2006understanding} defines Collaborative design as "the process in which actors from different disciplines share their knowledge about both the design process and the design content". This happens to increase the team understanding to help them to design a new product based on all of their knowledge and experience.

% Collaborative design is the process in which actors from different disciplines share their knowledge about both the design process and the design content. They do that in order to create shared understanding on both aspects, to be able to integrate and explore their knowledge and to achieve the larger common objective: the new product to be designed.

%Actors share design knowledge through design communication, which means communication about the design content (Chui, 2002), (Valkenburg, 2000)

%The first aspect is the question of whether actors share information or knowledge. Court (1997) makes a distinction between these two concepts. Court offers multiple definitions of information. Most definitions focus on the aspect that information is data, which the receiver understood, or data that have a meaning for the receiver. Court also sees information as an activity of informing and becoming informed. This makes information dynamic. 

%Knowledge, on the other hand, is a state which actors can record or register to remember later. According to Court, knowledge is more than just recorded information. It is the mental state of ideas, facts, concepts, data and techniques recorded in the individual's memory. 

%During collaborative design, the individual actors will use these ideas, facts and concepts to perform their individual tasks. In other words, information flows restructure or change the knowledge bases of the individual actors. It is the knowledge base of an actor which influences his actions (Nonaka, 1994). Based on the preceding reasoning, the concept of knowledge sharing is preferred to the concept of information sharing in this thesis.

%The second aspect is that during collaborative design actors have to share their individual knowledge bases, which is a team activity. Additionally, collaborative design is also an activity of knowledge creation and integration (Sonnenwald, 1996). The difference between knowledge creation and integration is that knowledge creation is a divergent activity, while knowledge integration is a convergent activity. Knowledge creation and integration are important for collaborative design. Therefore, both knowledge sharing as well as knowledge creation and integration should be part of the definition of collaborative design. 

%Keeping the preceding aspects in mind (combined with the definition of Chui (2002)), the definition of collaborative design reads as follows: 



