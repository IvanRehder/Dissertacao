Mental workload is one of the main concepts studied in Human Factors \cite{stanton2004handbook}.

In order to explain it, \citeonline{stanton2004handbook} propose an analogy with the concept of physical workload. When an athlete must lift a dumbbell (one of those gym weights bars), the strength demand from the athlete is proportional to the dumbbell’s mass being lifted. If the dumbbell is lighter than the athlete’s capability, then it is easy enough for him to lift it. If the athlete is strong enough to carry the dumbbell, he does not feel a physical demand bigger than his capabilities. In this case, the physical workload of this activity is properly fitted for this athlete. If the dumbbell is heavier than what the athlete can lift, then two things can happen: either the athlete adapts to lift that dumbbell using tools (adjust the strategy), or the athlete is not able to lift completely the dumbbell (performance degrades). It corresponds to the case of a person executing a task that is not fitted for his capabilities.

The mental workload is similar to the physical workload, but refers to the mental capacity necessary to perform a task. Each human being has a finite mental capacity. When the mental demand is higher than the operator’s capacity, the person needs to adapt in order to finish the task or the overall performance of the task is compromised. Otherwise, if the mental workload is too low, the operator may get bored and easily distracted and so could also fail or not process the task’s information.

It’s important to say that mental workload is unique within each individual and is influenced by his/her perception of the task but is also impacted by other factors outside the task itself and more related to the operator (like its skill, age, education, training) or the environment (like noise, heat and toxicity)  \cite{cain2007review, fallahi2016effects, cardoso2012evaluation}.

The mental workload is not a quantitative resource or something that one can directly measure, but a number of different methods have been proposed in the literature to infer it. Figure \ref{fig:mwl_overview} illustrates three different classes of methods to evaluate mental workload: methods based on task performance, methods based on physiological measures, and methods based on subjective questionnaires.
        
    \begin{figure}[!htb]
    \centering

    \tikzstyle{arrow} = [rounded corners, line width = 1mm, ->]
    
    \resizebox{0.85\width}{!}{
    \begin{tikzpicture}[node distance=1cm]
        
        \node (mwl) {\includegraphics[width=.15\textwidth]{Fundamentação/Carga Mental/mwl.png}}
        node(t_mwl)[below of = mwl,yshift=-0.75cm] {Mental}
        node(t_mwl2)[below of = t_mwl,yshift=0.5cm] {workload};
        
        \node (demand) [above of=mwl, xshift=3cm, yshift=2.5cm] {\begin{tikzpicture}[node distance=1cm]
    \centering
    
    \node (worker) {\includegraphics[width=.15\textwidth]{Fundamentação/Carga Mental/work.png}};
    
    \node (energy1) [above of = worker, xshift = 0.5cm, yshift = 0.05cm]
     {\includegraphics[width=.060\textwidth]{Fundamentação/Carga Mental/bolt.png}} 
    node(energy2) [right of=energy1, xshift=-0.5cm, yshift = 0.3cm] {\includegraphics[width=.060\textwidth]{Fundamentação/Carga Mental/bolt.png}}
    node(energy3) [left of = energy1, xshift=0.5cm, yshift = 0.3cm] {\includegraphics[width=.060\textwidth]{Fundamentação/Carga Mental/bolt.png}};;
    
\end{tikzpicture}}
        node(t_demand)[below of = demand,yshift=-0.75cm] {Task}
        node(t_demand2)[below of = t_demand,yshift=0.5cm] {Demand};
        
        \node (capacity) [above of=mwl, xshift=-3cm, yshift=2.5cm] {\includegraphics[width=.15\textwidth]{Fundamentação/Carga Mental/full-battery.png}}
        node(t_capacity)[below of = capacity,yshift=-0.75cm] {Mental}
        node(t_capacity2)[below of = t_capacity,yshift=0.5cm] {Capacity};
        
        \node (tasks) [left of=mwl, xshift=-4cm, yshift=-6cm] {\includegraphics[width=.15\textwidth]{Fundamentação/Carga Mental/multitasking.png}}
        node(t_tasks)[below of = tasks,yshift = -0.75cm] {Primary and} 
        node(t_tasks2)[below of = t_tasks,yshift = 0.5cm] {secondary tasks};
        
        \node (physiological) [below of=mwl, yshift=-5cm,] {\includegraphics[width=.15\textwidth]{Fundamentação/Carga Mental/physiological.png}} 
        node(t_physiological) [below of = physiological, yshift = -0.75cm]{Physiological}
        node(t_physiological2) [below of = t_physiological, yshift = 0.5cm]{measurements};
        
        \node (subjective) [right of=mwl, xshift=4cm, yshift=-6cm] {\includegraphics[width=.15\textwidth]{Fundamentação/Carga Mental/subjective.png}}
        node(t_subjective) [below of = subjective, yshift = -0.75cm]{Subjective} 
        node(t_subjective2) [below of = t_subjective, yshift = 0.5cm]{measurements};
    
    
        \draw [arrow] (t_mwl2.south) to ++(0,-0.75) to +(-5,0) to (tasks.north);
        \draw [arrow] (t_mwl2.south) to (physiological.north);
        \draw [arrow] (t_mwl2.south) to ++(0,-0.75) to +(5,0) to (subjective.north);
        \draw [arrow, sharp corners] (capacity.east) to +(1.6,0) to (mwl.north);
        \draw [arrow, sharp corners] (demand.west) -- +(-1.35,0) to (mwl.north);
    
        
    \end{tikzpicture}
    }
    \caption{A overview of mental workload and the techniques to infer it.}
    \label{fig:mwl_overview}
\end{figure}    
    
    \subsubsection*{Methods based on task performance}
    %\label{subsec:task_performance}
    
        If the mental workload influences the task performance, then it would be possible to infer it using the performance’s variation of a task. Because there are cases where the user’s mental capacity is too high for only one task, a common approach is to add a secondary task. In this case, the user is asked to maintain a good performance level and still try to execute both tasks. Both tasks should use the same kind of skills \cite{stanton2004handbook, sanders1998human}.
        
        For example, an experiment to assess mental workload in a flight simulator may use two tasks, where the primary task is to fly the aircraft maintaining a given performance level, while the secondary task may be to mentally sum two random numbers that appear on the screen and, if the numbers’ sum is odd, then the pilot should press left on the keyboard else he/she should press right. If the pilot’s performance in the secondary task is too low, it means that the mental demand from the first task is too high for him/her to be able to pay attention to the secondary task \cite{mohanavelu2020cognitive}.
        
%    \item Physiological measures;
    \subsubsection*{Methods based on physiological measurements}
    %\label{subsec:physiological_measures}
    
        There are many physiological measurements that can be used to assess mental workload. Among the most common ones are the heart and brain activity \cite{chakladar2020eeg, orlandi2018measuring}, skin conductance, eye movement, pupillary contraction \cite{stanton2004handbook, rodriguez2015pupillometry}. These measurements are considered a good, unbiased assessment method \cite{fallahi2016effects}, but, still, it is recommended that they are evaluated alongside another method, as they may be influenced by unknown variables and external factors.

        This work uses the heart activity, obtained from an electrocardiogram (ECG) sensor, and the electrodermal activity, obtained from a galvanic skin response (GSR) sensor, as physiological measurements to assess mental workload.
                
        The electrocardiogram (ECG) is a recording of the heart’s electrical activity. From it, it is possible to determine the intervals between heartbeats and the corresponding frequency (heart rate, HR). Other common variable is the heart rate standard deviation (heart rate variability, HRV) \cite{cain2007review}. The heart activity is controlled by the sympathetic and parasympathetic nervous systems \cite{stanton2004handbook}. During a task, the heart activity changes with the mental demand of the task. The heart rate is expected to increase with the mental workload, while the heart rate variability is expected to decrease. These are consequences of two reactions in our system when in a mental demand situation \cite{stanton2004handbook}: a decrease in the parasympathetic nervous system activity and an increase in sympathetic nervous system activity. The ECG is a simple and non-invasive method used in many experiments to evaluate mental workload and other human factors’ \cite{mohanavelu2020cognitive, mansikka2016fighter, zhang2014detection}.
         %(DEFINIR MELHOR)
                
         The skin electrodermal activity is affected by the person’s sweating and the level of moisture in the environment. It can be used to reveal changes in our sympathetic system \cite{nourbakhsh2012using, shi2007galvanic}. It has been used in the literature as an assessment method for stress and arousal \cite{nourbakhsh2012using, stanton2004handbook, shi2007galvanic}, the usability of human-computer systems \cite{shi2007galvanic} and also mental workload \cite{zhang2014detection, borghini2014measuring}.
    
    \subsubsection*{Methods based on subjective questionnaires}
    %\label{subsec:subjective_measures}    

        The use of subjective questionnaires to assess mental workload has been extensively discussed in the literature \cite{sanders1998human, stanton2004handbook}. They are sensitive to perceived difficulty, automation, concurrent activities and demand for multiple resources. The questionnaires can be unidimensional, which are simpler but has only a general workload score 

        It is discussed if one should only use subjective measures to measure MWL \cite{sanders1998human, stanton2004handbook}. They are sensitive to perceived difficulty, automation, concurrent activities and demand for multiple resources. These test can be unidimensional, which are simpler but has only a general workload score \cite{stanton2004handbook},or multidimensional. Examples of multidimensional questionnaires are the Subjective Workload Assessment Technique (SWAT) and the NASA Task Load Index (NASA-TLX). SWAT decomposes the mental in three dimensions: time load; mental effort load; and psychological stress. NASA-TLX, a questionnaire created by \citeonline{hart1988development}, uses 6 different dimensions, as described in Table \ref{tab:nasa_dimensions}. These questionnaires were proposed evaluate only one task/activity. If the user has performed two tasks (a primary and secondary task), he/she should be oriented to answer about the primary task, not a combination of both of them \cite{sanders1998human}.
        
        \begin{table}[htb]
            \centering
            \caption{NASA-TLX dimensions and the description of each dimension. \cite{stanton2004handbook}.}
            \label{tab:nasa_dimensions}
                \begin{tabular}{|l|l|}

                    \hline
                    \textbf{Dimension}   & \textbf{Explanation}                                                                                                                                                   \\ \hline
                    Mental demand (MD)   & \begin{tabular}[c]{@{}l@{}}The mental and perceptive activity\\ demanded by the task (chose, decide,\\ think, calculate, search, etc.).\end{tabular}                       \\ \hline
                    Physical demand (PD) & \begin{tabular}[c]{@{}l@{}}The physical activity demanded by\\ the task (pull, lift, spin, drag, etc.).\end{tabular}                                                       \\ \hline
                    Temporal demand (TD) & \begin{tabular}[c]{@{}l@{}}The time pressure felt by the user.\\ A rating the leverages the time \\ available and the time necessary to\\ completed the task.\end{tabular} \\ \hline
                    Performance (PE)     & \begin{tabular}[c]{@{}l@{}}The user's satisfaction with its \\ performance or result the task.\end{tabular}                                                                \\ \hline
                    Effort (EF)          & \begin{tabular}[c]{@{}l@{}}A rating of the effort necessary \\ to achieve that performance felt by\\ the user.\end{tabular}                                                 \\ \hline
                    Frustration (FR)     & \begin{tabular}[c]{@{}l@{}}A rating of stress, annoy or irritation\\ felt by the user throughout the task.\end{tabular}                                                    \\ \hline
                \end{tabular}
        \end{table}
            
        Finally, it is important to highlight that, in order to have a comprehensive evaluation of the mental workload, it is recommended not to choose only one method, but to combine methods from the three classes (task performance, physiological measurements and subjective questionnaires). Mental workload is multidimensional and can reflect partially or differently in each of the methods \cite{sanders1998human}.