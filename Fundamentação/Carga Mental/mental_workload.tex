
Mental workload (MWL) is one of the main concepts studied in Human Factors  and is not a familiar concept to the most people \cite{stanton2004handbook}. A good way to explain it is with a analogy with physical workload \cite{stanton2004handbook}. When a athlete must lift a dumbbell (one of those gym's weights bars). The strength's demand from the athlete will be proportional with the the dumbbell's mass he/she is lifting. If the dumbbell is lighter than the athlete's capability, then it will be easy enough for him/her to lift it. So if the athlete is strong enough to carry the dumbbell, he/she will not feel a physical demand bigger than his/her capabilities. So the physical workload of this activity is properly fitted for this athlete.

If the dumbbell is heavier than the athlete can lift then two things can happen:
\begin{itemize}
    \item Or the athlete adapts to lift that dumbbell using tools (adjust the strategy)
    \item Or the athlete will not be able lift completely the dumbbell (performance degrades)
\end{itemize}

This is a scenario that represent an user, or operator, executing a task that is not fitted for their capabilities.

It is the same with MWL. Each human being has a finite mental capacity and can only use it with a limited number of tasks at the same time. If the sum of these mental demands are higher than the user's capacity, the user will need to adapt in order to finish those task, otherwise he/she will compromise the overall performance of those tasks.

Although, if the workload is too low, the same operator may get bored and easily distracted and so could also fail or not process the task's information.  

It's important to say that MWL is unique within each individual and is influenced by his/her perception of the task`s workspace but is also impacted by other factors outside the task itself and more related to the operator (like it's skill, age, education, training) or to the environment (like noise, heat and toxicity) \cite{cain2007review, fallahi2016effects, cardoso2012evaluation}.

MWL is not a quantitative resource or something that one can directly measure, but is has methods to infer it. The Figure \ref{fig:mwl_overview} has an overview of MWL and its measureament methods.
        
    \begin{figure}[!htb]
    \centering

    \tikzstyle{arrow} = [rounded corners, line width = 1mm, ->]
    
    \resizebox{0.85\width}{!}{
    \begin{tikzpicture}[node distance=1cm]
        
        \node (mwl) {\includegraphics[width=.15\textwidth]{Fundamentação/Carga Mental/mwl.png}}
        node(t_mwl)[below of = mwl,yshift=-0.75cm] {Mental}
        node(t_mwl2)[below of = t_mwl,yshift=0.5cm] {workload};
        
        \node (demand) [above of=mwl, xshift=3cm, yshift=2.5cm] {\begin{tikzpicture}[node distance=1cm]
    \centering
    
    \node (worker) {\includegraphics[width=.15\textwidth]{Fundamentação/Carga Mental/work.png}};
    
    \node (energy1) [above of = worker, xshift = 0.5cm, yshift = 0.05cm]
     {\includegraphics[width=.060\textwidth]{Fundamentação/Carga Mental/bolt.png}} 
    node(energy2) [right of=energy1, xshift=-0.5cm, yshift = 0.3cm] {\includegraphics[width=.060\textwidth]{Fundamentação/Carga Mental/bolt.png}}
    node(energy3) [left of = energy1, xshift=0.5cm, yshift = 0.3cm] {\includegraphics[width=.060\textwidth]{Fundamentação/Carga Mental/bolt.png}};;
    
\end{tikzpicture}}
        node(t_demand)[below of = demand,yshift=-0.75cm] {Task}
        node(t_demand2)[below of = t_demand,yshift=0.5cm] {Demand};
        
        \node (capacity) [above of=mwl, xshift=-3cm, yshift=2.5cm] {\includegraphics[width=.15\textwidth]{Fundamentação/Carga Mental/full-battery.png}}
        node(t_capacity)[below of = capacity,yshift=-0.75cm] {Mental}
        node(t_capacity2)[below of = t_capacity,yshift=0.5cm] {Capacity};
        
        \node (tasks) [left of=mwl, xshift=-4cm, yshift=-6cm] {\includegraphics[width=.15\textwidth]{Fundamentação/Carga Mental/multitasking.png}}
        node(t_tasks)[below of = tasks,yshift = -0.75cm] {Primary and} 
        node(t_tasks2)[below of = t_tasks,yshift = 0.5cm] {secondary tasks};
        
        \node (physiological) [below of=mwl, yshift=-5cm,] {\includegraphics[width=.15\textwidth]{Fundamentação/Carga Mental/physiological.png}} 
        node(t_physiological) [below of = physiological, yshift = -0.75cm]{Physiological}
        node(t_physiological2) [below of = t_physiological, yshift = 0.5cm]{measurements};
        
        \node (subjective) [right of=mwl, xshift=4cm, yshift=-6cm] {\includegraphics[width=.15\textwidth]{Fundamentação/Carga Mental/subjective.png}}
        node(t_subjective) [below of = subjective, yshift = -0.75cm]{Subjective} 
        node(t_subjective2) [below of = t_subjective, yshift = 0.5cm]{measurements};
    
    
        \draw [arrow] (t_mwl2.south) to ++(0,-0.75) to +(-5,0) to (tasks.north);
        \draw [arrow] (t_mwl2.south) to (physiological.north);
        \draw [arrow] (t_mwl2.south) to ++(0,-0.75) to +(5,0) to (subjective.north);
        \draw [arrow, sharp corners] (capacity.east) to +(1.6,0) to (mwl.north);
        \draw [arrow, sharp corners] (demand.west) -- +(-1.35,0) to (mwl.north);
    
        
    \end{tikzpicture}
    }
    \caption{A overview of mental workload and the techniques to infer it.}
    \label{fig:mwl_overview}
\end{figure}    
    
    \subsection{Task Performance}
    \label{subsec:task_performance}
    
        If the MWL influences on the task perfomance, then it would be possible to infer it using the performance's variation of a task. Because there are cases that the user's mental capacity is too high for only one task, two tasks are designed. In these evaluations, the user is asked to maintain a good perfomance level and still try to execute both tasks. Both tasks are similar and use the same kind of skill. \cite{stanton2004handbook, sanders1998human}.
        
        For example, an experiment to assess MWL in a flight simulator that uses two tasks:
        \begin{itemize}
            \item Fly a fighter aircraft and maintain a good performance level;
            \item Mentally sum two random numbers that appear on the screen. If the numbers' sum is odd, then the pilot should press left on the keyboard, if the result is even the he/she should press right.
        \end{itemize} 
    
        If the pilot's performance at the second task is too low, it means that the demand from the first task is too high for him/her to be able to pay attention on it, than it means that the MWL at the flight was high \cite{mohanavelu2020cognitive}.
        
%    \item Physiological measures;
    \subsection{Physiological measures}
    \label{subsec:physiological_measures}
    
        There are many physiological reflexes that one can use to assess MWL. These measures are a good, unbiased method to assess MWL \cite{fallahi2016effects}, but, still, it is recommended that they are evaluated alongside other method. It is possible to extract MWL information from the heart and brain activity \cite{chakladar2020eeg, orlandi2018measuring}, skin conductance, eye movement, pupillary contraction \cite{stanton2004handbook, rodriguez2015pupillometry} This master's thesis it is used heart activity and skin conductance.
        
        \subsubsection{Heart rate and heart variability with electrocardiogram (ECG)}
        \label{subsubsec:ecg}
        
            Electrocardiogram is a recording of the heart's electrical activity. With this recording one can verify the heart's interval between heartbeats and frequency (heart rate, HR), and other statistical parameters such as the standard deviation and the mean error (heart rate variability, HRV) and these are a good way to assess MWL \cite{cain2007review}. This is a simple and non-invasive method used in many human factors' experiments \cite{mohanavelu2020cognitive, mansikka2016fighter, zhang2014detection}.
        
            The heart activity is controlled by the sympathetic and parasympathetic nervous systems. These systems are responsible to control many of the body's autonomous activities \cite{stanton2004handbook}. (DEFINIR MELHOR)
        
            During a task that has a mental demand the user's heart activity changes with MWL. The higher the MWL, higher the HR and lower the HRV. This happens because of the mechanism that controls our heart activity. These are consequences of two reactions in our system when in a mental demand situation \cite{stanton2004handbook}.:
            
            \begin{itemize}
                \item A decreased parasympathetic nervous system activity and;
                \item An increase sympathetic nervous system activity.
            \end{itemize}
    
        \subsubsection{Electrodermal response with galvanic skin reaction (GSR)}
        \label{subsubsec:gsr}
        
        One of the electrodermal activity that can happen in our skin is controlled by the the sweating and the moisture level and both can be used to reveal changes in our sympathetic system \cite{nourbakhsh2012using, shi2007galvanic}. So its origin lies solely in the sympathetic branch of the autonomic nervous system as is MWL \cite{stanton2004handbook}. EDA is being used to assess stress, emotion, arousal, mental strain and cognitive activity \cite{nourbakhsh2012using, stanton2004handbook, shi2007galvanic}m also used to evaluate the usability of HCI systems \cite{shi2007galvanic} and some are to assess the mental workload \cite{zhang2014detection, borghini2014measuring}.
    
    \subsection{Subjective measures}
    \label{subsec:subjective_measures}    

        It is discussed if one should only use subjective measures to measure MWL \cite{sanders1998human, stanton2004handbook}. They are sensitive to perceived difficulty, automation, concurrent activities and demand for multiple resources. These test can be unidimensional, that are simplier but has only a general workload score \cite{stanton2004handbook}, or multidimensional. Some example of the latter is the Subjective Workload Assessment Technique (SWAT) and the NASA Task Load Index (NASA-TLX), both multidimensional tests. SWAT treats MWL as a load defined by three dimensions: time load; mental effort load; and psychological stress. In this test the user score each of these dimensions based on a 3-point scale while NASA-TLX uses 6 different dimensions.
        
        \subsubsection{NASA-TLX}
        \label{subsec:nasa_tlx}
        
            NASA-TLX is a questionnaire created by \citeonline{hart1988development}. It is answered by an user who has just completed a task/activity that someone wish to infer its MWL. This questionnaire will assess the task's MWL felt by that user with 6 rating scales and each of these is explained, ideally, at the experiment's briefing. The Table \ref{tab:nasa_dimensions} presents each scale with a description of it.
        
            \begin{table}[htb]
                \centering
                \caption{NASA-TLX dimensions and the description of each dimension. \cite{stanton2004handbook}.}
                \label{tab:nasa_dimensions}
                    \begin{tabular}{|l|l|}

                        \hline
                       \textbf{Dimension}   & \textbf{Explanation}                                                                                                                                                   \\ \hline
                        Mental demand (MD)   & \begin{tabular}[c]{@{}l@{}}The mental and perceptive activity\\ demanded by the task (chose, decide,\\ think, calculate, search, etc.).\end{tabular}                       \\ \hline
                        Physical demand (PD) & \begin{tabular}[c]{@{}l@{}}The physical activity demanded by\\ the task (pull, lift, spin, drag, etc.).\end{tabular}                                                       \\ \hline
                        Temporal demand (TD) & \begin{tabular}[c]{@{}l@{}}The time pressure felt by the user.\\ A rating the leverages the time \\ available and the time necessary to\\ completed the task.\end{tabular} \\ \hline
                        Performance (PE)     & \begin{tabular}[c]{@{}l@{}}The user's satisfaction with it's \\ perfomance or result the task.\end{tabular}                                                                \\ \hline
                        Effort (EF)          & \begin{tabular}[c]{@{}l@{}}A rating of the effort necessary \\ to achieve that perfomance felt by\\ the user.\end{tabular}                                                 \\ \hline
                        Frustration (FR)     & \begin{tabular}[c]{@{}l@{}}A rating of stress, annoy or irritation\\ felt by the user throughout the task.\end{tabular}                                                    \\ \hline
                    \end{tabular}
            \end{table}
            
            This questionnaires evaluate only one task/activity. So if the user executed two tasks (like a primary and secondary tasks), he/she should be oriented to answer about primary task only, not a combination of both of them \cite{sanders1998human}.

        To measure mental workload, it is recommended not to chose only one measuring method, but more. MWL is multidimensional and can reflect partially or differently in each of the methods \cite{sanders1998human}.


            
    

   

    
