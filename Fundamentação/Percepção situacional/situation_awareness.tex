Situation awareness (SA) can be defined as "the perception of the elements within a volume of time and space (Level 1), the comprehension of their meaning (Level 2), and the projection of their status in the near future (Level 3)”. The term was first written in the aeronautic sector. Today is a key factor when designing complex and dynamic systems, i.e the aeronautic and automotive, medical and nuclear power plant systems \cite{endsley1995measurement}. 
%SA is the perception, comprehension and projection of the future of the elements inside an environment within a time and space interval. 
It is an important factor to make sure that the user will be capable to take important decisions correctly and achieve high-performance \cite{endsley1988design, endsley2018automation}. 

For example, when an air traffic controller looks at a radar display, he/she seeks to understand the aircraft's position and speed and then predict its position in the near future (5, 10 or 15 minutes after) \cite{sanders1998human}, or when a pilot reads the cockpit panel, understands their data then he/she can predict the next reading of that same instrument or some other status of the aircraft after a couple of minutes.
    
As is with MWL, SA is not a quantitative subject. The most common way to measure SA is by subjective methods, for example the Situation Awareness Rating Technique, but is not reliable since it can distort the participant's answer \cite{stanton2004handbook}, and the Situation Awareness Global Assessment Technique (SAGAT). The Figure \ref{fig:sa_overview} represents an overview of SA.

\begin{figure}[!htb]
    \centering
    \tikzstyle{arrow} = [rounded corners, line width = 1mm, ->]

    \resizebox{0.85\textwidth}{!}{
    \begin{tikzpicture}[node distance=3.3cm]
        \centering
        
        \node (info) [fill=white] {\includegraphics[width = 0.20\textwidth]{Fundamentação/Percepção situacional/information.png}}
        node(t_info)[below of = info,yshift=1.3cm] {\Large Information};
        
        \node (arrow) [right of=info, xshift=7.0cm, yshift = -0.5cm] {\begin{tikzpicture}[node distance=5cm, line width = 2.5mm]
    \centering

    \node (seta1) [] {};
    \node (seta2) [right of = seta1, xshift = 5.0cm] {};
    \node (seta3) [above of = seta2, yshift = -2.5cm] {};
    \node (seta5) [below of = seta1, yshift = -1.5cm,] {};
    \node (seta6) [right of = seta5, xshift = 5.0cm] {};
    \node (seta7) [below of = seta6, yshift = 2.5cm] {};
    \node (seta4) [right of = seta1, below of = seta1, yshift = 1.75cm, xshift = 10.0cm] {};

    \draw (seta1.center) -- (seta2.center);
    \draw (seta2.center) -- (seta3.center);
    \draw (seta3.center) -- (seta4.center);
    \draw (seta4.center) -- (seta7.center);
    \draw (seta7.center) -- (seta6.center);
    \draw (seta6.center) -- (seta5.center);
    \draw (seta5.center) -- (seta1.center);
    

\end{tikzpicture}
};
        
        \node (perception) [right of=info, xshift=2.5cm, draw, line width=2mm] {\begin{tikzpicture}[node distance=5cm]
    \centering
    
    \node (comprehension) {\includegraphics[width=3.0cm]{Fundamentação/Percepção situacional/eye.png}};

    \node(info) [] {\includegraphics[width=1.0cm]{Fundamentação/Percepção situacional/information}};

\end{tikzpicture}}
        node(t_perception)[below of = perception, yshift=0.9cm] {\Large 1st Level}
        node(t_perception2)[below of = t_perception, yshift=2.5cm] {\Large Perception};

        \node (comprehension) [right of=perception, xshift=0.25cm, draw, line width=2mm] {\begin{tikzpicture}
    \centering
    
    \node (comprehension) {\includegraphics[width=3.0cm]{Fundamentação/Percepção situacional/idea.png}};
    
    \node(info) [yshift = 0.5cm] {\includegraphics[width=0.75cm]{Fundamentação/Percepção situacional/information}};

\end{tikzpicture}}
        node(t_comprehension)[below of = comprehension, yshift=0.9cm] {\Large 2nd Level}
        node(t_comprehension2)[below of = t_comprehension, yshift=2.5cm] {\Large Comprehension};

        \node (projection) [right of=perception, xshift=3.75cm, draw, line width=2mm] {\begin{tikzpicture}[node distance=5cm]
    \centering
    
    \node (comprehension) {\includegraphics[width=3.0cm]{Fundamentação/Percepção situacional/future.png}};

    \node(info) {\includegraphics[width=1.5cm]{Fundamentação/Percepção situacional/information}};

\end{tikzpicture}
}
        node(t_projection)[below of = projection, yshift=0.9cm] {\Large 3rd Level}
        node(t_projection2)[below of = t_projection, yshift=2.5cm] {\Large Projection};

        \node (decision) [right of=projection, xshift=5.0cm] {\includegraphics[width = 0.30\textwidth]{Fundamentação/Percepção situacional/decision.png}}
        node(t_decision)[below of = decision, yshift=0.5cm] {\Large Decision}
        node(t_decision2)[below of = t_decision, yshift=2.5cm] {\Large making};

    \end{tikzpicture}
    }
    \caption{An overview of situation awareness and the SAGAT.} %and the methods to infer it
    \label{fig:sa_overview}
\end{figure}

    \subsection{SAGAT}
    \label{subsec:sagat}
    
        The Situation Awareness Global Assessment Technique is a method developed by \citeauthor{endsley1988situation} in \citeyear{endsley1988situation}. It is based on how the information is processed inside the user's mind. The test application is made by stopping the operator activity, usually made in a simulation, then asking the user some questions that were previously created based on the user's activity. These questions should be as similar as possible to how the person thinks when thinking about that information to avoid extra effort in understanding it \cite{stanton2004handbook}.

       Although the stopping during the activity may sound troublesome for the testing, empirical work has shown that that doesn't interfere with the user performance and the user memory can withstand a break as long as 5 to 6 min \citeauthor{endsley1995measurement}
