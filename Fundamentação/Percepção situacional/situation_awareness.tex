Situation awareness can be defined as “the perception of the elements within a volume of time and space (Level 1), the comprehension of their meaning (Level 2), and the projection of their status in the near future (Level 3)” as illustrated in Figure \ref{fig:sa_overview}. One example is when an air traffic controller looks at a radar display (Level 1). He/she seeks to understand the aircraft’s position and speed (Level 2) and then predict its position in the near future, 5, 10 or 15 minutes after (Level 3) \cite{sanders1998human}. Similarly, when a pilot reads the cockpit panel (Level 1), and understands their data (Level 2) then he/she can predict the next reading of that same instrument or some other status of the aircraft after a couple of minutes (Level 3).

The term “situation awareness” was first proposed for the Aeronautics domain and today is considered a key factor for designing complex and dynamic systems also from other domains, such as automotive, medical and nuclear \cite{endsley1995measurement}. It is an important factor to make sure that the user will be capable to make important decisions correctly and achieve high-performance \cite{endsley1988design, endsley2018automation}.

\begin{figure}[!htb]
    \centering
    \tikzstyle{arrow} = [rounded corners, line width = 1mm, ->]

    \resizebox{0.85\textwidth}{!}{
    \begin{tikzpicture}[node distance=3.3cm]
        \centering
        
        \node (info) [fill=white] {\includegraphics[width = 0.20\textwidth]{Fundamentação/Percepção situacional/information.png}}
        node(t_info)[below of = info,yshift=1.3cm] {\Large Information};
        
        \node (arrow) [right of=info, xshift=7.0cm, yshift = -0.5cm] {\begin{tikzpicture}[node distance=5cm, line width = 2.5mm]
    \centering

    \node (seta1) [] {};
    \node (seta2) [right of = seta1, xshift = 5.0cm] {};
    \node (seta3) [above of = seta2, yshift = -2.5cm] {};
    \node (seta5) [below of = seta1, yshift = -1.5cm,] {};
    \node (seta6) [right of = seta5, xshift = 5.0cm] {};
    \node (seta7) [below of = seta6, yshift = 2.5cm] {};
    \node (seta4) [right of = seta1, below of = seta1, yshift = 1.75cm, xshift = 10.0cm] {};

    \draw (seta1.center) -- (seta2.center);
    \draw (seta2.center) -- (seta3.center);
    \draw (seta3.center) -- (seta4.center);
    \draw (seta4.center) -- (seta7.center);
    \draw (seta7.center) -- (seta6.center);
    \draw (seta6.center) -- (seta5.center);
    \draw (seta5.center) -- (seta1.center);
    

\end{tikzpicture}
};
        
        \node (perception) [right of=info, xshift=2.5cm, draw, line width=2mm] {\begin{tikzpicture}[node distance=5cm]
    \centering
    
    \node (comprehension) {\includegraphics[width=3.0cm]{Fundamentação/Percepção situacional/eye.png}};

    \node(info) [] {\includegraphics[width=1.0cm]{Fundamentação/Percepção situacional/information}};

\end{tikzpicture}}
        node(t_perception)[below of = perception, yshift=0.9cm] {\Large 1st Level}
        node(t_perception2)[below of = t_perception, yshift=2.5cm] {\Large Perception};

        \node (comprehension) [right of=perception, xshift=0.25cm, draw, line width=2mm] {\begin{tikzpicture}
    \centering
    
    \node (comprehension) {\includegraphics[width=3.0cm]{Fundamentação/Percepção situacional/idea.png}};
    
    \node(info) [yshift = 0.5cm] {\includegraphics[width=0.75cm]{Fundamentação/Percepção situacional/information}};

\end{tikzpicture}}
        node(t_comprehension)[below of = comprehension, yshift=0.9cm] {\Large 2nd Level}
        node(t_comprehension2)[below of = t_comprehension, yshift=2.5cm] {\Large Comprehension};

        \node (projection) [right of=perception, xshift=3.75cm, draw, line width=2mm] {\begin{tikzpicture}[node distance=5cm]
    \centering
    
    \node (comprehension) {\includegraphics[width=3.0cm]{Fundamentação/Percepção situacional/future.png}};

    \node(info) {\includegraphics[width=1.5cm]{Fundamentação/Percepção situacional/information}};

\end{tikzpicture}
}
        node(t_projection)[below of = projection, yshift=0.9cm] {\Large 3rd Level}
        node(t_projection2)[below of = t_projection, yshift=2.5cm] {\Large Projection};

        \node (decision) [right of=projection, xshift=5.0cm] {\includegraphics[width = 0.30\textwidth]{Fundamentação/Percepção situacional/decision.png}}
        node(t_decision)[below of = decision, yshift=0.5cm] {\Large Decision}
        node(t_decision2)[below of = t_decision, yshift=2.5cm] {\Large making};

    \end{tikzpicture}
    }
    \caption{An overview of situation awareness and the SAGAT.} %and the methods to infer it
    \label{fig:sa_overview}
\end{figure}

As it is for the mental workload, situation awareness is not a quantitative subject. The most common way to measure it is using subjective methods, among which one of the most famous is the Situation Awareness Global Assessment Technique (SAGAT). It was proposed by \cite{endsley1988design} and is based on how the information is processed inside the user’s mind. The test application is made by freezing the operator activity, usually made in a simulation environment, and, then, asking the user some questions that were previously defined based on the user’s activity. These questions should be as similar as possible to how the person thinks when reasoning about the situation, in order to avoid extra effort in understanding it \cite{stanton2004handbook}.  Although freezing the activity may sound troublesome, empirical work has shown that it doesn’t interfere with the user performance and the user memory can withstand a break as long as 5 to 6 min \cite{endsley1988design}.
